% This is part of the Avaneya Project Crew Handbook.
% Copyright (C) 2010-2017 Cartesian Theatre™ <info@cartesiantheatre.com>.
% See the file Copying for details on copying conditions.

% Specialities chapter...
\StartChapter{Specialities}
It should be of no surprise to anyone that a project of this size calls upon a variety of disciplines. There is also nothing to stop someone from being involved in more than one capacity if they would like to. If you want to model, but also happen to be fluent in Italian, by all means consider translating as well.

\StartSection{Audio Engineer}

These people take audio samples or other sources and creatively master them into something useful for the game. An example would be sampling the squeak of a chair or a car driving by with a high end portable microphone, then remixed into the low rumble of a diesel engine idling on Mars.

These people perfect their craft with {\it libre} tools like Ardour, Rosegarden, and Audacity to name a few.

%\StartSection{Cinematic Artist}

%Cinematics play an important role in games, though perhaps less so today now that realtime renderers can out perform the pre--rendered cinematics of games in decades past. Cinematics prepare the player in setting the stage in ways that might be difficult to do during normal game play. 

%Cinematic artists work with tools like Blender, Cinelerra, and Lombard, to name a few. We will have more to say on this in the future because this is a very complex field involving many.

\StartSection{Conceptual Illustrator}

Illustrators prepare high quality concept artwork. This can include environmentals, machines, timeline scenes, character portraits, cinematic storyboards, and so on.

\StartSection{Engineer}

Engineers design and implement the engine specification. They do not only write engine code, but sometimes also the Lua scripts that drive it (game behaviour). They work mostly in the language of C++ and, in the case of writing shaders, GLSL. To get started as an engineer, see \in{chapter}[Engineer Contributors].

\StartSection{Modeller}

Modellers produce the 3D game models the users see during game play. They work with the 2D artists to ensure models are properly textured and lit. They work with Blender, Wings 3D, or any other modelling program that can export to standardized non--proprietary formats.

\StartSection{Musician}

Music falls into two categories with Avaneya, ambient and non--ambient. Ambient music is the music the user passively listens to during actual game play. Non--ambient music is more actively listened to during navigation menus, an opening cinematic, and so on. 

More likely the latter, but possibly the best of both, we will consider releasing at a later date on a separately released game score soundtrack. This could either be as a separate download, or possibly even on redbook.

\StartSection{Package Maintainer}

Package maintainers are responsible for building and preparing Avaneya packages. They need to be familiar with the standard method of rolling out software for a given platform. See \in{chapter}[Package Maintainer Contributors] for more information on how we plan to do this.

\StartSection{Quality Reassurance}

Quality reassurance looks for game play issues and stress tests our code looking for weaknesses. This is important to help developers find nasty bugs and triage project resources appropriately. They should be comfortable generating stack traces and using the Launchpad bug tracker.

Note that we used the terminology of quality reassurance and not quality assurance. This is because quality is best assured as a design time consideration. Quality reassurance reassures everyone that it still is after implementation.

\StartSection{Researcher}

Researchers provide the background information and attention to detail that makes this game so rich. Perhaps they have an interest in areology\index{areology} (the study of Mars), terraformation, atmospheric geography, simulation and complex modelling, social, political, and economic issues (e.g. the GPI), game mechanics, or what have you. They have some valuable area of expertise they would like to apply, even if they may not have the direct technical expertise to implement it themselves.

\StartSection{Script Writer}

Script writers author code in Lua that drives the game engine. They work closely with the engineers to ensure that the API's they need are exposed from the engine. Besides implementing game logic, they also implement the CEGUI graphical user interface logic.

\StartSection{SysOp}

System operators administer server side software and oversee the facilities open to the public. They also can act as moderators over our forums, IRC, mailing lists, Launchpad, and eventually the Solnet cluster described in \in{section}[Multiplayer: Solnet] to be rolled out.

\StartSection{Translator}

Translators localize software to meet the needs of a specific culture. This includes familiarity with the language, writing direction, idiom, and so on. 

Translators localize cinematic subtitles, documentation, the graphical user interface, website, and other interactive elements. They work with any tools that support standard GNU gettext message catalogues, such as Launchpad Translations. 

See \in{section}[i18n & L10n] for more information on internationalization and localization.

%\StartSection{Typeface Designer}

%Typeface designers are involved in all aspects of producing the high quality Avaneya Font Family\index{Avaneya Font Family}. This font family is used mainly at higher resolutions, the user interface, printed media, some documentation, DVD jewel case, and so on. They are familiar with the different font rasterization technologies that control the way fonts render on screen, as well as the {\it TrueType}\index{TrueType} font format and it's successor, {\it OpenType}\index{OpenType}. The fonts produced must ultimately be compatible with {\it SDL_ttf}\index{SDL_ttf}.

\StartSection{Voice Actor}

Cinematics and in game audio need real people to play a role of a given character. We have a number of leading characters, and therefore a need for a number of people.

\StartSection{Web Developer}

Web developers are familiar with the relevant international standards. They work with technologies like HTML5, PHP, CSS3, jQuery, MySQL, JavaScript, and SVGA in building and maintaining our website. These are important because they are the best way to ensure a reliable and consistent user experience -- so far as one can hope for.

\StartSection{Writer}

Our writers are either of the creative or technical kind, or both. Those more interested in elements of story, dialogue, characters, in game fictional literature, and other creative aspects of the game are creative writers.

Technical writers, on the other hand, are responsible for writing and maintaining technical documentation. They work with powerful document engineering software like \BIBTEX\ and \CONTEXT. These are needed for typesetting our literature into books like the one you are reading now.

To get started as either, see \in{chapter}[Writer Contributors].

\StopChapter

