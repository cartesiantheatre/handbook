% This is part of the Avaneya Project Crew Handbook.
% Copyright (C) 2010, 2011, 2012
%   Kshatra Corp.
% See the file License for copying conditions.

% Internationalization and Localization section...
\StartSection{i18n & L10n}

What do {\it i18n} and {\it L10n} stand for? They are old industry acronymns for internationalization and localization respectively. Internationalization is the process of adapting software so that it can accommodate different localities. Localities, or sometimes just called locales, specify data specific to a given locality, region, or culture, of the world. This usually includes the user's preferred language, writing direction, numeral system, and so on. Thus, localization is the process of creating data, for a specific locale, for a specific internationalized application. It involves the work of translators.

\startitemize[4]
\head {\em GNU gettext}

GNU gettext is used for localization and translation to different human languages by making it possible to substitute strings that are marked for translation in our C++ code with a string from another language.

\head {\em lua-gettext}

The lua-gettext API is a Lua package that acts as a Lua wrapper for gettext bindings.

\head {\em lua-xgettext}

The lua-xgettext tool is a small program for message extraction of marked strings from Lua code so they can be made available for translation to other languages. It is similar to GNU xgettext, but far more primitive; it just extracts the strings and prints them out without any additional information. It was written because GNU `xgettext` did not support Lua at the time it was created.
\stopitemize

