% This is part of the Avaneya Project Crew Handbook.
% Copyright (C) 2010, 2011, 2012 Cartesian Theatre <kip@thevertigo.com>.
% See the file Copying for details on copying conditions.

% Communication section...
\StartSection{Communication}

\StartSubSection{Satellite Uplink With Earth}
Communicating with Earth is unfortunately not without high latency. It takes between 4 to 20 minutes to communicate between the two planets, depending on the position of their orbits. When communication is brokered by satellite, they will lose their connection when their orbit takes them to the far side of the planet.

\StartSubSection{Personal Citizen Band Radio}
In terms of personal communication on the surface of the planet, a 2 m antenna should work in the 144 MHz band (VHF). Line-of-sight communication is required, but it is useless past 40 km because the planet is not very large.

\StartSubSection{Ham Radio}
In the event that satellites are not available, either because they fail or are too expensive, ham radio is a cheap and viable backup. Ham radio works by reflecting radio signals off of the ionosphere in the shortwave radio band. The {\it Mariner 9} and {\it Viking Mission} orbiters and landers provided us with a great deal of information on the Martian ionosphere which consists of about 90\% \chemical{O_2} cations, so we know this is possible to do.

Being able to communicate with a ham radio on Mars will be dependent on the electron density peak which will determine the usable frequencies. This is directly proportional to the square root of the electron density. On Earth, this is as high as 20 MHz which provides a great deal of bandwidth. On Mars, this will vary based on the time of day. During the day, the electron density at an altitude of 135 km is 200,000 \math{e/cm^{3}}, providing 4 MHz of bandwidth. At night, the electron density is reduced to 5,000 \math{e/cm^{3}} at an altitude of 120 km. That affords only 700 KHz of bandwidth. That is pretty low for video, but sufficient for voice and any engineering telemetry.

One of the added benefits of ham radio on Mars is that, unlike on Earth, signals suffer very little attenuation. This is because there is less interference since there are no distant thunderstorms or other radio stations.

