% This is part of the Avaneya Project Crew Handbook.
% Copyright (C) 2010, 2011, 2012
%   Kshatra Corp.
% See the file License for copying conditions.

% Scenarios section...
\StartSection{Scenarios}

Scenarios are at the heart of Avaneya. They can be selected specifically through the single player's custom mode, enjoyed with your friends over a LAN, or incorporated in creative ways in the single player's campaign mode -- which is itself just a series of well scripted scenarios. But not all scenarios can be played in all game modes. Some are available only in the single player campaign mode, while others only for multiplayer over LAN.

These scenarios seek not only to entertain, but also to inform by drawing upon issues in the environmental, social, political, economic, agricultural, and other realms. But each scenario should never draw strictly from only one of the aforementioned. Remember that, as we discussed back in \in{section}[Leitmotifs], everything is connected -- even if not always apparent. As an example, the issue of water privatization is relevant to discussions in at least all of the aforementioned.

In all scenarios, the user is responsible for directly controlling, working with, or maintaining Arcadians and their infrastructure. At other times, however, they may be micromanaging only a handful of units perform a task, such as assisting a remote survey team locate water ice as opposed to erecting or managing an entire city.

Avaneya is different from other strategy games in that you are not the only source of influence over the units you think you are in control of. Even though you at times may be functioning like a mayor of an Arcadian city, there are other influences that will attempt to reshape the world you are immersed in. These could be, depending on the scenario, transplanetary corporations, the United Nations, Terran intelligence agencies, and others, with the lines frequently blurring between them. At times you will feel powerless. At other times you will know exactly what you need to do.

Scenarios can incorporate multiple goals. These goals, depending on how the scenarios are scripted, play a role in conditionally determining whether the user has successfully completed a scenario or not. In the case of the single player's campaign mode, a scenario must be completed before the user can advance to the next one.

Let us now take a look at some of the general types of scenarios. Bear in mind that this is only a subset of all the possibilities, since new ones can and will be written to drive the engine in novel ways. You will note that the Genuine Progress Indicator (GPI) described \in{section}[Genuine Progress Indicator] is a recurring theme.

\startitemize[4]
\head {\em No Goal} 

A scenario may be goalless. In this scenario, the user erects their city and is responsible for its management without thought of any terminal condition for success, such as a time limit. But as with any other scenarios, the user can still strive to maximize their GPI.

\head {\em Achieve Minimal GPI}

The user must achieve a minimal GPI. This may or may not involve a time constraint. Although the GPI is a recurring theme in Avaneya, scenarios that require this goal bring greater attention to it than others.

%\head {\em Free Trade}

%The North American Union would like Arcadia to become a signatory to the North American Union Free Trade Agreement.

\head {\em Improve Energy Supply}

Your city needs energy and it may not have enough to power everything on the grid. The user must adjust factors in their city that will improve the situation. This may mean building a nuclear plant, a geothermal well, erecting photovoltaic panels, or perhaps looking for ways to increase efficiency, such as examining how much energy is required to produce the different types of things that people eat and exploring alternatives.

\head {\em Improve Public Health}

Many factors effect public health, among the most important are what the public consumes. But other factors contribute as well, such as education, annual take home income, available free time, and many others. Scenarios that depend on this goal require the user to improve the public's health.

\head {\em Improve Transportation Network}

As a city grows, efficiency of movement can become a problem. This goal requires the user to have their city's transportation network increase in efficiency.

\head {\em Natural Disasters}

The user must rebuild their city in response to damage sustained from solar flares, micro meteorite strikes, land slides, marsquakes, or other natural disasters. Yes, Mars experiences all of the aforementioned. But some \quote{natural} disasters can be caused through human mishandling of the natural environment. For example, if an entire city block is swallowed whole into the ground, chances are the aquifer it was sitting on was depleted. In many situations, the user may be given a chance to avert the disaster entirely.

\head {\em Prevent Staged Terror}

Staged terrorism is as common on Mars as it is on Earth. The CIA will work tirelessly to stage acts of terror to be blamed on arbitrary groups -- which are usually just fronts. The user must use whatever means necessary to prevent acts of staged terror. If they fail to do this, as an example, the NAU could claim sabotage of a mineral extraction site they are leasing from your city as justification for a troop deployment.

\head {\em Protect Another City}

The user must come to the assistance of another Arcadian city as it comes under siege at the hands of an armed foreign aggressor, such as North American Union troops, the United Nations peacekeepers, CIA backed contras, private security contractors, or what have you. Constraints may vary, such as preventing the city from exceeding a certain threshold of material loss, the protection of its inhabitants, and so on.

\head {\em Protect From Corporate Media}

People can never be any more free than they are uninformed. They make decisions that are influenced by what they know and the quality of information they are fed, among other things. If the source of the information has a conflict of interest, those that pay heed will operate with information that probably does not consider their best interests.

You may need to reduce the influence corporations wield in socially engineering public opinion. Their influence may be overt, such as through advertising, or it may be more subvert, such as through funding junk science\index{Junk science} at a university.

\head {\em Protect Human Rights}

Articles XI --- XXVI of the Rubicon Act, as described on \at{page}[item:rubicon_act_fundamental_rights_first], enumerate the fundamental freedoms and human rights Arcadians enjoy with respect to their new republic. The user is responsible for ensuring that they are upheld.

\head {\em Recover From GMO Terminator Gene}

People that have nothing to eat will eventually die. If the city's greenhouses produce nothing after having changed little other than adopting new genetically modified seed stock, chances are the greenhouse did what farmers had been doing for centuries. They stored seed and planted it with the expectation that life would grow, only to find that the seeds had been genetically programmed by the vendor to be non-reusable. This way, your city's food supply is held hostage to the vendor since it will have to buy new seeds annually, even though there are still plenty in stock.

Scenarios that depend on this goal require the user to restore their annual agricultural yield to a variable minimum and to ensure that the food is safe to eat.

\head {\em Recover From Water Privatization}

Without water, you cannot survive. A corporation such as Bechtel-Biwater\index{Bechtel-Biwater} may attempt to privatize the artesian aquifers or water ice sites your city depends on. A judicial system may be of use, or you may have to recover the sites through use of force if it proves corrupt.

\head {\em Reduce Crime}

The user must reduce crime by addressing factors, such as education, affordable living, health, and other fundamental factors and environmental conditions that give rise to crime.

\stopitemize

