% This is part of the Avaneya Project Crew Handbook.
% Copyright (C) 2010, 2011, 2012
%   Kshatra Corp.
% See the file License for copying conditions.

% Contact section...
\StartSection{Contact}

\StartTimelineDate{12 Leo, 2 A.R.}
Elements of the {\it 1\high{st} People's Own Arcadian Mechanized Infantry} alert their chain of command that they have detected a non-Arcadian nav-aid transponder signal within their demarcation while conducting a routine dismounted patrol.
\StopTimelineDate

\StartTimelineDate{20 Leo, 2 A.R.}
Arcadia comes under fire from what appears to be orbital shelling directed by UNMEPF's forward observers\footnote{A forward observer is an artilleryman who is in a forward position directly overlooking the area to be shelled. By communicating with an artillery unit, he directs and corrects their fire. He is needed because sometimes the actual artillery does not have direct line of sight with the target.} on the ground. The event marks the first shots ever fired in anger on Mars. The original {\it Arcadia Hall} is successfully targeted and destroyed, though most of the city is deliberately left intact. 

Having realized that the city had been evacuated, MARSCOM relies on their {\it Bhadra I} photographic reconnaissance satellite to identify all possible locations suspected of harbouring the settlers' emergency underground shelter. Phantom targets across the surface of {\it Arcadia Planitia} are hammered from what appears to be the fast moving Phobos for more than an hour every sol over the next twenty sols. There are no casualties.

Between shellings, Arda has personnel from the {\it 3\high{rd} People's Own Arcadian Combat Support Engineers} erect a rudimentary high-gain directional antenna. This circumvents the need for access to a satellite communication relay normally necessary to transmit back to Earth by instead relying on S-band microwaves. 

Arda explains to anyone able to receive him on Earth what is happening on Mars. He suspects that the average Terran has probably been left in the dark by their governments and mainstream media. He describes the United Nations' shelling campaign as an attack on peaceful and defenceless civilians levied from a manned orbital installation on Phobos. The installation apparently functioning as an orbital artillery battery. He claims the UN's intentions are to force the settlers to surrender so that they can resume applying themselves and their extensive expertise as needed for the resumption of exports under a newly restructured administration.

Arda references a document hacked from a UNSA intranet as evidence of the battery's existence. He had received it prior to Arcadia having lost solnet connectivity when UNSA's {\it METO} communications satellite was taken out. The document was part of the dossier he received from the anonymous tip-off that had revealed the United Nations' intentions for use of force. It is titled {\it Project Horizon: A UNSA Study for the Establishment of a Phobos Artillery Battery}.

The detailed white paper, complete with schematics, describes the first orbital artillery battery ever put into service. The classified document details an ambitious covert UNSA project to construct what it refers to as the {\it Thor Battery}. The outpost is apparently manned by a garrison of a single platoon of thirty soldiers, powered by three nuclear reactors, and located on and within the Martian moon Phobos. 

\startTimelineGeneralDocument
...The Phobos outpost is required to develop and protect strategic UNSA administered interests on Mars; to develop techniques in Phobos and general orbital-based surveillance of Mars, in communications relay, and in operations on the surface of Mars; to serve as a base for exploration of Phobos, for further exploration into space and for military operations on Phobos if required; and to support scientific investigations on Phobos...

\hskip 1.5cm {\it -- Project Horizon: A UNSA Study for the Establishment of a Phobos Military Outpost}
\stopTimelineGeneralDocument

Construction was feasible due to the moon's composition. From the surface down to to the first one-hundred metres, the moon is covered with a very fine regolith. The remaining solid material beneath it is of a mostly porous composition. These two characteristics allowed for a straightforward excavation in ensuring that most of the installation could remain subterranean.

Arda provides high resolution imagery taken by Senka from the Martian surface a month prior with the transmission as evidence. The images clearly depict an outpost still under construction within the moon's Stickney crater\index{Stickney crater}. He explains that the outpost was originally not intended for active service yet, but rather as part of a long term United Nations' insurance policy for dealing with \quote{\it the distinct possibility of future settlement anomalies}, should the need ever arise. This option was on the table, but only to be taken seriously well after, as MARSCOM had expected, a straightforward capture of Arcadia with the expectation that its inhabitants would not be capable of defending themselves.

Arda illustrates that by exploiting a legal loophole in the {\it Outer Space Treaty}, the {\it Anti-Ballistic Missile Treaty}, and the {\it Mars Treaty}, whose draftings predated the advent of kinetic bombardment weapons, UNSA believes it is justified in targeting the surface with hypersonic shells containing tungsten cores. The shells are capable of hitting targets with surgical precision at downward velocities approaching 10 km/s. The treaties do not observe kinetic bombardment as weapons of mass destruction since they require no explosive payload, though the shells filled with more massive cores can deliver sufficient kinetic energy in the order of a small tactical nuclear weapon.

The transmission is picked up on Earth by amateur radio operators and quickly disseminated over solnet peer-to-peer networks. Though mainstream media do not acknowledge the transmission, dissemination is sufficiently rapid and widespread that Terran citizenry are best described as mostly of a mixture of fury and disgust at the actions of their respective government's involvement. Amnesty International calls for a public inquiry with powers of subpoena as well as the presence of Red Cross observers on the ground.

UNSA dismisses Arda's claims as \quote{\it baseless allegations propagated by Terran conspiracy theorists, intertwined with a misunderstanding of the nature of meteorite strikes on Mars, used to justify the theft and reckless destruction of a costly set of off-world public assets}. A UNSA spokesman refuses to respond to existential questions over with the now controversial Phobos based outpost. 

The subject is controversial since the exposure of a clandestine orbital installation, if the sensitive document Arda had cited is genuine, provides strong evidence that there were ulterior motives behind the {\it Avaneya Initiative}, given that the purpose of the {\it Thor Outpost} clearly has little to do with sustainability research. This observation increases in popularity on Earth and helps to promote Terran support for Arcadians as the public consciousness revisits Arcadia's rationale for their declaration of independence.
\StopTimelineDate

% Constraint: 20 sols after Thor Outpost commenced shelling...
\StartTimelineDate{40 Leo, 2 A.R.}
A column of an UNMEPF armour and a battalion's worth of mechanized infantry are spotted approaching the city by an Arcadian Dragoons armoured reconnaissance platoon commanded by Henrik. The Dragoons have strict rules of engagement to not engage UNMEPF forces unless they cross the perimeter extending two kilometres outward from Arcadia's surrounding impact crater's elevated rim.

The UNMEPF column halts in response to Henrik ordering the Dragoon's lead tank to fire a warning shot. The shell lands several hundred metres directly ahead of the column's line of movement. Arda addresses UNMEPF's commanding officer over the tactical radio's common band. He advises him that UNMEPF forces will be met with a measured response if they continue to advance and do not immediately withdraw outside RAP's demarcation. Brigadier General Dragov responds by identifying himself as the commanding officer and the brigade he is responsible for. Dragov asserts that if the settlers' units surrender themselves immediately and disclose the location of their emergency shelter, they will be given quarter. Arda repeats the original directive and breaks radio contact.

Having not anticipated the presence of a militia, Dragov liaises over satellite with MARSCOM's Joint Chiefs of Staff at Peterson Air Force Base, North American Union, for orders. He is advised that MARSCOM's photographic reconnaissance satellite {\it Bhadra I} shows only a single platoon of Dragoons of four light tanks on the inside lip of the impact crater's rim. Dragov orders a tactical retreat of the brigade to withdraw eight-hundred meters backward.
\StopTimelineDate

\StartTimelineDate{41 Leo, 2 A.R.}
Dragov takes advantage of {\it Thor Outpost}'s two and a half hour launch window. A UNMEPF Weapons Control Officer operating from in orbit requests that the forward observers provide grid references for the Dragoons. The orbital artillery battery loads shells containing smaller calibre tungsten cores to minimize damage to the adjacent city. The battery fires a salvo.

The {\it 3\high{rd} People's Own Arcadian Combat Support Engineers} alert the Dragoons that an infrared signature in orbit has been detected. The engineers know through previous experience observing {\it Thor Outpost} that, though unlocalizable, the signature is characteristic of an orbital kinetic launch which always precedes bombardment. The Dragoon's platoon commander, Capt Henrik Nørgaard, instructs his crews to make use of their vehicles' grenade-launchers to fire smoke canisters containing an infrared reflective additive. The canisters are fired directly above themselves to screen their movement from orbital surveillance. Dismounting the vehicles, Capt Henrik Nørgaard leads the platoon to seek refuge within tunnels previously excavated for the occasion. The platoon rendezvous two-hundred metres below the surface with an anti-tank platoon of the {\it 1\high{st} People's Own Arcadian Mechanized Infantry}.

Eleven minutes later the salvo completes its atmospheric entry, destroying the tanks with surgical precision. Falling rocks rolling down the slope of Arcadia's impact crater and chunks of shredded airborne permafrost moving at high velocity depressurize an evacuated habitat cluster within Arcadia's Third Precinct. There are no casualties.

Arda contacts Dragov over the tactical radio's common band. He advises the commander that the Arcadian Militia are prepared to surrender, provided Dragov will provide quarter. Dragov is agreeable and advises a jubilant Joint Chiefs of Staff over satellite of Arda's disposition. Arda provides a grid reference within Arcadia's impact crater of their underground emergency shelter and advises UNMEPF to proceed through the city's main entrance on the rim of the crater. 

As darkness descends rapidly, Dragov reasons it prudent in not holding fast until the morning. He is weary of taking the brigade through Arcadia's main entrance along the crater's rim, suspecting it mined. He advances the brigade without contest from eight-hundred metres outside the two kilometre perimeter previously withdrawn to, forward to meet with the base of the slope of the impact crater's ridge leading into the city. Dragov instructs his officers that this is necessary to feign committing themselves to the city's main entrance. The column performs a ninety degree turn at the base of the slope, altering its course to carry on clockwise along the base of the ridge rapidly for two kilometres where its apex meets the site of the destroyed dragoons.

Dragov, having earlier studied {\it Bhadra I} photographic reconnaissance imagery, noted that the site of the destroyed Dragoons on the impact crater's ridge cleared enough of the outer ridge away to allow armour and the rest of the brigade to proceed with an element of surprise through it. The route can lead them along the inside of the ridge for about eight-hundred metres, then down the route the Arcadian Dragoons probably had ascended up based on, what is clear now to Dragov, what must have been armoured rovers' tank tracks leading somewhere within the city.

Having arrived at the base of the destroyed Dragoons, UNMEPF forces rapidly negotiate the incline of the ridge. Their armoured squadron penetrates through the ridge with three battalions of mechanized infantry and combat support behind it. By the time the last of the brigade passes through the blasted rim, the front reaches the end of the path along the inner ridge as it prepares to descend down into the city. Moving with haste, the column fires flares over the less illuminated portions of the city. 

The convoy passes by an anti-tank platoon of the {\it 1\high{st} People's Own Arcadian Mechanized Infantry} concealed under thermal reflective blankets. The lead tank at the fore of the UNMEPF column is rendered immobile as it makes contact with a mine, unravelling itself of its tank tracks. At that moment, the rearmost combat support vehicle is hit simultaneously and taken out with the platoon having engaged them with several rounds of their recoilless shoulder fired rifles from elevated firing positions. The lead tank's main gun is then knocked out. Multiple vehicles within the column report contact as they come under fire.

With the lead tank destroyed and the engine block blown out of the rear vehicle, the UNMEPF column is immobilized between the two. The column's mechanized infantry immediately dismount their armoured personnel carriers. They attempt to take cover behind the only available, the vehicles themselves. Arcadian infantry release several salvos of cryonades down the ridge at the halted column. The explosive ordnance, having a fragmentation range more than twice that of Earth and virtually no atmospheric attenuation, send an array of vehicles rolling down the incline -- including Dragov's staff rover. Many of those having sought protection behind the vehicles are knocked down the ridge, pinned under them.

UNMEPF forces attempt to return fire, but have difficulty sighting well dug in targets in the very low Martian night light. Illumination flares prove of little use as the brigade becomes immersed in its own gas clouds as pressure suits are rapidly punctured, degas, and the occupants bleed out in the near vacuum of \math{-72^{\circ}}C.

MARSCOM receives UNMEPF's contact report and responds with only satellite assisted target acquisition -- the beligerants being too proximate for orbital shelling fire support. In the six minutes that pass from the time of the brigade's contact report transmission and its reception on Earth, the brigade's second in command in the second vehicle nearest the rear, Major General Nilhara's updated situation report is already {\it en route} back to Earth. Nilhara informs the Joint Chiefs of Staff that Dragov has been killed in action, that they have sustained heavy casualties, and that UNMEPF is preparing to negotiate its unconditional surrender to Arcadian forces.
\StopTimelineDate

