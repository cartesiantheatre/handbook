% This is part of the Avaneya Project Crew Handbook.
% Copyright (C) 2010-2016 Cartesian Theatre <info@cartesiantheatre.com>.
% See the file Copying for details on copying conditions.

% Game Mechanics chapter...
\StartSection{Workforce Interaction}


%--------------> WORKFORCE UNIT DECISION MAKING <------------------------------
\StartSubSection{Workforce Unit Decision Making}


%--------------> GROUP DYNAMICS <---------------
\StartSubSection{Group Dynamics}

A group is 2 or more units together within a set vicinity undertaking the same task/action. Groups can be longer term, called static, or brief and fleeting, called fluid.

%Static vs Fluid
Static groups are groups a unit will be in over an extended period of time, they will either meet with members or this group regularaly or seek them out. Examples of static groups are families, co-workers at work, and friends. 

Fluid groups are groups formed spontaneiously and that only last for a short duration or single event. Examples of fluid groups are people meeting in the street and chatting, or large group events such as protests or concerts.


%properties of a group
-Id number [group location and time]
-Units in group.
-Interests that group shares
-Variables that group shares

The group identity number is the key unique identifier of the group that is used to tag it to units associated with it.

The units in the group is a list of the unit id variable of each unit tagged into this group.

Interests and variables are properties of a group in order for the group to be compared by a unit against any other group to determine if it is dropped from the permenant list.



%Forming friendship groups / fluid groups on like interests.
Friendship groups form when units meet by chance or as part of another group and have ... number of workforce variables in common, more variables leeds to a stronger friendship. Same interests increase the friendship further. 
%Figure

%Example
%Figure

%Example
%Figure

%Example
%Figure



%Keeping friendship groups.
Each unit is only allowed ... number of static friendship groups. When this number is reached units will not have additional friendship groups unless the new group is of higher quality. The quality of friendship groups is determined by the number of interests and variables that units share, with more being of greater quality and interests being of greater quality that variables.
%Example
%Figure

%Permanent friendship groups
Units will make or attempt to make contact with well liked units they are in a friendship group in order to meet belonging need. 

The number of interests/variabls shared by the unit with others in the group determine the order in which units seek to meet up. Better quality friendship groups are sought after in descending order.
%Figure
%Example

This is the order in which groups are sought to meet belonging needs, however they do not effect how fast the belonging need is met. It is thought that while this makes people feel like they belong more, it also meens they would want to be with these people more. And since the belonging need is the key factor in determining when units seek other units as company, both attributes of a good quality group cancle eachother out to make the unit continue to seek this group out. 


%Organising group meetups and how this happens.
%Also relevant for rallies, concerts, etc.
%Group dynamics in emergencies or scale events
Units can make contact with other units to meet up. They can attempt to contact 1 or more other units in the group to meet up. Methods of doing this vary and depend on what is available to each unit as well as technology that is available. There are ... broad means of communicating with units at a distance in order to meet up, these are:

Vicinity communication: this entails organising to meet up where units are in the immediate vicinity, either face to face or within ear shot.

One way communication: these methods require the organising unit to send a message, then wait for a reply. The reply usually takes some time. Examples include notice boards, email, snail mail. These can be sent to multiple or single units.   

Two way communication: these methods utilize an originating unit who sends the message to a unit or several units who recieve the message and are able to immediately reply. Examples of this type of service include phones, messenger apps, and radios (2 way radio). 

While in all three categories of communication their is a sender and receiver, the main differences between the groups are the distance the sender can be from the receiver, and the time it takes the receiver to get a message back to the sender. 

These broad groups leave the simulator open enough to allow for changes in technology but clear enough on inital construction to create the obvious, already established norms of communication being things like: phones, email, and post mail.
%Figures
%Examples
%x3



%Groups or socializing at a distance
In the same way that units can organise to meet up, they can also use technology to socialise as a group. The same drive that causes a unit to seek contact with another unit to meetup can lead to socialising over a phone or similar. The diffence is in many cases, interests shared will not be enjoyed or engaged in alongside the socialising. If two units meet up and go bowling, when the same two units communicate over the phone they will obviously not be bowling at the same time. Therefore by communicating via distance means units may not be fulfilling interests or needs while socializing.  

The aspect of socializing, that is the meeting of the belonging need, is met when the unit receives communication from another unit as a part of a group. 


%Units forming new groups from larger groups
Units meet, if interests of some members align so that they would have more common interests, this will form an additional group or, for those units that already have max out number of groups, it will replace the original group.
%Example
%Figure


%Friendships denied because of other friendship groups present
Friends don't happen cause one unit will not value to new friendship as its current friendships are maxed and are better.
%Example Figure




%Family Groups
%Units given group ID but with family id variable ie F[gametime][location]
%When family groups split off the form sub groups of that ID, F[gametime][location][.01]
Family groups are unit groups where all the units are related. More specifically they are units that are part of group, some of whom they live together and associate with. So initally a family group is formed by two units when either: they intimately live together; or, one or both has a child. The family group will grow from this point. (Note: by having a child, this includes adoption) The family group ID of both parent units is shared to the child unit at birth, similarily when two units intimately live together they will also share eachothers family gorup ID for the period they do so.

In the first instance, kinda like an Adam & Eve scenario. Adam and Eve form the family group, have a kid called Gen, now this kid is in the family group. When this kid grows up and forms its own intimate relationship with Bob bob enters the family group. Gen and Bob have a kid, Judy, who gains both family groups, which on Gen's side will include Gen, Adam, and Eve.
%Fix example
%Figure/Chart

In an extended example, if Adam and Eve have 3 kids, the family group will extend to be Adam, Eve, Gen, Rick, and Michael. When Gen enters the personal relaitonship with Bob, Bobs family group will encompass all this, likewise when Gen and Bob have Judy, Judy's family group includes Gen, Bob, Michael, Rick, Adam, and Eve.   
%Fix example
%Figure/Chart

%Define Assoicate
Some family groups will be larger than other due to the degree of association between family members. Units of the same family are deemed to be assoicating when they are part of a group in the immediate vicinity of one another or attempt to do so. Factors that drive family units to associate are that they live together, they seek to meet their belonging need via family members, the have common interests (ie meet various needs via interests together), or they meet up similar to how friendship groups meet up. 
%Example of living
%Example of interests
%Example of friendship in families

%Support from family
In this way units of a family group can assist other family units to meet various needs, therefore a family unit is seen to be offering support and the degree of family support a unit receives is measured by home many family units form the family group. 

If family members do not associate, the links between family members are severed and the group either shrinks or splits.

%Example of group shrinking - ie intimate relation ends.
%Figure

%Example of group splitting - ie two halfs of family no longer see eachother.
%Figure

This enables enough flexibility for families to become or be created into anything, from linear mum, dad plus 2 kids, to complex meshes of extended families including inlaws. Therfore family groups grow and shrink as members are born, adopted, married/partnered, seperate, lose contact and die. Some families will become extensive over time, becoming large groups who associate frequently while other will be smaller and many will fluctuate.

%Intimate Friends
%2 units in same family will not sex unless previously had intimate friends tag. Once units are in same fmaily group can't get intimate friends tag.

%at some point in these next three paragraphs need to go into variables of mate selection

Intimate friendships differ from other friendship groups in that they lead to new workforce units being created, and are often more enduring. Intimate relationships can be fleeting one night stands, short term relationhips where units live apart, or long term relationships that lead to living together and starting a family. Intimate relationships are not restricted to couples with instances where groups of 3 or more being a single intimate relationship possible.   

Two (or more) units that are in either a static or fluid friendship group have the potential to beomce intimate friends as long as they share a mutal gender sex attraction (be it hereosexual, homosexual, etc). The likelihood of units forming intimate friendships increases as they share interests and display gender specific sex characteristics. 

Generally the process works like this: Friend -> Gender Sex Attraction -> Sex Characteristics -> Shared Interests -> Intimate Friends

Interests and sex characteristics have two functions. First they decide what units that meet stay together and for how long. Second they filter possible matches to only those units within a certain range or eachother.

%Formulas for determining relationship match range (broadens as age increases
The following shows the method in which units determine if another unit is appropriate for intimate friendship.

%Formula where A less sexy than B
sA        <       sB
s+(i/2)A =>       sB

%Formula where B less sexy than A
sA        >       sB
sA       =< s+(i/2)B

%Formulas where A and B equal sexy (at least 1 shared interest)
sA       =        sB

A = unit A
B = unit B
s = sex characteristics,
i = matched interests


%1
if sex A < sex B 
+ number of shared interests / 2 
 
%2
 then
  if sex A + shared interests / 2 >= B - (age bracket) <- B lowers its standard based on its age.
  match

%3
else if sex A = sex B && shared interests => 1
  match

%4
else if sex A > sex B
 then
  if sex B + shared interests / 2 => A - (age bracket)
  match


Age Bracket Variables
New Born		-
Toddler			-
Child			-
Adolescent		2
Young Adult		3
Middle Aged		4
Late Middle Aged	5	
Elderly			6
Elderly Infirmed	7
	
The first section shows that only if sex characteristic of one unit are lower than the other will we calculate their shared interests

Section two shows that one unit's sex characteristics are added to the number of shared characteristics divided by two (X+(Y/2)) then compared to the other units sex characterists level minus its age bracket. If unit A's value is equal to or greater than B's the units will be a match. 

Section three shows that if two units match in the number of sexual characteristics and share at least one interest there will be a match.

Section four shows that the same as section two but with the roles reversed.

%Relationship types and potentials (ie one night stand, short term, etc).
Once an intimate friends match is made these units will remain as eachothers intimate friend while the original conditions mentioned above are in place. 
If units interests change, or sex characteristics change at the unit or city level, this may cause the conditions underlying the match to change and therefore end the relationship.

After ... period of time, an intimate relationship can be switched by a better option and match coming along. In this case the unit who is matched will switch intimate friendship to the new unit who offers better conditions.

These factors vary the number of relationships a unit forms, allows for units to seek ideal mates, and maintains some form of stability as units settle into mate relationships. This naturally creates relationships of various durations. 

However as relationships age the above conditions become superficial and the relationships will be more enduring. In one of the above examples (#1,#2,#3,#4) where a match is calculated, additional factors are added to the relationship selection variable for existing relationships. There are: 

1) Relationship duration
2) Living together
3) Child or children
4) Age bracket

Relationship duration refers to the length of time that has elapsed since the relationship match occured. A base numeric value is added to the above calculation depending on segements of time that have elapsed.
%Table depicting elapsed time and relationship duration value
Duraiton	  Value
< 1 day		    0
1 day - < 1 mth     1
1 mth - < 1 yr	    2
1 yr  - < 3 yr	    3
3 yr  - <10 yr	    4
>10yr		    5

Living together
%what determines if units live together.


(relationship duration + relationship strength ((M)) + (cultural ratio)) * co-habitation economic benefit ratio
or
(R+S+C)*E


economic benefit: financial, or goods and services access, difference between living together and not. (effects polygamy also)
if mean resource access increases 	- units percentage of outlay reduces
					- units standard of living increases
An economic benefit if deemed to have occured if the units access to goods and services increases in either quality or quantity.

In this instance, dealing with mated units living together, units will share the costs of some things when the live in the same dwelling. Exactly what these are will depend on each units consumption patterns, but they will generally be reduced to the expenses of renting/owning/running a dwelling. 

Therefore examples include costs of: rent, mortage, energy and other bills, rates.

In addition government kick backs or incentives for shared dwellings add additional amounts to the economic benefit.

As can be seen, it will usually be the case that units benefit from living together, however the model needs to reflect that more than economic benefits determine if units live together. While this is partially covered by the other variables of relationship duration, strength, and the cities culture, in addition we will use a ration of the increase in resource access that would occur vs the units standard access.

%calculating resource access 
In situations where units use money, the base value is easy to calculate as the unit access to resources is an exact numerical value, being the amount of money he has. While this will make up most of the instances in avenaya, some cities, or groups within cities, may have instances where they access resources via other means such as in the trading of goods or services for other goods or services. If units don't use money to access resources they will use other resources. In all three cases a base numerial value can be calucated, either the amount of money, the amount of each good used, or the amount of time spent providing a service.

Section [...] explains how units will allocate funds for purchasing of goods and services to meet needs and interests. This methods determines what units have access to and calculates how much access they have to resources. 

%-->find this section mentioned above ^ and create formula and copy into here.

%co-habitation economic benefit ratio
While we have determined a numerical value for resource access, this base value of resource access does not weigh evenly on whether a unit co-habitates with another unit. We must factor in the extent of change over current level of purchasing as the greater the amount the unit can already purchase, the less significant small changes in living situations matter.

prospective resource access / current resource access = co-habitation economic benefit ratio

clearly, the higher this ratio the more insentive there is for the two units to live together based on economic means. The co-habitation economic benefit ratio is multiplied by other factors so values less than 1 will lower the incentive for units to live together.



%how do units choose where to live? 



-relationship duration
relationship duration + relationship strength

-children
relationship duration + relationship strength + children

-cultural conditions (ie marriage vs cohabitation)
cultural ratio between culture strength and other living conditions.

-economic conditions (ie save living costs etc)

if cultural conditions permit, ratio of economic benefit vs relationship duration.

 
Child or Children
If units have a child or children together this also increases the relationship's strength, and increases as units have more children to a point (source 10). This value increases a large amount with 1 child and drops off rapidly as more children are added to the relationship. 
%Table depicting number of children and relationship value
Children	Value 
 0		0	
 1		2
 2		3
 3+		4
		



Age Bracket
- age bracket <- needs to be offset, people will not just keep going for younger people all the time.
Units in a higher age bracket will have more open standards on mate selection allowing them to secure a mate prior to reproductive benefits closing. However this also makes these unit's switching mates more frequent based entirely on their age. Therefore the units age variable is not included in calculating relationship change. This means as units age they will be more likely to hold down a catch, and only take up new relationships that have overwhelmingly better value. 


%final formula for relationship stay/change - that includes above variables (duration, living, child, age).

M = S + ( I / 2 ) - D - L - C - A

match =  sex characteristics + (shared interests / 2) - duration value - living together value - child value - age value





Intimate friend will be filled after match.
Remains in place until better match, unless: Breakup Ratio Factors: Length, Children, Change in Sex Characteristics

why do units one night stand?
-sex as interest
-belonging need low
-form intimate friend but conditions lost
-

%Table depicting interest and sex charactersitics and relationship length.





Sex characteristics are both predetermined and culturally determined. Things like .. and ... are predetermined, biological, sex characteristics. Things like ... or ... and culturally determined. Predetermined characteristics will be the same in every city, whereas culturally determined characteristics will vary between cities. 
%List of predetermined sex characteristics
%List of cultural sex characteristics





can we have intimate friendship groups of 3 or more? -> factor in homesexuality variable.
how is this recorded (tag)
how do units become intimate friends.
how do units sex etc
how do units end up living together
how do units split up
arranged marriages?


!!!The culture determines the sex characteristics.

social pressure from family and friends.

sex/personality traits determine relationship type, length, etc.

marriage or intimate friends as mating relationships.

similarity overwhelmingly the rule in human mating - hieght, weight, personality, intelligence. Buss 1985 (source 1)

Similarity, equity, and proximity (source 2)

females look for certain traits, males look for certain traits (characteristics).(source 2)

greater investment in offspring = greater selectivity in mate selection.

fertility and reproductive value (source 2)

settling for lower standard because of what is available (source 3)

personality characteristics correlate with significance. also age, religion, drug use. (source 4) 


->Polygomy etc and multiple sex partners.
Resource benefits signifcant;y greater in polygamy relationship.

Culture of polygamy.
reproductive success, both conceive and assist in rearing.

Male: Polygyny > Monogamy > polyandry. Female: Polyandry > monogamy > Polygyny for genetic survival (source 5).

Polandry more likely to occur when female has access to large resources, male likewise (source 5).

Polygny threshold where resources of polygnous relationship has more value to female then monogamous (vice verse with males of polyandrous). (source 6).

Polygamy related to productive capacity of local ecology "The logic is that where there is less gatherable food, women rely more on male hunting and fishing, so that monogamy is ecologically enforced at higher latitudes."(Marlowe, 2007)(Source 6).

Female may choose polygny if males gene of exceptional quality (source 6).

Male vs Female sex ratios, if one too much greater than the other monogamy does not work (source 6).

Cowives achieve equivalent reproductive success or inclusive fitness by sharing resources of a wealthy man as they would from a poor man monogamously (source 7).

Hames, 1996; Symons, 1979 - source 7

monogamy greater in culture with sexual division of labour marlowe 2007 (source 7)

stronger pair bond = less polyamgy, weaker pair bond = more polygamy. this assoicated with sexual division of labour or couples working together in tougher conditions (source 7).

polygyny negative correlated with female literacy (source 7) (access to resources?).

polygyn not related to dominent males and wife beating, religion etc (source 7).

--> one gender having unequal acess to resources
--> fe/male scarcity

polygynous relationships enterred into on account of their functional benefits (source 7).

-->if sex A holds resources sig greater this will gain multiple sex B if B need this access to resources, cannot get their own resources, and are not required to produce resources. if B resources contributed by A and B equal then monogamy more likely.

polyandry to avoid dividing family estates (source 8).

where land can only support small density population, institutionalised polyandry way to keep growth in check (source 8).

sex ratios (source 8).

if men are required to provide economic susistance, polyandry (source8) (must depend on how they contribute as conflicts with polygyny).  

-->quality of mates in vicinity determine selection of intimate friend. changes in quality may cause change in intimate friends choice. current intimate friend choice becomes disadvantageous (ie access to resources, change in interests) then will separate.


% source  1 - marriage 1 - human mate selection, Buss, 1985
% source  2 - marriage 6 - sexual strategies theory: an evolutionary perspective on human mating, Buss & Schmitt, 1993
% source  3 - marriage 7 - different cognitive processes underlie human mate choices and mate preferences, Todd et al, 2007
% source  4 - marriage 8 - personality and mate preferences: five factors in mate selection and marital satisfaction, Botwin Buss & Shackelford, 1997
% source  5 - polygamy 5 - sexual conflict and the polygamy threshold, davies, 1989
% source  6 - polygamy 6 - why behavior matches ecology: adaptive variation as a novel solution, barber, 2015
% source  7 - polygamy 7 - explaining cross-national differences in polygyny intensity, barber, 2008
% source  8 - polygamy 8 - exmploration into human polygamy: an evolutionary examination of the non-classical cases, starkweather, 2010
% source  9 - polygamy 9 - serial monogamy as polygyny or polyandry, mulder, 2009
% source 10 - marriage 9 - the impact of children on divorce risk, Yu & Qiu, 2015

%Work Groups
%Work group and interests meeting things like friendship groups, ie if people at work have some variables/interests then its as if they are with friends.

Static groups things like work and family, remain for longer periods of time.

multiple interests alike, the units will be more friendly and therefore make permient friend groups.
Perma group eating and another unit attends and has same interests, this unit will join group

?Larger groups require more interests the same?
?Do interests, degree or amounts of importance or involvment that draws groups?

some groups will have members come and go, like companies. 
Some groups will disolve.

can multiple groups cross over or are groups exclusive? Probably need to have cross over, or can groups be focused on current activity.


%Groups being companies/Businesses as groups
Units who enter work group/business group contribute to productivity. 



%--------------> WORKFORCE UNIT INTERACTION <----------------------------------
\StartSubSection{Workforce Unit Interaction}

