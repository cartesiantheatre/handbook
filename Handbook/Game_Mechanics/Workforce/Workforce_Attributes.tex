% This is part of the Avaneya Project Crew Handbook.
% Copyright (C) 2010-2016 Cartesian Theatre <info@cartesiantheatre.com>.
% See the file Copying for details on copying conditions.

% Game Mechanics chapter...

\StartSection{Workforce_Attributes}

%--------------> NEEDS, INTERESTS, TYPES AND VARIABLES <---------------------
\StartSubSection{Needs, Interests, Types and Variables} %Variables vs Needs vs Interests vs Types (all explained)
The primary means of describing unit actions and abilities is through the terms, needs, interests, types, and variables. Some of these related to eachother more than others, but they can all come across as overlapping or conflicting. While all these are described in more detail below, here is a brief description on each for ease of reference as well as to show the difference between them.

Needs are the fundamentals of a unit's existence being, in order of hierarchy: oxygen, water, food, security, and belonging.

Interests are preferential actions units undertake.

Types are notations specific to a unit's skills; both employable and not. 

Variables are all labels that specify the physical conditions, skills and traits of the unit, as well as unique identifiers. Variables encompass interests, types, and needs, as well as other things such as gender, race, skin tone, location, current action, and other values used to determine and track a unit.

Needs are what drives a unit to act, where applicable a unit's interests determine how it meet its needs. Sometimes units need to meet needs they have no interests for or are unable to meet with at that time, other time units will act on interests when no needs are required to be met.

Types are the units skills, what they can do, what jobs they can get, and how good they are at these. Types determine how the unit acts on its interests to meet its needs. This is usually via employment for work to earn money to pay for things.

Variables are notations of each unit.



%--------------> WORKFORCE UNITS NEEDS <-------------------------------------
\StartSubSection{Workforce Units Needs}
As mentioned briefly above, a cities workforce is made up of many individual units. Each of these units can be a part of one or more workforce types. These workforce types require certain conditions or needs met in order to be created, maintained, and have degrees of functionality. 

The needs a unit has determine where it will go, what it will do, and how it will do it. If we take section [Maslow's] accuracy into account we can view humans in a somewhat tabula rasa state, with no ill will except to maintain there own needs. Further to this, it can be seen that the nature in which humans interact and the means in which they satisfy their needs is frequently determined by something outside of there biological state. 

The human needs consist of oxygen, water, food, security, and belonging.



Oxygen: breathable air, a human can survive for less than 3 minutes without suffering potentially serious consequences. 
%Expand and example.

Water: clean drinking water. The human body can survive upto a week without water (provided good conditions). 
%Expand and example.

Food: sustainence and nutrtion. It is essential to not only have something to eat but that it contains the various essential nutrients to support human life. A human can last several weeks without food provided they have water.
%Expand and example.

Security: shelter/housing, feeling safe in surroundings, and feeling secure about the future.
%Expand and example.

Belonging: feeling of making a contribution, having people to talk to, being a part of something more than itself, sex/workforce unit creation.
%Belonging- family; friends; partner; own family; nation/state. (hierarchy or random?)
%Expand and example.



In light of the above paragraphs, it can be seen that it is relatively simple to satisfy a humans security needs. However, continually bombard it with messages of insecurity and threats despite evidence to the contrary and it will seek ever greater means to satisfy this need, perhaps waging unnecessary wars against otherwise peaceful peoples. This is just an example of how a player can create or solve disonance in human needs and scenarios the game can implement to challenge players.


%--------------> CALCULATING UNITS NEEDS <---------------------------------------------
\StartSubSection{Calculating Units Needs}

Units seek to fulfil their needs continually. Some needs are satisfied longer than others and this satisfaction is affected by the method in which the needs are satisfied.

The need for oxygen is an example of a need that is continually being sought, however unless the unit is in a critical life or death situtaion this need will be replenished continually from the available air supply.

A workforce unit's needs in each category (oxygen, water, etc) are based on its age bracket, with needs stepping up at each level, then gradually falling off as units age. Table ... refers to this.

%Table showing needs of units at each age bracket. (draft numbers)	
			Need	
Age			Oxygen	Water	Food	Security	Belonging
New Bown		1	0	0	0		0
Toddler			1	1	1	0		0
Child			1	2	2	0		0
Adolescent		2	3	3	0		0
Young Adult		2	4	4	0		0
Middle Aged		2	4	4	0		0
Late Middle Aged	1	4	4	0		0
Elderly			1	3	3	0		0
Elderly Infirmed	1	3	3	0		0


Age also changes the type or quality of needs a unit will seek. For instance a new born or toddler unit may have 1 belonging need, being a family, but as the unit ages eventually into adolescents, it will require more complex needs of family and friends. So while most needs are of a quantitative value (such as food being energy in and energy out), some will have added qualitative factors. 


Added to this is additional needs that units have based on their activity level and interests undertaken. This breaks down to a units current employment and interest types adding various demands on oxygen, water and food consumption.
%This will be referred to spreadsheet containing all possible types of employment.

An example of added amount is depicted in tables ... and ... below.

%Table showing example workforce types and impact on unit needs.
		Need
Workforce Type	Oxygen	Water	Food	Security	Belonging		
Unemployed	0	0	0	0		0
Factory Worker	1	1	1	0		0
Clerical Staff	0	1	1	0		0
Carpenter

%Table showing example interests and impact on unit needs.
	Need
Interest	Oxygen	Water	Food	Security	Belonging
Jogging	2	2	2	0	0
Powerlifting	1	2	3	0	0
Eating*	0	0	1	0	0
Video Games	0	0	0	0	0	


Some interests, although not physically demanding, may request more intake from the unit of a specific need in order to help the unit participate in this activity.

Generally the security and belonging needs remain unchanged as a result of workforce and interests types.


%Calculating immediate gratification



%Calculating long term rise and fall of needs
% - ie input over the course of a week.

As mentioned above a unit's needs are determined by the factors of age, workforce type, and activity (such as interests). These numbers represent the expenditure of each need on a daily basis. To maintain these numbers unit will therefor need to replenish needs with various consumerables on a daily basis.

Calculating needs and average needs.



%Determining when a units seeks to meet its own needs or whether another unit will seek to meet this need for it.
Unit able to meet needs if actions in age range make it possible or if it has access to money required. All actions are listed in the Workforce Change - Incluencing workforce section under table ... This table shows what actions can be undertaken and at what age. From this exhaustive list it can be seen that units from all ages can undertake an action of some sort in an attempt to meet its needs. 

A unit will be able to meet its own needs if the actions open to it relevant to its age bracket make it possible for the unit to meet its needs. There are instances where needs are required to be met with money by default, and a unit will rely on other units to provide for the need via money. A unit may have the action ability to meet a hunger need by stealing food but this is not the default action if the unit has a family member who can buy it food.

If the units age is greater than adolescent it will have all the actions required to meet its needs provided it has adequate money. In this case the utility ratio calcuation takes place and the unit undertakes the appropriate actions to meet its needs.

Otherwise whether or not the unit is able to meet its needs depends on the actions that the unit has available to it given its age bracket. Generally if the unit can meet its needs using its actions alone it will do so. If the units is younger the a young adult it will use a family member who will either directly provide the need or purchase items to meet the unit's needs. Failing a family member to provide this, the unit will enter starvation mdoe (see below).

Micro Economics - Budget Constraints and Preference section details more of the financial side of units determine purhcases of goods.

%Flow chart showing this process.


%Unit Actions in Starvation Mode
Starvation mode is when a unit cannot meet one of its needs and undertakes drastic action to see that this need is met. The severity of the action that will be undertaken depends on how fundamental the need is (a unit that needs air will murder to get it) and how long the unit has been without the need (a hungry unit may eat another human). Therefore unit will not kill to speak to someone to meet a belonging need, for instance. 



%--------------> WORKFORCE INTERESTS <-------------------------------------------------
\StartSubSection{Workforce Interests}%Interests interacting with needs.
Each human workforce unit has a variety of interests which dictates to the simulation its consumption pattern, or how the unit will react in certain situations. This will determine the direction it will take in its spare time as well as the way in which it satisfies its needs. 

The majority of workforce unit changes occur at the unit level with each individual unit deciding which direction it will take in furthering (or lessening) its development. The primary way this is undertaking is the unit seeking to meet its units which are its primary goal. Some of its interests are redimentary and based around being more complex way of meeting its needs, such as eating at fine resturants with friends, other interests will be education or career specific. 

Interests are based around the units needs. There are interests for food, shelter, belonging, and less so for water and oxygen.

Units will have a set number of open slots that can acquire interests based on: (1) Their environmental exposure (friends, family, community, etc); (2) Previous variables (education, other interests, age).

%Example: having certain worforce types will open up influences/interests. In this light not all external influences will influence the interests of all workforce units. 

As the unit goes through age brackets which change it, these slots will become open and then filled with new or old interests. This may occur either immediately and gradually depending on the strength/frequency of the signal that the unit is receiving. In some cases this will grow over time as the unit engages in the influence. The stronger the strength of the interest the harder it is to be replaced.

Once gained interests occupying a slot in the unit will be given a strength number, level, or similar indicator to determine how involved, addicted, interested, needy, etc, the unit is of that interest. 

Various influences will determine this strength. Length of exposure, exposure by family while growing up, exposure by peers, acceptance in society, are examples of stronger influences that create an interest quicker. In addition the interest itself may have a degree of strength imposing itself onto the unit, such as an addictive drug. Different influences affect their allocated interest at different rates. For example the food need influences of junk food (food high in fats and sugars) increase at a rate quicker than interests of healthy food. Influences created with social groups also increase quicker, such as sharing meals with family, or seeing movies with friends.


Replacing an interest occurs through exposure to other interests. Lower level, seasonal, or trend based interest's will be replaced relatively easily compared to interests which are higher order. This represents interests that are ingrained both individually and socially. 
%Expand.

In the simulation this represents changes in trends, the availability of items, and with the units internal interests, such as education, the units ability to "self examine" and view needs to change. With this in mind there will be a link between many interests or interests will be in categories, allowing units to jump from interest to interest as part of its modeled "internal dialogue".
%Expand and explain.

To aid in interest generation through the ages, units will have slots of the next age open in a hidden manner, the slots are fed information to help determine what interests will fill those slots when they become open. In this way we can model the influence of domestic violence, household drug use, etc, as well as positive influence's such as parents who play instruments, exercise, etc, and even neutral influences, eg food consumption, clothes shopping.

Some interests will only be available or last during certain age groups. A child who liked "Random Childhood Show" and moves up into an age bracket which the show doesn't represent will have the interest slot that this show represents re-opened. 

Figure [Workforce_Interests.gv] displays the method in which units gain and replace interests. From the graph we can see that there are specific external variables which cause the unit to generate interests. Primarily there are various media forms as well as other workforce units performing actions. Media will largely be in the form of advertisements or programs with "content". Units, who perform behaviours will be "demonstrating" behaviours that influence other units. All behaviours, media, and externalities based around interest generation will have an interest category, a broad interest variable such as "Consumption Interest" or "Social Interest". Within each of these categories are content types which is the specific interest that is being displayed or is to be generated for the perceiving unit. If the interests category is full then the unit can still generate an interest if this new interest is able to replace an old one. If the interests category is not full a new interest will be generated with the unit under that category (e.g Consumption Interest > Rap Songs) provided the unit has perceived the externality for the required duration for that interest.
%Example: influences -> interests -> needs -> unit.
%Example: social influence "A" -> social interest "1" -> belonging need -> unit.

Interest Externalities and Perception Times
%Expand.

%->SPREADSHEET -> List of interest by category with times required to be viewed.


As a unit acts, moves, etc, it will have a record of all influences on it. If a unit has an interest slot open each time it comes across an influence associated with that same interest, than the influence is recorded and summed over time until the interest is finally generated or filled by another influence (if it is easier for processing speed these lists can be terminated after a certain amount of time of no increase in influence).

Each unit has an amount of consumption interests determined by the age variable of the unit, this is to limit a toddler having a complete set of consumption interests too early. The interests are continually changing throughout the units lifespan and are based on the highest averages of external influences which cause consumption interest generation. A units strongest consumption interest might be accredited organic food, the next might be "clown foods" inc, then shot guns, pure water, and finally fast cars. In addition as each in-game second passes the units average yearly ratios of influence are being calculated, with the variables changing places or being replaced completely.   
 

%--------------> WORKFORCE INTEREST CREATION <---------------------------------------
\StartSubSection{Workforce Interest Creation}
Figure [... (picture and flowchart of unit interacting with environmeny at various stages, underneath is an example representation of influences impacting the unit and how the simulation is tracking them)]
%Expand and explain.

%Example 1
A unit will have a variety of influences impacting it. As it walks down the street it is impacted by a billboard advertisement, a buscar, and a group of people eating (having a good time).

%Example 2
The unit arrives at work and is influenced by a group of people drinking coffee, a person eating, and a radio advertisement. Notice that different level of influence the lone person eating has in this image compared to that in the previous image of a group of people eating and enjoying themselves.

%Example 3
Next the unit engages in its employed office based work. In this example the unit has no influences impacting it while working. This is not universal but in this case there are no influences.



Units with interests in areas impact other units.
A unit with an interest will act on that interest therefore influencing other units.
%Expand and explain.
%Example.



%--------------> WORKFORCE TYPES <------------------------------------------------
\StartSubsection{Workforce Types}
The dynamic and multifaceted nature of workforce may become more evident when reading about the various workforce types. From the list below it can be seen that workforce comprises what the population is doing in reaction to the cities governance or that of other cities and also what stage of development a unit is at. Some types of workforce are easy to create and maintain in a state desired by players. Others are more specific and complex requiring specific circumstances surrounding a population such as the right levels of affluence and education. In addition workforce types applied to a workforce unit can overlap. A unit could be tertiary educated but unemployed, or a physical labourer with a secondary education. 

Units may move between workforce types, as in moving through education levels or going from unemployed to employed. Some factors which change workforce types are done through time, others done through city management.


% -> SPREADSHEET -> Workforce Types 
-Homeless
-Unemployed
-Physical labour (varying levels/classes, ie production, mining)
	(a physical labour type for each task needed in a city) 
-Mental labour (various)(as above)
-Education (primary, secondary, tertiary, post tertiary [not complete, see education types])
-Employment skill (varying types depending on field, ie years of experience)
-Training (varying types, ie separate training that a business or government may provide)
-Criminal (varying types)
-Militia
-Consumer (varying types depending on influences, ie social eco status, cultural)
-Mechanical (varying)(only applies to machines)

-Protester (various depending on cause)
-Age (varying)(only one, changes)
-Drug user (various)
-Drug addicted (various)
-Rioter (various depending on cause)



%--------------> TRANSITIONING BETWEEN WORKFORCE TYPES <---------------------------
\StartSubSection{Transitioning Between Workforce Types}
Unit gets skill and either upgrades workforce type or adds a workforce type.

Workforce units can have multiple workforce types noting the variety of abilities they have. The methods in which the units gain these varies from default/automatic changes (such as units changing types as they move through different age brackets giving them certain effects, abilities, needs, and wants that is particular to that age bracket and to a particular cities style) through to workforce types specifically chosen by the unit such as further education or changing jobs. 

Figure [Workforce_Types_Changes.gv] dicpicts the change of a workforce unit over time. We can see in this figure that certain aspects of the unit remain at certain stages of its lifespan. In the first period of "time lapse" in the example the unit moves from secondary education into tertiary education while still working in a factory (as a line worker). During this period the unit has aged (age group 2 has changed to 3), gained experience in line working (Line Worker 1 has changed to 2, that is moving from level 1 to 2) and has completed 2 units of tertiary education giving it the extra workforce types Economics 1 and Literature 1. We can also see that during the first time lapse the unit has not lost his secondary education workforce type even though he is no longer undertaking this education type as this represents skills and knowledge the unit has gained. 

More time has now passed and the unit will have aged but not so much so as to have moved into a new age group. Therefor it has retained age group 4. The unit has also studied further at a tertiary level gaining economics 4 and literature 2. Note that these have replaced economics 4 and literature 2 as they represent levels or improved skill in these areas, while they appear to replace the previous economics 1 and literate 1, they are upgrades of these with economics 4 having all that was in economics 1 and more (the same goes for literature 2). This is similar to how the unit gained further experience as a line worker in the previous example. The unit has also started work as an economist. The "Economist 1" represents that the unit is level 1 at being a working economist. Also note that although the unit is currently working as an economist at level 1, he has retained the line worker level 2 workforce type as this is a skill the unit will retain, similar to his secondary education, as well as his education of economics 4 and literature 2. Given more time this unit may gain levels in Economist and will also change age groups. 

From this example we can see that workforce units are gaining experience over time in a variety of areas depending on what they are doing over time and that each type of experience is categorised as a workforce type which can either change, improve, or increase in number over time. 

A workforce type does not note when the unit is currently working only what skills he/she has. The unit variables (see[subsection]), discussed later, contains the information of where the unit is currently working, among other things.

How units get skills - interests, needs, continual work, education, age, handicaps.
%Expand and explain.
%Example.

How units move between jobs?
%Expand and explain.
%Example.

How units choose an education type?
%Expand and explain.
%Example.

%--------------> WORKFORCE VARIABLES <---------------------------------
\StartSubSection{Workforce Variables}

Age:
Workforce units change age types as the unit ages, being the changes in the length of time since it the unit was created. 

As mentioned above, workforce units will change one of there types as the unit ages. The subsection on unit variables notes that all units have a time stamp noting there creation, from this time stamp the game will calculate what age this unit is and therefore what age bracket it falls within. The age bracket is just another workforce type which allows the game to give attributes or apply conditions to the unit. New borns have different needs and will effect the world differently than middle aged units which in turn will be different from the elderly. 

Specifically there will be 9 age groups being: the age groups will be new born, toddler, child, adolescent, young adult, middle aged, late middle aged, elderly, elderly infirmed. A unit may not reach all of these age groups but those it does reach it must pass through sequentially.


Calamity:
Units will come across certain situations which will cause them injury and/or disabilty which could be temporary or permanent. This change in workforce is not specifically chosen but an outcome of unit performing certain actions or encountering certain things. These units will automatically be given this type which will be listed on its variable.  


%Variables which uniquely identify the unit and determine its capabilites.

Unique ID
[Unique alpha-numeric key](unique to each unit)
[Date and time stamp of birth](auto generates secondary age variable)[>>][Age]
-These above two generate a unique unit, allowing for keys to be recycled, but date/time stamp will change.

%[First Name and Last Name](may become unique to a cities cultural influences)
[Race/Ancestry]
[Income](fluctuates)
[Workforce Types](multiple, fluctuates)
%[Health Effects](multiple)
[Social Category](?)

%[Political](?)maybe just have complex interaction of existing interests with politcal ideals.

%SAMPLE UNIT VARIABLES
Here is a sample list of a units variables:
[UNIQUE ID][DATE/TIME O.B][LATITUDE:LONGITUDE]
[CURRENT ACTION][EMPLOYMENT][OXYGEN][WATER][FOOD][SECURITY][BELONGING]
[NAME][FATHERS RACE][MOTHERS RACE][SEXUALITY][STR/BI/GAY][HEALTH][HEALTH][EDUCATION TYPE][EDUCATION TYPE][WORKFORCE TYPE][WORKFORCE TYPE]
[WORKFORCE TYPE][WORKFORCE TYPE][WEALTH][SOCIAL INTEREST][SOCIAL INTEREST][SOCIAL INTEREST][WORK INTEREST][CONSUMPTION INTEREST][CONSUMPTION INTEREST][CONSUMPTION INTEREST][CONSUMPTION INTEREST][CONSUMPTION INTEREST][CONSUMPTION INTEREST]

And here is a sample list of an average labour worker within sub-par city with several Earth influences filling those boxes:
[A245JK89DJ102SK][2013/17/03:2146.46.12][-37.048601,143.739624][MARCHING IN PROTEST][C&R PTY WELDING LINES][O:9][W:4][F:5][S:4][B:8]
[ABUDL JHAKA][MIDDLE EAST][EURASIAN][OVERWEIGHT][UNFIT][PRIMARY][SECONDARY][MANUAL LABOUR][MANAGING][PROTESTER][TAXI DRIVER][31000][COFFEE][CINEMA][RESTURANT][JOURNALISM][FAST FOOD][MOVIES][BANANAS][WATCHES][DESIGNER BOOTS][RIFLES]



In this example the unique ID identifies the unit along with its game time date and time stamp. The games longitutude and latitude place and track where the unit is, current action notes what the unit is involved in and the employment variable shows where he is working.
Next we can see the unit has plenty of air (9/10), is a little thirsty, will soon need food, is not feeling secure which is probably why he is protesting, and is fitting in with the crowd with a beloning level of 8. Optional for the game is the tracing of the units bloodlines, showing mothers and fathers ancestry. Next are its health effects which reflect its consumption choices and how the unit may eventually being spending it money (ie on healthcare or medication) or how it will eventually die. Next we can see this unit has obtained general primary and secondary education types which influences the basic skill set it has and therefore determines the work it can obtain. Next we have the workforce types it is involved in which denotes where is has been working, where it is working, so its skills, along with any other non-social activity, in this case he is actively protesting. We can then see he is currently earning the equivalent of 31,000 a year in wages and prefers to socialise by having coffee and going to movies resturants. He also has an education preference for journalism which means if it was available and he could still meet his needs he would be studying journalism instead of what he is currently doing. He has 
a consumption preference for fast food, movies, bananas, watches, designer boots, and rifles; this means he will spend his disposable income on these goods in no particular order. Note that in this above exmaple, education types does not accurately depict a units education, education types will actually be made up of many types forming an education as chosen by the player. Primary, secondary, etc are just the games default educational system in place without any player management.

Of the above variables, some are automatically asigned to the unit, such as ID, name, date of birth, and location, however others are generated and changed throughout the game. 

%Expand and explain.
%Examples.


%Health (remeber prenatal stuff also)
%Expand and explain.
%Examples.

