% This is part of the Avaneya Project Crew Handbook.
% Copyright (C) 2010, 2011, 2012
%   Kshatra Corp.
% See the file License for copying conditions.

% Senka Rukavina section...
\StartSection{Senka Rukavina}

Senka was born 13 B.R. in Dalmatia\index{Dalmatia}, Mediterranean Union\index{Mediterranean Union}. This is a region located within what was known prior to 31 B.R. as Croatia\index{Croatia}. 

Senka's parents were organic olive oil and lavender farmers who had to cope with the economic uncertainties of life in the Mediterranean Union. With virtually no domestic market for their commodities, they sought export markets in the European Union to secure themselves a very modest subsistence.

As part of life on a farm, Senka was exposed to large industrial and agricultural machinery at a very young age. She rapidly developed an aptitude for trouble shooting and solving practical mechanical problems as they arose in the field. She also had a curiosity for geology which was nurtured with her grandfather's explanations of the many different naturally occurring clays and other materials the machines would inadvertently extract while tilling the earth. 

Senka was able to take advantage of these interests in applying her knowledge while still a teenager to locate underground artesian aquifers. The Dalmatia region is fairly arid with water being a scare resource. The aquifers were vital in providing the family's crops with much needed irrigation. Without sufficient water, it would have been next to impossible for the family to provide for themselves.

Senka's grandfather had been a geology professor long before Senka was born and was forced to leave his position when the economy had taken a turn. Since then, he had worked the farm for all the years following. He had watched Senka grow and noted, as others in her family had, her intelligence and great potential. He sat her down one day as she was one year away from completing her secondary education. He was honest with her and described what he saw as a bleak future, should she choose to remain toiling in the soil of her family's farm. He encouraged her to consider a higher education and to forget the farm and, in the frank words of one old enough to remember the nation's former title, \quote{Croatia's ruined economy}.

On her grandfather's advice, Senka applied for and was accepted into the University of Oxford's\index{University of Oxford} Merton College\index{Merton College}. This was made possible almost solely on the basis of scholarships. She completed her masters in planetary geology\index{Planetary geology} where she studied under a number of notable areologists. Following that, she then went on to complete a DPhil in mechanical engineering.

Upon wrapping up her studies at Oxford, Senka found herself at a crossroad in life as she tried to determine what to do next. At the suggestion of a close friend, she submitted her paperwork, though somewhat in jest, in answer to UNSA's request for applications for the Avaneya Initiative. To her surprise, she was summoned to participate in more than twenty levels of interviews before finally being sent to Antarctica and Huelva in the European Union. There, she underwent final selection and training. Although noted for her strong and independent character, there were occasions when the selection committee made note of her preference for working independently as opposed to as a member of a group.

Having achieved final selection, she was informed that she would be joining the Avaneya crew as their Chief Flight Engineer\index{Chief Flight Engineer}, a responsibility which includes piloting the Avaneya spacecraft to Mars. Once on the surface, she would take advantage of her background as a planetary geologist in leading the Survey Team in their search for building materials, water ice, artesian aquifers, viable locations for erecting structures, and other mission critical responsibilities.

