% This is part of the Avaneya Project Crew Handbook.
% Copyright (C) 2010, 2011
%   Kshatra Corp.
% See the file License for copying conditions.

% The Game chapter...
\StartChapter{The Game}

\startTimelineCorrespondenceDocument
    \startlines
    \input Source/Timeline/Arda_Independence_Letter.tex
    \stoplines
\stopTimelineCorrespondenceDocument
\page

\StartSection{Why}
Too often, people have come out of experiencing a great dystopian science fiction novel or film only to say to themselves, Thank goodness we don't live in {\it that} world. This needs to change, and if anything could have deserved a finger pointed scornfully at it for having prompted this undertaking, that, more so than anything else, probably would have been to blame.

The problem is that dystopian science fiction is among the most honest kind of story telling when examined in the context of history. But unless people can see the pedagogical relevance, analogies will remain vauge and metaphors too cryptic to be useful. Perhaps it could be said that {\it responsible} dystopian science fiction tactfully withholds the link far less.
\placefigure
    [right,2*hang]
    {Captured by Viking orbiter I on 22 February 1980. Valles Marineris is the scar on Mars over 3000 km long and up to 8 km deep.}
    {\externalfigure[Source/The_Game/Images/Valles_Marineris.png][][width=.5\textwidth]}
Still, responsible dystopian science fiction, or whatever one calls it, ought to strive to be well received. One of the ways it does this is in being only partially fictional, because it is the plausible scientific dimension that usually attracts people to it in the first place.

But science, by its nature, is a thinking and knowledge oriented enterprise. It is the study of reality, and yet fiction, by definition, is outside of reality. So science fiction must be a mediator between the two, often acting as an {\it avant-garde} for both, with the dystopian variety debatable which end it is more proximate.

This can set the bar high for what its patrons expect and requires designers to pay a great deal of attention to detail. A certain degree of creative license is expected, but people still prefer it to be as consistent with what we already know to be true, or what is at least reasonably plausible.

Another important reason why this project is necessary is that there is very little, if any, {\it free}, commercial, games for the GNU operating system. When people use GNU, they are treated as second class citizens in many respects - not least of which is the availability of good games. Not only are there very few higher production titles that are available for GNU, those that exist are usually proprietary, and even then, generally bad ports using deprecated APIs, poorly packaged,\footnote{Assuming the platform's native package manager was even used at all.} and integrating horribly into the user's desktop - ignoring  useful community driven conventions like {\it freedesktop.org}. 

In some sense, Avaneya is part of an overall effort to complete the GNU operating system. When the GNU project set out to replace the body of non-free software required to use a computer, Richard Stallman once remarked that it would also require games since a complete operating system needed games too.\footnote{\fullahref{http://www.gnu.org/gnu/linux-and-gnu.html}} One of the aims of this project then is to lend a hand in that respect by helping to reward the many good people who commit themselves to free software for ethical reasons.

A final reason for this project is found in the existing proprietary games themselves. They tend to appeal more to the mainstream proprietary user. They do not share the {\it software libre} community's values of freedom, and not simply with respect to how it is licensed. In the proprietary user's mind, conscious computing is irrelevant because machines are simply tools capable of existence and evaluation external of any social context. Such a superficial perception of machines is well illustrated in the open source school of thought in contrast to its cousin, {\it software libre}.

In other words, we assert that the choice between using free and proprietary software is not akin to, say, the choice of chocolate versus vanilla, since the decision says something more meaningful about a user's constitution. That is, users of free software, for the purpose of freedom, are different from proprietary users, and not merely in their software preferences. Therefore, a game for that audience should go beyond merely a distinction of licensing and reflect the audience in spirit as well.

\StartSection{Classification}
People tend to struggle to classify Avaneya. It is what it is, but the closest traditional categories that form a {\it subset} of it would be the classic city builder and management simulations and the real time strategy.

\StartSection{Likely Users}
The game so far has attracted a fairly large base of followers. From what can be observed at this time, the game appears to appeal to those with an interest in:

\startitemize[3]
\item
{\it games that are not only fun, but useful}
\item
{\it challenging the consensus of reality}
\item
{\it software libre}
\item
{\it a social conscience}
\item
{\it science fiction}
\item
{\it an interconnectedness of everything}
\stopitemize

The game may take place in the future, but it deals with current problems. The best way to get an idea of the intended audience is to quickly scan both \in{chapter}[Core Leitmotifs] and \in{section}[Resources For Everyone]. You will be in a better position to try and gauge the type of audience that this game resonates with after doing that. 

\StartSection{Unlikely Users}
Avaneya is a {\it sui generis}.\footnote{{\it "Literally meaning of its own kind / genus or unique in its characteristics. The expression is often used in analytic philosophy to indicate an idea, an entity, or a reality which cannot be included in a wider concept,"} (Wikipedia).} It is not like other games, and thus it is not for all people. It does not try to be, nor will it ever be.

Those with a brief attention span, believe that things originate in cans with little appreciation for process, and accidentalists will probably not enjoy this game. There are already many such games that appeal to that type of audience, so that need not be our aim here.

This game will challenge you to think, and possibly even offend you. It challenges the consensus of reality, and therefore, potentially, your world view. Consequently, some have accused Avaneya of being a vehicle for culture jamming and political commentary. This project is shamelessly guilty as charged---like the newspapers, film, television, games, and other mainstream media that saturate us. 

The only difference is that, unlike those mediums, the very presence of a normative bias in Avaneya is not subject to dispute and is self evident. Other mediums sometimes pretend to not have one. In any case, you would be very hard pressed to try to find any classical work of science fiction, or really any kind of fiction for that matter, that did not. Moreover, that in itself is not necessarily a bad thing.

\StartSection{Availability}
We do not believe that deliberately {\it restricting} users to a specific platform or hardware is ethical for any other reason than practical technical limitations of the alternatives, such as the absence of a programmable shader interface under {\it any} license. We do, however, strongly {\it promote} a specific platform, that being GNU and any available hardware it can run on capable of providing for the technical requirements of Avaneya.

Since we consider it unethical to encourage people to use non-free software, we will not ourselves ever be the primary maintainers undertaking such an endeavour. However, it would also be unethical to deliberately design it in such a way so as to cripple the efforts of others from doing so. They have the freedom to disagree. Thus, since Avaneya relies on portable libraries, it should not be unreasonable for someone to do this if they do not share our values.

Other GNU distributions that provide alternatives to Linux as the kernel are also considered fair game. These include GNU/kFreeBSD, GNU/Hurd, GNU/NetBSD, and possibly others, depending on their level of maturity, technical feasibility, and demand from the community. As Linux continues to deviate\footnote{\fullahref{http://fsfla.org/svnwiki/anuncio/2010-03-Linux-2.6.33-libre}} from the free (and even open source) philosophy, it becomes increasingly important to encourage and support alternative choices as they become viable.

Some have suggested that maintaining for free platforms can be rather self limiting in terms of user outreach. There was a time when this view had validity. Now, with tens of millions of users worldwide running some flavour of GNU, such as Ubuntu, the perception of a non-existent or marginal demographic  is rather antiquated. Consider that gaming under non-free operating systems such as OS X is still active and lively today, and yet Ubuntu alone (not including other GNU distributions) has arguably surpassed it in terms of popularity with a user base in the tens of millions.\footnote{\fullahref{http://www.freesoftwaremagazine.com/columns/ubuntu_surpasses_mac_osx}} GNU long ago stopped being merely an operating system for hobbyists. It is so ubiquitous, you will find it anywhere from the Parliaments of the world to up in orbit.

\StopChapter

