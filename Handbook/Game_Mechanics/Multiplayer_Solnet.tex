% This is part of the Avaneya Project Crew Handbook.
% Copyright (C) 2010-2017 Cartesian Theatre™ <info@cartesiantheatre.com>.
% See the file Copying for details on copying conditions.

% Multiplayer: Solnet section...
\StartSection{Multiplayer: Solnet}

Another multiplayer mode is {\it Solnet}\index{Solnet}. Solnet is the online service that allows players to join a much larger group for cooperative play, potentially in the thousands, over the internet. It is Avaneya's feature attraction. Players having purchased a reasonable monthly subscription gain access to the official Solnet server which hosts, at least initially, a sector from {\it Arcadia Planitia} in the fictional timeline during the Arcadian Diaspora.

Playing over Solnet offers another distinct advantage over LAN multiplayer -- {\it persistence}. Consider the problem of what to do with the game world after the player has disconnected from a multiplayer session. This is not a problem for a game hosted on the LAN because players already have a reasonable expectation that the game is over after their friends pack up. The logical cluster of game world space, objects, and states is destroyed and no machine need labour to maintain it in their absence.

But in the case of Solnet, the world {\it must} remain. It is persistent because the player may be sharing the world with potentially thousands of others who would not appreciate it if it was abruptly destroyed simply because some of its players in another timezone went to bed. The world hosted on Solnet carries on with a life of its own -- even in the player's absence.

The Solnet subscriptions are absolutely critical for ensuring the official server runs on reliable high quality hardware that can handle large concurrent loads made accessible through sometimes costly, albeit necessary, hosting. This is also vital for promoting additional development of the game's engine, artwork, music, and other graphical media that continue to make the game better and sustainable.

The following is a summary of all of the different modes of play we have just covered.

\startitemize[4]
    \item Single Player
        \startitemize[4]
        \item Tutorial
        \item Campaign
        \item Custom
        \stopitemize

    \item Multiplayer
        \startitemize[4]
        \item Local Area Network (LAN)
        \item Solnet (Internet)
        \stopitemize
\stopitemize

This concludes a general overview of who can play with who and over what medium. You can see \in{section}[Networking] if you are curious about the technology that makes multiplayer possible. Now we will turn our discussion to what it is that players actually do, either by themselves or with others.

