% This is part of the Avaneya Project Crew Handbook.
% Copyright (C) 2010, 2011, 2012
%   Kshatra Corp.
% See the file License for copying conditions.

% Scenarios section...
\StartSection{Scenarios}

Scenarios are at the heart of Avaneya. They can be specifically selected through the single player's custom mode, enjoyed with your friends over a LAN, or experienced through the single player's campaign mode (which is itself a series of scenarios). 

These scenarios seek not only to entertain, but also to inform by drawing upon issues in the environmental, social, political, economic, agricultural, and other realms. But each scenario never really draws strictly from any one of the aforementioned. This is impossible since everything is connected. As an example, the issue of water privatization is relevant to a discussion in at least all of the aforementioned. 

In all scenarios, the user is responsible for directly controlling, working with, or maintaining Arcadians and their infrastructure. At other times, however, they may be managing only a handful of units perform a task such as assisting a remote survey team locate ice water as opposed to erecting an entire city.

Avaneya is different from other strategy games in that you are not the only source of influence over the units you think you are in control of. You may be president of the republic, but there are other influences that will attempt to reshape the world you are emersed in, such as transplanetary corporations, the United Nations, Terran intelligence agencies, and others, with the lines frequently blurring between them. At times you will feel powerless.

Scenarios can incorporate multiple goals. These goals, depending on how the scenarios are scripted, play a role in determining whether the user has successfully completed the scenario. Let us take a look now at some of the general types of scenarios. Bear in mind that this is only a subset of all the possibilities, since new ones can be written to drive the game engine in novel ways.

You will note that the \about[definition:Genuine Progress Indicator] (GPI) is a recurring theme. This concept was described on \at{page}[definition:Genuine Progress Indicator].

\startitemize[4]
\setupwhitespace[big]
\head {\em No Goal} 

A scenario may be goalless. In this case, the user erects their city and is responsible for its management without thought of a terminal condition for success, such as a time limit. But as with any other scenarios, the user can still strive to maximize their GPI.

\head {\em Achieve Minimal GPI}

The user must achieve a minimal GPI. This may or may not involve a time constraint. Although the GPI is a recurring theme in Avaneya, scenarios that require this goal bring greater attention to it than others.

%\head {\em Free Trade}

%The North American Union would like Arcadia to become a signatory to the North American Union Free Trade Agreement.

\head {\em Improve Energy Supply}

Your city needs energy and it may not have enough to power everything on the grid. The user must adjust factors in their city that will improve the situation. This may mean building a nuclear plant, a geothermal well, erecting photovoltaic panels, or perhaps looking for ways to increase efficiency, such as examining how much energy is required to produce the different types of things that people eat and exploring alternatives.

\head {\em Improve Public Health}

Many factors effect public health, the most important of which is what the public eats and drinks. But other factors contribute as well, such as the quality of life, education, annual take home income, available free time, the presence of narcotics one's government deals covertly, and many others. Scenarios that depend on this goal require the user to improve the public's health.

\head {\em Improve Transportation Network}

As a city grows, efficiency of movement can become a problem. This goal requires the user to have their city's transportation network increase in efficiency.

\head {\em Natural Disasters}

The user must rebuild their city in response to damage sustained from solar flares, micro meteorite strikes, land slides, marsquakes, or other natural disasters. Yes, Mars experiences all of the aforementioned. But some \quote{natural} disasters can be caused through human mishandling of the natural environment. For example, if an entire city block is swallowed whole into the ground, chances are the aquifer it was sitting on was depleted. In many situations, they may be able to avert the disaster entirely.

\head {\em Prevent Staged Terror}

Staged terrorism is as common in Arcadia as it was on Earth. NAU-CIA will work tirelessly to stage acts of terror to be blamed on arbitrary groups, such as the Red Unionists contras which are essentially a NAU-CIA backed front. The user must use whatever means necessary to prevent acts of staged terror. If they fail to do this, the NAU may claim sabotage of their mineral extraction sites as justification to increase their military presence with a troop deployment.

\head {\em Protect Another City}

The user must come to the assistance of another Arcadian city in repelling an attack of a foreign aggressor. The city has come under siege at the hands of Terran influences using the United Nations. Constraints may vary, such as preventing the city from exceeding a threshold of damage, protecting its inhabitants, defending a mineral extraction site, and so on.

\head {\em Protect From Corporate Media}

People are less free when they are not informed. They make decisions that are influenced by what they know and the quality of information that they are fed, among other things. If the source of the information has conflicts of interest, those that pay heed will operate with information that is not in their best interest. 

Scenarios that depend on this goal require the user to reduce the influence of corporations in socially engineering public opinion. Their influence may be overt, such as through advertising, or it may be more subvert, such as funding junk science\index{Junk science} at a university. 

\head {\em Protect Human Rights}

Articles XI --- XXVI of the Rubicon Act, as described on \in{item}[item:rubicon_act_fundamental_rights_first], enumerate the fundamental freedoms and human rights Arcadians enjoy with respect to their new republic. The user is responsible for ensuring that they are upheld.

\head {\em Recover From GMO Terminator Gene}

People that have nothing to eat will eventually die. If those working in the city's greenhouses report that their crops have yielded nothing because of the adoption of new genetically modified seed stock, chances are they did what farmers have been doing for millenia. They stored seed and planted it with the expectation that they would grow, only to find that the seeds had been genetically programmed by the distributor to not be re-usable. This way, farmers are held hostage to the distributor by having to buy new seeds annually, even though they already saved some.

Scenarios that depend on this goal require the user to restore their annual agricultural yield to a variable minimum and to ensure that it is not genetically modified.

\head {\em Recover From Water Privatization}

Without water, you cannot survive. A corporation such as Bechtel-Biwater\index{Bechtel-Biwater} may attempt to privatize the artesian aquifers or water ice sites your people depend on for survival and their livelihoods. A judicial system may be of use, or you may have to recover the sites through use of force if they are corrupt.

\head {\em Reduce Crime}

The user must reduce crime by addressing factors, such as education, affordable living, health, and other fundamental factors and environmental conditions that give rise to crime. Sometimes people may not be doing anything particularly harmful to society, such as using marijuana, but they may be convicted and sent to prisons that are privatized and profit off of each incarceration. They do this by having the corporations that own them bribing judges.

\stopitemize

