% This is part of the Avaneya Project Crew Handbook.
% Copyright (C) 2010, 2011
%   Kshatra Corp.
% See the file License for copying conditions.

% Terms & Concepts chapter...
\StartChapter{Terms & Concepts}

The following is a collection of terminology used throughout the handbook and the game world. They range from the everyday colloquial to the technical. Some of it is factual and some of it is fictional (e.g. the {\it \about[Jenya]}). Getting familiar with the terminology before you read on might be useful.

% Albedo...
\StartSection{Albedo}
The proportion of incoming light reflected. It is the ratio of reflected light to total incident light. That is, a material with an alebdo of 1 is a perfect mirror whereas a material with an albedo of 0 absorbs all incident light.

% ATATÜRK
\StartSection{ATATÜRK}
NAU CIA's cryptonym for Arda Baştürk. See \in{section}[Arda Baştürk] for more on Arda.

% Bancor...
\StartSection{Bancor}
A global currency proposed in the early 1940s by the British economist John Maynard Keynes. On 13 April 2010, the {\it Strategy, Policy and Review Department of the International Monetary Fund} published a paper {\it Reserve Accumulation and International Monetary Stability}\footnote{\fullahref{http://www.imf.org/external/np/pp/eng/2010/041310.pdf}} recommending the world adopt the bancor and create an international central bank to administer it. The Avaneya game universe adopts the small Greek beta character \quote{β} as its currency symbol.

% Cornucopianism...
\StartSection{Cornucopianism}
The term {\it cornucopia} arrises from the Greek legend surrounding the magical horn of plenty which was capable of providing the bearer with an ambundance of whatever was desired.

Cornucopianism is a school of thought that argues that increased demand for resources as a function of time can always be met with increases in the means of acquiring them. As an example, advocates for cornucopianism sometimes argue that Terran peek oil is not a problem because either it can never run out, new innovations will reveal new reserves, or future technological advances will provide alternative forms of energy in time. 

The doctrine asserts that systems of infinite growth aspirations will never meet any limiting constraints, either because the environment it operates within is believed to be infinite, or for some other reason. This is often contrasted with the doctrine of \goto{Malthusianism}[Malthusianism].

% Corporate personhood...
\StartSection{Corporate Personhood}
A corporate charter creates a legal fiction with many of the basic rights of an actual human being. These may include the rights to own and sell assets, to sue and be sued, and so on. Critics argue that this ensures the physical human beings responsible for decisions are buffered from ever having to be held personally accountable. Advocates argue it promotes economic prosperity.

% Deuterium...
\StartSection{Deuterium}
All chemical elements have one or more variants that occur less frequently in nature than their most common forms as noted on the periodic table. When elements share the same atomic number, but vary only in the number of neutrons they contain, they are called {\it isotopes\index{isotope}}.

Hydrogen is denoted as \chemical{H} and is known to occur naturally as either of two isotopes. The more common form has one proton, one electron, and no neutrons. This form is called {\it protium\index{protium}} and is sometimes denoted \chemical{^1H} when distinguishing from another type of hydrogen isotope. Usually when someone is talking about hydrogen, they are talking about protium. The number one located in the notation means that it has an atomic mass of one. It is one because electrons have nearly zero mass, neutrons a mass of one, and protons a mass of one. This means that isotopes of the same element still vary in mass.

The only other known isotope is {\it deuterium} and is denoted \chemical{^2H}. It has a neutron that protium does not, hence the atomic mass of two in its notation. When two deuterium atoms are combined with an oxygen atom, the resultant is deuterium oxide \chemical{^2H_2O}, or simply {\it heavy water\index{heavy water}}. The name derives from it appearing exactly the same as normal water, only heavier.

On Earth, deuterium is very rare and accounts for only 0.0156 \% of all the hydrogen that naturally occurs in our planet's oceans. On Mars, however, it is more plentiful by a factor of five. About 166 in every million hydrogen atoms are deuterium on Earth, but on Mars that number is about 833. 

The \goto{RWGS}[Reverse-Water-Gas-Shift Reactor] reactor heavily relied upon by Arcadians to produce water from the Martian atmosphere can also produce deuterium as an inadvertent byproduct. If the water emerging from the reactor is split through electrolysis, hydrogen fuel and oxygen will result. For every 6,000 kg of \chemical{H_2O} the reactor produces, 1 kg of that will contain deuterium. Since heavy water takes slightly longer to split with electrolysis than regular water, if you keep cycling the water that has not yet been split back through electrolysis, eventually the concentration of heavy water increases until it is almost totally pure since the normal water gasses off faster.

By the late 20\high{th} century, deuterium was valued on Earth at about 70 \% that of gold per kilogram. It has wide ranging applications from nuclear fusion and fission reactors, nuclear weapons, medical, biochemical, environmental, and more. As an example, if it is contained as heavy water, it can be used in non-uranium enriched fission reactors that poor Terran countries may benefit from.

Since Mars is only a third as massive as Earth, only a fourth the energy needed to exceed Earth's escape velocity is actually required on Mars to get things off the planet. This makes deuterium \goto{railgun}[Railgun] exports from Mars back to Earth feasible where it is a highly valued commodity. Since all water, oxygen, and many other Arcadian industrial processes relying on the RWGS reactor produce deuterium as an inadvertent byproduct, the settlement theoretically would always have something valuable that is sought after in the Terran market.

% Endothermic reaction...
\StartSection{Endothermic Reaction}
A chemical reaction that requires energy to be put into it in order to react. An example would be cooking an egg. The egg does not change unless you apply heat to it. This is contrasted with an \about[Exothermic Reaction].

% Exothermic reaction...
\StartSection{Exothermic Reaction}
A chemical reaction that releases energy as part of its reaction. An example would be burning wood. As the wood changes, it releases energy in the form of mostly heat.

% ERV...
\StartSection{ERV}
Earth return vehicle. These are vehicles capable of return to Earth directly or for perform a rendezvous docking in orbit above the Martian surface. They can refuel themselves automatically from the Martian atmosphere using the \goto{Sabatier}[Sabatier Reactor] technology.

% FAP...
\StartSection{FAP}
Free Arcadia partisans. A political and ideological movement originating in the Arcadian Settlement in the years leading up to its independence. It was instrumental in the drafting of the Constitution and encouraging an Arcadia free of Terran political and economic control and influence. Prior to year zero, it maintained its presence underground out of UNSA observation.

% Fifth column...
\StartSection{Fifth Column}
A clandestine force that undermines a greater power from within. The term originated during the Spanish Civil War on Earth in reference to a \quote{fifth column} of supporters within the city of Madrid that would support the four columns of military forces attempting to place the city under siege.

% Genuine Progress Indicator...
\StartSection{Genuine Progress Indicator}
The GPI\index{GPI} is a system of true cost accounting which is intended to be a replacement to the GDP\index{GDP}, gross domestic product, sometimes called the gross national product, GNP. The GPI works by taking into account all costs of an activity to society and provides a net balance sheet. The latter, on the other hand, only functions as an income sheet by tabulating the total amount of goods and services produced in a year. 

An example where the difference between the two is well illustrated is in society's relation with the tobacco industry. The GDP accounts for the value of all cigarettes sold, a dollar figure greater than zero. Its functionality is limited more or less to that of an income sheet.

Conversely, the GPI, like the GDP, would also account for the total value of all cigarettes sold. Where it differs is it then goes on to subtract the dollar figures socialized to everyone in the form of health costs, fires, garbage collection, environmental toxicity, and the deaths of approximately 5,000,000\footnote{See {\it Merchants of Doubt} in \in{section}[Resources For Everyone].} people every year. The GDP calculated a gain. The GPI, however, revealed it was actually a deficit.

% In situ...
\StartSection{In situ}
From the Latin, in position. When used in space-related contexts, it denotes a situation where something is not done or is available in the actual field, not prior.

% Jenya...
\StartSection{Jenya}
{\it Article VII} of Arcadia's {\it Rubicon Act\index{Rubicon Act}} superannuated the Terran \goto{bancor}[Bancor]\index{bancor} fiat currency\index{fiat currency}\footnote{A fiat currency is a currency that has only the endorsement of an authority, such as a government, to back it. It is neither composed of nor redeemable in any material perceived of as valuable and held in reserve .} with the {\it jenya\index{jenya}}. It is the first currency to be backed by a predominantly \goto{rhodium}[Rhodium] standard\index{rhodium standard}, with the remainder by other precious metals, such as gold and silver. The jenya became the exclusive legal tender within Arcadia when the Act was passed. This required all Terran interests to acquire Arcadian goods in jenyas only. The currency symbol for the jenya is \quote{\externalfigure[jenya_inline]}.

\placefigure
    [here]
    {Currency symbol of the Arcadian jenya.}
    {\externalfigure[Source/Terms_and_Concepts/Images/Jenya_red.svg][][width=3cm]}

The word {\it jenya} is Sanskrit\index{Sanskrit}. It means {\it of noble origin, genuine, or true wealth}. The idea being that since rhodium is considered precious, indeed, more so than gold on both Earth and Mars, distribution of jenyas across a society implied a redistribution of real monetary wealth.

% Malthusianism...
\StartSection{Malthusianism}
The term {\it Malthusianism} is used to denote the doctrines of the 18th century English political economist, Reverend Thomas Malthus. The school of thought argued that populations tend to increase at a rate that exceeds their environmental carrying capacity and which always results in human suffering. 

The 19th century French mathematician, Pierre-François Verhulst, after having read Malthus' seminal works, {\it An Essay on the Principle of Population}, produced his famous Verhulst equation shown here (\in[formula:Verhulst equation]) solved for population as a function of time.
\crlf

\placeformula[formula:Verhulst equation]
\startformula
\math{P(t) = \frac{K P_0 {\it e}^{rt}}
                  {K + P_0({\it e}^{r t} - 1)}}
\stopformula
\startlegend
\leg P \\ population at time {\it t} \\ \\
\leg t \\ time \\ \\
\leg K \\ maximum carrying capacity \\ \\
\leg P_0 \\ initial population \\ \\
\leg r \\ growth rate \\ \\
\stoplegend
\crlf

Now note how formula \in[formula:Limit of Verhulst equation] asserts that as time goes on, the population will always asymptotically approach {\it K}. That is, it keeps getting closer to {\it K}, but can never actually surpass it. This is known as the {\it law of population growth}.
\crlf

\placeformula[formula:Limit of Verhulst equation]
\startformula
\math{\lim_{t\to\infty} P(t) = K.\,}
\stopformula
\crlf

Formula \in[formula:Verhulst equation] when graphed out looks like figure \in[figure:Verhulst equation]. Note again that as time progresses, the population is limited by the maximum carrying capacity {\it K}. It starts off with near exponential growth, but then rapidly decelerates.

\placefigure
    [here, force]
    [figure:Verhulst equation]
    {The Verhulst equation illustrating the law of population growth.}
    {\externalfigure[Source/Terms_and_Concepts/Images/Law_of_Population_Growth.svg][][width=.5\textwidth]}

The Malthusian doctrine acknowledges the inherent dangers of systems of infinite growth aspirations operating within the constraints of a finite environment. This is often contrasted with \goto{cornucopianism}[Cornucopianism].

% Mars Direct...
\StartSection{Mars Direct}
Mars Direct is a \type{$}50 billion dollar plan proposed by an American aerospace engineer named Robert Zubrin\index{Robert Zubrin} (born April 19, 1952) as an alternative to the prohibitively costly \type{$}450 billion dollar mission to Mars proposed by NASA in consultation with its government.

The then incumbent President of the United States President of the United States\index{President of the United States}, George H. W. Bush\index{George H. W. Bush}, announced the government's proposal in 1989 as the {\it Space Exploration Initiative\index{Space Exploration Initiative}}. It called for the creation of the {\it Space Station Freedom\index{Space Station Freedom}} and a permanent Lunar base\index{Lunar base} as intermediate steps for an ultimate destination to the Red Planet. If implemented, it was to be rolled out over the process of several decades.

Zubrin reasoned that it is totally unnecessary to construct giant space stations in low earth orbit, useless Lunar bases on a barren moon, and massive spacecraft carrying hundreds of people to achieve a manned mission to Mars. That, along with transporting all that is necessary to get there, survive there, and return safely. He argued that the government prefers an intentionally bloated approach because it creates the illusion of progress and productivity through countless jobs, contracts, bureaucratic expansion, and so on. But it comes at the cost of enormous waste, misdirected resources, and through increased complexity, an increased likelihood of disaster.

Zubrin compared their approach to the failed Arctic explorer, Sir John Franklin\index{Sir John Franklin}, who, with government assistance in 1845 took two ships, the {\it Erebus} and {\it Terror}, each displacing more than 300 tonnes in an effort to navigate through the Northwest Passage\index{Northwest Passage}. His ships carried all manner of useless items, including heavy English silverware, but spared many of the critical items necessary for survival. 

The crew met a bitter end as they dragged heavy iron and oak sleds across the Arctic ice, having abandoned their ships that were stranded. With shotguns useless in the Arctic and other heavy and inappropriate equipment, all 127 men ended up perishing to the combined efforts of the elements and scurvy. It never occurred to them to take advantage of {\it in situ} resources, like fur coats, seals, and fish.

Zubrin argued that the Space Exploration Initiative's mission architecture is an absolute inverse of a sound engineering approach. He outlined cogently in his book {\it The Case For Mars} for a very reasonable, well thought out, minimalistic approach of travelling light, living off the land, and manufacturing the necessary rocket propellant for the return trip {\it in situ}. This is akin to the efforts of early Terran settlers, like those who pushed through the American Western Frontier, or virtually ever other civilization's successful effort at settling a distant land. Going to another planet is, according to him, no different. Indeed, the travel time to Mars is comparable to that of navigating the Northwest Passage.

This trans-planetary travel to Mars is possible because Mars is so opulent. It has an abundance of natural resources necessary for creating rocket fuels, water, plastic polymers, alloyed metals, glass, gasses like oxygen, semi-conductors, ceramics, and just about everything else one might require. All this, he calculated, at a fraction of the cost of NASA's proposal, and using technology that has been around since the mid-19\high{th} century.\footnote{See the \about[Reverse Water Gas Shift Reactor] in \in{section}[Reverse Water Gas Shift Reactor] and the \about[Sabatier Reactor] in \in{section}[Sabatier Reactor], for instance.} His plan could allegedly be realized in less than a decade with current technology, as opposed to requiring several decades.

% Money as free speech...
\StartSection{Money As Free Speech}
In 1976 A.D., the Supreme Court of the former United States ruled in {\it Buckley v. Valeo} that, while there must be limits on contributions to political campaigns, spending money to influence elections without limit is permissible. The reason being that since corporations were considered legal persons, and since the rights of persons were constitutionally protected under the 14\high{th} ammendment, and since corporate media had the right to freedom of the press (speech), corporations, therefore, ought to have freedom of speech as well. Since they are not capable of speech in the physical sense, the closest analogue is money.

% Pressure rating...
\StartSection{Pressure Rating}
The mean sea level atmospheric pressure on Mars ranges from 30 Pa (pascals) to 1135 Pa, which is about the same as one would find at 36 km above the Earth's surface. The mean sea level atmospheric pressure on Earth, by contrast is 101,300 Pa (101.3 kPa). This means that the surface pressure on Mars is only about 1\type{%} of the mean experienced on the surface of Earth.

This has an impact on the way buildings must be engineered on Mars. The main difference between inflatable buildings is their pressure rating. A lower pressure rating means the fabric can be thinner, the building lighter, and therefore lower in cost to manufacture. The pressure rating also determines whether you need to wear a full pressure suit, just a respirator, or nothing at all.

The {\it Armstrong Limit\index{Armstrong Limit}} of 626 Pa is the lowest the human body can survive before the vapour pressure of all exposed liquids (but not liquids like blood within your skin's pressure barrier), such as tears, saliva and the liquid wetting the alveoli within the lungs exceeds that of its surrounding atmospheric pressure. They will begin to boil away at this point. On Earth, the Armstrong Limit begins at about 19 km above the surface. On Mars, it is already well exceeded at the surface.
\crlf
\crlf

\placetable[force][table:Pressure Ratings]{Martian building pressure ratings.}
{
    \bTABLE[split=repeat,option=stretch]
    \setupTABLE[column][4]
        [width=.50\textwidth,
        align=yes]
    \setupTABLE[row][each][align=center]
    \setupTABLE[4][1][align=center]

    \bTABLEhead
    \bTR[bottomframe=on]
      \bTH  Pressure \eTH
      \bTH  Respirator \eTH
      \bTH  EVA Suit \eTH
      \bTH  Description \eTH
    \eTR
    \eTABLEhead

    \bTABLEbody
    \bTR
      \bTC 6.8 kPa \eTC
      \bTC Needed \eTC
      \bTC Needed \eTC
      \bTC These buildings are attractive because they are economical and very light to pack, requiring fabric only 0.2 mm in thickness. For plants, they are fine since plants require only 5.0 kPa of pressure. But for humans, they need at least 17.0 kPa to be able to live. \eTC
    \eTR

    \bTR
      \bTC 17.0 kPa \eTC
      \bTC Needed \eTC
      \bTC Unneeded \eTC
      \bTC These buildings cost a little bit more, but you can work in them without wearing a pressure suit. You still need to wear a respirator in order for the gas exchange taking place in your lungs to still work, otherwise you will quickly pass out. \eTC
    \eTR

    \bTR
      \bTC 34.0 kPa \eTC
      \bTC Unneeded \eTC
      \bTC Unneeded \eTC
      \bTC These buildings cost a little bit more, but you can work in them without wearing a pressure suit or respirator, although the O₂ partial pressure levels still need to be enriched. The other main advantage is that the pressure can also be equalized with a habitat making movement easier. As an added bonus, bees can polinate better at this pressure coupled with the lower gravity which makes it excellent for greenhouses. \eTC
    \eTR

    \bTR
      \bTC 100.0 kPa \eTC
      \bTC Unneeded \eTC
      \bTC Unneeded \eTC
      \bTC These buildings cost the most, but they offer at least the same pressure as on Earth. Since everything needs to be three times as heavy as it needs to be, it is a waste of resources, too costly, and unnecessary. \eTC
    \eTR
    \eTABLEbody

\eTABLE
}

% Railgun...
\StartSection{Railgun}
By applying a magnetic field to a conductive mass mounted on a rail, it can be accelerated to a supersonic velocity. The drastic acceleration experienced by the mass, though high enough to crush the skull of a man, can be used for Martian material exports when brought to a speed exceeding the minimum Martian escape velocity.

Arcadians use railguns to export \goto{deuterium}[Deuterium], geochemically rare elements, and other materials back to Earth. Although they are energy demanding, they are more economical to use on Mars because only a fourth the energy required to escape the gravitational field on Earth is required on the less massive planet.

% Regolith...
\StartSection{Regolith}
What most refer to as dirt. More technically, it is the the loose heterogeneous mixture of material that blankets the solid rock of a planet. Although the term may seem redundant, note that earth refers to regolith from the planet Earth, while dirt can sometimes mean regolith that contains organic materials.

% Rhodium
\StartSection{Rhodium}
An elemental chemical denoted with the symbol {\it Rh} on the periodic table and given atomic number 45. It is a member of the platinum family and considered to be the most precious metal of that family. It is one of the rarest precious metals and costs more than any other, including gold.

Generally the only way of acquiring any kind of high quantity mineral is from high-grade ore. High-grade ore only exists when complex hydrological and volcanic processes have occurred. In our solar system, this has taken place only on Mars and Earth and hence why the Moon is barren. But unlike the Earth, Martian deposits have never been exploited.

% RWGS reactor...
\StartSection{Reverse-Water-Gas-Shift Reactor}
The reverse-water-gas-shift reactor is a method of producing oxygen (\chemical{O_2}) from carbon dioxide (\chemical{CO_2}). This is useful because the latter is plentiful in the Martian atmosphere at 95\type{%}.

%\int_0ˆ1 xˆ2 dx
\placeformula[formula:Reverse Water Gas Shift Reaction]
\startformula
\inlinechemical{CO_2,+,H_2,+,->,H_2O,+,CO}
\stopformula

The process has been known since the mid 1800s and works by reacting carbon dioxide and hydrogen gasses together over a copper-on-alumina catalyst. Aqua (liquid water) and carbon monoxide gas are produced as byproducts. The aqua is split via electrolysis to produce hydrogen and oxygen gasses. The hydrogen can then be recycled back into the reactor and the carbon monoxide purged out into the atmosphere.

The reactor needs to be at \math{400\,^{\circ}{\rm C}} and at low pressure. It requires about 180 watts of power, or about 3 \Square \Meter of solar panels on a fully sunny day's average solar flux. At that energy rate, you can expect to produce about 1 kg per day of oxygen, which is sufficient for a single person. The reactor requires power because it is an \about[Endothermic Reaction]. However, it is possible to use a \about[Sabatier Reactor] in tandem, which is an \about[Exothermic Reaction], to provide the heat required to drive the RWGS reaction.

To start the process, only a small amount of water is required which acts as a reagent. By importing hydrogen from Earth, it acts to the colonists' advantage in allowing it to be leveraged in the creation of water, or hydrogen gas if needed.

% Sabatier reactor...
\StartSection{Sabatier Reactor}
A chemical process for creating methane \chemical{CH_4} from \chemical{CO_2} and hydrogen. This is useful because carbon dioxide gas is plentiful in the Martian atmosphere at 95 \type{%}.

\placeformula[formula:Sabatier Reaction]
\startformula
\inlinechemical{CO_2,+,4H_2,->,CH_4,+,2H_2O,+,heat}
\stopformula

The reactor needs to be at \math{400\,^{\circ}{\rm C}} and at low pressure. This makes it almost the same as the \about[Reverse Water Gas Shift Reactor] except that it uses a different catalyst to make methane instead of carbon monoxide. You can either use nickel, which is cheap, or ruthenium-on-alumina, which is safer, but more expensive.

% Sierra
\StartSection{Sierra}
MARSCOM phonetic alphabet designation for a \quote{settler} of the Arcadian Settlement.

% Sol...
\StartSection{Sol}
Short for solar day, the length of time a planet takes to rotate completely on its polar axis with respect to the sun. Terrans call this a day, Martians a sol. See also {\it yestersol}.

% Specific impulse...
\StartSection{Specific Impulse}
Written \math{I_{sp}}, the specific impulse is a useful metric for comparing rocket efficiency. Whenever you see the word "specific" in a physics context, it means something per unit of mass. The units are in seconds. It measures the amount of time that one pound of fuel will burn for, producing one pound of thrust (higher being better). This can be calculated using either SI or Imperial units, but the end result is usually expressed in seconds. 

As an example, compare the specific impulse of some of the different types of rockets.

\placetable[here][table:Specific Impulse Comparison]{Comparison of different rocket specific impulses.}
{
    \bTABLE[split=repeat,option=stretch]% head on every page, stretch columns
    \setupTABLE[row][each][align=center]

    \bTABLEhead
    \bTR
      \bTH Rocket Type \eTH
      \bTH Fuel \eTH
      \bTH \math{\bf I_{sp}} \eTH
    \eTR
    \eTABLEhead
    \bTABLEbody
    %
    \bTR
      \bTC Ancient Chinese Rocket \eTC
      \bTC Gunpowder \eTC
      \bTC 80 \eTC
    \eTR
    \bTR
      \bTC Modern Rocket (e.g. ICBM) \eTC
      \bTC Solid \eTC
      \bTC 250 \eTC
    \eTR
    \bTR
      \bTC Saturn V \eTC
      \bTC LOx / kerosene \eTC
      \bTC 260 \eTC
    \eTR
    \bTR
      \bTC Space Shuttle Main Engine \eTC
      \bTC LOx / \chemical{H_2} \eTC
      \bTC 400 \eTC
    \eTR
    \bTR
      \bTC Nuclear Thermal \eTC
      \bTC Solid \eTC
      \bTC 800 \eTC
    \eTR
    \bTR
      \bTC Nuclear Thermal \eTC
      \bTC Liquid \eTC
      \bTC 1300 \eTC
    \eTR
    \bTR
      \bTC Jet Engine \eTC
      \bTC Compressed Air \eTC
      \bTC 3000 \eTC
    \eTR
\eTABLEbody
\eTABLE
}

Note how high the specific impulse a jet engine offers. This is because it is has an unlimited supply of free air from the atmosphere to feed the air compressor so it does not have to carry its own supply.

% Yestersol...
\StartSection{Yestersol}
The sol preceding the current one. This is the Mars analogue to the Terran yesterday, but different since the length of a sol on both worlds is different.

\StopChapter

