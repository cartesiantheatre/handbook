% This is part of the Avaneya Project Crew Handbook.
% Copyright (C) 2010, 2011, 2012
%   Kshatra Corp.
% See the file License for copying conditions.

% Leitmotifs chapter...
\StartChapter{Leitmotifs}

This is probably the most important chapter in the entire book in helping you to decide whether this is a project for you to be involved in or not. Avaneya has several recurring ideas or {\it leitmotifs}\index{leitmotif} within it that will help you determine whether it resonates with you.

We would like to keep the leitmotif set as minimal as reasonably possible, with everything else that we felt relevant, it hopefully already being possible to derive when obvious from that which is already present. However, if you still feel the set should be modified in some way, please bring it up on any of the project crew communication methods related in \in{section}[Communication]. We are open minded but, like us, you must also be prepared to reasonably substantiate your argument.

The following form the core leitmotif set the project felt were important, given in no particular order.

\startitemize[4]
\item
Most illegal narcotics on the street originate from government.\cite[cia_drug_plane_crash]\cite[top_mexican_drug_lord]\cite[afghan_opium_kingpin]\cite[cia_drugs_list]\cite[vicente_rule]\cite[ruppert_confronts_deutch]

\item
People are never free. To be free is to be independent of influence, which no one is.\footnote{See {\it The Equality Trust} in \in{chapter}[Resources For Everyone].} We are all subject to influence, otherwise there would be no such thing as advertising. Everything is connected and nothing happens in a vacuum.

\item
All of the major axioms of neoclassical economics, the backbone of modern economic theory, are dangerously wrong. People do not always have rational preference, do not always seek self maximization, and do not always act on the basis of full and relevant information. Markets are not natural systems comparable to thermodynamics, do not always correct themselves, and their growth is not always good. Competition does not always trump cooperation. The assumptions are copious in number. It is a pseudo-science field that is rife with fraud.\footnote{See {\it Adbuster's} \href{http://anticap.wordpress.com/2010/10/25/jamming-neoclassical-economics/}{Jamming neoclassical economics} campaign, with prejudice for \href{http://www.adbusters.org/cultureshop/backissues/85}{Issue \#85}.}

\item
Government is among the leading causes of death.\cite[statistics_of_democide]

\item
Most acts of terrorism are staged by government.\cite[debunking_911_debunking]\cite[letter_to_minister_regarding_911]

\item
Government is generally there to protect the interests of a few unelected principle benefactors, not society.\cite[report_from_iron_mountain]

\item
Most health related deaths are rooted in diet.\cite[the_china_study]\cite[statscan_leading_causes_of_death] According to {\it The China Study}, the top three causes are entirely dietary.}

\item
Agriculture is our principle means of interacting with the planet. Changing what we eat solves many global problems.

\item
Usury, which includes fractional reserve banking, is slavery.\cite[the_creature_from_jekyll_island]

\item
Giving corporations the rights of human beings was catastrophic.\cite[the_corporation]

\item
Very little of what we call human nature is actually genetic. Genes specify human needs, not behaviour.\footnote{See \in{section}[Socioeconomic Simulation].}
\stopitemize

If you have gotten thus far, and you are still comfortable being involved in this project, then you will probably find it rewarding. Otherwise, there is no sense in being here. There is no requirement that you believe anything or share our views. No one is forcing you to do anything. This is not a project for everyone and there are countless other software {\it libre} projects that could use the talents of those with reservations.

But regardless of whatever our readers choose to do, creativity is required whenever we present the aforementioned. We always try to encourage the users to do some thinking of their own. As Socrates once said, {\it I cannot teach anyone anything. I can only make them think}.

\StopChapter

