% This is part of the Avaneya Project Crew Handbook.
% Copyright (C) 2010, 2011, 2012
%   Kshatra Corp.
% See the file License for copying conditions.

% Game Mechanics chapter...
\StartChapter{Game Mechanics}
Having familiarized yourself with the game world, leading characters, and the background fictional timeline, you are probably still wondering the role of the user. This chapter will attempt to provide a complete picture of the interactive aspects of the game. Lets start by discussing the different ways users can play.

Avaneya allows users to either play single player or multiplayer. Single player, as the name suggest, the user plays alone. Well, actually, with the computer and the challenges it presents. This mode includes both a tutorial and a campaign option. In tutorial mode, the user is shown how to use the user interface and interact with elements of the game. In campaign mode, the user assumes the role of Arda Baştürk (see \in{section}[Arda Baştürk]), or any of the other leading characters as they progress through the campaign. They also have the option of loading from a previous saved game.

In multiplayer, the user can play over the local area network (LAN) with their friends. As in the case of the single player campaign, the user has the option of loading from a previous saved game.

But multiplayer offers another mode in addition to the former. Solnet\index{Solnet} is the online service that allows a user to join a much larger group for internet play. Either method allows users to plays cooperatively with each other, but this latter mode on a much grander scale. Users are required to purchase a reasonable monthly Solnet subscription to gain access to the official server. These subscriptions are useful for promoting additional development of the game's engine, artwork, music, and other graphical media. But they also ensure the official server runs on reliable high quality hardware that can handle large concurrent loads. After all, there is no sense on playing online if you are the only one.

But there is another very important distinction between playing over the LAN and Solnet. Consider the problem of what to do with the user's assets when they sign off. This is not a problem for a game hosted on the LAN because it is expected that no one cares after you and your friends pack up and have had enough. That universe is destroyed. But in the case of playing over Solnet, the world is persistent because you are sharing the same universe with potentially many others. A user who logs off has their city remain intact, but frozen in a non-interactive state, until they sign on again.

%\input Source/Gameplay/Unit_Tree.tex

\StopChapter

