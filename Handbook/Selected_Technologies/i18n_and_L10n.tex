% This is part of the Avaneya Project Crew Handbook.
% Copyright (C) 2010-2013 Cartesian Theatre <info@cartesiantheatre.com>.
% See the file Copying for details on copying conditions.

% Internationalization and Localization section...
\StartSection{i18n & L10n}

What does {\it i18n} and {\it L10n} stand for? They are old cryptic industry acronymns for \quote{internationalization} and \quote{localization} respectively. Internationalization is the process of adapting software so that it can accommodate different localities. This logic or general program behaviour is implemented by engineers.

Localities, or sometimes just called locales, specify data specific to a given locality, region, or culture, of the world. This usually includes the user's preferred language, writing direction, numeral system, and so on. The authoring of this data is the task of translators.

\startitemize[4]
\head {\em GNU gettext}

GNU gettext is used to implement {\it i18n} by providing C functions for string substitution or \index{translation}translation into any available human language. Strings in code are marked and substituted appropriately at runtime based on the user's preferred locale.

\head {\em lua--gettext}

The lua--gettext API is a Lua package that acts as a Lua wrapper for gettext bindings.

\head {\em lua--xgettext}

The {\tt lua-xgettext} tool is a small program for message extraction of marked strings from Lua code so they can be made available for translation to other languages. It is similar to GNU {\tt xgettext}, but far more primitive; it just extracts the strings and prints them out without any additional information. It was written because GNU {\tt xgettext} did not support Lua at the time it was created.
\stopitemize

