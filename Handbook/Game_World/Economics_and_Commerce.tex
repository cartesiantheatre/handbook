% This is part of the Avaneya Project Crew Handbook.
% Copyright (C) 2010, 2011, 2012
%   Kshatra Corp.
% See the file License for copying conditions.

% Economics & Commerce section...
\StartSection{Economics & Commerce}

Trade and commerce in general are central to gameplay mechanics in Avaneya. It also forms a cornerstone of multiplayer play. This section will discuss relevant design considerations.

\StartSubSection{Genuine Progress Indicator}
The GPI\index{GPI} is a system of true cost accounting which is intended to replace the gross domestic product (GDP\index{GDP}), or sometimes called the gross national product. The GPI works by taking into account all monetary costs of an activity to produce a net balance sheet. The GDP, on the other hand, only functions as an income sheet by tabulating the total amount of goods and services produced in a year.\footnotecite[cobb1999]

An example where the difference between the two is well illustrated is with the tobacco industry. The GDP accounts for the value of all tobacco products sold, a dollar value greater than zero. The GPI, like the GDP, would also account for the total value of all tobacco products sold. Where it differs is it would then go on to subtract the true cost socialized to society in the form of health costs, fires, garbage collection, environmental toxicity, and the deaths of approximately 5,000,000\footnotecite[merchants_of_doubt] people every year who are no longer productive. 

Another examples is in the cost of fast food. The upfront cost the consumer pays for a hamburger may be only 99 cents, but the actual true cost socialized to society and the environment is roughly \type{$}200.\footnotecite[extras={ p.~46.}][raj2010]

In both cases, the GDP erroneously determined a gain while the GPI revealed the more honest and closer approximation to the truth, namely, one of great deficits. The GPI is a powerful tool that is seeing more and more practical applications amongst sociologists and true cost economists.\footnotecite[costanzaa2004]

\StartSubSection{Commodities}
Commodities that can be bought and sold will vary, though they should all be as data driven as possible by using the engine's Lua interface. The data should include various parameters, graphics, and any scripted behaviour the commodity may require. All of the resources a city and its industries have available to them, supply, demand, and inter--city relations between industries are some factors in determining the complete list at any given time.

The user should be given as much information as possible when working with a given commodity. This information should include the required energy to produce and transport it. It is sometimes quite revealing to see that it might take 35 megajoules of energy to produce 500 mL of water. In other cases it, it takes only a fourth the energy to lift off from Mars than on Earth, giving the former a major economic advantage.

A very short list of commodities would include materials useful for agricultural purposes, such as smectite clays, nitrates, fertilizer, machinery, and so on. Deuterium, gold, magware (magnesiumware), food and beverages, such as Arcadian Manowar, high--grade mineral ores, geochemically rare elements, communications and shallow radar equipment useful for surveyors, and many others items form just a small sample of the possibilities.

Generally the only way of acquiring any kind of high quantity mineral is from high-grade ore. High-grade ore only exists when complex hydrological and volcanic processes have occurred. In our solar system, this has taken place only on Mars and Earth, hence why the Moon is barren. But unlike the Earth, Martian deposits of precious metal ore have never been exploited and thus present a commercial opportunity.

Other commodities can contribute to underground blackmarkets as well, such as NAU--CIA narco-traffickers. They may be providing a supply from off--world imports, or possibly manufacturing synthetic narcotics {\it in situ}, such as cocaine. If there is a demand for something, chances are there is someone willing to provide for it in the right circumstance.

Commodities can move in various ways. These include through airports, ground vehicles and railways, or even railguns for off--world exports to Earth.

\StartSubSection{Exchange Parties}
There are multiple parties that can engage in trade. These should include cities or outposts other than the user's, industries within a city, and even parties on Earth.

\StartSubSection{Industries}
The industries are many in number. Construction is one example of a Martian industry because it can transform undeveloped locations into tillable or marketable property. Prospecting for precious metals would also be lucrative because they are more abundant on Mars than on Earth. Supplying prospectors with replacement gear and supplies is another. Well--drilling for water would be yet another. Be creative.

\StartSubSection{Jenya}
Rhodium is an elemental chemical denoted with the symbol {\it Rh} on the periodic table and given atomic number 45. It is a member of the platinum family and considered to be the most precious metal of that family. It is one of the rarest precious metals and costs more than any other, including gold.

{\it Article VII} of Arcadia's {\it Rubicon Act\index{Rubicon Act}} superannuated the Terran \goto{bancor}[Bancor]\index{bancor} fiat currency\index{fiat currency}\footnote{A fiat currency is a currency that has only the endorsement of an authority, such as a government, to back it. It is neither composed of nor redeemable in any material perceived of as valuable and held in reserve .} with the {\it jenya\index{jenya}}. It is the first currency to be backed by a predominantly \goto{rhodium}[Rhodium] standard\index{rhodium standard}, with the remainder by other precious metals, such as gold and silver. The jenya became the exclusive legal tender within Arcadia when the Act was passed. This required all Terran interests to acquire Arcadian goods in jenyas only at the time. The currency symbol for the jenya is \externalfigure[jenya_inline].

\placefigure
    [here]
    [figure:Jenya_red]
    {Currency symbol of the Arcadian jenya.}
    {\externalfigure[Terms_and_Concepts/Images/Jenya_red.svg][][width=3cm]}

The word {\it jenya} is Sanskrit\index{Sanskrit}. It means {\it of noble origin, genuine, or true wealth}. The idea being that since rhodium is considered precious, for whatever reason, indeed, more so than gold on both Earth and Mars, distribution of jenyas across a society implied a redistribution of monetary wealth.

\StartSubSection{Banks & The Money Supply}

Arcadia normally functions on a rhodium standard, but this could change depending on political circumstance. Banks are generally hostile to the concept that they are responsible for its creation, but the truth is, they are.\footnotecite[the_creature_from_jekyll_island] In Avaneya, money can function either as a medium of exchange (a utility), or it can function as a dangerous weapon that will enslave Arcadia when its creation is controlled by nefarious banks.

How we model banks is important. We do not want to over--complicate things, but a certain degree of over--simplification may be necessary to convey only the most important ideas, such as the mechanics of fractional reserve lending. We should be able to do this without overwhelming the user. We can start, at least in part, by modelling a bank by using the basic accounting equation found in \in{formula}[formula:basic accounting equation].

\crlf
\placeformula[formula:basic accounting equation]
\startformula
\math{Assets = Liabilities + Capital}
\stopformula
\crlf

A bank's assets include three types of things. These would be loans to clients, reserves, and securities. Loans do not actually exist as assets, but banks maintain that they would be if they got them back. Their value is in the collateral, such as someone's home, vehicle, business, or whatever the loan was used for.

Reserves represent the amount of cash that they have in hand, the amount deposited in a central bank, and the money made from customers. They do not collect interest on this. 

Securities represent a wide range of investments. These include stocks, bonds, notes, limited partnership interests, and so on.

A bank's liabilities include two types of things, deposits and their shareholder's capital. Deposits consist of two types. They can be other peoples' money, which the bank pays interest on, or they can be money lent {\it to} the banks {\it by} the customers to fill their accounts. The bank considers these liabilities because they are liable to have to return them to the customer since it is not theirs.

The shareholder's capital is the equity or amount they put forward to start up the bank.

In fractional reserve banking, banks create new money with every bank loan. When this new money is created, the existing supply of money is devalued, thus, stealing from everyone. This is troublesome because inflation punishes in particular those who save by reducing the value of what they had saved (stealing) in a non--obvious manner. Since all money begins in a bank before making its way to citizens through wages, salaries, and dividends, it means that people are constantly being plundered instantaneously with every new bank loan.

The graphical user interface should show a selected bank's balance sheet and any fractional reserve information. This should include a user definable reserve requirement.

\StartSubSection{Inflation}
We can define inflation as the ratio of the total amount of money in citizens' hands with respect to the total cost of all available goods on the market. Contrary to popular believe, inflation is not a devaluing in the currency, although that is certainly a consequence of it, but rather an increase in the supply of money. 

\StartSubSection{Labour}
Without people providing services or producing things, there would be no monetary economy. On Mars, perpetual labour shortages would be a problem which is dissimilar to Earth. This is dissimilar to Earth because it frequently experiences a labour surplus. Even settled Martian areas will experience labour shortages because the skilled labourers will be out trying to stake a new existence on the frontier. As a result, one would expect to frequently see higher wages.

\StartSubSection{Taxation}
Taxes need to be raised in order to build public infrastructure. The user should be given a choice to experiment with different models of taxation, such as a flat or progressive model as applied to either individuals and businesses.

\StartSubSection{Railguns}
By applying a magnetic field to a conductive mass mounted on a rail, it can be accelerated to a supersonic velocity. The drastic acceleration experienced by the mass, though high enough to crush the skull of a man, can be used for Martian material exports when brought to a speed exceeding the minimum Martian escape velocity.

Arcadians use railguns to export \goto{deuterium}[Deuterium], geochemically rare elements, and other materials back to Earth. Although energy demanding, they are more economical to use on Mars because only a fourth the energy required to escape the gravitational field on Earth is required on a planet less massive.

