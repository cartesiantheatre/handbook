% This is part of the Avaneya Project Crew Handbook.
% Copyright (C) 2010, 2011, 2012
%   Kshatra Corp.
% See the file License for copying conditions.

% Sources, Mutators, Resources, & Sinks section...
\StartSection{Sources, Mutators, Resources, & Sinks}
The set of diagrams in this section show how resources can be transformed through the model the game's underlying resource processing pipeline uses. This relationship between {\it sources}, {\it mutators}, {\it resources}, and {\it sinks} is illustrated in \in{figure}[figure:Resource_Legend].

\placefigure
    [force,here]
    [figure:Resource_Legend]
    {A resource pipeline.}
    {\externalfigure[Source/Game_Mechanics/Images/Legend.pdf][][width=1.0\textwidth]}
    
A {\it source} could be the atmosphere, the ground, solar energy, or perhaps even imported from another city or off world. But except in the case of imports, sources usually contain resources in unrefined states. Think of them as suppliers of any number of resources. We denote them on our diagrams as tables with the set of resources they contain inside of them.

A {\it resource} can be anything that is, was, or could be of use to someone living on Mars. It could be a kind of energy, such as thermal, chemical potential, electrical, or what have you. Or it could be organic waste. Resources are denoted on our diagrams as octagons.

A {\it mutator} is an abstract entity that transforms one type of resource into another. They are denoted on our diagrams as three dimensional boxes. Usually these mutators are physical buildings, but they do not have to be. Note how the mutator in \in{figure}[figure:Resource_Legend] combines several resources to produce a new one.

A {\it sink} is the terminal end of a resource's journey. We denote them on our diagrams as inverted triangles. The source and sink may sometimes be related. Depending on the resource and the sink, any number of things could happen when a resource enters a sink. For example, consider Earth's atmosphere as a sink with its uptake of anthropogenic carbon dioxide increasing the amount of carbonic acid available in a source, the ocean, which in turn diminishes its ability to support life.\footnote{Neoclassical economists refer to this concept euphemistically as an {\it externality}\index{externality}.} Sinks can be defined in the game as a set of rules and behaviours for the reception of various types of resources.

Whenever a resource enters a sink and efficiently replenishes all of a source's resources needed in its creation, we call this recycling. Whenever this is not the case and a resource is non-renewable, we know that the source's supply of the resource will eventually become depleted. This is one of the many reasons why the models adopted by neoclassical economists are wrong since they presuppose the viability of linear transformation pipelines that can operate forever.

As an example, consider the Martian atmosphere as a source. We can do this because it contains resources which are useful to us, such as \chemical{CO_2}. This \chemical{CO_2} can be fed into the \about[definition:Reverse-Water-Gas-Shift Reactor] which is described on \at{page}[definition:Reverse-Water-Gas-Shift Reactor] to produce new resources, \chemical{CO}, or carbon monoxide, hydrogen, and oxygen. The carbon monoxide could, in turn, undergo further refinement when combined with other resources and mutators to produce yet additional resource such as methanol, plastics, or whatever is possible. Let us assume for the sake of example that the hydrogen was cycled back into the reactor with perfect efficiency as a reagent and the oxygen was used to breath. The lungs, a resource mutator, produce \chemical{CO_2} from the oxygen and the food you ate, with the \chemical{CO_2} vented back out into the atmosphere.

Note that not all resources require processing to be considered a resource. For instance, although the atmosphere is mostly \chemical{CO_2}, that gas is already a resource in and of itself since it has many applications, even though it could still undergo further refinements.

Since all of Avaneya's sources, resources, mutators, and sinks are data driven, to add more one simply provides new ones through the Lua interface. 

These resource processing pipelines can be as complex or over simplified as one chooses. Although there is no theoretical limit to their diversity, the game comes with a core set. Let us take a look at an over simplified model of how they work in \in{figure}[figure:Resources_Pipeline].

\FullPageDiagram
    {figure:Resources_Pipeline}
    {The resource pipeline of all the core sources, resources, mutators, and sinks.}
    {Source/Game_Mechanics/Images/Resources.pdf}

