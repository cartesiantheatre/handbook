% This is part of the Avaneya Project Crew Handbook.
% Copyright (C) 2010, 2011, 2012
%   Kshatra Corp.
% See the file License for copying conditions.

% Promises Made section...
\StartSection{Promises Made}

\StartTimelineDate{44 Aquarius, 48 B.R.}
Noted for his modest and generally apolitical demeanour, renowned Nobel prize winning French astrophysicist and areologist, Dr Richard Lamothe, publishes his autobiography. It contains the following passage which quickly becomes cause for controversy:

\startTimelineGeneralDocument
{\it "I cannot say with certainty with what public rationale the states of tomorrow will commit their public's attention to a human settlement of the Red Planet, only that it will be a lie."}

\hskip 1.5cm {\it - Reflections: Memoirs of Richard Lamothe, p.220.}
\stopTimelineGeneralDocument
\StopTimelineDate

% Provides a hint to astute reader of rough order of magnitude of time between now and story...
\StartTimelineDate{48 Aries, 15 B.R.}
The last of the {\it International Space Station} crew return to Earth with the station subsequently decommissioned, deorbited, and incinerated during atmospheric entry.
\StopTimelineDate

\StartTimelineDate{12 Cancer, 14 B.R.}
NASA's {\it Zubrin} spacecraft lifts off atop a 140 tonne capable Enigma-D heavy-lift booster from its launchpad at Cape Canaveral, North American Union. The manned module separates from the booster in low Earth orbit. It carries a crew of four with two flight engineers and two mechanics with combined specialities in geology and biogeochemistry. This is the first attempted piloted mission to Mars, implementing the Mars Direct mission architecture as a direct launch. Its objective is to determine primarily if the Red Planet ever harboured life, as well as to survey and explore.
\StopTimelineDate

% This should be 180 days after departing Earth for a fast conjunction class manoeuvre...
\StartTimelineDate{2 Libra, 14 B.R.}
{\it Zubrin} completes its conjunction class transit by performing a deceleration manoeuvre, aerobreaking through the Martian atmosphere. It enters orbit to prepare for the lander stage of its mission which carries a 25 tonnes payload of a habitat module and an Earth return vehicle (ERV).
\StopTimelineDate

\StartTimelineDate{3 Libra, 14 B.R.}
Mars Direct crew contact Mission Control from the surface of Mars. They report the ERV's chemical processing plant has been successfully brought online, producing oxygen, liquid water, and the methane/oxygen return propellant.
\StopTimelineDate

% Constraint: Should be 910 days total mission time since departure, with 550 days of Mars stay time...
\StartTimelineDate{24 Scorpius, 13 B.R.}
{\it Zubrin ERV} completes its mission objectives and lands at the Kennedy Space Center, North American Union. It provided an extensive mineralogical data profile, but failed to detect any recognizable biosignatures in the Martian regolith.
\StopTimelineDate

\StartTimelineDate{40 Sagittarius, 13 B.R.}
The \goto{bancor}[Bancor] is adopted as the new global currency and implemented via the respective central banks of the North American Union, European Union, African Union, and Asian Union.
\StopTimelineDate

% Don't use provisional designation for asteroid since includes fixed Gregorian date in name...
\StartTimelineDate{50 Sagittarius, 13 B.R.}
Professor Ramraj discovers a near-Earth, C-type Amor II asteroid, with a size comparable to Phobos from the Lincoln Near-Earth Asteroid Research laboratory, Socorro, North American Union. The discovery is rapidly assigned the designation {\it 52048 Varuna} by the {\it International Astronomical Union}. His findings are chronicled in the {\it Minor Planet Circular} where he calculates a near collision trajectory with Mars.
\StopTimelineDate

\StartTimelineDate{32 Capricorn, 13 B.R.}
Wikileaks publishes a leaked list of 163 purported attendees of the annual Bilderberg\index{Bilderberg} conference held three days prior at the Hotel de Crillon, Paris, European Union. Dr. Samuel Lieberman, director of the National Institute of Standards and Technology, and Adriaan Janssen, the Secretary-General of the United Nations, are among those listed.
\StopTimelineDate

\StartTimelineDate{2 Pisces, 13 B.R.}
The United Nations holds the Second International Mars Summit in Geneva, North American Union, with the purpose of discussing the options on the table for the second manned mission to Mars. The attendees number in the thousands, representing the states of the African Union, North American Union, European Union, and the Asian Union. Unlike the First Summit held in the years prior, this occasion has the aforementioned political presence. In addition, thousands of scientists, engineers, and philosophers attend as either independent presenters or as part of any number of associations ranging from former Case For Mars conference members to The American Astronautical Society, NASA, the RANND Corporation, and others.

The minimalistic Mars Direct\footnote{See \about[Mars Direct] for more information on Mars Direct.} approach of travelling light, living off of the land, and using indigenous materials to produce the fuel necessary for the journey home is again a central theme, but with more emphasis on the nature of the duration on Mars and material infrastructure required.
\StopTimelineDate

\StartTimelineDate{28 Pisces, 13 B.R.}
Leonard Kissinger, President of the {\it Council of Foreign Relations}, holds a symposium at the Harold I. Pratt House in New York City, North American Union.
\StopTimelineDate

\StartTimelineDate{1 Aries, 13 B.R.}
The {\it Trilateral Commission} holds its biannual meeting several months in advance, Tokyo, Asian Union.
\StopTimelineDate

\StartTimelineDate{2 Aries, 13 B.R.}
The Second International Mars Summit concludes and publishes its findings in the {\it Report on the International Proceedings of the Mars Summit}. The report contains the following:

\startTimelineGeneralDocument
...Zero capability missions have no value. The Mars Direct mission, having already demonstrated successfully the feasibility of sending humans to Mars, the bar now can and ought to be raised. The attendees concluded nearly unanimously that it is not prudent to return merely for the sake of demonstrating that a safe return is possible, that having already been established, but to create a permanent settlement on Mars. 

A suitable upper Amazonian geologically classified region where the presence of underground liquid water reservoirs has already been observed is an ideal candidate at this time. 

Such a mission architecture would be primarily concerned with providing Earth with a unique, cutting edge, off-world, research facilities to address major global issues afflicting all nations. Secondarily, it would also serve to bootstrap the first non-terrestrial expansion of human civilization. Both of these goals are in the service of all mankind...
\stopTimelineGeneralDocument

The report's abstract outline for a new mission architecture is as follows:

\startitemize[n]
\item Deploy a space station in low Earth orbit to act as a construction berth for trans-Martian spacecraft supplied by efficient single-stage-to-orbit (SSTO) vehicles.

\item Launch a spacecraft constructed within the space station that would transport only the minimum necessary to bootstrap a settlement, such as cryogenic \chemical{H_2} needed to initiate the \goto{Sabatier}[Sabatier Reactor] and \goto{RWGS}[Reverse Water Gas Shift Reactor] reactors to produce fuel and oxygen, {\it in situ}, respectively, a nuclear reactor, excavation equipment, and additional equipment.

\item Use a chemical propulsion system of \chemical{H_2/O_2} to get to Mars, then use indigenous resources to supply the return propellant of \chemical{CH_4/O_2}. This requires the mission payload to include propellant for the outbound trans-Martian injection stage only.

\item The crew would deploy construction equipment and machinery to process indigenous materials to create the {\it Arcadian Settlement}, a permanent outpost located in the Arcadia Planitia region and falling within a United Nations extraterritorial mandate.

\item The spacecraft would return to Earth unmanned, but fully fuelled with {\it in situ} produced methane/oxygen propellant. This would allow subsequent shipments to depart Earth bearing spare parts, machinery, additional excavation equipment, and whatever else the outpost could not have determined necessary {\it a priori}.

\item Personnel ferries departing Earth at regular intervals would continue to augment the settlement capability through providing additional manpower. This would increase the number of available habitats, greenhouses, and other vital infrastructure necessary to support the research being conducted.
\stopitemize

A preliminary draft of the {\it Mars Treaty} is produced using a revised {\it Outer Space Treaty} of 1967 A.D. as an adaptable boilerplate. The treaty provides the contractual basis for all signatories, outlining the road map, responsibilities, and party resource allocations necessary to bring about the terraformation and first manned mission to Mars operating under a United Nations mandate. The treaty also provides an agreement governing the activities of states on Mars and its two moons, Phobos and Deimos.

The terraformation strategy is to be initiated by intercepting {\it 52048 Varuna} with a nuclear warhead and deflecting it into a Martian collision trajectory. Being a C-type asteroid, it contains high levels of hydrated (water-containing) minerals which, if released into the atmosphere as aerosols, have been calculated to greatly increase the total available cloud condensation nuclei.

Data was presented at the summit predicting that the direct effect of the introduction of these new, dark, carbonaceous, aerosols would be a decrease in \goto{albedo}[Albedo] since they absorb more solar radiation. This would cause a cascading series of reactions, resulting in increases in the mean surface temperature and atmospheric pressure.

The deflection strategy, if executed as calculated, would have the asteroid impact directly over the Martian south pole. The south pole contains a large sheet of buried \chemical{CO_2} (dry ice) approximately 8 metres thick which, when sublimated, would theoretically act as a powerful greenhouse gas, thickening the atmosphere, along with the underlying water ice beneath it.

While the legalities of the project are defined in the {\it Mars Treaty}, the international collaborative project it implicates is known as the {\it Avaneya Initiative}.
\StopTimelineDate

\StartTimelineDate{6 Cancer, 12 B.R.}
{\it United Nations General Assembly Resolution 5571} is adopted, ratifying the Mars Treaty. The resolution augments {\it Chapter III} of the {\it United Nations Charter} to beget its seventh principle organ, the {\it Space Agency} (UNSA). Its mandate reifies the treaty, overseeing the implementation of the Avaneya Initiative.

The United Nations Office for Outer Space Affairs\index{United Nations Office for Outer Space Affairs} (UNOOSA\index{UNOOSA}) is dissolved. The former responsibilities that it held since its formation in 1958 A.D. are amalgamated under the new UNSA.

The resolution requires members to accede state space agencies, such as the North American Union Aeronautical and Space Agency (NAUSA), European Space Agency (ESA), Asian Aerospace Exploration Agency (AAXA), and others under a centralized UNSA administration. The public rationale given that space exploration and settlement ought to be an endeavour for all mankind, best realized through the cooperative aggregation of resources.

Lieberman is nominated to the Office of the President of the UNSA, resigning his position as director for the National Institute of Standards and Technology.
\StopTimelineDate

\StartTimelineDate{12 Cancer, 12 B.R.}
The UNSA's Office of the Avaneya Initiative disseminates more than ten-thousand requests for quotations.
\StopTimelineDate

\StartTimelineDate{29 Aquarius, 12 B.R.}
UNSA's Office of the Avaneya Initiative announces successful bids. The {\it Iterum Shipyard} contract is jointly awarded to Lockheed Martin-Boeing and Mercedes-Pratt & Whitney to robotically build a station remotely in low Earth orbit at a cost of β2.2 billion. The station is to be used as a general purpose interplanetary space vehicle construction platform.

% Lockheed Martin's failed X-33 design would be a good starting point for a model...
% TODO: Get the Sanskrit devangari characters to display some how...
Lockheed Martin-Boeing is awarded a contract to provide three, leased, single-stage-to-orbit (SSTO\index{SSTO}), manned, re-usable, {\it VentureStar VII} class suborbital spaceplanes. These vehicles, known as {\it Aravans},\footnote{Sanskrit अरवन् for a steed or horse.} provide UNSA with heavy lift workhorses for ferrying personnel and materials to the {\it Iterum Shipyard}. They require no external solid booster rockets or external fuel tanks and each operate at less than a fifth of the cost of the retired Space Shuttles.

Lockheed Martin-Boeing is also awarded β21 billion worth of contracts to incrementally provide four direct-launch to Mars {\it Tarikin}\footnote{Sanskrit तरीकिन for a ferry-man.} ferries capable of carrying payloads of 400 personnel each, plus provisions. The {\it Tarikins} will shuttle settlers to Mars using a chemical \chemical{H_2/O_2} propulsion system for the outbound trajectory, but then provided in Martian orbit with the return \chemical{CH_4/O_2} propellant manufactured {\it in situ}. The {\it Tarkins} will depart Earth every 780 day minimum-energy launch window, taking 180 days to arrive at Mars in constant rotation with each other.

Mitsubishi-Saab's is awarded a contract to robotically construct {\it Avaneya} remotely within the {\it Iterum Shipyard} orbital berth at a cost of β7.8 billion. {\it Avaneya} will carry the bootstrap crew and large amounts of construction equipment.

Volvo-John Deere is awarded a contract to provide chemical propulsion systems for {\it Avaneya}, {\it Aravans}, and {\it Tarikins} at a cost of β920 million. It is used for the outbound trans-Martian trajectory, with the return methane/oxygen (\chemical{CH_4/O_2}) propellant manufactured {\it in situ}.

{\it Carlyle Holdings'} {\it Bronfman-Murdoch Aerospace} is awarded a contract to provide four {\it Mars Positioning System} satellites and the {\it Mars Enhanced Telecommunications Orbiter} carrying an optical relay and other instrumentation at a cost of β750 million. The five satellites are carried as part of the {\it Avaneya} mission payload
\StopTimelineDate

% Huelva pronounced \ˈwel-vä, ˈhwel-\
\StartTimelineDate{4 Pisces, 12 B.R.}
Construction of the last of UNSA personnel training facilities in Antarctica and a red-tinted river basin in Huelva, European Union, are completed.
\StopTimelineDate

\StartTimelineDate{40 Pisces, 12 B.R.}
The first crew candidates arrive and report to UNSA training facilities. Orientation and courses begin two days later.
\StopTimelineDate

\StartTimelineDate{15 Virgo, 11 B.R.}
The {\it Iterum Shipyard} is completed remotely in low Earth orbit. The station orbits Earth at an altitude of 340 km, travelling at a speed of 27,400 km/h, and taking one and a half hours to complete one orbital revolution about the Earth.
\StopTimelineDate

\StartTimelineDate{19 Scorpius, 11 B.R.}
The crew selection is completed with nearly two-hundred personnel encompassing a broad range of specialities. These include flight engineers, artificial intelligence specialists with knowledge engineers among them, astrogeophysicists, cyberneticists, chemical and civil engineers, cold weather construction experts, mechanics, biogeochemists, geologists, areobotanists, a single xenobiologist, and others.

\goto{Arda Baştürk}[Arda Baştürk] is named Mission Commander; Henrik Nørgaard as Chief Medical Officer and Chief Field Science Officer; Senka Rukavina as Flight Engineer, Mechanics Team Lead, and Survey Team Lead; Khalid Zafar as Systems Team Lead; and Nayana Rai as Greenhouse Team Lead and Terraformation Team Lead. Further, Leonard Kissinger is appointed United Nations Deputy Secretary-General and will accompany the crew as a viceroy of sorts.
\StopTimelineDate

\StartTimelineDate{11 Sagittarius, 11 B.R.}
{\it Mars Science Laboratory Curiosity XI}, an unmanned autonomous aerial vehicle, surveys potential landing sites in Arcadia Planitia. The integrated onboard artificial intelligence is instructed to evaluate sites based on average available sunlight, underground water ice revealed by airborne shallow radar, mineralogy, surface geography, and other simultaneously weighted factors.
\StopTimelineDate

\StartTimelineDate{49 Sagittarius, 11 B.R.}
The {\it Internet Assigned Numbers Authority} allocates {\tt A001:CA7:3134::/48} IPv6 address block for general Martian use. 
\StopTimelineDate

\StartTimelineDate{16 Capricorn, 11 B.R.}
{\it Aravan III} departs UNSA's Cape Canaveral launchpad en-route to the {\it Iterum Shipyard}, low Earth orbit, with a mission payload of materials for {\it Avaneya}. This includes the disassembled aeroshield, whole food, potable water, lab equipment, and its Haliburton nuclear reactor.
\StopTimelineDate

\StartTimelineDate{12 Aries, 11 B.R.}
{\it Avaneya} construction is completed remotely in orbit at the {\it Iterum Shipyard}, low Earth orbit.
\StopTimelineDate

\StartTimelineDate{8 Taurus, 11 B.R.}
Launched from the Kennedy Space Center, North American Union, the {\it Aravan II} transporting all {\it Avaneya} crew docks successfully with the {\it Iterum Shipyard}, low Earth orbit.
\StopTimelineDate

\StartTimelineDate{10 Taurus, 11 B.R.}
{\it Avaneya} completes all system checks and disembarks the {\it Iterum Shipyard} berth. It commences its six month journey with a delta-v from low Earth orbit injecting itself into a trans-Martian conjunction class orbital manoeuvre.
\StopTimelineDate

\StartTimelineDate{8 Cancer, 10 B.R.}
Henrik, Nayana, and others commence scheduled in-transit experiments in space medicine, life sciences, astronomy, physical sciences, meteorology, and human research.
\StopTimelineDate

\StartTimelineDate{2 Leo, 10 B.R.}
{\it Avaneya} engages a short burn of its manoeuvring thrusters to performs a debris collision avoidance in response to a warning Senka receives emanating from the {\it Solar and Heliospheric Observatory}. The observatory is stationed at the Lagrange \math{L_1} gravity well stationed between the Sun and the Earth.
\StopTimelineDate

\StartTimelineDate{44 Virgo, 10 B.R.}
The crew wrap up all scheduled in-transit experiments in space medicine, life sciences, astronomy, physical sciences, meteorology, and human research.
\StopTimelineDate

\StartTimelineDate{9 Leo, 10 B.R.}
Khalid instructs {\it Avaneya} to deploy all four of the {\it Mars Positioning Satellites} (MPS) the ship is carrying into high Martian orbit.
\StopTimelineDate

% Constraint: 180 days after launch with a departure velocity of 3.34 km/s...
\StartTimelineDate{10 Leo, 10 B.R.}
Assisted by retrorockets, {\it Avaneya} performs orbital capture by aerobraking into Martian orbit. The ship's various instrumentation subsystems update the onboard areology database on the most recently available surface geography, weather dynamics, mineralogy, and other available telemetry.

The {\it Mars Enhanced Telecommunications Orbiter} is released and deploys itself into areosynchronous orbit, 17,065 kilometers above the equator.

\StartTimelineDate{11 Leo, 10 B.R.}
Khalid advices UNSA that the {\it Mars Enhanced Telecommunications Orbiter} (METO) has successfully completed all self diagnostics. The satellite registers itself into UNSA's {\it Interplanetary Internet} as an available communications node. This augmented form of the internet extended over to Mars gives rise to what becomes colloquially known as {\it solnet}. Terran downlink passes through a ground station at UNSA's Jet Propulsion Laboratory staffed by Bronfman-Murdoch Aerospace contractors and provides tier 1 network access.

Brokered by a satellite uplink with METO, Khalid's Systems Team establishes contact with Mission Control routed over the solnet. The solnet becomes the defacto standard for Earth-Mars intercommunication and comes to replace the term \quote{internet} in the everyday parlance.
\StopTimelineDate

