% This is part of the Avaneya Project Crew Handbook.
% Copyright (C) 2010-2017 Cartesian Theatre™ <info@cartesiantheatre.com>.
% See the file Copying for details on copying conditions.

% Signing the Code of Conduct section...
\StartSection{Signing the Code of Conduct}

All contributing members of our community need to sign the Avaneya Code of Conduct presented in \in{section}[Avaneya Code of Conduct]. If you would like to be one of them, you will need an OpenPGP compliant key pair in order to do this. Usage of OpenPGP is beyond the scope of this book, but if you need help, consult any of the many online tutorials available. {\it Just remember to make sure you backup your key pair.}

If you are not using a GNU operating system, such as Ubuntu, do not worry. GnuPG is almost guaranteed to be available for it -- possibly even with an attractive, easy to use, GUI that integrates into your shell like Seahorse.\footnotecite[seahorse] If this is the case, you will have to consult with the appropriate documentation or tutorial for your environment to achieve the same goal attempted here.

Once you have read the Avaneya Code of Conduct, you can generate your digital signature using GnuPG as follows. Of course, replace \quote{FirstName} and \quote{LastName} with your first and last names respectively.

\startCodeExample
$ cd Avaneya/Documentation/Contributors/Code\ of\ Conduct/Signatures/
$ gpg --armour --output "FirstName LastName.gpg" --clearsign \
    ../Code\ of\ Conduct
\stopCodeExample

Note that the trailing \quote{\type{\}} character should not appear when you type it, but is written above just to signal a line break to you.

This command generates a digital signature within the {\tt Signatures/} directory. This file can then be added and commit into the repository, or bundled as part of a larger commit, as described in \in{section}[Using Bzr]. 

The signature is also human readable ({\tt --armour}) and clearsign ({\tt --clearsign}) format. This means that the file you just generated contains both a copy of the original document within it as well as your digital signature. This makes it easier for others.

To verify someone's signature, run the following. You will need the signer's verified public key available in your keyring in order for GnuPG to check the file's authenticity against it. If you are verifying your own signature, then you probably already have it installed.

\startCodeExample
$ cd Avaneya/Documentation/Contributors/Code\ of\ Conduct/Signatures/
$ gpg --verify "Dicky Chow.gpg"
gpg: Signature made Sun 08 Apr 2012 09:40:48 PM PDT using DSA key ID ABCD1234
gpg: Good signature from "Dicky Chow <dicky_chow@some_domain.cn>"
\stopCodeExample

In the above example, the signature appeared to be authentic.

Note that if the contents of the Avaneya Code of Conduct ever changes and you are still a contributor, you will need to sign the document again using the previously described method to update your signature.

