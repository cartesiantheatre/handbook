% This is part of the Avaneya Project Crew Handbook.
% Copyright (C) 2010, 2011, 2012
%   Kshatra Corp.
% See the file License for copying conditions.

% Environmental section...
\StartSection{Environmental}

This section covers some of the relevant environmental characteristics of Mars. This is important for shader writers and artists in particular.

\StartSubSection{Atmosphere}
The atmosphere is 95\% \chemical{CO_2}. The mean sea level pressure ranges from 30 Pa to 1135 Pa. This is about the same at 36 km above the Earth's surface. 

Because there is very little atmospheric buffering, the temperature will rise and fall in rapid synchronicity with its proximity to the Sun determined by its orbital position. It ranges from a minimal \math{-90^{\circ}}C to a maximum of \math{20^{\circ}}C and with an average of only \math{-63^{\circ}}C. But despite the extreme temperature, the greater threat to human is actually the low atmospheric pressure. 

The {\it Armstrong Limit\index{Armstrong Limit}} of 626 Pa is the lowest the human body can survive before the vapour pressure of all exposed liquids (but not liquids like blood within your skin's pressure barrier), such as tears, saliva and the liquid wetting the lung's alveolar membrane exceed that of its surrounding atmospheric pressure. They will begin to boil away at this point. On Earth, the Armstrong Limit begins at about 19 km above the surface. On Mars, it is already well exceeded everywhere.

\StartSubSection{Desert Threats}
Deserts are dangerous and the vast ones of Mars are no exception. It is not difficult for one to get lost in the event that their MPS stops working due to a hardware failure, a solar flare affecting a relay satellite, a cosmic ray affecting a receiver, or software defects in the proprietary software used in the communications equipment.

Compounding things, there is very little light at night, so it is pitch black. One can also freeze to death in the \math{-90^{\circ}}C weather if their suit loses its power or the integrated heating element fails for whatever reason. If they are out for long enough, they could run out of oxygen.

One could also run out of water which they need about 1 kg per day of under normal circumstance. To extract the water from the soil, perhaps in an emergency, one could setup a greenhouse--tent to condense the water in the regolith which it contains anywhere from 3\% to as high as 60\%. If the equipment is available, a photovoltaic oven could be used to melt the water in the soil. If whatever process is involved in extracting water from the soil uses distillation, then salinity will not be a problem.

\StartSubSection{Regional Characteristics}

The planet's north and south poles are different. The north is mostly water ice, containing about 821,000 km\high{3} of it. It has a more moderate climate than the south and is wettest in the spring. The south is mostly \chemical{CO_2} ice and colder. However, both locations have gale strength winds and thick permafrost mantles.

Climate varies throughout Mars, like on Earth, not just at the poles. The northern hemisphere is advantageous because it has less seasonal variation in temperature. This is because the planet is closer to the Sun in the northern winter and farther during its summer. There are also many more lower altitude locations so pressure would be highest there when mean atmospheric pressure is increased through terraformation.

It was for these reasons that the Arcadian Settlement sought a mid--latitude location in {\it Arcadia Planitia}, roughly half way between the north pole and the equator. The region of {\it Arcadia Planitia} was originally region of Ancient Greece. It was in turn named after the Greek legend of Arcas\index{Arcas}.

The centre of {\it Arcadia Planitia} is mostly uniform in appearance, with its centre roughly at \math{46.7^{\circ}}N \math{192.0^{\circ}}E. The seasons are not only more predictable, but there is a great deal of water ice and sunlight for the greenhouses year round.

Images of a fresh meteorite crater 12 m across taken in 2008 revealed under the surface a massive blanket of water ice.\footnotecite[water_ice_on_mars] It also turns out fortuitously that this ice is almost completely pure, with only about one percent of it dirt.

Even though there is a great deal of ice on Mars, it is unstable in the thin Martian atmosphere and you rarely ever see it anywhere other than in the polar regions. This is because the ice rapidly sublimates\footnote{When a solid material sublimates, it means it skips melting to a liquid and turns directly into vapour.} as soon as it is exposed.

The region has also experienced recent lava flows. By recent, in a geological time scale, that means within the last few hundred-million years. Therefore, Mars most likely is still volcanically active and still has a molten core. It does not, however, appear to spin around its core which is why there is no magnetic field -- rendering a magnetic compass useless.

{\it Arcadia Planitia's} windswept landscape consists of a vast, mostly flat, pale tan coloured plain. It has sand dunes of modest height, never approaching anything higher than a few feet, with small uniformly sized rocks littering the surface. As the prevailing theory goes, these rocks are remnants of some of the underlying bedrock which is an older layer of solidified lava. Every time an asteroid impacts, penetrating the younger upper layer, the underlying bedrock ejecta is scattered everywhere.

Like all other explored regions, it undergoes constant sterilization through intense ultraviolet radiation. This ensures that there is no {\it known} life on the {\it surface} of Mars.

Artists and shader writers are encouraged to draw on the resources provided by our {\it Viking Lander Remastered Archive} described in detail in \in{chapter}[Viking Lander Remastered].

\StartSubSection{Radiation}

Radiation is a concern on Mars since its levels are 50 times those of Earth. It is usually measured in either rems or Sieverts (sv in SI). One rem is the equivalent of \math{1/100} sv.

The amount or dosage received, combined with various weighting factors, determine how effective the radiation is likely to cause in terms of biological damage.

Exposure can be grouped into two categories, acute and extended. Acute or prompt exposure is shorter than the time of cellular regeneration. Anything less than 50 rems is typically sub--clinical and should not produce anything other than blood changes. 50 to 200 rems can cause illness but will rarely be fatal. 200 to a 1,000 rems will cause serious illness and with anything greater than 450 usually killing half of all those exposed. Anything greater than 1,000 rems and it is almost always fatal.

Extended exposures are any exposures that took longer than the normal time for cellular regeneration to take place. A low exposure will not make you sick, while a medium dose will increase the likelihood of cancer.

Sources of natural radiation on Mars are solar flares and cosmic rays. Solar flares are of no danger to people on Mars, but only to space travellers, since even though the planet's atmosphere is thin, it is still enough to protect you. Even then, although they are mostly unpredictable, they average only one per terrestrial year. They tend to happen more frequently during the planet's solar max and less frequently during the solar minimum. They can last for several hours at several thousand rems and consist mostly of protons with a few million volts. Shielding with 12 cm of water or comparable mass of \math{12g/cm^{2}} is sufficient from protection.

Cosmic rays are another significant source of radiation, but to Martians as well as to space travellers. Unlike solar radiation, their particles have energies in the billions of volts and no one is certain where they come from -- though we are certain they do not come from our Sun. Interstellar crew would be subjected to about 30 rems per year. That is about 21 rems for an 8.5 month type--II Hohmann transfer.

Despite all of the emphasis NASA gives the dangers of cosmic radiation on Mars, there are several reasons to not be alarmed. Half of the sky is always blocked out to those on it. The remaining radiation is reduced by roughly a factor of half to only about 10 rems per year reaching the surface. Proper shielding of buildings makes this not an area of concern. 

Moreover, we know now that cancer on Earth is nearly always dietary induced and so nutrition needs to be given more than a head nod.\footnotecite[the_china_study] Many foods and drinks can even help repair cellular DNA damage caused by radiation, such as certain fermented beverages like kombucha.\footnotecite[cavusoglu2009]

\StartSubSection{Sky}
Mars has a beautiful sky. At night time, it may be possible to see auroras. During the day, one can see \chemical{CO_2} and water clouds. Artists are encouraged to examine the images provided by our {\it Viking Lander Remastered Archive} described in detail in \in{chapter}[Viking Lander Remastered] to get an idea of their appearance.

