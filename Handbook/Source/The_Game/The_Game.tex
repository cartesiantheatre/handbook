% This is part of the Avaneya Crew Handbook.
% Copyright (C) 2010, 2011
%   Kshatra Corp.
% See the file License for copying conditions.

% The Game chapter...
\startchapter{The Game}

\startsection{Why}
Too often, people have come out of experiencing a great dystopian science fiction novel or film, only to say to themselves, Thank goodness we don't live in {\it that} world. That needs to change.

The problem is that dystopian science fiction is among the most honest kind of story telling when examined in the context of history. But unless people can see the pedagogical relevance, analogies will remain vauge and metaphors too cryptic to be useful. Perhaps it can be said that {\it responsible} dystopian science fiction makes the connection.

But good science fiction is only {\it partly} fiction because it is the {\it science} portion that attracts people to it in the first place. Science, by its nature, is a thinking and knowledge oriented enterprise. Science is the study of reality, and yet, fiction is outside of reality. So science fiction is a mediator between the two, often acting as an {\it avant-garde} for both.

This can set the bar high for what its patrons expect and requires designers to pay a great deal of attention to detail. A certain degree of creative license is expected, but people still prefer it to be as consistent with what we already know to be true, or what is at least reasonably plausible.

Another important reason why this project is necessary is that there is very little, if any, {\it free}, commercial, games for the GNU operating system. When people use GNU, they are treated as second class citizens in many respects - not least of which is the availability of good games. Not only are there very few higher production titles that are available for GNU, they are usually proprietary, and even then, generally bad ports using deprecated APIs, poorly packaged, \footnote{That is, if they even bother to use the platform's native distribution's package manager at all.} and integrate horribly into the user's desktop - ignoring the usual human factors community driven conventions.

In some sense, Avaneya is part of the effort to complete the GNU operating system. This is because {\it a complete system needs games too}, Richard Stallman once remarked.\footnote{\fullahref{http://www.gnu.org/gnu/linux-and-gnu.html}}

But in terms of the games' subject matters themselves, they tend to appeal more to the mainstream proprietary user. They do not share the {\it software libre} community's values of freedom. To them, conscious computing is irrelevant since machines are just tools existing outside of a social context. 

Worded differently, the choice between using free and proprietary software is not in the same class of arbitrary choices, like chocolate versus vanilla. Users of free software are qualitatively different from proprietary users. Therefore, a game for that audience should reflect this.

\startsection{Classification}
People have struggled in the past to classify Avaneya. It is what it is, but the closest traditional categories that form a subset of it are the traditional city builder and management simulations and the real time strategy.

\startsection{Likely Users}
The game so far has attracted a fairly large base of followers. From what can be observed at this time, the game appears to appeal to those with an interest in:

\startitemize[3]
\item
{\it challenging the consensus of reality}
\item
{\it software libre}
\item
{\it a social conscience}
\item
{\it science fiction}
\item
{\it an interconnectedness of everything}
\stopitemize

The game may take place in the future, but it deals with current problems. The best way to get an idea of the intended audience is to quickly see both \in{chapter}[Core Leitmotifs] and \in{chapter}[Resources For Everyone]. You will be in a better position to try and gauge the type of audience that this game probably resonates with after doing that. 

\startsection{Unlikely Users}
Avaneya is a {\it sui generis}.\footnote{{\it "Literally meaning of its own kind / genus or unique in its characteristics. The expression is often used in analytic philosophy to indicate an idea, an entity, or a reality which cannot be included in a wider concept,"} (Wikipedia).} It is not like other games, and thus it is not for all people. It does not try to be, nor will it ever be.

Those with a brief attention span, believe that things originate in cans with little appreciation for process, and accidentalists will probably not enjoy this game. There are already many such games that appeal to that type of audience, so that need not be our aim here.

This game will challenge you to think, and possibly even offend you. It challenges the consensus of reality, and therefore, potentially, your world view. Consequently, some have accused Avaneya of being a vehicle for culture jamming and political commentary. This project is shamelessly guilty as charged---like the newspapers, film, television, games, and other mainstream media that saturate us. 

The only difference is that, unlike those mediums, the very presence of a normative bias in Avaneya is not subject to dispute and is self evident. Other mediums sometimes pretend to not have one. In any case, you would be very hard pressed to try to find any classical work of science fiction, or really any kind of fiction for that matter, that did not. Moreover, that in itself is not necessarily a bad thing.


