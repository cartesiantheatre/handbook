% This is part of the Avaneya Project Crew Handbook.
% Copyright (C) 2010-2014 Cartesian Theatre <info@cartesiantheatre.com>.
% See the file Copying for details on copying conditions.

% How To Read This Book chapter...
\StartChapter{How To Read This Book}

It should not be necessary for readers to feel compelled to read the whole book, as specific sections are intended for specific audiences. Readers and contributors do not need to be familiar with every aspect of this creative and technical endeavour, any more so than we should expect a theatre patron to be familiar with all of the details of stage lighting, acoustics, pyrotechnics, carpentry, costume design, or what have you.

Having said that, contributors, or actually anyone for that matter, will have nothing to lose in reading as much or as little as they like. We tried to efficiently organize material in such a way that everyone would benefit the most when chapters are explored in the order presented. However, if you happen to be a contributor, some portions will probably be of greater interest than others.

If you happen to be a contributor, we describe who you might be later in \in{chapter}[Specialities].

If you happen to be an editor, you should turn right away to \in{section}[Editing This Handbook] for instructions on how to properly go about doing this.

If you are a passive, casual, or non--technical reader, simply wanting the \quote{Coles Notes} on this project, we encourage you to check out chapters \in[Leitmotifs], \in[The Game], \in[Leading Characters], and \in[Timeline]. This should get you up to speed in as little time as possible.

If you already are, or would like to consider becoming a contributing member of our community, you would probably do well to read as much as you like. One noteworthy exception is \in{chapter}[Engineer Contributors], unless you would like to get involved in an engineering capacity, or simply curious of the types of problems the engineers must deal with.

With all of that out of the way, on we go!

\StopChapter

