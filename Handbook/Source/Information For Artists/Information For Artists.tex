% Information For Artists chapter...
\chapter{Information For Artists}

% The Martian Landscape
\section{The Martian Landscape}

Getting the game experience right means replicating the visual environmental conditions of the Martian landscape as closely as we can. Since the artists use tools like Blender to create an aesthetic experience, not readings from a mass spectrometer, it is more important that they know what Martian rocks and sunrises would actually look like if they were standing there than, say, the peroxide concentration of Martian regolith (dirt). This means they will need lots of high resolution reference photographs to look at.

The Avaneya settlers erected the settlement of Arcadia in the @dfn{Arcadia Planitia} region. It is named after the Arcadia region of Ancient Greece, so, in turn, named after the legend of Arcas. 

Mid-latitude Martian areas like this have their benefits, such as year round sunlight and plenty of water ice. The centre of Arcadia Planitia is mostly uniform in appearance, with its centre at roughly @math{46.7^{\circ}}N @math{192.0^{\circ}}E.

Its windswept landscape consists of a vast, mostly flat, pale tan coloured plain. It has sand dunes of modest height, never approaching anything higher than a few feet, with small uniformly sized rocks littering the surface. Like all other explored regions, it is well sterilized with intense ultraviolet radiation, ensuring that there is no known life on the Martian surface.

Images of a fresh meteorite crater 12 metres across taken in 2008 revealed that under the surface reveal a blanket of ice.@footnote{See @url{http://www.cbc.ca/news/technology/story/2009/09/24/tech-space-water-mars-crater.html}.} It also turns out fortuitously that this ice is almost completely pure, with only about one percent dirt.

Even though there is a great deal of ice on Mars, it is unstable in the thin Martian atmosphere and you rarely ever see it anywhere other than in the polar regions. This is because the ice rapidly sublimates@footnote{Skips melting to a liquid and turns directly into vapour.} as soon as it is exposed.

Arcadia Planitia has experienced recent lava flows. By recent, we mean in geological time - As in the last few million years.

@node Viking Landers
@subsection Viking Landers

At present, no lander has explored Arcadia Planitia returning colour images. However, the neighbouring plain of Utopia Planitia is very similar in appearance and we have a plethora of images captured by the Viking II Lander at @math{47.7^{\circ}}N @math{225.9^{\circ}}W which it explored back on September 3, 1976.

A bit of background. In 1976 NASA placed two spacecraft into Martian orbit. The Viking Orbiters both carried their own landers which they deployed. They were capable of roaming the surface of the planet and relaying their findings through the orbiters back to Earth. They continued to do this without any issues for four years. Of the data collected, the useable colour images have been prepared by the Avaneya crew in the @i{Mars Viking Lander Colour Image Restoration Archive}. They were painfully restored from what was recovered into the archaic VICAR format from even more archaic old, rotting, magnetic tapes encoded in formats few have access to.
@sp 1


