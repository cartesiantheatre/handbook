% This is part of the Avaneya Project Crew Handbook.
% Copyright (C) 2010-2013 Cartesian Theatre <kip@thevertigo.com>.
% See the file Copying for details on copying conditions.

% Multiplayer: Solnet section...
\StartSection{Multiplayer: Solnet}

Another multiplayer mode is {\it Solnet}\index{Solnet}. Solnet is the online service that allows users to join a much larger group for cooperative play, potentially in the thousands, over the internet. Users having purchased a reasonable monthly subscription gain access to the official Solnet server which hosts, initially at least, a sector from {\it Arcadia Planitia} in the fictional timeline during the Arcadian Diaspora. See \in{section}[Sectors] for more on sectors.

But playing over Solnet offers another distinct advantage over multiplayer via LAN -- persistence. Consider the problem of what to do with the world the user plays in after they have signed off from a multiplayer session. This is not a problem for a game hosted on the LAN because it is reasonable to expect that no one cares after you and your friends pack up and have had enough. The logical cluster of game world space, objects, and states is destroyed and no machine need labour to have it persist in the players' absence.

But in the case of Solnet, the world {\it must} remain. It is persistent because the user may be sharing the world with potentially thousands who would not appreciate it if it was abruptly destroyed simply because some of its users in another timezone went to bed. Thus, the world hosted on Solnet carries on with a life of its own -- even in the user's absence.

The Solnet subscriptions are absolutely critical for ensuring the official server runs on reliable high quality hardware that can handle large concurrent loads made accessible through sometimes costly, albeit necessary, hosting. This is also vital for promoting additional development of the game's engine, artwork, music, and other graphical media that continue to make the game better and sustainable.

Before we move on, let us review a summary of all of the different modes of play we have covered.

\startitemize[4]
    \item Single Player
        \startitemize[4]
        \item Tutorial
        \item Campaign
        \item Custom
        \stopitemize

    \item Multiplayer
        \startitemize[4]
        \item Local Area Network (LAN)
        \item Solnet
        \stopitemize
\stopitemize

That concludes a general overview of who can play with who and over what medium. You can see \in{section}[Networking] if you are curious about the technology that makes multiplayer possible. But now we must turn our discussion to what it is that the user actually does, either by themselves or with others.

