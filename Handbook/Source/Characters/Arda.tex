% This is part of the Avaneya Project Crew Handbook.
% Copyright (C) 2010, 2011, 2012
%   Kshatra Corp.
% See the file License for copying conditions.

% Arda Baştürk section...
\StartSection{Arda Baştürk}
Arda was born 6 Capricorn, 81 B.R. in Istanbul, European Union, to Dursun and Ayla Baştürk. His mother, Ayla, was an assistant to the manager of a local and independently operating bank. His father was an aeronautical engineer. Both were non-practising secular Muslims. The latter initially began his career as a military intelligence officer before completing his engineering studies and securing employment with NASA's Jet Propulsion Laboratory. 

Shortly after Arda was born, what was then Turkey, became a member of the European Union. This was a time marked by enormous social, political, and economic change. The times were further accentuated when the European Union underwent a transformation from being merely a confederation of states to a nation in and of itself in 76 B.R.. This left the former Turkey and all other member states amalgamated, and thus dissolved with respect to their traditional geopolitical boundaries. With political leaders around the world pointing to the transformation as a story of great success, the North American Union and the other Unions rapidly followed in its example within several years.

Dursun had been deeply engaged as a political activist in opposing Turkey's membership, both before and after it actually gained it. Several attempts had been made on his life, culminating in a car bomb that had left him paralysed in a wheel chair. He died less than a year later. However, an autopsy revealed he had not succumbed to the condition the accident had left him in, but the result of polonium-210 induced radiation poisoning.

As Arda grew, Ayla was reluctant to discuss politics or the identity of those she suspected of involvement in her husband's death. She did not wish to risk putting her son in danger and felt it best that everyone simply move on. Nevertheless, Arda, being a precocious child, assembled enough information from the world and family relatives to sense his mother knew something beyond the apparent. He came to believe that she had always known that the attempts on his father's life had probably not been the work of marginalized extremists, but of influential elements acting within their own government.

By the time he came of age, he was made aware of the rumours linking his father's death to those with relations to a powerful banking cartel of international financiers, the House of Rothschild. However, Arda found the world of banking uninteresting, convoluted, and of little relation to his principle devotion, the world of natural phenomena. As such, by the time he reached his early teens, he conceded to his mother's advice and shifted his interests to the more pressing matters of life.

As part of a working class family, with only his mother to support him, Arda had few opportunities that would not come without extending great effort on his part. In an era where scarcity had become a way of life, they were of modest means and his mother knew it was important to continue imparting Dursun's philosophy of a positive work ethic and the pursuit of a higher education. With food shortages rampant, high unemployment, and riots on a weekly basis in Istanbul, she was convinced a higher education was her son's only way out - should he have any at all.

Arda went on to study at Istanbul Technical University, made possible through a combination of scholarships and his salary as a part time army reserve officer cadet. He graduated with a degree in aeronautical engineering where he was then commissioned into the army as a lieutenant. He completed his research doctorarate in heliophysics two years later and his habilitation one year after that. He gained notoriety in having solved a central problem plaguing his habilitation thesis one day while deployed on a field exercise with the army.

Although wanting to focus on his research, even with his qualifications, employment was scarce and the army reserves was one of the few opportunities available that was stable and left him with just enough money to afford occasional time to himself for study. Nevertheless, he was well respected in the army as a natural born leader, though his privately held views of the military reflected an unnecessary institution that he resented, saw as corrupt, and susceptible to gross misuse.

He later underwent commando training with the 3rd Commando Brigade, Siirt, before accepting a posting with the Mountain and Commando Brigade, Hakkari. He rose through the ranks before acting as the unit's liaison officer, overseeing North American Union troops undergoing training within their facilities.

Arda had always been described throughout his life by both his peers and colleagues as both highly perceptive and intelligent, although generally reserved. Prior to his time on Mars, directing his above average intellectual gifts with passion to anything other than his principle love of science was uncommon. He was generally unconcerned with applying his intellect to solving issues pertaining to social phenomena or pursuing studies in the humanities. Nevertheless, he was not without compassion. With the exception of what little he knew of his father, he felt the political landscape was largely unproductive and there were more interesting things to concern himself with. That, combined with his privately held conviction that the world was irreversibly falling apart, encouraged him to find comfort in discovering the constants of the universe through its natural phenomena and the forces governing it.

