% This is part of the Avaneya Project Crew Handbook.
% Copyright (C) 2010, 2011, 2012 Cartesian Theatre <kip@thevertigo.com>.
% See the file Copying for details on copying conditions.

% How To Read This Book chapter...
\StartChapter{How To Read This Book}

Be honest, large books can be intimidating. Fortunately, most people need not read this one in its entirety. This is because one does not need to be any more familiar with every aspect of this creative and technical endeavour any more so than we should expect a theatre patron to be familiar with all of the details of stage lighting, acoustics, pyrotechnics, carpentry, costume design, and what have you. If you are not responsible for the play's production and only wish to view it, such details may be uninteresting. However, if you happen to be directly involved in the production, or are simply curious, some of these aspects will probably be of interest.

Having said that, contributors, or actually anyone for that matter, will have nothing to lose in reading the whole book. We also tried to efficiently organize it in such a way that everyone would benefit the most when its chapters are explored in the linear order presented.

Depending on the audience, different parts of this book will appeal more so to some than others. If you happen to be a contributor, we describe what type later in \in{chapter}[Specialities]. But if you happen to be an editor, you should turn right away to \in{section}[Editing This Handbook] for instructions on how to properly go about doing this.

If you are a passive, casual, or non--technical reader, simply wanting the \quote{Coles Notes} on this project, we encourage you to check out chapters \in[Leitmotifs], \in[The Game], \in[Leading Characters], and \in[Timeline] at a minimum. This should get you up to speed in as little time as possible.

If you already are, or would like to consider becoming a contributing member of our community, you would probably do well to read most of the book. One noteworthy exception is \in{chapter}[Engineer Contributors], unless you would like to get involved in an engineering capacity, or simply curious of the types of problems the engineers will have to deal with.

With all of that out of the way, on we go!

\StopChapter

