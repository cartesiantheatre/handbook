% This is part of the Avaneya Project Crew Handbook.
% Copyright (C) 2010-2018 Cartesian Theatre™ <info@cartesiantheatre.com>.
% See the file Copying for details on copying conditions.

% Resources section...
\StartSection{Resources}

There are many types of resources in Avaneya. As described in \in{section}[Commodities], they should all be as data driven as possible by using the engine’s Lua interface. The data should include various parameters, artwork, and any scripted behaviour the resource may require. This section will describe a small subset of all the possibilities of the many types of resources. It exists to encourage more ideas.

Some of these resources can be used as fuels. Fuels are necessary for moving anything from rockets and rovers to motorbikes. Different types are available, but each one described below has some unique considerations. These include carbon monoxide, dimethyl ether, hydrazine, and methane / oxygen as some examples.

\StartSubSection{Aluminum}
Aluminum is plentiful on Mars in the form of the tough oxide alumina (\chemical{Al_2O_3}). This makes up about 4\% of the surface material. To extract just the aluminum from the alumina is energy intensive and requires about 20 kWh per kg (e.g. very endothermic). If a base had a 100 kW nuclear reactor it could only provide enough energy to produce about 123 kg of aluminum per day.

Aluminum is the second most important metal on Earth. It is useful for wiring and flight system components, being that it is a good electrical conductor. On the other hand steel could be used instead since it is easier to make on Mars and weighs about the same as aluminum would back on Earth.

\StartSubSection{Bricks}
Baked bricks are easy to make from fine dust or regolith. You start by wetting it, mould under mild compression, then bake the result at a minimum of \math{300^{\circ}}C. The baking could be done very efficiently by using cheap solar reflectors. By adding shredded parachute fibre the brick's strength can be increased.

Frozen bricks are also simple to make. The process is similar to baked bricks but we skip the baking and just freeze the wet regolith in a mould into hard permafrost blocks. The bricks can be sealed together using water as though it were a mortar or cement. This process works fine, but only if the building does not have to be heated -- lest it turn to mud. This could be useful for storage sheds for instance.

A word of caution on bricks. They have minimal tension strength. You can only compress them safely. Buildings that are built out of them need to be reinforced on their outside with two and a half metres of regolith. Otherwise the building's internal atmospheric pressure will blow itself apart.

\StartSubSection{Carbon Dioxide}
Most of the Martian atmosphere is \chemical{CO_2} at 95\%. It has plenty of uses. It can be used as a solvent to extract useful materials from rocks. These include magnesium, hydrogen rocket fuel, breathable oxygen, and water. It can also be used in a number of different chemical reactions for producing a variety of new resources.

\StartSubSection{Carbon Monoxide Fuel}
Carbon monoxide (\chemical{CO}) is a poor but usable fuel. It is poisonous but useful in emergencies.

\StartSubSection{Ceramics}
Clay minerals are all over the planet. When processed into ceramics they have many uses, such as in \index{pottery}pottery. They are easy to make and do not require much work for the first settlers.

\StartSubSection{Copper}
Copper's presence on Mars is about the same as on Earth at about 50 ppm in the regolith. It is more efficient if extracted from high grade ore concentrate. This takes the form of copper sulphide (\chemical{Cu_2S} or \chemical{CuS}). You can find this ore at the base of any of the many lava flows found on Mars. The copper can then be extracted from the ore in the same way as on Earth through smelting or leaching.

\StartSubSection{Deuterium}
All chemical elements have one or more variants of themselves that occur less frequently in nature, even though only their most common form is denoted on the periodic table. When elements share the same atomic number but vary only in the number of neutrons they contain, they are called {\it isotopes\index{isotope}}.

Hydrogen is denoted as \chemical{H} and is known to occur naturally as either of two isotopes. The more common form has one proton, one electron, and no neutrons. This form is called {\it protium\index{protium}} and is sometimes denoted \chemical{^1H} when distinguishing from another type of hydrogen isotope. Usually when someone is talking about hydrogen, they are talking about protium. The number one located in the notation means that it has an atomic mass of one. It is one because electrons have nearly zero mass and neutrons and protons a mass of one. This means that different isotopes of the same element will still vary in mass.

The only other known isotope of hydrogen is \index{deuterium}{\it deuterium}. It is denoted \chemical{^2H}. It has a single neutron that protium does not, hence the atomic mass of two in its notation. When two deuterium atoms are combined with an oxygen atom, the resultant is deuterium oxide \chemical{^2H_2O}, or simply {\it heavy water\index{heavy water}}. The name derives from it appearing exactly the same as normal water, only heavier.

On Earth deuterium is very rare and accounts for only 0.0156\% of all the hydrogen that naturally occurs in our planet's oceans. On Mars it is more plentiful by a factor of five. About 166 in every million hydrogen atoms are deuterium on Earth, but on Mars that number is about 833. 

The reverse--water--gas--shift (RWGS) reactor described in \in{section}[Reverse-Water-Gas-Shift Reactor] is heavily relied upon by Arcadians. It produces water from the Martian atmosphere, but inadvertently produces deuterium as a byproduct. If the water emerging from the reactor is split through electrolysis, hydrogen fuel and oxygen will result. For every 6,000 kg of \chemical{H_2O} the reactor produces, 1 kg of that will contain deuterium. Since heavy water takes slightly longer to split with electrolysis than regular water, if you keep cycling the water that has not yet been split back through electrolysis, eventually the concentration of heavy water increases until it is almost totally pure. This works because the normal water gasses off faster.

By the late 20th century deuterium was valued on Earth at about 70\% that of gold per kilogram. It has wide ranging applications from nuclear fusion and fission reactors, nuclear weapons, medical, biochemical, environmental, and more. If contained as heavy water, it can be used in non--uranium enriched fission reactors that poor Terran countries could benefit from.

Since Mars is only a third the mass of Earth, only a fourth the energy is needed to exceed the escape velocity to get things off the planet. This makes deuterium railgun exports from Mars back to Earth feasible where it is a highly valued commodity. Since all water, oxygen, and many other Arcadian industrial processes relying on the RWGS reactor produce deuterium as a byproduct, the settlement theoretically would always have something valuable that is sought after in Terran markets as a staple export.

\StartSubSection{Dimethyl Ether Fuel}
For higher peak energy uses like for bulldozers and drilling rigs, dimethyl ether \chemical{(CH_3)_2O} (DME) is a good fuel. It can be manufactured {\it in situ}. It is a clean burning fuel that burns in diesel engines, but actually performs better than diesel. It will not freeze at Martian temperatures. It is relatively non--toxic, but highly flammable.

\StartSubSection{Electricity}
See \in{section}[Power] for more on the different ways electrical energy can be generated on Mars. Like other resources electricity too can have a commercial use.

\StartSubSection{Glass}
Glass is useful for many applications from tableware and windows to fibre glass. Silicon dioxide (\chemical{SiO_2}), or silica, is the main ingredients. We know from the {\it Spirit} rover that large deposits of silica in the form of bright white soil are all over Mars and present at about 40\% of the regolith.

The problem is that to make clear glass you need to be free of contaminants. Because iron oxide (\chemical{Fe_2O_3}) makes up about 17\% of the regolith the glass would have a reddish tint. There are two options for dealing with this problem, assuming you want colourless glass. One approach is to use quartz which we do not at present know where to locate on Mars. Another approach is to rely on the hot carbon monoxide (\chemical{CO}) waste gas that the RWGS reactor produces. This waste gas when applied to tainted silica would produce iron, \chemical{CO_2} gas, and \chemical{SiO_2}. The gas is not a problem. The iron can easily be pulled out with just a magnet and saved for other things like making steel.

Even then the tinted iron--red glass is not always a problem. One might want glass that colour for whatever reason, say, for decorative purposes in habitat zones or in civic buildings. Alternatively you could use it to produce fibre glass for some other purposes.

\StartSubSection{Hydrazine Fuel}
Hydrazine (\chemical{N_2H_4}) is a popular monopropellant that sees use frequently in the small thrusters responsible for maintaining attitude control on spacecrafts. It is chemically attractive because it allows for long term storability and simplicity of use since all that it needs for combustion can be contained within a single storage vessel. That is, it does not require separate tanks for both a fuel and oxidizer. This is why it is called a monopropellant.

Hydrazine reserves can be used to supply a spacecraft, aircraft, or ground vehicle with emergency power for short periods of time.

Hydrazine can be dangerous if mishandled because its combustion is extremely exothermic, meaning it releases a very large amount of energy in a very short time.

More work needs to be done to determine how it could be synthesized {\it in situ}, but it should be possible.

\StartSubSection{Iron}
If it was not for all of the iron oxide or hematite (\chemical{Fe_2O_3}) dust all over the planet, Mars would not have the reddish hue that it does. Because it is so common, so is commercial grade iron ore. 

There are two methods for extracting the iron out of the hematite. You could remove the \chemical{O_3} by using the same method used in \in{section}[Glass] for glass by running hot \chemical{CO} gas through it to break it down into iron and \chemical{CO_2}. Another approach is to react it with hydrogen gas to produce iron and water. By using a condenser to capture the waste water it can be electrolysed and cycled back into the reaction. Both methods are almost energy neutral, so you just need enough energy to get the reactions started and very little after that to keep them going.

\StartSubSection{Magnesium}
Magnesium has many uses and is an excellent lightweight alternative to aluminum for many applications on Earth. This includes at least in engine blocks, as an electrical conductor, a rocket fuel, in signal flares, as fire starter, a structural metal, and in aerospace construction. It can also be used for many everyday household items, such as tables, sinks, cups, faucets, and more. A Martian society might call these commodities {\it magware}.

Magnesium is highly flammable when exposed to oxygen. This is why pure magnesium products are never used in Earth's oxygenated atmosphere. On Mars it needs to be alloyed with another material to prevent autoignition if used indoors. Otherwise, as long as it remains outside, this would not happen because there is too little free molecular oxygen in the atmosphere.

Magnesium is likely found in the Martian regolith and could be extracted with a solvent like \chemical{CO_2}.

\StartSubSection{Methane}
Methane (\chemical{CH_4}) has many uses on Mars from manufacturing to rocket fuel. It can be manufactured {\it in situ} by using the Sabatier reactor described in \in{section}[Sabatier Reactor].

\StartSubSection{Methane / Oxygen Fuel}
Methane (\chemical{CH_4}) as a fuel and oxygen as the oxidizer make excellent propellants for use in rockets, reconnaissance drones, and even in internal combustion engines. When used in internal combustion engines it needs to be diluted with atmospheric \chemical{CO_2} or it will burn too hot. The exhaust is just more \chemical{CO_2} with some water in it. The water could be captured with a condenser to recover up to 90\% of it for other uses.

The saved water could be used to feed a Sabatier reactor like the one described in \in{section}[Sabatier Reactor], to produce additional methane fuel. The oxygen needed could be produced by using the RWGS reactor described in \in{section}[Reverse-Water-Gas-Shift Reactor].

\StartSubSection{Methanol}
Methanol (\chemical{CH_3OH}) is necessary for methanol / oxygen fuel cells found in ground vehicles and EVA suits to produce electrical energy. A chemical reaction describing one approach to its synthesis is described in \in{reaction}[reaction:Methanol Synthesis].

\placeformula[reaction:Methanol Synthesis]
\startformula
\inlinechemical{CO,+,2H_2,->,CH_3OH}
\stopformula

If you fill a reactor with copper--on--zinc oxide pellets, heat it to \math{250^{\circ}}C, and provide it with carbon monoxide and hydrogen gas at about 2,000 kPa, you will produce methanol. A full reaction does not happen in a single pass so you have to recycle the unreacted gas back into the reactor until it has all been consumed.

\StartSubSection{Oxygen}
Oxygen is both plentiful and not on Mars, depending on how you look at it. There is no breathable molecular oxygen (\chemical{O_2}) in the atmosphere, but there is plenty in the form of carbon dioxide (\chemical{CO_2}) at 95\% of the atmosphere. This oxygen can be extracted from the atmosphere by using the RWGS reactor described in \in{section}[Reverse-Water-Gas-Shift Reactor]. Another approach is to wet unprocessed regolith containing peroxide so that it evolves \chemical{O_2}.

\StartSubSection{Plastics}
Plastics have all the same uses on Mars as they do on Earth. These range from textiles, housewares, equipment, storage containers, and more.

To produce we react methanol with itself in a \math{400^{\circ}}C reactor filled with cheap gamma--alumina pellets at 100 kPa. This will produce dimethyl ether or DME, \chemical{(CH_3)_2O}, which is a useful intermediate or precursor to other organic compounds. As described in \in{section}[Dimethyl Ether Fuel], DME is also useful as a fuel.

The DME is then fed into a second reactor which contains a common zeolites catalyst (ZSM--5) at a temperature of \math{400^{\circ}}C--\math{450^{\circ}}C at a pressure between 100--200 kPa. The DME will turn into either ethylene (\chemical{C_2H_4}) at low pressure or propylene (\chemical{C_3H_6}) at higher pressure. This ethylene could be used as a welding fuel at this point. If one continues to heat either at a still higher pressure they will polymerize into polyethylene or polypropylene respectively.

For simpler uses we can use polyethylene plastic. For applications requiring a higher quality plastic we can expend more energy and use polypropylene.

\StartSubSection{Portland Cement}
Gypsum (\chemical{CaSO_4·2H_2O}) is plentiful on Mars. It is the mineralogical variant of calcium sulphate. To create lime all you have to do is bake the gypsum. The lime can then be mixed with finely grounded regolith. You are left with a high quality Portland cement that can be used in building construction, roads, or what have you.

\StartSubSection{Precious Metals}
As mentioned in \in{section}[Economics & Commerce], generally the only way of acquiring any kind of high quantity mineral is from high--grade ore. High--grade ore only exists wherever complex hydrological and volcanic processes have occurred. In our solar system this has taken place only on Mars and Earth. This is why the Moon is barren and the Earth is not. But unlike the Earth, Martian deposits of precious metal ore have never been exploited.

These deposits may be near the surface and exist in large quantities. These might include, but are not limited to, cerium (\chemical{Ce}), europium (\chemical{Eu}), gadolinium (\chemical{Gd}), gallium (\chemical{Ga}), germanium (\chemical{Ge}), gold (\chemical{Au}), hafnium (\chemical{Hf}), lanthanum (\chemical{La}), rhenium (\chemical{Re}), rubidium (\chemical{Rb}), samarium (\chemical{Sm}), silver (\chemical{Ag}), and the platinoids, such as palladium (\chemical{Pd}), iridium (\chemical{Ir}), platinum (\chemical{Pt}), and rhodium (\chemical{Rh}). Rhodium is used to back Arcadia's indigenous currency as described in \in{section}[Jenya].

\StartSubSection{Silane}
Silane (\chemical{SiH_4}) burns in \chemical{CO_2} so it makes for an excellent fuel for rocket propulsion or supersonic combustion ramjet engine on Mars as described in \in{section}[Aviation]. But more work needs to be done to determine how best to synthesize it {\it in situ}.

\StartSubSection{Silicon}
Silicon is the third most important metal after steel. Aluminum being the second. It is needed in the manufacture of all electronics, photovoltaic panels for collecting solar energy, and many other uses. Fortunately silica (\chemical{SiO_2}) is present at 40\% of the regolith.

To extract just the silicon out of the silica, take the silica and mix with carbon. The carbon can be made just through the pyrolysis of a Sabatier reactor's methane. Heat this mixture together in an electric furnace for a carbothermal reduction reaction. The resulting byproducts leave us with metallic silicon and carbon monoxide gas. This process is described in \in{reaction}[reaction:Silicon Extraction]. 

\placeformula[reaction:Silicon Extraction]
\startformula
\inlinechemical{SiO_2,+,2C,->,Si,+,2CO}
\stopformula

This works well but it is a very energy intensive process (endothermic). Nevertheless it is still less than aluminum and, where you desire pure metallic silicon, you usually do not need very much of it anyways. 

If the resulting silicon is to be used in microchips and solar panels this cannot be done without further refinement because it still contains hematite impurities. By bathing it in hot \chemical{H_2} gas it will turn into silane (\chemical{SiH_4}). This is a gas at room temperature. You start by piping the silane through a reactor to decompose it under high temperature so as to reduce it to pure \chemical{Si} and \chemical{H_2} gas. 

The resulting silicon can be doped with phosphorus or some other impurity. Doping is done to change an extremely pure semiconductor's electrical properties into whatever kind of semiconductor needed. Alternatively one could have liquefied the silane and stored it for use as either a rocket or ramjet combustion engine fuel if preferred.

\StartSubSection{Steel}
Steel is the most important metal on Earth. It is the main material we can use for high strength structures on Mars.

As you saw in \in{section}[Iron], iron is the most accessible industrial metal present on the planet. When it is alloyed with some other element, usually carbon, you have steel. The type of material the iron is alloyed with and the ratios used determine the type of steel. The four most common types of materials to alloy with are all common on Mars. These are carbon, manganese, phosphorus, and silicon.

\StartSubSection{Water}
Without water there would be no human life. Water is a vital resource but it will take more work to get than on Earth. The best source is found in \index{artesian wells, underground}underground artesian wells. The next best is water ice which is almost pure. The next best source is in the \index{permafrost}underground permafrost. For extraction by means of artesian aquifer, see \in{section}[Geothermal Power].

Extraction from the permafrost is not very difficult. You could use a transparent, lightweight, tensile fabric structure. This greenhouse--tent will warm the first few centimetres of regolith above zero which is all that is needed to make the water degas. This could be very useful in survival situations.

Another approach to the permafrost is to use a photovoltaic oven if available. This requires a lot of energy. The water that melts out of it still needs to undergo distillation because of its salinity.

\StartSubSection{Wood}
It may come as a surprise to many, but wood is still very useful on another planet like Mars. Bamboo for instance could be grown in a greenhouse. It can be cut into sheets and planks, used in laminating floors, paper, landscaping, and many other uses. It can grow up to 100 cm or more per day. Orchards from a colony's greenhouse could provide wood for furniture in addition to producing fruit. 

The cellulose waste has other uses too. This material could be used in the production of ethylene plastics. This greatly increases the number of different types of plastics that can be produced.

