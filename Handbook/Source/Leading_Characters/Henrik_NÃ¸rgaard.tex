% This is part of the Avaneya Project Crew Handbook.
% Copyright (C) 2010, 2011, 2012
%   Kshatra Corp.
% See the file License for copying conditions.

% Henrik Nørgaard section...
\StartSection{Henrik Nørgaard}

Henrik was born 49 Aries, 32 B.R., in Copenhagen, European Union. He attended Lund University\index{Lund University} where he completed his undergraduate education in cellular microbiology. Following that, he went on to complete a Masters in Molecular Biology and Microbiology. After completing his doctoral studies, he pursued research opportunities with the Max Planck Institute for Medical Research in Heidelberg\index{Max Planck Institute+for Medical Research}, European Union. 
\placetable[force,nonumber,right]{}
{
    \SetupCharacterTable

    \bTABLEbody

        \bTR 
            \bTD[nc=2] \midaligned{\rotate[rotation=42]{\color[red]{\bft Todo: Character's image goes here.}}} \eTD 
        \eTR

        \bTR 
            \bTD[nc=2] Henrik Nørgaard \eTD 
        \eTR

        \bTR
            \bTC Born \eTC
            \bTC 49 Aries, 32 B.R. \eTC
        \eTR

        \bTR
            \bTC Birthplace \eTC
            \bTC Copenhagen, European Union \eTC
        \eTR
        
        \bTR
            \bTC Gender \eTC
            \bTC Male \eTC
        \eTR
            
        \bTR
            \bTC Nationality \eTC
            \bTC European Union \eTC
        \eTR
        
        \bTR
            \bTC Ethnicity \eTC
            \bTC Danish \eTC
        \eTR
        
        \bTR
          \bTC Hair \eTC
          \bTC Light Blond \eTC
        \eTR
        
        \bTR
            \bTC Eyes \eTC
            \bTC Blue \eTC
        \eTR

        \bTR
            \bTC Age (Selection) \eTC
            \bTC 21 MYrs / 40 Yrs \eTC
        \eTR

        \bTR
            \bTC Age (Year Zero) \eTC
            \bTC 32 MYrs / 62 Yrs \eTC
        \eTR

        \bTR
            \bTC Education \eTC
            \bTC 
                \startitemize[4]
                \startpacked
                \item Cellular Microbiology
                \item Molecular Biology &\\Microbiology
                \item Orthomolecular Medicine
                \stoppacked
                \stopitemize
            \eTC
        \eTR
        
        \bTR
            \bTC Occupation \eTC
            \bTC 
                \startitemize[4]
                \startpacked
                \item Naturopathic Physician
                \item Scientist
                \stoppacked
                \stopitemize
            \eTC
        \eTR
        
        \bTR
            \bTC Mission Titles \eTC
            \bTC 
                \startitemize[4]
                \startpacked
                \item Chief Field Science Officer
                \item Chief Medical Officer
                \item Experimental Team Lead
                \stoppacked
                \stopitemize
            \eTC
        \eTR
    \eTABLEbody

\eTABLE
}

While in Heidelberg, Henrik's principle area of investigation was in the field of orthomolecular medicine\index{Orthomolecular medicine}.\footnote{Orthomolecular medicine is a branch of medicine concerning itself with the relation of nutrition and illness, especially using the former in the treatment of the latter.} During his time with the institute, Henrik had more than 80 peer reviewed papers published on topics ranging from clinical trials involving the use of traditional herbal remedies, the biochemistry of fermented foods, and the efficient recycling of biomass.

After several years with the institute, Henrik went on to acquire his status as a naturopathic physician at the University of Copenhagen while lecturing part time at the university. As a lecturer, he was noted for his satirical sense of humour, brilliance, and an extensive set of mushrooms and herbs from virtually every corner of the Earth in his office.

Following his accreditation as a naturopath, Henrik went on to open a medical clinic in Copenhagen. Shortly after his acceptance of a UNSA offer to venture to Mars as the Avaneya Initiative's Chief Medical Officer and Chief Field Science Officer (Experimental Team Lead), Henrik passed over the care of his clinic under the supervision of a close colleague.

