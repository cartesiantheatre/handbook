% This is part of the Avaneya Project Crew Handbook.
% Copyright (C) 2010-2013 Cartesian Theatre <kip@thevertigo.com>.
% See the file Copying for details on copying conditions.

% Militia & Combat section...
\StartSection{Militia & Combat}
The option of raising a militia in protecting the user's city forms an aspect of Avaneya's gameplay. Keeping with a sandbox style of play, the use of militia units will be a dynamic process involving varying degrees of player input at the players own discretion. In this section various types of militia units are discussed as well as their creation, movement, and available behaviours. Aspects of these gameplay elements are also discussed in \in{section}[Interface & Interaction]}.



\StartSubSection{Militia Training} 
If the history of strategy games has shown anything, it is that people need enough diversity to keep a game interesting, but having too many choices of units can detract from the experience. From another point of view, the more units and options that are put into a game, the more it needs to be tested since there is greater room for error and abuse. It could be argued that what makes a lot of massively expansive games a disappointment is that they are untested, making some units an obvious choice while creating dead end paths which inhibit experimentation. Expansive games need to be well tested, and while they can offer experimentation, a fine line between creativity and aimlessness needs to be tread carefully. The idea of players being able to create city management options, defending a city, and encouraging research and innovation, needs to be constrained initially as they explore their options. By this it is meant that too many options too soon leads to being overwhelmed.

This is most true for the training and use of military units which may change the tide of a games balance, especially on Solnet. Militia units can be trained at the user's prospective military academy, producing personnel as affected by the underlying social conditions and values. This might include effecting how frequently they can be trained and the degree to which this can occur. It may be very hard for cities with a farming or labour workforce to produce officers with highly technical knowledge, whereas they may be better or naturally adept at operating large military machinery. 

Machinery on the other hand can be produced by a city's population or imported from another city. Production at a local level requires the degree and type of knowledge appropriate for the particular mechanical unit that is wanted. This will require a {\it workforce} capable of both the technical knowledge to implement the work and the skills to conduct the work. Local production also requires the appropriate machinery/industry or construction equipment to create the desired units. Importing miltary equipment only requires the player knowing another city who has the ability to produce desried equipment that is willing to sell to the player. In this instance other factors come into play such as transporting equipment, payments, costs, and diplomacy.   

The use of this machinary also requires a degree of technical knowledge from the population, namely militia units specially trained to fly aircraft, drive tanks, fire artillery, or operate hi-tech weaponry. This is another factor in raising a militia as conditions to have a number of militia units trained in using specialized equipment need to be met. It is therefore more cotly to operate a diversified militia with hi-tech weaponry and equipment than it is to operate an infantry based militia. While a basic militia may only require a military academy or similar, a more complex militia may require: military academy; repair factories; officer training; aircraft facilities; and pilot/operator training. Also, as diversity increases the number of each of these requirements may multiply. For instance a militia with jets and tanks may require separate facilities to train pilots and tank drivers. Further, advanced training may require even more facilities, as will using more technologicially advanced equipment and tactics. A closing point for this is that militia/military units and therefore combat is extremely costly.  

%Stealing equipment or ambushing on transport. in similar vain to resource transport or
%  lifecycle, can this and other equipment be visible on transport? Can players steal this from 
%  computer players or other players? Transport of vehicles such as tanks etc requires other
%  machinery and complex transport networks.  

% Adam: Will this machinery be useable on its own or require trained militia? 
% Kip: Requires personnel with appropriate training. This is why infantry is so 
%  versatile, because it is cheap to raise and doesn't require the depth of 
%  technical knowledge that crews on naval destroyers or in bombers require.


\StartSubSection{Milita Units}
%Discuss various unit types. Below list to be edited and described.

%The three regiments Arda has raised (3 Capricorn, 0 A.R., p.93) and
%figure 9.6 (p.180) should be the basis for now of the type of equipment
%a typical Arcadian city with a militia would be furnished with. Here is
%a list of some of the units:


Armoured Dragoons Regiment:

- Armoury (an actual regiment's building)

- Armoured reconnaissance rover

- Rover tanks in light, medium, and heavy form factors.

Artillery Regiment:

- Self propelled artillery (powerful, but slow and not well armoured)

- Artillery (light, medium, and heavy)

Combat Support Engineers:
- Armoury (an actual regiment's building)

- Logistical support rover (sapper crew repairs equipment in the field)

- Bridging vehicle (e.g. like the M104 Wolverine)

- Reconaissance UAV

- Combat UAV

Training:

- Drill hall (needed for everyone)

- Academy (needed for officers and NCOs)

Mechanized Infantry:

- Armoury (an actual regiment's building)

- Infantry formations in the size of fire team, section, platoon,
company, and battalion.

- No special forces at this time.

- APC rovers for troop transport.

- Command vehicle rover.

- Mobile infirmary.


\StartSubSection{Attacking and Defending}
When a unit deemed an enemy is within line of sight and range of a unit they will engage provided they have the ability. They will also alert other units in there vicinity and through radio where applicable causing other units to engage in group tactics. Units can also be told to attack units by selecting the unit then right clicking the enemy unit, this can also be done by selecting the unit, left clicking the attack function on the user interface then left clicking the enemy unit or unit you want attacked. Depending on the units current stance the manner in which they engage the enemy will vary. As described above, various movement styles cause units to act differently towards the enemy. Also relevant is defensive or combat stances. Units can be put into aggressive, defensive, or general combat stances.

Units with certain weapons or abilities can also be given unique commands. Depending on situations they may utilize these abilities automatically but players can choose to use these at a time and in a manner that they see fit. 

Examples of unique commands may be:
\startitemize[16]
 \item[suppressive fire]
 \item[draw fire]
 \item[ignore enemy fire]
 \item[grenade or various explosives]
 \item[charge]
 \item[hide]
 \item[take cover]
 \item[destroy cover]
 \item[assist]
 \item[mantle]
 \item[dismantle]
 \item[force reload]
 \item[regroup]
 \item[repair]
 \item[medical attention]
 \item[call fire support]
\stopitemize



\StartSubSection{Attacking and Defending Cities}
While units movement in general will be the same, moving through hostile terrain can be a time consuming process. This is multiplied by moving through enemy cities or cities held by another player. City streets hold lots of hiding places and spots for cover, but also hide ambushes, militant civilians, and multiple story buildings. Fast movement through city spaces leaves flanks open and areas unturned. 

On Solnet, due to its MMO gameplay, this is very relevant. While players cities can be attacked when they are offline they will not be able to be completely destroyed/captured due to the nature of street by street warfare. The aim is to make attacking and defending cities a gradual and expensive process (as it is) so that a player can log onto Solnet once per week and still defend his city or continuing the attack on someone else. 

With this in mind infantry movement through city streets will take a gradual, street by street clearance approach unless being ordered otherwise. The aim is to make attacking cities a long, drawn out affair with little strategic room for error by the attacker, or by defender reclaiming lost land/streets/buildings. Some methods in which this can be accomplished may be:
 
Multiple rooms and stories in buildings which require clearing and where fighting can get bogged down.
 Civilians left behind or angry which will fight back or cause ambushes.
 Media and civil unrest at attackers homeland due to civilian casualties.
 The need to import resource for attacking units.
 Ease of defender sending in resource and units due to proximity.
 Large city scape extrapolates issues. 
 Day night cycle effecting combat conditions and strategic styles.
 Line of sight conditions making long city streets vulnerable, requiring slow movement.
 General expseniveness straining alliances and creating vulnerabilities in attackers cities. 




\StartSubSection{Artillery}
%This section will deal with mechanics of explosive warfare and its balance on Solnet.

\StartSubSection{Unit Noise Exposure}
When units move, shoot, and generally perform actions they make noise, the degree of this noise may alert other units to their presence. A method of display noise is a sound wave indicator at the locations vicinity and on the mini--map. Focusing on the location will play the sound for players attention.

