% Communication section...
\startsection{Communication}
The team uses two primary means of intercommunicating. The first is through the crew mailing list, and the second is over IRC. The Code of Conduct applies to both. @xref{Avaneya Code of Conduct}.

\startsubsection{Mailing Lists}
Mailing lists have been around for literally decades. They lack the cosmetics and ease of use of a forum, but until our website is ready, it will suffice for the time being.

Avaneya has three mailing lists. The first is a unidirectional announcement mailing list. It is low volume and only intended to communicate from the project to the general public about major project press releases. Anyone can subscribe to it.
@sp 1

Subscribe:
@sp 0
@smallexample
@url{https://www.avaneya.com/lists/?p=subscribe&id=1} 
@end smallexample
@sp 1

Browse the archives:
@sp 0
@smallexample
@url{https://www.avaneya.com/news/announcements/}
@end smallexample
@sp 1

The second @email{avaneya@@lists.avaneya.com} is for anyone to discuss anything related to the project. It is open to everyone.
@sp 1

Subscribe:
@sp 0
@smallexample
@url{http://lists.avaneya.com/listinfo.cgi/avaneya-avaneya.com} 
@end smallexample
@sp 1

Browse the archives:
@sp 0
@smallexample
@url{http://lists.avaneya.com/pipermail/avaneya-avaneya.com/}
@end smallexample
@sp 1

The third @email{avaneya-private@@lists.avaneya.com} is open only to project crew. Topics range from game storyline spoilers, unreleased conceptual art, security vulnerabilities, and so on. If you are a volunteer, you should have been sent a subscription invitation.
@sp 1

Subscribe:
@sp 0
@smallexample
@url{http://lists.avaneya.com/listinfo.cgi/avaneya-private-avaneya.com} 
@end smallexample
@sp 1

Browse the archives:
@sp 0
@smallexample
@url{http://lists.avaneya.com/private.cgi/avaneya-private-avaneya.com/}
@end smallexample
@sp 1

When using either of the latter two mailing lists, you may find the following tips useful.
@sp 1

\itemize @bullet
\item
Do not ever send HTML email. Some peoples' mail clients cannot render it. And even if they can, there is no guarantee it will come out the same. People with visual disabilities may not have their speech synthesizers work properly, since HTML email is much harder to parse. Others may have to pay for additional bandwidth to retrieve your email, since HTML email is larger in size than plain text.@footnote{For more information, consider reading @url{http://www1.american.edu/cas/econ/htmlmail.htm}}
@sp 1

\item
When you reply to a post, remember to reply to the list and not just the original sender privately. Unless you had intended to, the mailing list is setup so that everyone who subscribes to it may benefit from productive communication on it. Sometimes this may not happen until years later when a new subscriber searches through old archives to find a solution to a problem they were having that was solved long ago.
@sp 1

\item
When replying to a post, if you have your subscription configured to use batch digest mode,@footnote{When batch digest mode is enabled, the server will "batch" together emails into groups and then send it to you as a single compilation to cut down on the amount of email you receive.} you do not need to copy the whole digest. Just quote the minimum needed for context.
@sp 1

\item
Check the subject heading of your reply to a message posted on the list to make sure it still reflects the original post. Some mail readers, if you have batch digest mode enabled in your subscription, will change the heading to reflect the batch digest's subject heading, instead of the specific message within it you are replying to.
@sp 1

\item
When you reply, remember to reply at the bottom and not at the top of the message. Top posting is generally not encouraged because it makes preservation of chronological order difficult to follow for readers.@footnote{This explains why top posting is a bad idea: @url{https://secure.wikimedia.org/wikipedia/en/wiki/Top_posting#Top-posting}.}
@sp 1

@end itemize

\startsubsection{Internet Relay Chat (IRC)}

IRC is among the oldest forms of realtime chat over the internet. Avaneya has a channel (@strong{#avaneya}) on the Freenode server (@strong{irc.freenode.net}). You can use whatever client you like, but it is recommended you use one that supports SSL.

Make sure you register your chosen nick name with the @i{nickserv} on Freenode. This ensures you are consistently identifiable to others in the chat room.

Whenever you would like to send someone a message publicly in the channel, you should precede your message with their nick name. This is because many people have their IRC clients configured to alert them audibly when that happens, as opposed to every time anyone says anything in the channel. Usually you only need to type the first few letters of their nick name and hit tab to have your client complete it.


