% This is part of the Avaneya Project Crew Handbook.
% Copyright (C) 2010, 2011
%   Kshatra Corp.
% See the file License for copying conditions.

% Timeline chapter...
\StartChapter{Timeline}
\startlines
The world Avaneya takes place in is highly self-referential. A fictional timeline is positioned relative to a {\it year zero\index{year zero}}, with years preceding suffixed {\it B.R.\index{B.R.}}, before the republic; while those following are suffixed {\it A.R.\index{A.R.}}, after the republic. This is the method adopted by \goto{Arda}[Arda Baştürk] and his settlers around the time they declare their independence.

Note that a year is defined as a Martian year (MYr) and is not the same as a Terran year. You may want to refer to \in{section}[Time] for a detailed explanation of how Arcadians keep track of time on Mars.

Year zero's placement relative to the Gregorian calendar\index{Gregorian calendar} is deliberately never explicitly given. But inferentially, one can reason it is proximate enough for a contemporary audience to feel relevant, yet distant enough to afford some creative license. This is because, although Avaneya takes place in the future, it deals with many contemporary issues.
\crlf

% Provides hint to astute reader of rough order of magnitude of time between now and story...
\StartTimelineDate{(date here) B.R.}
The last of the {\it International Space Station} crew return to Earth with the station subsequently decommissioned, deorbited, and incinerated during atmospheric entry.
\StopTimelineDate

\StartTimelineDate{B.R.}
NASA's {\it Zubrin} spacecraft lifts off atop a 140 tonne capable Enigma-D heavy-lift booster from its launchpad at Cape Canaveral, North American Union. The manned module separates from the booster in low Earth orbit. It carries a crew of four with two flight engineers and two mechanics with combined specialities in geology and biogeochemistry. This is the first attempted piloted mission to Mars, implementing the Mars Direct mission architecture as a direct launch. Its objective is to determine primarily if the Red Planet ever harboured life, as well as to survey and explore.
\StopTimelineDate

% This should be 180 days after departing Earth for a fast conjunction class manoeuvre...
\StartTimelineDate{B.R.}
{\it Zubrin} performs a deceleration manoeuvre by aerobreaking through the Martian atmosphere and entering orbit to prepare the lander stage which carries a 25 tonnes payload of a habitat module and Earth return vehicle (ERV).
\StopTimelineDate

\StartTimelineDate{B.R.}
Mars Direct crew report ERV's chemical processing plant successfully brought online, producing oxygen, liquid water, and the methane/oxygen return propellant {\it in situ}\footnote{From the Latin, in position. It is used in space-related contexts to denote a situation where something is not done until in the actual field, not prior.}.
\StopTimelineDate

% Constraint: Should be 910 days total mission time since departure, with 550 days of Mars stay time...
\StartTimelineDate{B.R.}
{\it Zubrin ERV} completes its mission objectives, including extensive mineralogical data and an unsuccessful search for life. It lands at the Kennedy Space Center, North American Union.
\StopTimelineDate

\StartTimelineDate{B.R.}
The \goto{bancor}[Bancor] is adopted as the new global currency and implemented via the respective central banks of the North American Union, European Union, African Union, and Asian Union.
\StopTimelineDate

% Don't use provisional designation for asteroid since includes fixed Gregorian date in name...
\StartTimelineDate{B.R.}
Professor Ramraj discovers a near-Earth, C-type Amor II asteroid, with a size comparable to Phobos from the Lincoln Near-Earth Asteroid Research laboratory, Socorro, North American Union. The discovery is rapidly assigned the designation {\it 52048 Varuna} by the {\it International Astronomical Union}. His findings are chronicled in the {\it Minor Planet Circular} where he calculates a near collision trajectory with Mars.
\StopTimelineDate

\StartTimelineDate{B.R.}
Wikileaks publishes a leaked list of 163 purported attendees of the annual Bilderberg conference held three days prior at the Hotel de Crillon, Paris, European Union. Dr. Samuel Lieberman, director of the National Institute of Standards and Technology, and Adriaan Janssen, the Secretary-General of the United Nations, are among those listed.
\StopTimelineDate

\StartTimelineDate{B.R.}
The United Nations holds the Second International Mars Summit in Geneva, North American Union, with the purpose of discussing the options on the table for the second manned mission to Mars. The attendees number in the thousands, representing the states of the African Union, North American Union, European Union, and the Asian Union. In addition, thousands of scientists, engineers, and philosophers attend as either independent presenters or as part of any number of associations ranging from former Case For Mars conference members to The American Astronautical Society, NASA, the RANND Corporation, and others.

The minimalistic Mars Direct\footnote{See \about[Mars Direct] for more information on Mars Direct.} approach of travelling light, living off of the land, and using indigenous materials to produce the fuel necessary for the journey home is again a central theme, but with more emphasis on the nature of the duration on Mars and material infrastructure required.
\StopTimelineDate

\StartTimelineDate{B.R.}
Leonard Kissinger, President of the {\it Council of Foreign Relations}, holds a symposium at the Harold I. Pratt House in New York City, North American Union.
\StopTimelineDate

\StartTimelineDate{B.R.}
The {\it Trilateral Commission} holds its biannual meeting several months in advance, Tokyo, Asian Union.
\StopTimelineDate

\StartTimelineDate{B.R.}
The International Mars Summit concludes and publishes its findings in the {\it Report on the International Proceedings of the Mars Summit}. The report contains the following:

\startTimelineGeneralDocument
Zero capability missions have no value. The Mars Direct mission, having already demonstrated successfully the feasibility of sending humans to Mars, the bar now can and ought to be raised. The attendees concluded nearly unanimously that it is not prudent to return merely for the sake of demonstrating that a safe return is possible, that having already been established, but to create a permanent settlement on Mars. 

A suitable upper Amazonian geologically classified region where the presence of underground liquid water reservoirs has already been observed is a strong candidate location. 

Such a mission architecture would primarily provide unique, cutting edge, off-world, research facilities to address major global issues afflicting all nations. Secondarily, it would also serve to bootstrap the first non-terrestrial expansion of human civilization. Both of these goals are in the service of all mankind.
\stopTimelineGeneralDocument

The report's abstract outline for a new mission architecture is as follows:

\startitemize[n]
\item Deploy a space station in low Earth orbit to act as a construction berth for trans-Martian spacecraft supplied by efficient single-stage-to-orbit (SSTO) vehicles.

\item Launch a spacecraft constructed within the space station that would transport only the minimum necessary to bootstrap a settlement, such as cryogenic \chemical{H_2} needed to initiate the \goto{Sabatier}[Sabatier Reactor] and \goto{RWGS}[Reverse Water Gas Shift Reactor] reactors to produce fuel and oxygen, {\it in situ}, respectively, a nuclear reactor, excavation equipment, and additional equipment.

\item Use a chemical propulsion system of \chemical{H_2/O_2} to get to Mars, then use indigenous resources to supply the return propellant of \chemical{CH_4/O_2}. This requires the mission payload to include propellant for the outbound trans-Martian injection stage only.

\item The crew would deploy construction equipment and machinery to process indigenous materials to create the {\it Arcadian Settlement}, a permanent outpost located in the Arcadia Planitia region and falling within a United Nations extraterritorial mandate.

\item The spacecraft would return to Earth unmanned, but fully fuelled with {\it in situ} produced methane/oxygen propellant. This would allow subsequent shipments to depart Earth bearing spare parts, machinery, additional excavation equipment, and whatever else the outpost could not have determined necessary {\it a priori}.

\item Personnel ferries departing Earth at regular intervals would continue to augment the settlement capability through providing additional manpower. This would increase the number of available habitats, greenhouses, and other vital infrastructure necessary to support the research being conducted.
\stopitemize

A preliminary draft of the {\it Mars Treaty} is produced using a revised {\it Outer Space Treaty} of 1967 A.D. as an adaptable boilerplate. The treaty provides the contractual basis for all signatories, outlining the road map, responsibilities, and party resource allocations necessary to bring about the terraformation and first manned mission to Mars operating under a United Nations mandate. The treaty also provides an agreement governing the activities of states on Mars and its two moons, Phobos and Deimos.

The terraformation strategy is to be initiated by intercepting {\it 52048 Varuna} with a nuclear warhead and deflecting it into a Martian collision trajectory. Being a C-type asteroid, it contains high levels of hydrated (water-containing) minerals which, when released into the atmosphere as aerosols, have been calculated to greatly increase the total available cloud condensation nuclei.

Data was presented at the summit predicting that the direct effect of the introduction of these new, dark, carbonaceous, aerosols would be a decrease in \goto{albedo}[Albedo] since they absorb more solar radiation. This would cause a cascading series of reactions, resulting in increases in the mean surface temperature and atmospheric pressure.

The deflection strategy, if executed as calculated, would have the asteroid impact directly over the Martian south pole. The south pole contains a large sheet of buried \chemical{CO_2} (dry ice) approximately 8 metres thick which, when sublimated, would theoretically act as a powerful greenhouse gas, thickening the atmosphere, along with the underlying water ice beneath it.

While the legalities of the project are defined in the {\it Mars Treaty}, the international collaborative project it implicates is known as the {\it Avaneya Initiative}.
\StopTimelineDate

\StartTimelineDate{B.R.}
{\it United Nations General Assembly Resolution 5571} is adopted, ratifying the Mars Treaty. The resolution augments {\it Chapter III} of the {\it United Nations Charter} to beget its seventh principle organ, the {\it Space Agency} (UNSA). Its mandate reifies the treaty, overseeing the implementation of the Avaneya Initiative.

The United Nations Office for Outer Space Affairs\index{United Nations Office for Outer Space Affairs} (UNOOSA\index{UNOOSA}) is dissolved. The former responsibilities that it held since its formation in 1958 A.D. are amalgamated under the new UNSA.

The resolution requires members to accede state space agencies, such as the North American Union Aeronautical and Space Agency (NAUSA), European Space Agency (ESA), Asian Aerospace Exploration Agency (AAXA), and others under a centralized UNSA administration. The public rationale given that space exploration and settlement ought to be an endeavour for all mankind, best realized through the cooperative aggregation of resources.

Lieberman is nominated to the Office of the President of the UNSA, resigning his position as director for the National Institute of Standards and Technology.
\StopTimelineDate

\StartTimelineDate{B.R.}
The UNSA's Office of the Avaneya Initiative disseminates more than ten-thousand requests for quotations.
\StopTimelineDate

\StartTimelineDate{B.R.}
UNSA's Office of the Avaneya Initiative announces successful bids. The {\it Iterum Shipyard} contract is jointly awarded to Lockheed Martin-Boeing and Mercedes-Pratt & Whitney to robotically build a station remotely in low Earth orbit at a cost of β2.2 billion. The station is to be used as a general purpose interplanetary space vehicle construction platform.

% Lockheed Martin's failed X-33 design would be a good starting point for a model...
% TODO: Get the Sanskrit devangari characters to display some how...
Lockheed Martin-Boeing is awarded a contract to provide three, leased, single-stage-to-orbit (SSTO), manned, re-usable, {\it VentureStar VII} class suborbital spaceplanes. These vehicles, known as {\it Aravans},\footnote{Sanskrit अरवन् for a steed or horse.} provide UNSA with heavy lift workhorses for ferrying personnel and materials to the {\it Iterum Shipyard}. They require no external solid booster rockets or external fuel tanks and each operate at less than a fifth of the cost of the retired Space Shuttles.

Lockheed Martin-Boeing is also awarded β21 billion worth of contracts to incrementally provide four direct-launch to Mars {\it Tarikin}\footnote{Sanskrit तरीकिन for a ferry-man.} ferries capable of carrying payloads of 400 personnel each, plus provisions. The {\it Tarikins} will shuttle settlers to Mars using a chemical \chemical{H_2/O_2} propulsion system for the outbound trajectory, but then provided in Martian orbit with the return \chemical{CH_4/O_2} propellant manufactured {\it in situ}. The {\it Tarkins} will depart Earth every 780 day minimum-energy launch window, taking 180 days to arrive at Mars in constant rotation with each other.

Mitsubishi-Saab's is awarded a contract to robotically construct {\it Avaneya} remotely within the {\it Iterum Shipyard} orbital berth at a cost of β7.8 billion. {\it Avaneya} will carry the bootstrap crew and large amounts of construction equipment.

Volvo-John Deere is awarded a contract to provide chemical propulsion systems for {\it Avaneya}, {\it Aravans}, and {\it Tarikins} at a cost of β920 million. It is used for the outbound trans-Martian trajectory, with the return methane/oxygen (\chemical{CH_4/O_2}) propellant manufactured {\it in situ}.

{\it Carlyle Holdings'} {\it Bronfman-Murdoch Aerospace} is awarded a contract to provide four {\it Mars Positioning System} satellites and the {\it Mars Enhanced Telecommunications Orbiter} carrying an optical relay and other instrumentation at a cost of β750 million. The five satellites are carried as part of the {\it Avaneya} mission payload
\StopTimelineDate

% Huelva pronounced \ˈwel-vä, ˈhwel-\
\StartTimelineDate{B.R.}
Construction of the last of UNSA personnel training facilities in Antarctica and Huelva, European Union, are completed.
\StopTimelineDate

\StartTimelineDate{B.R.}
The first crew candidates arrive and report to UNSA training facilities. Courses begin the following day.
\StopTimelineDate

\StartTimelineDate{B.R.}
The {\it Iterum Shipyard} is completed remotely in low Earth orbit. The station orbits Earth at an altitude of 340 km, travelling at a speed of 27,400 km/h, and taking one and a half hours to complete one orbital revolution about the Earth.
\StopTimelineDate

\StartTimelineDate{B.R.}
The crew selection is completed with a broad range of personnel including flight engineers, artificial intelligence specialists with knowledge engineers among them, astrogeophysicists, cyberneticists, chemical and civil engineers, cold weather construction experts, mechanics, biogeochemists, geologists, areobotanists, and a xenobiologist, and many others. 

\goto{Arda Baştürk}[Arda Baştürk] is named Mission Commander. 

Leonard Kissinger will accompany the crew as the newly appointed United Nations Deputy Secretary-General.
\StopTimelineDate

\StartTimelineDate{B.R.}
{\it Mars Science Laboratory Curiosity XI}, an unmanned autonomous aerial vehicle, surveys potential landing sites in Arcadia Planitia. The integrated onboard artificial intelligence is instructed to evaluate sites based on average available sunlight, underground water ice revealed by airborne shallow radar, mineralogy, surface geography, and other simultaneously weighted factors.
\StopTimelineDate

\StartTimelineDate{B.R.}
The {\it Internet Assigned Numbers Authority} allocates {\tt A001:CA7:3134::/48} IPv6 address block for general Martian use. 
\StopTimelineDate

\StartTimelineDate{B.R.}
{\it Aravan III} departs UNSA's Cape Canaveral launchpad en-route to the {\it Iterum Shipyard}, low Earth orbit, with a mission payload of materials for {\it Avaneya}. This includes the disassembled aeroshield, whole food, potable water, lab equipment, and its Haliburton nuclear reactor.
\StopTimelineDate

\StartTimelineDate{B.R.}
{\it Avaneya} construction is completed remotely in orbit at the {\it Iterum Shipyard}, low Earth orbit.
\StopTimelineDate

\StartTimelineDate{B.R.}
Launched from the Kennedy Space Center, North American Union, the {\it Aravan II} transporting all {\it Avaneya} crew docks successfully with the {\it Iterum Shipyard}, low Earth orbit.
\StopTimelineDate

\StartTimelineDate{B.R.}
{\it Avaneya} completes all system checks and disembarks the {\it Iterum Shipyard} berth. It commences its six month journey with a delta-v from low Earth orbit injecting itself into a trans-Martian conjunction class orbital manoeuvre.
\StopTimelineDate

\StartTimelineDate{B.R.}
The Avaneya crew commence scheduled in-transit experiments in space medicine, life sciences, astronomy, physical sciences, meteorology, and human research.
\StopTimelineDate

\StartTimelineDate{B.R.}
{\it Avaneya} engages a short burn of its manoeuvring thrusters to performs a debris collision avoidance in response to a warning emanating from the {\it Solar and Heliospheric Observatory} stationed at the Lagrange \math{L_1} gravity well stationed between the Sun and the Earth.
\StopTimelineDate

\StartTimelineDate{B.R.}
The Avaneya crew complete all scheduled in-transit experiments in space medicine, life sciences, astronomy, physical sciences, meteorology, and human research.
\StopTimelineDate

\StartTimelineDate{B.R.}
{\it Avaneya} deploys all four {\it Mars Positioning Satellites} (MPS) into high Martian orbit.
\StopTimelineDate

% Constraint: 180 days after launch with a departure velocity of 3.34 km/s...
\StartTimelineDate{B.R.}
Assisted by retrorockets, {\it Avaneya} performs orbital capture by aerobraking into Martian orbit. The ship's various instrumentation subsystems update the onboard areology database on the most recently available surface geography, weather dynamics, mineralogy, and other available telemetry.

The {\it Mars Enhanced Telecommunications Orbiter} is released and deploys itself into areosynchronous orbit, 17,065 kilometers above the equator.

\StartTimelineDate{B.R.}
The {\it Mars Enhanced Telecommunications Orbiter} (METO) successfully completes all self diagnostics and registers itself into UNSA's {\it Interplanetary Internet} as an available communications node, giving rise to what becomes colloquially known as {\it solnet}. Terran downlink passes through a ground station at UNSA's Jet Propulsion Laboratory staffed by Bronfman-Murdoch Aerospace contractors and provides tier 1 network access.

Brokered by a satellite uplink with METO, the systems team establish contact with Mission Control routed over the solnet. The solnet becomes the defacto standard for Earth-Mars intercommunication and comes to replace the term \quote{internet} in the everyday parlance.
\StopTimelineDate

\StartTimelineDate{15 Leo 8 B.R.}
Arda issues coordinates to the loadmaster for a preselected drop site in {\it Arcadia Planitia}, one of the several candidates previously catalogued by {\it Curiosity XI}. The cargo of mostly construction equipment, cryogenic liquid hydrogen, water, and other provisions, are jettisoned and parachute to the surface with no material loss, save one asset due to an attitude control computer malfunction.
\StopTimelineDate

% Leo 13, about the peak of the Martian Spring with clear skies and low winds with the 
%  weather at its finest...
\StartTimelineDate{B.R.}
All crew alight {\it Avaneya}, boarding the {\it Manu} landing craft. No one is left onboard exposed to further solar flares and cosmic radiation.

The {\it Mars Positioning Satellites} provide a guided landing by tracking the critical manoeuvres of {\it Manu's} terminal descent. The communications uplink with Mission Control is maintained throughout over a 4 minute delayed solnet connection.

{\it Manu} makes contact and reports a successful soft landing where they are met with an outside temperature of \math{-70^{\circ}}C and an atmospheric pressure of 30 Pa. The time is local noon, allowing for maximum photovoltaic use.

Team briefings are conducted within {\it Manu} at the drop site. Concurrently, the recovery team's rovers guided by scouts on methanol motorbikes are dispatched to begin asset recovery across {\it Arcadia Planitia}, or colloquially known simply as \quote{the planitia}.

The base's nuclear reactor is brought online, along with several temporary mobile portable dynamic isotope power systems.

The construction team begin minor excavation for anchoring and erecting temporary 34.0 kPa \goto{rated}[Pressure Rating] aluminium strut reinforced inflatable polypropylene tents for the Command Centre and habitats. The greenhouse tent is rated 6.8 kPa, sufficient for plant life, but requiring personnel to don EVA suits.
\StopTimelineDate

\StartTimelineDate{B.R.}
Gas extractors are brought online and run at full capacity capturing liquid oxygen, liquid nitrogen, argon, and carbon dioxide. 

The \goto{Sabatier}[Sabatier Reactor], \goto{RWGS}[Reverse Water Gas Shift Reactor], and methanol reactors successfully create methane, oxygen, hydrogen, methanol, and aqua successfully.
\StopTimelineDate

\StartTimelineDate{B.R.}
The recovery team complete their asset recovery of all undamaged parachuted cargo within a 92 kilometre radius from {\it Manu's} landing site.
\StopTimelineDate

\StartTimelineDate{B.R.}
Vehicular onboard artificial intelligence and system firmware is upgraded remotely by UNSA's Jet Propulsion Laboratory over the solnet.

The mechanics team complete the necessary preparation of the backhoes, front end loaders, bulldozers, tractors, graders, water ice processors, dump trucks, and other vehicles now available for construction team use.

Mining and excavation operations begin through a mixture of directly manned, remotely manned, and autonomous operation.
\StopTimelineDate

\StartTimelineDate{B.R.}
The {\it Mars Ascent Vehicles' Alpha} and {\it Bravo} alternate launches to perform rendezvous dockings with {\it Avaneya} in low Martian orbit. As methane/oxygen fuel stocks are manufactured {\it in situ}, they provide the ship with the propellant required for its return journey back to {\it Iterum Shipyard}.
\StopTimelineDate

% Constraint: 550 days after first arrival...
\StartTimelineDate{B.R.}
Arda directs the flight engineers to remotely issue the necessary command sequence for the fully refuelled {\it Avaneya} to begin its unmanned return to {\it Iterum Shipyard}.
\StopTimelineDate

\StartTimelineDate{B.R.}
{\it 52048 Varuna} is intercepted by UNSA's ion-drive propelled impactor, {\it Don Quixote V}. The spacecraft's payload of a nuclear warhead is detonated above the asteroid's surface, successfully altering its trajectory into a collision course with the Martian south pole.
\StopTimelineDate

\StartTimelineDate{B.R.}
{\it Tarikin I} and its crew begin preparation to depart Cape Canaveral, North American Union, in response to Arda's UNSA advisory report of the completion of additional settlement infrastructure. This marks the beginning of the {\it Tarikins'} continual supply of 400 new colonists every 780 day launch window.
\StopTimelineDate

% At 800 new arrivals every year, each Aravan capable of transporting
%  400 persons twice every year, this would take about ten years...
\StartTimelineDate{B.R.}
The Arcadian Settlement's annual census reports a population now in excess of 8,000 colonists and growing.
\StopTimelineDate

\StartTimelineDate{A.R.}
Originally published by the {\it RANND Corporation}, Edward Geisler's seminal work, {\it Transplanetary Commerce with Mars: The Red Horn of Plenty}, is hailed by industry leaders for the influence and achievement the white paper, long esteemed as a classic of great notoriety, culminates in as the first of a series of private launches en route to Mars departs Earth. Beginning from Kiruna, European Union, the heavy lift spacecraft {\it Arrogo} bears a payload of both personnel and cargo to be received at the starport in the Arcadian Settlement. It is already validated with an approved UNSA application.

The launch is celebrated by industry leaders as a landmark transformation as the first transnational corporations promote themselves to the first transplanetaries. The mission stakeholders consist of an international private consortium from a variety of sectors ranging from soft drink vendors to agricultural biotechnology.

Geisler's paper discussed the means and opportunities off-world commerce with Mars theoretically would provide and draws historical analogies with the European expansion into the \quote{new world} of the Americas and the great wealth that was tapped. The paper is identified as the original driving force behind the industry's efforts that succeed in the day's launch.
\StopTimelineDate

\StartTimelineDate{A.R.}
Arda receives a UNSA directive to terminate an orthomolecular laboratory and several other alternative medical research facilities. UNSA intends to lease the vacated facilities to GlaxoKlineSmith, a pharmaceutical company.
\StopTimelineDate

\StartTimelineDate{A.R.}
Arda receives a UNSA directive to reduce greenhouse water usage so the savings can support a new Coke bottling facility.
\StopTimelineDate

\StartTimelineDate{A.R.}
Arda receives a UNSA directive to provide logistical support to Clown Food contractors. They require assistance in erecting several kilometers of facilities for the production of genetically modified beef, pork, and potatoes required by newly opened restaurants across the settlement and its starport.
\StopTimelineDate

\StartTimelineDate{A.R.}
Arda receives a UNSA directive to terminate several mushroom ranches. The ranches were used to recycle plant waste into high quality sources of edible protein. The electricity used to maintain ranch temperature is reallocated for railgun use. All plant waste destined for the ranches or as greenhouse fertilizer now end up in newly created outdoor garbage dumps on the planitia where atmospheric conditions and cosmic radiation ensure organic matter will neither degrade nor become reusable.
\StopTimelineDate

\StartTimelineDate{A.R.}
Arda receives a UNSA directive to secure all new settlement energy requirements from private energy contractor Haliburton. Haliburton will provide UNSA with nuclear reactors that will superannuate the settlement's standard practise of expanding primarily through new geothermal wells and secondarily through additional photovoltaic panels. The directive specifies that these purchases are to be absorbed by the settlement's Terran sustainability research budget.
\StopTimelineDate

\StartTimelineDate{B.R.}
Arda receives a UNSA directive to terminate several settlement research labs currently investigating Terran sustainability. Some of these include key biophotoreactor plants that produce kelp fertilizer for greenhouse use. The resources freed are to be redirected to increased deuterium production and \goto{\it railgun}[Railgun] use. Arda reluctantly complies with the directive. UNSA cites technical reasons while Arda cites settlement debt.
\StopTimelineDate

\StartTimelineDate{B.R.}
The first of a long series of partisan meetings for settlement independence is held in secret with Arda in attendance.
\StopTimelineDate

\StartTimelineDate{B.R.}
Arda wins a landslide election and accepts executive office as President of a newly created interim settlement government. Arda and his staff will operate from within the newly erected {\it Arcadia Hall}. UNSA neither authorizes the interim government nor acknowledges the election.

Kissinger strongly cautions Arda against {\it "forming a redundant administration inconsistent with UNSA interests"}. On Earth, the Security Council holds an emergency meeting in New York on the situation.
\StopTimelineDate

\StartTimelineDate{B.R.}
Promoted from the position of United Nations Deputy Secretary-General, Security Council Resolution 12661 is adopted appointing Kissinger to the newly created {\it Office of the Governor of the Arcadian Settlement} in response to the settlement's self initiated election. The appointment asserts the Governor's seniority to any of the settlement's self elected public officers, though UNSA still does not acknowledge the election. Arda does not make a public statement.
\StopTimelineDate

\StartTimelineDate{A.R.}
The asteroid {\it 52048 Varuna} aerobreaks into the Martian atmosphere, disintegrating as predicted. Meteor showers are reported impacting the south pole's dry ice sheets by the terraformation team.
\StopTimelineDate

\StartTimelineDate{A.R.}
Arda puts the issue of settlement independence to a publicly held vote of 15,343 colonists. The vote carries.
\StopTimelineDate

% Year zero...
\StartTimelineDate{15 Virgo 0 A.R. (Year Zero)}
Intending to reach both planets, Arda formally addresses the Secretary-General of the United Nations through an open letter transmitted over the solnet. He reflects on the predicament faced by his fellow settlers.

\startTimelineCorrespondenceDocument
    \input Source/Timeline/Arda_Independence_Letter.tex
\stopTimelineCorrespondenceDocument

Arda's transmission over the solnet is available for public consumption, is accompanied with digital copies of all settlement research to date, and includes a copy of the {\it Rubicon Act}. 

The latter is also known as the {\it Constitution of the Republic of Arcadia Planitia}. It is the first of the settlement's self-declared legislation and aims to initiate and secure its independence. The document was the product of years of drafting through covertly held partisan meetings and, if successful, would beget the first off-world autonomous city-state in human history. Notable portions summarized follow:

\startTimelineGeneralDocument
    \input Source/Timeline/Rubicon_Act_Summary.tex
\stopTimelineGeneralDocument

A failed denial-of-service attack lodged against the Terran solnet downlink station at UNSA's Jet Propulsion Laboratory on Earth attempts to disrupt reception of Arda's transmission. Station technicians determine the attack originated from within the station's own local intranet from behind a restricted access zone using several zero day vulnerabilities in the station transceiver's software.

Arda's independence letter propagates rapidly over peer to peer networks. Responses on Earth among Terrans that have followed Martian events are predominantly in favour of the settlement's commitment to independence. Mainstream media relate the situation as a costly mission in crisis with a Mission Commander of recently revealed questionable associates and background. Alternative media consign themselves to a mixture of speculation and scepticism.
\StopTimelineDate

\StartTimelineDate{A.R.}
The former NASA had long ago salvaged aluminium recovered in 2001 A.D. from the ruins of the former World Trade Centers in lower Manhattan for reuse in the Mars rover {\it Spirit's} cable shield. To remind future Arcadians of the reason for independence, on Arda's directive, the {\it Constitution} is engraved in aluminium recovered from the derelict {\it Spirit} as it came to finally rest at its UNESCO World Heritage Site on Mars. 

NASA had originally selected the metal for reuse with {\it Spirit} as a symbol of perseverance in the face of evil. Arda selected it for reuse as a symbol of perseverance in the face of government.
\StopTimelineDate

\StartTimelineDate{A.R.}
A survey team is deployed on a routine assignment using its ground based shallow radar to scout for artesian aquifiers. Travelling solo, Arda departs the Arcadian Settlement to rendezvous later with the team as planned. Midway, he performs an EVA to dismount the rover and check one of its tires. While outside, he is knocked from his feet as an explosion occurs on the other side of the rover from within the cabin. Although his forearm is shattered, he sustains no life threatening injuries. With the rover's first aid kit destroyed, he improvises a make shift caste around the suit's elasticized mesh to hold his forearm in place by applying liquid water to coatings of regolith, hardening it into permafrost.

Arda then inspects the rover to find that the cabin is fully depressurized and the airlock's seal too badly damaged to maintain positive pressure. The rover's electronic controls are non-responsive, rendering the machine derelict in the middle of a Martian desert. The nearest rescue effort that acknowledges his emergency VHF transponder is an aquifier drilling rig crew at least 42 hours away. With only 35 hours of liquid oxygen available for his respirator, he extends it in time by siphoning cryogenic liquid oxygen from the rover's propulsion system.

The damaged rover is later towed back to a service bay in the city. Mechanics find fragments of a container with a gaseous methane / oxygen (\chemical{CH_4 / O_2}) signature that had been under up to 35,000 kPa of pressure. The mechanics rule out the rover's own equipment since its fuel cells are methanol based (\chemical{CH_3OH}) and no other known onboard equipment requires methane. In an effort to determine who last serviced the vehicle, the mechanics note that the electronic maintenance log is missing. Remnants in the cabin of what appear to be fragments of a timer based detonator are recovered with its components not matching any of the rover's.
\StopTimelineDate

\StartTimelineDate{A.R.}
{\it Arcadia} amends the Constitution by passing the {\it Humanoid Act}, stripping corporate legal fictions of the rights of human beings. Transplanetaries operating within Arcadia Planitia's territorial region do so now without the safety of corporate personhood.\footnote{A corporate charter creates a legal fiction with many of the basic rights of an actual human being. These may include the rights to own and sell assets, to sue and be sued, and so on. Critics argue that this ensures the physical human beings responsible for decisions are buffered from ever being held personally accountable.}
\StopTimelineDate

% Coke pulled out of Vietnam the year before the United States military did...
\StartTimelineDate{A.R.}
{\it Coke} quietly closes down its production and bottling facilities. Several other transplanetaries, including the successful fast food franchise {\it Clown Food}, are also seen at Starport Arcadia making preparations to return to Earth. No public statements are issued.
\StopTimelineDate

\StartTimelineDate{A.R.}
\goto{No. \#2 Railgun}[Railgun] is taken offline for unscheduled rail and insulator surface repair by Terran UNSA contractor, {\it Solar Urban Moving Systems}.
\StopTimelineDate

\StartTimelineDate{A.R.}
No. \type{#}2 Railgun's blockhouse reports an explosion that has taken out its compulsator power supply. The single largest loss of life ever to occur on Mars unfolds as two-hundred and fifty three people are killed and seven more injured as an adjacent residential district's inflatable dome in Arcadia's Fourth Precinct depressurizes from shrapnel.

Arda convenes an emergency meeting with all state secretaries present at {\it Arcadia Hall}. Forensic evidence of a cryobomb's (\chemical{CH_4/O_2}) chemical residue is presented. Chemical analysis suggests that the recovered components are of probable non-Martian origin. 

Before the meeting adjourns, Terran media implicate the alleged identities of the \quote{railgun bombers} as partisan extremists with ties to Arda. 

Shortly after the meeting adjourns, Arda begins reading a presidential address from {\it Arcadia Hall} covering his staff's preliminary findings. However, the live broadcast is cutoff prematurely from reaching a Terran audience due to technical difficulties with the {\it Mars Enhanced Telecommunications Orbiter}. The satellite used to broker nearly all communication between the two planets is remotely taken offline for an unscheduled systems diagnostic by {\it Bronfman-Murdoch Aerospace} contractors at UNSA's Jet Propulsion Laboratory on Earth. Terran mainstream media networks reassure Arcadia Hall's media relations that Arda's presidential address will be rescheduled for replay later, simultaneous the Superbowl.

Kissinger meets with Arda privately and advises him to step down as President, a political position, in light of the optics of the situation amidst growing pressure from UNSA. However, he maintains that UNSA still backs him in the predominantly technical capacity of Mission Commander due to the influence and respect he wields amongst the settlement population, as well as to oversee the resumption of stable deuterium exports. The Governor, as a final suggestion, advises that all indigenous self-governance apparatus be immediately dissolved due to its redundancy with UNSA and inability to secure the safety of the colonists.
\StopTimelineDate

\StartTimelineDate{A.R.}
Arda notifies Starport Arcadia to begin fueling {\it ERV Bravo}, Earth-bound, to take advantage of a minimum-energy launch window. Kissinger's visa is confiscated, revoked, and destroyed by Arcadian officials. A security detail then escorts the irate Kissinger to the starport for an automated solo departure that will lift off later that evening.
\StopTimelineDate

\StartTimelineDate{A.R.}
Lockheed Martin-Boeing's Skunk Works advanced research facilities begin refitting {\it Tarikin III} into the {\it Yama}, a manned and remotely piloted spacecraft in response to a UNSA contract for a classified mission payload.
\StopTimelineDate

\StartTimelineDate{B.R.}
The terraformation team at the south pole report findings of dry ice sublimation in response to the asteroid impact.
\StopTimelineDate

\StartTimelineDate{A.R.}
United Nations Security Council resolution 12664 is unanimously adopted, beginning with:

\startTimelineGeneralDocument
The Security Council,

Reaffirming the principles and purposes of the Charter of the United Nations,

Determined to combat by all means threats to transplanetary peace and security caused by terrorist acts,

Deploring the gross and systematic violation of human rights, including the repression of peaceful demonstrators, expressing deep concern at the deaths of civilians, and rejecting unequivocally the incitement to hostility and violence against the civilian population excited through the local administration of the United Nations Arcadian Settlement,

Recognizing the inherent right of individual or collective self-defence in accordance with the Charter,

\startitemize[n]
\setupwhitespace[big]
\item Unequivocally condemns in the strongest terms the horrifying terrorist attacks which took place recently in the United Nations Arcadian Settlement, Mars, and regards such acts, like any act of international terrorism, as a threat to international and transplanetary peace and security;

\item Expresses its deepest sympathy and condolences to the victims and their families and of the United Nations Arcadian Settlement;

\item Calls on all Mars Treaty signatories to work together urgently to bring to justice the perpetrators, organizers and sponsors of these terrorist attacks and stresses that those responsible for aiding, supporting or harbouring the perpetrators, organizers and sponsors of these acts will be held accountable;

\item Calls also on all Mars Treaty signatories to redouble their efforts to prevent and suppress terrorist acts including by increased cooperation and full implementation of the relevant anti-terrorist conventions and Security Council resolutions;

\item Expresses its readiness to take all necessary steps to respond to the terrorist attacks of the Arcadian Settlement railgun bombing, and to combat all forms of terrorism, in accordance with its responsibilities under the Charter of the United Nations;

\item Authorizes Mars Treaty signatories that have notified the Secretary-General, acting nationally or through regional organizations or arrangements, and acting in cooperation with the Secretary-General and the Office of the President of the Space Agency, to take all necessary measures to protect civilians, civilian populated areas, and assets under threat of attack in the United Nations Arcadian Settlement, and requests those signatories concerned to inform the Secretary-General immediately of the measures they take pursuant to the authorization conferred by this paragraph which shall be immediately reported to the Security Council;

\item Decides to remain actively seized of the matter.\footnote{Most Security Council resolutions end with this phrase. Article 12 of the United Nations charter states that \quote{While the Security Council is exercising in respect of any dispute or situation the functions assigned to it in the present Charter, the General Assembly shall not make any recommendation with regard to that dispute or situation unless the Security Council so requests.}}
\stopitemize
\stopTimelineGeneralDocument

The resolution recommends the {\it United Nations Department of Peacekeeping Operations} (UNDPKO) place all four permanent standing {\it Rapid Reaction Force} battalions on high alert.

UNSA makes the announcement on the day of the Terran Superbowl where it is met with minimal public reaction.
\StopTimelineDate

\StartTimelineDate{A.R.}
Footage of alleged riots erupting across Arcadia with security forces responding disproportionately are aired across Earth. Critics noting the equivalence of Terran gravity in the footage are dismissed by mainstream media as \quote{cranks} and \quote{conspiracy theorists}.
\StopTimelineDate

\StartTimelineDate{A.R.}
Selected Rapid Reaction Force personnel are assigned to UNSA training facilities in Antarctica and Huelva, Spain.
\StopTimelineDate

\StartTimelineDate{A.R.}
Arda receives an asymmetrically encrypted message coded using his public key from his Mountain and Commando Brigade, Hakkari originated from Earth
\StopTimelineDate

\StartTimelineDate{A.R.}
Arda directs the systems team to take out UNSA orbital surveillance capabilities by destroying the {\it Mars Enhanced Telecommunications Orbiter} while they still have access. UNSA transmits a command sequence to the orbiter to revoke the settlement's keypair but is eight minutes too late. The orbiter burns up its entire attitude control hydrazine fuel reserves, causing it to deorbit itself and be incinerated during atmospheric entry. 

UNSA has the NAUSA JPL lock down all four {\it Mars Positioning Satellites} in anticipation of RAP denial-of-service attacks.

All solnet communications between Earth and Mars that had been brokered by METO are now unavailable. The orbiter had also been used for reliable inter-Arcadian use. Arcadians adapt in its absence with a combination of line-of-sight UHF and shortwave communication.
\StopTimelineDate

% Need a strategic transport...
\StartTimelineDate{A.R.}
{\it Yama's} launch window is missed due to low-Earth orbit overly saturated with space debris. UNSA engineers fear they may be approaching the Kessler effect of causing a runaway chain reaction which would reduce all objects in orbit. Since this would threaten not only the {\it Iterum Shipyard}, but all Terran space exploration, the departure is rescheduled for the next available launch window of 25 months.
\StopTimelineDate

\StartTimelineDate{A.R.}
Arda holds an emergency cabinet meeting at {\it Arcadia Hall} without issuing a public statement. A team of geologists and engineers are attached to a survey team and dispatched immediately on rovers on his directive.
\StopTimelineDate

% Expediency is plausible if partisan meetings had been held at such a location were surveying had already been done in the past...
\StartTimelineDate{A.R.}
Arda has Arcadia undergo an emergency evacuation over the following two weeks of all non-essential personnel to an undisclosed installation located within a subterranean lava tube.
\StopTimelineDate

\StartTimelineDate{A.R.}
{\it ERV Bravo} carrying Kissinger lands at Edwards Air Force Base, California, North American Union. The spacecraft is immediately seized as a UNSA asset. Kissinger leaves for Geneva the following day.
\StopTimelineDate

% Yama takes a free-return trajectory taking 180 days and drops off personnel and material
\StartTimelineDate{A.R.}
{\it Yama} completes its trans-Martian journey, aerobreaking into Martian geostationary orbit. The ship carries a payload of remotely operated equipment destined for Phobos, a photographic reconnaissance satellite for sun-synchronous orbit, as well as a single Rapid Reaction Force battalion. The battalion is to be deployed as the {\it United Nations Emergency Assistance Peacekeeping Force}. Its numbers are drawn principally from North American Union and European Union airborne light infantry units.
\StopTimelineDate

\StartTimelineDate{A.R.}
{\it Yama's} commanding officer Lieutenant-Colonel Dragov issues warning orders down the chain of command to prepare for insertion while in Martian geostationary orbit.
\StopTimelineDate

\StartTimelineDate{A.R.}
Arcadia's 3rd militia battalion's attached Signal Corps detects a non-RAP nav-aid transponder signal within the RAP demarcation.
\StopTimelineDate

\StartTimelineDate{A.R.}
{\it Arcadia Planitia} comes under fire by orbital shelling, marking the first off-world shots ever fired in anger. {\it Arcadia Hall} is successfully targeted and destroyed, though most of Arcadia is deliberately left intact. Outside of the city, all locations of the suspected subterranean RAP installation are hammered over the next month.

Between the shelling, Arda has the systems team erect a rudimentary high-gain directional antenna which circumvents the need for a satellite communication relay to transmit to Earth. He discloses directly via S-band microwave to the people of Earth what is happening on Mars, describing the UNSA shelling campaign as an attack on peaceful and defenceless civilians levied from a partially robotically operated orbital installation.

Arda provides a document hacked from a UNSA intranet with a detailed white paper, complete with schematics, of the outpost. The classified document titled {\it Project Horizon: A UNSA Study for the Establishment of a Phobos Military Outpost}, details an ambitious covert UNSA project to construct {\it Thor Outpost}. The alleged outpost is manned by a garrison of a single platoon of 30 soldiers, powered by three nuclear reactors, and located on the Martian moon Phobos. The mostly porous composition of the moon allows for the majority of the installation to remain subterranean. If the documents are genuine, then the installation had been kept from the public's knowledge. High resolution imagery taken from the Martian surface of the installation accompany the transmission as evidence.

\startTimelineGeneralDocument
...The Phobos outpost is required to develop and protect UNSA interests on Mars; to develop techniques in Phobos and general orbital-based surveillance of Mars, in communications relay, and in operations on the surface of Mars; to serve as a base for exploration of Phobos, for further exploration into space and for military operations on Phobos if required; and to support scientific investigations on Phobos...

\hskip 1.5cm {\it - Project Horizon: A UNSA Study for the Establishment of a Phobos Military Outpost}
\stopTimelineGeneralDocument

Arda explains that by exploiting a legal loophole in the {\it Outer Space Treaty}, the {\it Anti-Ballistic Missile Treaty}, and the {\it Mars Treaty} whose drafting predated the advent of kinetic bombardment weapons, UNSA believes it is justified in dropping hypersonic tungsten rods with surgical precision at downward velocities of 10 km/s on the surface. The treaties do not observe kinetic bombardment as weapons of mass destruction since they require no explosive payload, though they deliver sufficient kinetic energy in the order of a small tactical nuclear weapon.

The transmission is picked up on Earth by amateur radio operators and quickly disseminated over the solnet. UNSA dismisses Arda's claims as {\it "baseless allegations propagated by Terran conspiracy theorists, intertwined with a misunderstanding of the nature of meteorite strikes on Mars, and used to justify the theft and reckless destruction of the costly off-world public asset of the Arcadian Settlement"}. A UNSA spokesman refuses to respond to questions on the existence of the controversial outpost.
\StopTimelineDate

\StartTimelineDate{A.R.}
A subsonic aircraft crashes 40 km south of RAP Starport carrying 3600 kg of cocaine in the plains of {\it Arcadia Planitia}. The transportation secretary's investigators confirm that the aircraft's tailnumber was registered to the CIA for alleged UNSA diplomatic flights. Chemical analysis suggests it to be the first known synthetic cocaine and of Martian origin.
\StopTimelineDate

UNSA destroy food stocks and then provides aid with terminator gene gmo

photobioreactors converted for growing biofuels

red cross

\stoplines

\StopChapter

