% This is part of the Avaneya Project Crew Handbook.
% Copyright (C) 2010-2013 Cartesian Theatre <info@cartesiantheatre.com>.
% See the file Copying for details on copying conditions.

% The Game chapter...
\StartChapter{The Game}

\StartSection{Why?}
Although the answers are many, they are best considered in the context of another question. {\it Can a game be socially useful?}

Too frequently people come out of experiencing a great work of dystopian science fiction only to say to themselves, Thank goodness we don't live in {\it that} world. That needs to change. Probably nothing could have better deserved a finger pointed scornfully, for having prompted this ambitious technical and creative undertaking than that.

Dystopian science fiction may be among the most honest kind of story--telling when examined in the context of history. But unless its patrons can identify with it, analogies will remain vague and metaphors too cryptic for it to reach its maximum use. Perhaps it could be said that {\it constructive} dystopian science fiction is more reluctant to conceal its own relevance.

\placefigure
    [right,0*hang]
    {Captured by Viking orbiter I on 22 February 1980. \index{Valles Marineris}{\it Valles Marineris} is the scar on Mars over 3000 km long and up to 8 km deep.}
    {\externalfigure[The_Game/Images/Valles_Marineris.png][][width=.5\textwidth]}

Constructive dystopian science fiction (or whatever one calls it) still ought to strive to be well--received in its fictional characteristics as well. One of the ways it does this is in being only partially fictional because it is the plausible scientific dimension that usually attracts people to the genre in the first place.

To write good fantasy well is a challenge in itself, but it is different from hard science fiction. In fantasy, there are no experts in fire breathing dragons to count among its patrons simply because there are no fire breathing dragons. In hard science fiction, patrons often actually {\it are} areologists, astronomers, biologists, chemists, geologist, and so on. This is a major difference between hard science fiction and other fictional genres -- including even soft science fiction. 

Science, by its nature, is a thinking and knowledge--oriented enterprise. It is the study of reality, and yet fiction, by definition, is outside of reality. So science fiction must be a mediator or broker between the two -- often acting as the {\it avant--garde} for both.\footnotecite[shedroff2012] But the dystopian variety, we will leave to readers to ponder which end it imitates greater, reality or fiction.

But if reality, then the bar has been set high for what its patrons expect. This requires designers to pay a great deal of attention to detail. A certain degree of creative license is expected, but patrons of {\it hard} science fiction still prefer it to be at least consistent with what we already know to be true; what is reasonable or plausible to consider true, or what may one day become true.

Isaac Asimov, a man who needs no introduction in the world of science--fiction writing, so very aptly once described this problem. \quote{\it The role of a writer is hard, for on every hand he meets up with critics. Some critics are, I suppose, wiser than others, but there are very few who are so wise as to resist the urge to show off. Critics of science popularization always have the impulse to list every error they can find and trot them out and smile bashfully at this display of their own erudition. Sometimes the errors are egregious and are worth pointing out; sometimes the critic is indulging in nitpicking; and sometimes the critic inadvertently shows himself up.}\footnotecite[extras={, p.~73.}][asimov1981]

Another important reason why this project is necessary is that there are disappointingly very few {\it libre} games being actively developed for GNU.\footnotecite[xdg_games_inactivity] Even fewer, if any, {\it free} as in freedom,\footnotecite[what_is_free_software] commercial games for the GNU operating system. Even though there are many non--free games available for it, they ignore the issue of freedom.\footnotecite[nonfree_drm_games_on_gnu_linux] This is important to recognize because it was this issue of freedom that the GNU project set out to address. If we simply wanted an operating system that did not cost anything, options have long been available.\footnotecite[piratebay]

Not only are there very few higher--production titles available for GNU, those that do exist are usually proprietary. Even then, users are typically left with a bad port of an old game reliant on deprecated APIs, poorly packaged,\footnote{Assuming the platform's native package manager was even used at all.} and integrating horribly into the user's desktop -- ignoring useful community driven specifications like {\it freedesktop.org} in the process. Ethical software users are treated as though they were irrelevant -- even when they were under the impression that at last someone was willing to deliver. But despite this, the delivery usually ends up revealing itself to be little more than the table scraps and leftovers of the proprietary world, a world whose grapple over peoples' lives served to beget the GNU operating system in the first place -- an operating system now wanting of professionally developed games. When people use GNU, they are treated as second--class citizens in many respects -- not least of which is the availability of good games. It is as though these people do not even exist at worst,\footnotecite[gnu_linux_support_non_existent] or are considered creatures of novelty at best.\footnotecite[gnu_linux_support_as_novelty] 

A user should be able to simply play the game without having to worry about manually setting environment variables or having the right version of some support library installed to be able to enjoy it as its authors had intended.\footnote{Consistent audio support under GNU/Linux has historically been a nightmare, as one of many examples.} We believe things like this have gone on long enough and that GNU users ought to be treated as first--class citizens; the same as everyone else. The technology is there, the demand is there, and now those willing to deliver on the demand are there too.
\placefigure
    [left, 0*hang]
    [figure:Kip_Concept_Art]
    {Avaneya developer showcasing concept art before {\it Ubuntu Vancouver}. {\it Photo courtesy Reggie.}}
    {\externalfigure[The_Game/Images/Kip_Concept_Art.png][][width=0.5\textwidth]}

In some sense, Avaneya is part of an overall effort to complete the list of tasks the GNU operating system requires to bring it closer to completion. When the GNU project set out to replace the body of non--free software required to use a computer, Richard Stallman once remarked that it would also require games since a complete operating system needed games too.\footnotecite[linux_and_gnu] \quote{Even games are included in the task list--and have been since the beginning. Unix included games, so naturally GNU should too. But compatibility was not an issue for games, so we did not follow the list of games that Unix had. Instead, we listed a spectrum of different kinds of games that users might like.}\footnotecite[about_gnu_project] Indeed, his original announcement of the GNU project back in 1984 even called for a strategy game.\footnotecite[gnu_initial_announcement] One of the aims of this project then is to lend a hand by rewarding the many good people who commit themselves to free software for ethical reasons with a game that they might enjoy.

Another reason is to present a case for the responsible human settlement of Mars. This can be done by drawing on the existing research to date and creatively adapting it for use in a non--casual game. This doubles by encouraging a greater interest in science by showing how exciting and useful it can be to people of all ages.

A final reason for Avaneya is found in the existing proprietary games themselves. They tend to appeal more to the mainstream proprietary user. They do not share the software {\it libre} community's values of freedom and not merely in how they are licensed. The authors of these games tend to be comfortable with an implicit assumption that their users prefer to be entirely detached from, and unconcerned with, anything to do with reality at any meaningful level while playing. All games certainly need not offer a remedy; that would deprive us of the escapism they can offer which is harmless in moderation. But when all of the most prevalent share that assumption, and when one considers the amount of time a person can spend playing a game, games can become social liabilities.\footnotecite[wow_deaths]

Case in point, of all the contemporary video games that integrate the theme of terrorism in some way, and they surely number in the hundreds by now, of what proportion staged terror? Are such phenomena not of interest to the users of these games, or simply not to those authoring them? Although a good game can depart from reality, a better one might complement it. We must be progressive not only in our licensing, but also in our willingness to examine even the most dangerous of assumptions constructively. 

In the end, it is almost as though no user is truly getting anything particularly valuable. The {\it libre} user is taken for irrelevant. The proprietary user, provided with ample, yet taken for a fool by limiting games to only the most base of themes and memes.

The fact that so many of the proprietary games are so meaningless, without any real value or learning experience, is under appreciated.\footnotecite[blackwater_game]\footnotecite[hot_coffee_mod] Their authors are able to get away with this because they have managed to convince their users of another bad assumption, that technology is ethically neutral since machines are merely tools existing external of any social context. Such a superficial perception of technology is well illustrated in the open source school of thought in contrast to its cousin, the software {\it libre} movement.\footnotecite[open_source_misses_the_point] 

Proprietary technology developers usually create technology with the {\it profit motive}\index{profit motive} in mind. While it is important that free software be sustainable too, the {\it purpose motive}\index{purpose motive} is at least as fundamental. This is just one reason among many why free software tends to be of a higher degree of craftsmanship. Sometimes proprietary software, software which holds its users hostage to its vendor, can potentially even risk getting people killed when users cannot fix problems in mission critical environments.\footnotecite[campion_smith2012] Free software developers want to build powerful technology too, but they have a tendency to not ignore the social context of what they create the way other technologists sometimes do. When we ignore the social context, technologists can become dangerous when their superficial values are left to run their course.\footnotecite[rotten_fruit2011_death_toll]\footnotecite[black2012]\footnotecite[rotten_rotties_game]\footnotecite[rotten_fruit2011_alarm_over_suicides]\footnotecite[brew2012]\footnotecite[rotten_fruit2011_stand_24_hours]\footnotecite[rotten_fruit2011_refused_benefits]

The truth is, all technology we create has always had a social context. It can never exist suspended independently in a vacuum. Free games can take a different approach by not ignoring the social context, by not authoring software as though such an ideal were possible, and by not treating the user as though they were reductive abstract economic units, consumers, there to be exploited. Instead, it recognizes that they are living human beings where technology can responsibly integrate and enrich their lives. 

Thus, we assert that the choice between using free and proprietary software is not in the same class of decisions as chocolate versus vanilla, since such an arbitrary preference says nothing meaningful of the user's constitution. When users choose free software over non--free software for the purpose of freedom, they become different from proprietary users in terms of their values and not merely just in the software they opted for. Therefore, a game for such an audience ought to strive beyond merely a distinction in licensing and also reflect the progressive themes of {\it libre} culture in spirit as well. If you understand this, you will understand this game.

\StartSection{Classification}
People tend to struggle to classify Avaneya. It is what it is, but we might start with the traditional labels of the classic city builder, management simulation, and real time strategy. If we must exercise economy of labels, it would have to be just a {\it strategy} game in the most general sense. But upon closer examination, you will find it to be a political thriller, drama, an educational medium, science fiction, a metaphor, and perhaps more.

\StartSection{Likely Users}
The game so far has attracted a fairly large base of followers. From what can be observed at this time, it appears to appeal to those with an interest in:

\startitemize[4]
\item
{\it games that are not only fun, but considered useful by the users;}
\item
{\it challenging the consensus of reality;}
\item
{\it software libre;}
\item
{\it a social conscience;}
\item
{\it science fiction;}
\item
{\it and exploration of the interrelatedness of everything.}
\stopitemize

The game may take place in the future, but it deals with contemporary issues. Readers will be in a better position to determine more about who makes up this game's audience by examining \in{chapter}[Leitmotifs] and \in{section}[Resources For Everyone].

\StartSection{Unlikely Users}
Avaneya is a {\it sui generis}. \footnote{{\it "Literally meaning of its own kind / genus or unique in its characteristics. The expression is often used in analytic philosophy to indicate an idea, an entity, or a reality which cannot be included in a wider concept,"} (Wikipedia).} It is not like other games, and thus it is not for all people. It does not try to be, nor will it ever be.

Those seeking a casual game; those with a brief attention span, always seeking immediate gratification; those that believe everything they consume began its life in shrink wrap or in a can with little appreciation for process; and accidentalists, to name a few, will probably not enjoy this game. There are already many such games that appeal to that type of audience, so that need not be our aim here.

This game will challenge users to think and possibly even offend them. As already mentioned, it challenges the consensus of reality, and therefore, potentially, peoples' world views. As such, some have accused Avaneya of being a vehicle for culture jamming (reference: http://en.wikipedia.org/wiki/Culture_jamming) and political commentary. This project is shamelessly guilty as charged -- like the newspapers, film, television, games, and other mainstream media that saturate us on a daily basis. 

The only difference is that, unlike those mediums, the very presence of a normative bias in Avaneya is not subject to dispute and is pretty self evident. Other mediums sometimes pretend to not have one. In any case, you would be very hard pressed to try to find any classical work of science fiction, or really any kind of fiction among the classics for that matter, that did not. Besides, this in itself is not necessarily a bad thing anyways.

\StartSection{Availability}
We do not believe that deliberately {\it restricting} users to a specific platform or hardware is ethical for any other reason than practical technical limitations of the alternatives, such as the absence of a programmable shader interface under {\it any} license. We do, however, strongly {\it promote} a specific platform, that being GNU and any available hardware it can run on capable of providing for the technical requirements of Avaneya.

Since we consider it unethical to encourage people to use non--free software, we will not ourselves ever be the primary maintainers of such ports. However, it would also be unethical to deliberately design it in such a way so as to cripple the efforts of others from doing so. They have the freedom to disagree; thus, since Avaneya relies on portable libraries, it should not be unreasonable for someone to do this if they do not share our values, provided they respect the terms of our licensing (\in{chapter}[Licensing Rationale]).

Other GNU distributions that provide alternatives to Linux as the kernel we consider fair game. These include GNU/kFreeBSD, GNU/Hurd, GNU/NetBSD, and possibly others, depending on the level of maturity of their software interfaces, technical feasibility, and demand from our community. As Linux continues to deviate from the free (and even open source) philosophy,\footnotecite[linux_non_free] it becomes increasingly important to encourage and support alternative choices as they become viable.

Some have suggested that maintaining for free platforms can be rather self--limiting in terms of user outreach. There was a time when this view had validity. Now, with countless users worldwide running some flavour of GNU, such as Ubuntu, the perception of a non--existent or marginal demographic is rather antiquated. Consider that gaming under non--free operating systems, such as OS X, is still active and lively today, yet Ubuntu alone, not including other GNU distributions, has debatably surpassed it in popularity with an unregistered user--base numbering in the tens of millions.\footnotecite[ubuntu_surpasses_osx] GNU long ago stopped being merely an operating system for hobbyists. It is so ubiquitous, you will find it everywhere from the parliaments of the world to telecommunications satellites in orbit.

\StopChapter

