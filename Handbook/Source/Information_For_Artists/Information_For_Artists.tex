% This is part of the Avaneya Crew Handbook.
% Copyright (C) 2010, 2011
%   Kshatra Corp.
% See the file License for copying conditions.

% Information For Artists chapter...
\StartChapter{Information For Artists}

% The Martian Landscape
\StartSection{The Martian Landscape}

Getting the game experience right means replicating the visual environmental conditions of the Martian landscape as closely as we can. Since artists depend on their eyes to provide tools like Blender with an aesthetic experience, as opposed to readings from a mass spectrometer, it is more important that they know what Martian rocks and a sunrise would actually look like if they were standing there than, say, the peroxide concentration of Martian regolith. This means they will need lots of high resolution reference photographs to study. But Mars, like Earth, is a large place of greatly varied terrain and climate, so we must be more specific

The Avaneya settlers erected their settlement of Arcadia in the Arcadia Planitia region. It is named after the Arcadia region of Ancient Greece, so, in turn, named after the legend of Arcas\index{Arcas}. The region is located mid-latitude in the Northern hemisphere. Martian areas of that latitude have their benefits, such as year round sunlight and plenty of water ice. The centre of Arcadia Planitia is mostly uniform in appearance, with its centre roughly at \math{46.7^{\circ}}N \math{192.0^{\circ}}E.

Its windswept landscape consists of a vast, mostly flat, pale tan coloured plain. It has sand dunes of modest height, never approaching anything higher than a few feet, with small uniformly sized rocks littering the surface. Like all other explored regions, it is well sterilized with intense ultraviolet radiation, ensuring that there is no known life on the Martian {\it surface}.

Images of a fresh meteorite crater 12 metres across taken in 2008 revealed under the surface a massive blanket of water ice.\footnote{\fullahref{http://www.cbc.ca/news/technology/story/2009/09/24/tech-space-water-mars-crater.html}} It also turns out fortuitously that this ice is almost completely pure, with only about one percent of it dirt.

Even though there is a great deal of ice on Mars, it is unstable in the thin Martian atmosphere and you rarely ever see it anywhere other than in the polar regions. This is because the ice rapidly sublimates\footnote{When a solid material sublimates, it means it skips melting to a liquid and turns directly into vapour.} as soon as it is exposed.

Arcadia Planitia has also experienced recent lava flows. By recent, we mean in geological time (as in, the last few million years).

\StartSection{Viking Landers}

At present, no lander has explored Arcadia Planitia returning colour images. However, the neighbouring plain of Utopia Planitia is very similar in appearance and we have a plethora of images captured by the Viking II Lander at \math{47.7^{\circ}}N \math{225.9^{\circ}}W which it explored back on September 3, 1976.

A bit of background. In 1976 NASA placed two spacecraft into Martian orbit. The Viking Orbiters both carried their own landers which they deployed. They were capable of roaming the surface of the planet and relaying their findings through the orbiters back to Earth. They continued to do this without any issues for four years, providing us with copious amounts of data. 

\StartSection{Mars Viking Lander Colour Image Restoration Archive}

While researching the aesthetics of the Martian surface, we found it incredulous that in all the decades passed since the Viking Landers sent their images back to Earth, there does not appear to be so much as a single, complete, user friendly, archive of all the images. So you can forgot downloading a convenient archive of PNGs from NASA's website, for instance.

NASA was not thinking at the time of a need for a long term image archival strategy. This is understandable, given the overwhelming excitement and preoccupation that must have come with getting real data and real photographs of another world.

As a consequence, the archaic magnetic tapes the images were stored on began to rot as the years went on. Even if they had been safe on the magnetic tapes, the format they were encoded in is an ancient format that had its origins nearly half a century ago - the VICAR format\index{VICAR format}. Moreover, the few remaining functional VICAR image loaders the team has managed to access are for much newer file format versions than those of the Viking era.\footnote{ImageMagick 6.7.0 is one such example that can only read the newer format.}

And those we could not get our hands on because they were proprietary and their distribution severely restricted, your guess is as good as ours. NASA's Jet Propulsion Laboratory told us that they could {\it "only provide a royalty-free license to universities (signed by a department head) or a government subcontractor, where the requested software is required to support the effort. At the end of the contract, it must be deleted. [They] cannot provide software to individuals or for R&D purposes."}

In keeping with the {\it software libre} philosophy, we did not wait on proprietary software vendors to hold the data hostage. We have prepared a {\it Mars Viking Lander Colour Image Restoration Archive} of all of the colour image data that were successfully recovered. We did this because our artists should not have to perform digital archaeology to see what Mars looks like when the photos were already taken decades ago.

The source VICAR images were {\it not} from the Planetary Data System formatted volumes. They were produced by the Science Digital Data Preservation Task at NASA's Jet Propulsion Laboratory by copying data directly off of old, decaying tape media onto more stable optical media. They were not otherwise reformatted. They were provided courtesy of the NASA Planetary Data System.

The archive contains colour images from the Viking Lander High Resolution Mosaics, Stereo Images and Range Data Sets and the Viking Lander Processed Images. From the entire archive, we extracted images from the volumes containing only colour images. They were identified based on their description containing words like {\tt COLOR}, {\tt CLR}, {\tt IR} (for the colour products made from the infra-red filters), or {\tt RADCAM}, for the program used to generate colour products. All of the metadata the on-board instrumentation provided, such as whether the camera lens was cleared prior to shooting with compressed gas to clear off dust is gone. For our purposes, such metadata is not so useful.

\StopChapter

