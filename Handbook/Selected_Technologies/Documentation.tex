% This is part of the Avaneya Project Crew Handbook.
% Copyright (C) 2010, 2011, 2012
%   Kshatra Corp.
% See the file License for copying conditions.

% Documentation section...
\StartSection{Documentation}

\startitemize[4]
\head {\em \BIBTEX}

As you have probably noticed, this book has many references. \BIBTEX\ is a well proven tool that has been used for many years in various fields ranging from the humanities to the sciences. It keeps track of references, organizes them, and presents a standardized interface for typesetting software like \CONTEXT.

\head {\em \CONTEXT}

\CONTEXT\ is used for typesetting documents, such as this one. This document was compiled on \currentdate\ from Bzr revision \BzrRevisionClickable, typeset using \CONTEXT\ \contextversion, and rendered with \texenginename\ \texengineversion. One of the many beauties of using a programmable typesetting language is that documents written in them are diff friendly, making collaboration easier.

\head {\em Doxygen}

As software increases in its complexity, so does the necessity to document its internal workings. Doxygen is a tool that can generate engineering documentation on the fly. The documentation is derived from metadata hints at the source code level. It can even produce graphs that show the relationship between classes. This latter feature is actually implemented internally using the tool described next, Graphviz.

\head {\em Graphviz}

Graphviz is a powerful set of tools for drawing graphs specified in the DOT language. Among the suite it provides, {\tt dot} was used for several of our figures, like that in \in{figure}[figure:Resources_Pipeline].

\head {\em Umbrello}

Umbrello is a powerful diagramming tool which lends itself well to the creation of UML modelling. The AresEngine's architectural design was modelled in Umbrello as seen in \in{section}[AresEngine's Architecture]. The two projects compliment each other well since modelling the AresEngine put very heavy demands on Umbrello, thus allowing the stress testing to identify, solve, and enhance many aspects of Umbrello. We are very grateful to the very cooperative developers working on the Umbrello project.
\stopitemize

