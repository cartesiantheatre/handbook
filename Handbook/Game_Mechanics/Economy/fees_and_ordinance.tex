\StartSubSection{Fees and Ordinance}
Fees are closely related to taxation in that they take from individuals or business' for their actions or existence within the city. Ordinance are the guidlines, rules, or standards by which the city is to be held.

Fees generate income for the city whereas ordinance forms the rules or laws that allow fees to be obtained for or from certain activites. While ordinance can generate income, for the most part ordinance is seen as measures which improve a city's function and quality at a financial cost. Some examples of fees may include rates and road tolls whereas ordinance might include allocating special permits for land use and allocating resources for a neighbourhood watch.

Both fees and ordinance can be implemented in broad strokes to individuals and companies or can be micro managed to fine detail, focusing only on certain groups or city areas (similar to methods discussed in taxation).


%Fees Screen
This screen shows current fees in place, fee earnings, who pays, and various statistics.
%Ordinance Screen
This screen shows enacted ordinance and costs. From this screen players can reach all ordinance (broken down into categories) to enact further ordinance, as well as access screens showing various statistics. 


%%%%%%%%%%%%%%%%%%%%%%%%%%%%%%%%%%%%%%%%%%%%%%%%%%%%%%%%%%%%%%%%
%Implementing Fees (various examples)
%Figure/s showing how to get to main fees screen then figure showing the main screen screen itself.
From the gameplay screen players do ... ... ... to access the fees menu. From here players have access to various categories of fees. Within each category are various fees, the range of each depending on other factors previously implemented by the player. Some fees may be useful, some may not, and this usefullness may change through a city's life and evolution and new forces and technologies, both internal and external, come into play.

%Figure showing screen with various fees types.
%Fee Categories:
%Traffic Fees - fees relating to use and ownership of vehicles. 
%Domestic Fees - fees relating to property and household use and maintainence.
%Disorderly Fees - fees relating to an individual's behaviour.
%Business Fees - fees relating to business operations and dereliction.  

%Figure: traffic fees.
In this example we selected "Traffic Fees" which shows a list of (number of fees) that are applicable to the management of the city's traffic, both private and commercial. Various fees are shown, the list is scrollable and those fees enacted are marked (...) and display the current average income they generate for the designated time period(per annum?).  We can see in the list fees such as parking, speeding, and unregistered. To enact the fee, players simply select the (...) next to the fee they wish to enact.  

Next to each fee is a statistics(?) button. 
%Figure: fee statistics and numbers screen.
Within this screen players can, depending on the funding they have allocated to various research and the level of the cities complexity(?), see the effects that fees will have on various demographics. Within this screen players can also set the penalty levels, financial or otherwise, for each of the fines, and weigh the cost of implementing a fee with the estimated benefit.

%%%IMPLEMENTING


%%%%%%%%%%%%%%%%%%%%%%%%%%%%%%%%%%%%%%%%%%%%%%%%%%%%%%%%%%%%%%%%%
%Implementing Ordinance (various examples) -similar to taxation-
%Figure/s showing how to get to main fees screen then figure showing the main screen itself.
From the gameplay screen players click ... ... ... to access the ordinance menu. Within the ordinance menu players will see various categories of ordinance as well as see ordinance already enacted. Within each of the ordinance categories are various types of ordinance relating to that category. Specific ordinance will come into and out of play at various stages of a city's life. While the majority will generally be there the entire game, certain research, technological, trade, and diplomatic policies, among others, will alter the ordinance types that are available.

%Figure showing screen with various ordinance types.
%Ordinance Categories:
%
%
%

%Figure: trade ordinance.
In this example we selected "Trade Ordinance". In this screen we can see a list of various ordinance associated with trade, being the import and export of goods to and from the player's city.


%greater statistics can be seen by allocating more resources to city research and as the cities economy develops.
