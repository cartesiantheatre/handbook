% This is part of the Avaneya Project Crew Handbook.
% Copyright (C) 2010-2013 Cartesian Theatre <info@cartesiantheatre.com>.
% See the file Copying for details on copying conditions.

% The Game chapter...
\StartChapter{The Game}

\StartSection{Why?}
Although the answers are many, they are best considered in the context of another question. {\it Can a game be socially useful?}

Too frequently people come out of experiencing a great work of dystopian science fiction only to say to themselves, Thank goodness we don't live in {\it that} world. That needed to change. Probably nothing could have better deserved a finger pointed scornfully for having prompted this ambitious creative and technical undertaking.

Dystopian science fiction may be among the most honest kind of story--telling when examined in the context of history. But unless its patrons can identify with it, analogies will remain vague, metaphors cryptic, and whatever message the author might have intended greatly diminished. Perhaps it could be said that {\it constructive} dystopian science fiction is less reluctant to conceal its relevance.

\placefigure
    [right,0*hang]
    {Captured by Viking orbiter I on 22 February 1980. \index{Valles Marineris}{\it Valles Marineris} is the scar on Mars over 3000 km long and up to 8 km deep.}
    {\externalfigure[The_Game/Images/Valles_Marineris.png][][width=.5\textwidth]}

Constructive dystopian science fiction (or whatever one calls it) still ought to strive to be well--received in its fictional characteristics as well. One of the ways it does this is in being only partially fictional. This is because it is the plausible scientific dimension that usually attracts people to the genre in the first place, among other things.

To write good fantasy well is a challenge in itself, but it is different from hard science fiction. In fantasy, there are no experts in fire breathing dragons to count among its patrons simply because there are no fire breathing dragons. In hard science fiction, however, patrons often actually {\it are} areologists, astronomers, biologists, chemists, geologist, or what have you. This is a major difference between hard science fiction and other fictional genres -- including soft science fiction. 

Science, by its nature, is a thinking and knowledge--oriented enterprise. It is the study of reality, and yet fiction, by definition, is outside of reality. So science fiction has a special role as a mediator between the two -- often acting as the {\it avant--garde} for both.\footnotecite[shedroff2012] But the dystopian variety, we will leave to readers to ponder which end it imitates greater, reality or fiction.

But if reality, then the bar has been set high for what its patrons expect. This requires designers to pay a great deal of attention to detail. A certain degree of creative license is expected, but patrons of {\it hard} science fiction still prefer it to be at least consistent with what they already know to be true; what is reasonable or plausible to consider true, or what may one day become true. Of course, we know to be true sometimes ends up not being so, and in the history of science this happens to be a recurring theme.

Isaac Asimov, a man who needs no introduction, once described this problem in realizing that \quote{\it the role of a writer is hard, for on every hand he meets up with critics. Some critics are, I suppose, wiser than others, but there are very few who are so wise as to resist the urge to show off. Critics of science popularization always have the impulse to list every error they can find and trot them out and smile bashfully at this display of their own erudition. Sometimes the errors are egregious and are worth pointing out; sometimes the critic is indulging in nitpicking; and sometimes the critic inadvertently shows himself up.}\footnotecite[extras={, p.~73.}][asimov1981]

One of the many important reasons why this project is necessary is that there are disappointingly very few {\it libre} games being actively developed for GNU.\footnotecite[xdg_games_inactivity] Even fewer, if any, {\it free} as in freedom,\footnotecite[what_is_free_software] commercial games for the GNU operating system. Even though there are many non--free games available for it, they ignore the issue of freedom.\footnotecite[nonfree_drm_games_on_gnu_linux] This is important to recognize because it was this issue of freedom that the GNU project set out to address. If we simply wanted an operating system that did not cost anything, options have long been available to those willing,\footnotecite[piratebay] -- despite what the media would have us believe that GNU is for the homeless to use on scrap hardware.\footnotecite[straight2013_ubuntu_for_homeless]

Not only are there very few higher--production titles available for GNU, those that do exist are also almost always proprietary. But even that aside, users are typically left with a bad port of an old game reliant on deprecated APIs, poorly packaged,\footnote{Assuming the platform's native package manager was even used at all.} and integrating horribly into the user's desktop environment -- ignoring useful community driven specifications like {\it freedesktop.org} in the process. Ethical software users are treated as though they were irrelevant -- even when they were under the impression that at last someone was willing to deliver. But despite this, the delivery usually ends up revealing itself to be little more than the leftovers or hand--me--downs of the proprietary world -- one whose grapple over peoples' lives created the very reason for free software in the first place. 

When people use GNU, they are treated as second--class citizens in many respects. Although this is becoming less and less relevant as time goes on, it is still relevant with respect to the availability of good games. It is as though we do not even exist at worst,\footnotecite[gnu_linux_support_non_existent] or considered creatures of novelty at best.\footnotecite[gnu_linux_support_as_novelty] 

A user should be able to simply play the game without having to worry about manually setting environment variables or having the right version of some support library installed to be able to enjoy it as its authors had intended.\footnote{Consistent audio support under GNU/Linux has historically been a nightmare, as one of many examples.} We believe things like this have gone on long enough and that it is about time GNU users were treated like anyone else. The technology is there, the demand is there, and now those willing to deliver on the demand are there too if they have the support they need.
%\placefigure
%    [left, 0*hang]
%    [figure:Kip_Concept_Art]
%    {Avaneya developer showcasing concept art before {\it Ubuntu Vancouver}. {\it Photo courtesy Reggie.}}
%    {\externalfigure[The_Game/Images/Kip_Concept_Art.png][][width=0.5\textwidth]}

In some sense, Avaneya is one part of the overall effort to complete the list of tasks the GNU operating system requires to bring it closer to completion. When the GNU project set out to replace the body of non--free software required for a person to use a computer, Richard Stallman once remarked that it would also require games since a complete operating system needed games too.\footnotecite[linux_and_gnu] \quote{\it Even games are included in the task list--and have been since the beginning. Unix included games, so naturally GNU should too. But compatibility was not an issue for games, so we did not follow the list of games that Unix had. Instead, we listed a spectrum of different kinds of games that users might like.}\footnotecite[about_gnu_project] Indeed, his original announcement of the GNU project back in 1984 even called for a strategy game.\footnotecite[gnu_initial_announcement]

Presenting the case for the responsible human settlement of Mars is yet another reason. This can be done by drawing on the existing research to date and creatively adapting it for use in a non--casual game. This doubles by encouraging a greater interest in science by showing how exciting and useful it can be to people of all ages. 

But how? The decision for Mars to play a prominent role in Avaneya was neither arbitrary, nor prompted solely by the renewed interest in the Red Planet we see today. We live in a world where the consequences of tampering with, as an example, the hydrological cycle may be buffered with decades of latency, whereas in the distant and hostile world of Mars you must labour for everything from drinkable water to arable land. The latency separating cause and effect, and consequently the margin for error, is minimal and therefore the educational merit we felt at its greatest.

A final reason for Avaneya is found in the existing proprietary games themselves. They tend to appeal more to the mainstream proprietary user. They do not share the software {\it libre} community's values of freedom and not merely in how they are licensed. The authors of these games tend to be comfortable with an implicit assumption that their users prefer to be entirely detached from, and unconcerned with, anything to do with reality at any meaningful level while playing. All games certainly need not offer a remedy; that would deprive us of the escapism they can offer which is harmless in moderation. But when all of the most prevalent share that assumption, and when one considers the amount of time a person can spend playing a game, games can become social liabilities when they take more than they give.\footnotecite[wow_deaths]

No one ever broke free from prison without at least first coming to realize that they were in one. We need to be honest about our predicament if we are to improve it. Many of the things we have grown to depend on do not necessarily come from those who always have our best interests at heart.\footnotecite[adbusters_production_of_meaning]\footnotecite[santoso2008] Digital gaming is no exception. It is under appreciated how many of today's proprietary games provide little social value.\footnotecite[blackwater_game]\footnotecite[hot_coffee_mod] Authors are able to get away with this, perhaps among other reasons, because they have managed to convince users that technology is always ethically neutral -- machines being merely mechanical, value system agnostic tools that operate external of any social context, and therefore granting the author a license to never consider consequences.

This view that it is not necessary to see the forest for the trees with respect to technology is surprisingly a very popular one in the open source school of thought, in contrast to its cousin software {\it libre} -- albeit over some different issues.\footnotecite[open_source_misses_the_point] This is why Stallman once remarked that some of its advocates \quote{\it like to think that they can ignore politics. You can leave politics alone, but politics won't leave you alone.}\footnotecite[auza2008] 

Or it could be that the {\it authors} of today's multimedia technology are {\it themselves} solely to blame? This is not unreasonable to suspect. The once {\it avant--garde} digital subcultures of software cracking that flourished in Europe and the United States during the 1980s, which later paved the way for much of the world of interactive entertainment to come, including computer games, was not either. These subcultures were, with rare exception, not motivated by concerns for social justice.\footnotecite[extras={, p.~18.}][wasiak2012] This is understandable, given that their world was still a young one with the frontier still yet to be explored. We cannot say this today.

Case in point, of all the contemporary interactive games today that integrate themes of terrorism and counter--terrorism, and surely there are too many to count, of what proportion staged terror? Are such games not of interest to users, or simply not to those authoring them? Perhaps in the end, it is as though all users are losing. The {\it libre} user is treated as though they were irrelevant. The proprietary user provided with ample, yet still unstimulated with the authors limiting their experiences to only the most base of themes.

Although a good game can depart from reality, a better one might complement it. We must embrace progress not only in protecting a user's right to run, study, redistribute, and modify the non--casual games we author, but also in our willingness to examine the world we live in that ultimately finds its way into it in some form or another. The truth is sometimes stranger than fiction, and we believe that this can be a great start for a great game.

Commercial proprietary technology developers create technology with the {\it profit motive}\index{profit motive} in mind. While it is important that commercial free software be sustainable too, the {\it purpose motive}\index{purpose motive} is at least as fundamental. This is just one reason among many why free software tends to be of a higher degree of craftsmanship.\footnotecite[free_software_reliability]\footnotecite[free_software_reliability2] Sometimes proprietary software, software which holds its users hostage to its vendor, can potentially even risk getting people killed when users cannot fix problems in mission critical environments.\footnotecite[campion_smith2012] Free software developers want to build powerful technology too, but they have a tendency to not ignore the social context of what they create the way other technologists sometimes do. When we ignore the social context, technologists can become socially regressive,\footnotecite[newman2014_microsoft_women_computers_weddings_and_babies] absurd,\footnotecite[planck2013_silicon_valley_food] or even dangerous when their superficial values are left to run their course.\footnotecite[rotten_fruit2011_death_toll]\footnotecite[black2012]\footnotecite[rotten_rotties_game]\footnotecite[rotten_fruit2011_alarm_over_suicides]\footnotecite[brew2012]\footnotecite[rotten_fruit2011_stand_24_hours]\footnotecite[rotten_fruit2011_refused_benefits]\footnotecite[wilson2014_attention_span_electrocution]

The truth is that all technology has always had a social context. It can never exist suspended independently in a vacuum. Free games can take a different approach by not ignoring the social context, by not authoring software as though such an ideal were possible, and by not taking the user for a fool as though they were best modelled as reductive abstract economic units, {\it consumers}, there to be exploited. Instead, it recognizes that they are living human beings where technology can play a responsible role in their lives. 

We end with the consideration that the choice between using free and proprietary software is not the same kind of decision as deciding between chocolate and vanilla. These kinds of decisions say nothing about what the user values. When users choose free software over non--free software for the purpose of freedom, they become different from proprietary users in their values, not merely in the software they prefer. Therefore, a non--casual game for users who value freedom ought to strive beyond merely a distinction in copyright conditions and also reflect the progressive themes of {\it libre} culture in spirit as well. If you understand this, you will understand why we enjoy working on this game.

\StartSection{Classification}
People tend to struggle to classify Avaneya. It is what it is, but we might start with at least the traditional labels of the classic city builder social simulation, real time strategy, and cooperative multiplayer.

\StartSection{Likely Users}
The game so far has attracted a fairly large base of followers. From what can be observed at this time, it appears to appeal to those with an interest in:

\startitemize[4]
\item
{\it games that with unique themes;}
\item
{\it software libre;}
\item
{\it science fiction;}
\item
{\it and social simulations.}
\stopitemize

The game may take place in the future, but it deals with contemporary issues. Readers will be in a better position to determine more about who makes up our future audience by examining \in{chapter}[Leitmotifs] and \in{section}[Resources For Everyone].

\StartSection{Unlikely Users}
Avaneya is a {\it sui generis}. It is not like other games, and thus it is not for all people. It does not try to be, nor will it ever be. Those seeking a casual game, and there is nothing wrong with that, will probably not enjoy this one. There are already many, so that need not be our aim here.

This game will challenge users to think and possibly even offend them. As already mentioned, it challenges the consensus of reality, and therefore, potentially, peoples' world views. As such, some have accused Avaneya of being a vehicle for culture jamming and political commentary. We are guilty. 

\StartSection{Availability}
We do not believe that deliberately {\it restricting} users to a specific platform or hardware is ethical for any other reason than practical technical limitations of the alternatives, such as the absence of a programmable shader interface under {\it any} license. We do, however, strongly {\it promote} a specific platform, that being GNU and any available hardware it can run on capable of providing for our technical needs.

Having said that, since we consider it unethical to encourage people to use non--free software, we will probably not ourselves ever be the primary maintainers of ports to non--free platforms. However, it would also be unethical to deliberately design it in such a way so as to cripple efforts of others in their ports. With freedom comes also the freedom to disagree with us, and since Avaneya relies on portable libraries, it should not be unreasonable for someone to generously bring our work to another platform if they wish to.

Other GNU distributions that provide free alternatives to Linux as the kernel we consider fair game. These include GNU/kFreeBSD, GNU/Hurd, GNU/NetBSD, and possibly others, depending on the level of maturity of their software interfaces, technical feasibility, and demand from our community. As Linux continues to deviate from the free (and even open source) philosophy,\footnotecite[linux_non_free] it becomes increasingly important to encourage and support alternative choices as they become viable. We should note though that at the time of writing, at least some of these alternative are still far from the level of maturity ready to host a game like Avaneya.

Some have suggested that maintaining for free platforms can be rather self--limiting in terms of user outreach. There was a time when this view had validity. Now, with countless users worldwide running some flavour of GNU, such as Ubuntu, the perception of a non--existent or marginal demographic is rather antiquated. Consider that gaming under non--free operating systems, such as OS X, is still active and lively today, yet Ubuntu alone, not including other GNU distributions, has arguably surpassed it in popularity with an unregistered user--base numbering in the tens of millions.\footnotecite[ubuntu_surpasses_osx] GNU long ago stopped being merely a novelty operating system for hobbyists and academics. It is so ubiquitous, you will find it everywhere from the parliaments of our world to telecommunications satellites in orbit connecting your calls.

\StopChapter

