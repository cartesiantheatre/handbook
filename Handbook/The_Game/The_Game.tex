% This is part of the Avaneya™ Project Crew Handbook.
% Copyright (C) 2010-2017 Cartesian Theatre™ <info at cartesiantheatre dot com>.
% See the file Copying for details on copying conditions.

% The Game chapter...
\StartChapter{The Game}

\StartSection{Why?}
Too often people come out of experiencing a great work of dystopian science fiction only to say to themselves, Thank goodness we don't live in {\it that} world. That needs to change. Dystopian science fiction may be among the most honest kind of story--telling when examined in the context of history. But unless its patrons can identify with it, analogies will remain vague, metaphors cryptic, and whatever message the author might have intended gets lost.

\placefigure
    [right,0*hang]
    {Captured by Viking orbiter I on 22 February 1980. \index{Valles Marineris}{\it Valles Marineris} is the scar on Mars over 3000 km long and up to 8 km deep.}
    {\externalfigure[The_Game/Images/Valles_Marineris.png][][width=.5\textwidth]}

Dystopian science fiction when well--received offers a lesson. It does this is in being only partially fictional. To write high fantasy is a challenge, but a different one from hard science fiction. In high fantasy there are no experts in fire breathing dragons to count among its patrons or critics simply because they don't exist. In hard science fiction, however, patrons and critics often actually {\it are} engineers, areologists, astronomers, biologists, chemists, and geologist. This is the major difference between hard science fiction and other fictional genres -- including soft science fiction.

Science is the study of reality. And yet fiction, by definition, is not as concerned with reality. Science fiction therefore has a special role as a mediator between the two -- often acting as the {\it avant--garde} for both,\footnotecite[shedroff2012] with dystopian hard science fiction positioned closer towards reality.

A need for an element of reality can set the bar high. This requires the game designer to strike a balance between attention to detail and excessive pedanticism. A certain degree of creative license is expected, but patrons of {\it hard} science fiction still prefer it to be at least consistent with what they already know to be true; what is reasonable or plausible to consider true, or what might one day become true.

Isaac Asimov, a man who needs no introduction, once described this problem in realizing that \quote{\it the role of a writer is hard, for on every hand he meets up with critics. Some critics are, I suppose, wiser than others, but there are very few who are so wise as to resist the urge to show off. Critics of science popularization always have the impulse to list every error they can find and trot them out and smile bashfully at this display of their own erudition. Sometimes the errors are egregious and are worth pointing out; sometimes the critic is indulging in nitpicking; and sometimes the critic inadvertently shows himself up.}\footnotecite[extras={, p.~73.}][asimov1981]

When done well, hard dystopian science fiction of every kind can be stimulating. This is one of our reasons for creating this game. But there were others.

There are disappointingly very few {\it libre} games being actively developed for GNU.\footnotecite[xdg_games_inactivity] Even fewer, if any, {\it free} as in freedom\footnotecite[what_is_free_software] commercial games for the GNU operating system. Even though there are many non--free games available for it, they all ignore the issue of freedom.\footnotecite[nonfree_drm_games_on_gnu_linux] This is important to recognize because it was this issue of freedom that the GNU project set out to address. If we simply wanted an operating system that did not cost anything, which certainly can be socially beneficial for many,\footnotecite[straight2013_ubuntu_for_homeless] options have long been available to those willing.\footnotecite[piratebay]

Users are typically left with a bad port of an old game reliant on deprecated APIs, poorly packaged,\footnote{Assuming the platform's native package manager was even used at all.} and integrating horribly into the user's desktop environment. The installers, launcher, and other software ignore useful community driven specifications like {\it freedesktop.org}, among many others.

When people use GNU they can be treated as second--class citizens. Although this is becoming less and less relevant as time goes on, it is still relevant in considering the availability of professionally developed games. It is as if to some vendors GNU does not really exist at worst,\footnotecite[gnu_linux_support_non_existent] or is considered a creature of novelty at best.\footnotecite[gnu_linux_support_as_novelty] GNU users have had to be content with this situation for many years, despite numbering over 40 million worldwide.\footnotecite[ubuntu_statistics] They have a small selection of free games for their platform or non--free hand--me--downs of ports arriving years after they ceased being relevant.

Avaneya is part of the overall effort to complete the list of tasks the GNU operating system required to bring it closer to completion. When the GNU project set out to replace the body of non--free software required for a person to use a computer, Richard Stallman once remarked that it would also require games since a complete operating system needed those too.\footnotecite[linux_and_gnu] \quote{\it Even games are included in the task list--and have been since the beginning. Unix included games, so naturally GNU should too. But compatibility was not an issue for games, so we did not follow the list of games that Unix had. Instead, we listed a spectrum of different kinds of games that users might like.}\footnotecite[about_gnu_project] His original announcement of the GNU project back in 1984 even called for a strategy game.\footnotecite[gnu_initial_announcement]

And then there was the third reason. The decision for Mars to play a prominent role in Avaneya was not arbitrary, nor prompted solely by renewed interest in the Red Planet today. We live on a planet where the consequences of tampering with its natural cycles may not reveal themselves for decades. In the distant and hostile world of Mars one labours for everything from drinkable water to arable land. The latency separating cause and effect is minimal, narrowing the margin for error, and therefore creating a backdrop of great pedagogical and entertainment value.

\StopChapter

