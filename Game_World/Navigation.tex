% This is part of the Avaneya Project Crew Handbook.
% Copyright (C) 2010-2017 Cartesian Theatre™ <info@cartesiantheatre.com>.
% See the file Copying for details on copying conditions.

% Navigation section...
\StartSection{Navigation}

As we saw in \in{section}[Desert Threats] it is not difficult to get lost on Mars with so many ways to choose from. It is important to know on how to navigate. This section will be especially valuable for those designing a single player mode.

A nautical mile is defined as one minute of arc, or \fraction{1}{60} of a degree. On Earth this is 1,852 m. A convenience a smaller Mars offers over Earth is distances expressed in Martian nautical miles are almost equivalent to a kilometre (983 m).

There are several methods of navigation that can be used to determine one's location. The first is simple dead reckoning. Dead reckoning works by figuring out where you are based on your relation to prominent landmarks, such as a ravine, mountain, or what have you. 

Consider an example. Imagine you could see {\it Olympus Mons}, the highest volcanic mountain on Mars that towers at about 22 km. You manage to figure out your distance to it based on some simple trigonometry that takes into account the angle the horizon makes with respect to the mountain's peak, assuming you already know its elevation. 

Now assuming you know which way is north, your bearing and distance with respect to {\it Olympus Mons} determines your position with reasonable precision on a map with just a ruler and pencil. No sophisticated electronic equipment is necessary to do this. This process as described has not changed for centuries.

Of course, dead reckoning assumes that there is always a landmark to take a bearing off of. {\it Arcadia Planitia} is a flat windswept desert whose prominent landmarks that stick out of the horizon number in the few. But even if you had them, there is still the problem of determining the direction of north. 

Mars most likely is still volcanically active and still has a molten core. It does not, however, appear to spin around its core which is why there are only small localized magnetic fields and not at a global level. We know this because of {\it Mariner 4's} flyby. This is why a magnetic compass is useless on Mars. But if you had a clock you could find the direction of north based on the position of Phobos, the Sun, or Deimos as references. On Earth you have only the Sun as a reference whereas on Mars you have three.

That is fine for identifying north during the day but obviously useless during the pitch black night where the Sun is on the other side of the planet. In this situation celestial navigation is of great use. Take a look at \in{figure}[figure:celestial_navigation].

\vfill
\placefigure
    [here,force]
    [figure:celestial_navigation]
    {Positions of \alpha\ Cephei circled in red and Deneb in fixed rotation about the Martian northern celestial pole on the left. Seven hours later on the right. Images courtesy of {\it Stellarium}.}
    \startcombination[2*1]
    {\externalfigure[Game_World/Images/Celestial_Navigation.png][][width=.5\textwidth]}
    {}
    {\externalfigure[Game_World/Images/Celestial_Navigation_7_Hours_Later.png][][width=.5\textwidth]}
    {}
    \stopcombination
\vfill

Study the images carefully. You will note that both \alpha\ Cephei and Deneb are \quote{fixed} in the sky with respect to the \index{celestial pole}celestial pole, the axis the planet rotates on. Both stars are on opposite sides of the celestial pole and about equidistant from it. This means that celestial north is always found in almost exactly the half way point between these two stars.

Now that you have a bearing on celestial north, you need to know your latitude and longitude. The latitude can be determined by using a sextant\index{sextant} which is used to measure the distance between the celestial pole and the horizon. Sextants have been used for centuries by Terran sailors. 

The next step is to determine your longitude by once again comparing a clock with Phobos, the Sun, or Deimos as references. Once you have done that, you have both your latitude and longitude.

But this is all as a last resort, as it can be rather crude in an electronic age. Most of the time one should have access to a Mars Positioning System (MPS) receiver whenever dismounted or in a vehicle. 

Transponders would be of additional benefit too because the circuitry necessary to interrogate their signal is very simple and does not require the use of a satellite. They can be used to create unmarked roads and airways by installing or air dropping them on the ground at minimal expense. On--board vehicle artificial intelligence can then use them to self--navigate.

