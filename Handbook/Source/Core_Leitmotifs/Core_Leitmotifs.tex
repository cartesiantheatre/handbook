% This is part of the Avaneya Crew Handbook.
% Copyright (C) 2010, 2011
%   Kshatra Corp.
% See the file License for copying conditions.

% Core Leitmotifs chapter...
\StartChapter{Core Leitmotifs}

This is probably the most important chapter in the entire book in helping you decide whether this is a project for you to be involved in or not. 

Avaneya has several recurring ideas or {\it leitmotifs}\index{leitmotif} it communicates frequently to its users. The following form the core leitmotifs the project feels are important. As you consider each one, given in no particular order, you are also evaluating Avaneya itself in the process.

We would like to keep the leitmotif set as minimalistic as possible, with everything else we feel that is relevant, it being already possible to derive from the core set. However, if you still feel the set should be modified in some way, please bring it up on any of the crew communication methods related in \in{section}[Communication] and be prepared to support your argument on evidential grounds.

\startitemize[R]
\setupwhitespace[big]
\item
People are not free if they are not informed.\footnote{See {\it The Equality Trust} in \in{chapter}[Resources For Everyone].}

\item
The basic premises neo-classical economics operates on, such as an infinite growth model\footnote{See {\it Adbusters} in \in{chapter}[Resources For Everyone], with prejudice for \href{http://www.adbusters.org/cultureshop/backissues/85}{Issue \#85}.} operating within a finite environment, an absence of true cost accounting,\footnote{See {\it Food, Inc.} in \in{chapter}[Resources For Everyone] as well as the \about[Genuine Progress Indicator].} and so on, are fundamentally dangerous.\footnote{See Adbuster's \href{http://anticap.wordpress.com/2010/10/25/jamming-neoclassical-economics/}{Jamming neoclassical economics} campaign.} It is not a science, but a fraudulent pseudo science acting as a vehicle for dangerously unchecked assumptions.

\item
As government increases in size, it tends to become more dangerous, less useful, and the people less free, regardless of the paradigm it finds fashionable.\footnote{See {\it Statistics of Democide: Genocide and Mass Murder since 1900 (Macht Und Gesellschaft, Bd. 2)} in \in{chapter}[Resources For Everyone].}

\item
Most major acts of terrorism are staged by government.\footnote{See {\it Debunking 9/11 Debunking}, as well as the project lead's {\it Open Letter to Minister of Public Safety} in \in{chapter}[Resources For Everyone].}

\item
Governments are generally not there to make changes to serve the public interest. They are there to keep things the same.\footnote{See {\it Report From Iron Mountain: On The Possibility And Desirability Of Peace} in \in{chapter}[Resources For Everyone].}

\item
Most of the diseases that afflict mankind are caused by diet.\footnote{See {\it The China Study} in \in{chapter}[Resources For Everyone].}

\item
The majority of the presence of the most dangerous drugs in society are attributable to government, either overtly, as in the case of the most dangerous recreational narcotic, alcohol,\footnote{See {\it Drug Harms in the UK: a Multicriteria Decision Analysis} in \in{chapter}[Resources For Everyone]} or covertly, as in the case of substances like cocaine.\footnote{\href{http://afp.google.com/article/ALeqM5j6QonBKKMo2gw1e3ql-xUcQEZbVg}{Mexico Drug Plane Used for CIA rendition Flights. }{\it Associated Free Press}. 4 Sep. 2008.}

\item
Agriculture is our principle means of interacting with the planet. Change what you eat and you can solve many problems.\footnote{See {\it Food, Inc.} in \in{chapter}[Resources For Everyone].}

\item
Fractional reserve central banking is fraud.\footnote{See {\it The Creature From Jekyll Island} in \in{chapter}[Resources For Everyone].}

\item
Usury is the most ubiquitous, sophisticated, and least recognized form of contemporary slavery.\footnote{{\it Ibid.}}

\item
Bestowing corporations with the rights of human beings was dangerous.\footnote{See {\it The Corporation} in \in{chapter}[Resources For Everyone].}

\item
There is no such thing as human nature, only human behaviour.\footnote{See \about[Socioeconomic Modelling].}

\item
Everything is connected and nothing happens in a vacuum.\footnote{We think this is reasonable to take axiomatically.}
\stopitemize

If you have gotten thus far, and you are still comfortable being involved in this project, then you will probably find it rewarding. Otherwise, there is no sense in being here. No one is forcing you to do anything. This is not a project for everyone and there are countless other community driven projects that could probably use your talents.

But regardless of whatever {\it you} choose to do, creativity is required whenever we present the aforementioned. We should also always try to require the user to do some thinking of their own to arrive at these conclusions themselves. As the Buddhists say, {\it you cannot teach a person anything. They can only teach themself}.

\StopChapter

