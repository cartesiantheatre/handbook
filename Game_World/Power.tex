% This is part of the Avaneya Project Crew Handbook.
% Copyright (C) 2010-2017 Cartesian Theatre™ <info@cartesiantheatre.com>.
% See the file Copying for details on copying conditions.

% Power section...
\StartSection{Power}

There are several different options for generating electrical energy on Mars. Some are ideal for vehicles while others are best suited for non--mobile applications. Some are more practical than others. We will explore these options in this section.

\StartSubSection{Methanol / Oxygen Fuel Cells}
Methanol (\chemical{CH_3OH}) / oxygen fuel cells are ideally suited for use in ground vehicles or even as the integrated power supply in an EVA suit. They have a long engine life due in part to their low temperature. Methanol is also safe to handle. 

A bonus is that the reaction produces water as a byproduct. About 45\% of what goes into the fuel cells comes out as water. That is, for every 2 kg of methanol reacted with 3 kg of oxygen, you will produce 2 kg of drinkable water which is enough for two days. In a survival situation you could add another three days on top of the initial two because it takes three more to die of dehydration.

\StartSubSection{Microwave Power}
Microwave power works by having a nuclear fission reactor on the ground somewhere generating electricity. This energy would be transmitted by microwave from the surface to a satellite in areostationary orbit at an altitude of 16,600 km. The satellite would then transmit this energy back to a receiving station on the surface for local redistribution.

It may sound appealing, but it has its issues. Every time the microwave energy is transmitted, it loses half of its total available energy. Since it goes up and then comes down before reaching its destination it is reduced twice. This means it has only 25\% efficiency at best.

But even if it had 100\% efficiency, if the satellite ever veered slightly out of alignment due to any orbital or attitude corrections that it makes, the resulting inaccuracy could be catastrophic because microwave radiation can kill. Considering the distance is some 16,600 km away it would take only a very modest adjustment to veer as much as several kilometres off course.

\StartSubSection{Dynamic Isotope Power System}
A dynamic isotope power system (DIPS), sometimes called a radioisotope thermoelectric generator, works by converting the thermal radiation of a piece of nuclear material into electrical energy via a process known as the \index{Seebeck effect}{\it Seebeck effect}. This process requires a temperature differential in order to work.

Many existing spacecrafts, such as the Viking landers and orbiters, carried these generators to supply themselves with modest amounts of long--term, reliable, electrical power. They require no moving parts, but their power output will degrade as a function of time.

This is a useful device, but it does not provide much energy. Sometimes an accompanying deep--cycle battery can be used to buffer more energy to handle peak loads like during a transmission. This was how the problem was managed with both Viking landers.

The choice of radioactive materials used have both longevity and economic considerations. Plutonium--238 (\chemical{\high{238}Pu}) has a half--life of 88 years, but is very expensive. Strontium--90 (\chemical{\high{90}Sr}) or caesium--135 (\chemical{\high{135}Cs}) are cheaper alternatives that have half--lives of 30 years. These latter two can be acquired from nuclear waste.

\StartSubSection{Hydrazine Emergency Power}
Hydrazine reserves can be used to supply a spacecraft, aircraft, or ground vehicle with emergency power for short periods of time. A spacecraft could have several of these hydrazine based auxiliary power units (APUs). See \in{section}[Hydrazine Fuel] for more on this type of resource.

\StartSubSection{Hydroelectric Power}
Hydroelectric power is, for obvious reasons, not presently an option on Mars. With efforts at terraformation that could change. Mars once had a vast network of oceans and rivers. What happened to all of the water is still a mystery, but we do know that large quantities of it are still locked up underground and at the poles.

\StartSubSection{Nuclear Fission}
Nuclear fission reactors are common on Earth. They have unlimited mileage and are very efficient for large scale power use. They could be useful for the first settlers backed with substantial government funding. An initial 4,000 kg reactor might produce 100 kWe of electrical energy and 2 mW of thermal process heat. This thermal energy could be used to drive other endothermic reactions used to synthesize materials on Mars, heat buildings, or for other purposes.

The drawbacks are of course well known. They can be politically unpopular, are very expensive, and can be catastrophic in the event of a meltdown or other disaster.

\StartSubSection{Geothermal Power}
Geothermal energy works by tapping into hot artesian aquifers and extracting heat from them or possibly even converting it into electricity by having it drive a turbine. Besides energy, geothermal wells could also supply a colony with liquid water. It would still need to undergo desalination.

We know that Mars is still volcanically active and therefore still has a molten core. We know that Mars contains enormous stores of water. We also know that some liquid saline actually makes it to the surface from time to time. The {\it Mars Global Surveyor} has even identified signs of flowing water that appear to be recent in the {\it Cerberus} region near the equator of Mars. Even if this happened 10 million years ago this is still considered \quotation{recent} in geological time. Hot thermal wells are almost guaranteed to exist on the planet. This makes geothermal energy an attractive possibility.

Consider that on Earth our cities were erected first. Knowledge of geothermal came after and sometimes even by thousands of years. On Mars we have the advantage and convenience of the inverse. We can build settlements where we already know there to be hot thermal wells.

There is a relation between well depth, power output, and consequently the cost of installation. Locations with viable aquifers will vary in cost depending on how far down the drilling rig needs to bore. Even then it is still easier to drill deep on Mars than on Earth because the lower gravity reduces the amount of energy required to lift regolith out of the ground, in addition to having compressed it less.

Geothermal typically has a very low cost per kW hour to operate. A comparison done in 1996 determined that the cost of geothermal electricity was between 3--10¢, fossil fuels at 4--6¢, hydroelectric at about 3¢, burning of biomass at about 5¢, nuclear fission at about 5¢, tidal at least 8¢, solar thermal at about 9¢, photovoltaic between 25--35¢, and wind between 6--15¢.\footnotecite[fogg1996]

For more on the relationship between well depth and temperature on Mars see Zubrin's work.\footnotecite[extras={ p.~211.}][case_for_mars] For more on the research that has gone into considering geothermal energy on Mars see Fogg's work.\footnote{Fogg, {\it op cit.}, p.~404.} See DiPippo's for the underlying theory of how geothermal power generation works.\footnotecite[dipippo2012]

\StartSubSection{Solar}
The process of converting solar energy into electricity through photovoltaic panels has an excellent reputation on Earth. Unfortunately these panels are not as useful on Mars. This is because Earth receives about two and a half times the solar flux Mars does.

But even if it received the same, \index{dust storms}dust storms reduce what reaches the surface of the panel to only a tenth of its total potential for weeks -- or even months at a time. These dust storms usually occur around the time of the planet's perihelion -- when it would otherwise be getting the most sunlight. This is the position in its orbit when it is closest to the Sun. The planet receives 45\% more solar flux at its perihelion than when at its furthest, the aphelion. 

Even if dust storms did not occur, regular Martian winds would reduce the panels' effectiveness through the gradual deposition of very fine iron oxide dust. The panels would need to be brushed off regularly if they are to be of any use, even on a clear and bright afternoon.

At night time photovoltaic panels are useless for obvious reasons. A settlement would need another supply of power during the night.

The panels are expensive both to build as well as to replace. Even the best panels available on the market today still do not produce very much power. Nevertheless, they are useful for low power applications, mounted on vehicles, or in emergencies.

