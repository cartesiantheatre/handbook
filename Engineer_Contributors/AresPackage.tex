% This is part of the Avaneya Project Crew Handbook.
% Copyright (C) 2010-2017 Cartesian Theatre™ <info@cartesiantheatre.com>.
% See the file Copying for details on copying conditions.

% AresPackages subsection...
\StartSection{AresPackages}

\StartSubSection{Purpose}

The AresEngine uses a custom game archive format built using EBML, an extensible language akin to XML, but tailored to handling binary data. The Matroska multimedia container is its most prominent client.

You might be wondering why a game engine requires its media to be delivered to it in a custom archive format, as opposed to being exposed \quote{naked} directly through the platform's native file system. There are a number of benefits to using a custom archive format, as well as EMBL specifically.

\startitemize[4]
\item
The EMBL format enables extendability for future format changes.

\item
It can be easier to distribute a few files containing many, than many to the end user.

\item
As a single file (*.AresPackage) potentially containing many files, file seek, open, and load times are reduced.

\item
The EBML format already has a portable interface.

\item
Providing a layer of abstraction between the actual raw data and the client that requires it allows us to decompress compressed data on the fly. This results in a smaller file, which means a faster disk to RAM transfer. Remember that the disk is slow, while the CPU is much faster.
\stopitemize

Keep in mind that we might drop this idea. People have raised concerns that this may be a waste of time and it would be more straightforward to leverage an existing format, like a simple zip archive. Feedback is welcome.

\StartSubSection{Usage}
The {\tt ares-package} tool takes a package manifest file (XML) describing the contents of the package to output. The tool archives the requested files, along with whatever settings that may be required, and outputs the package.

\StartSubSection{Design}
A package manifest is checked against an XML schema ({\tt AresPackage.xsd}) to verify that it is syntactically correct. {\tt AresSamplePackage.xml} is a sample package conforming to the {\tt AresPackage.xsd} schema. Let's take a look inside of it.

\startCodeExample
<?xml version="1.0" encoding="UTF-8" ?>
<AresPackage Name="Title" SchemaFormat="1" Thumbnail="../Title/Artwork/Logo.png">
    <Files NamePrefix="Title/">
        <File Source="../Title/Scripts/Script.lua" Name="Title.luac">
            <Parameter Name="Compile" Value="true" />
        </File>
        <File Source="../Title/Artwork/*.png" Compress="0">
            <Parameter Name="Gamma" Value="0.5" />
        </File>
        <File Source="../Title/Artwork/Splash.mkv" Compress="0" />
    </Files>
</AresPackage>
\stopCodeExample

The following diagrams are a graphical representation of the schema, beginning with the root node.

\FullPageDiagram
    {figure:AresPackageManifestSchema_Root}
    {The schema for the {\tt AresPackage} root node of a manifest.}
    {Engineer_Contributors/Images/AresPackage/Root.svg}

\FullPageDiagram
    {figure:AresPackageManifestSchema_Files}
    {The schema for the {\tt Files} element of an AresPackage manifest (multiple File).}
    {Engineer_Contributors/Images/AresPackage/Files.svg}

\FullPageDiagram
    {figure:AresPackageManifestSchema_File}
    {The schema for the {\tt File} tag of an AresPackage manifest.}
    {Engineer_Contributors/Images/AresPackage/File.svg}

