% This is part of the Avaneya Project Crew Handbook.
% Copyright (C) 2010, 2011, 2012
%   Kshatra Corp.
% See the file License for copying conditions.

% Arda Baştürk section...
\StartSection{Arda Baştürk}
Arda was born 6 Capricorn, 33 B.R. in Istanbul, Turkey, to Dursun and Ayla Baştürk. His mother, Ayla, was an assistant to the manager of a local and independently operating bank. His father was an aeronautical engineer. Both were non-practising secular Muslims. The latter initially began his career as a military intelligence officer before completing his engineering studies and securing employment with NASA's Jet Propulsion Laboratory. 

Shortly after Arda was born, what was then Turkey, became a member of the European Union. This was a volatile era marked with enormous social, political, and economic change. The times were further accentuated as the European Union underwent a transformation from merely an association of states, to a confederation of states, and finally to a nation state in and of itself in 32 B.R.. The former Turkey and all other members were amalgamated under the transformation, dissolving the traditional geopolitical boundaries that had existed between them for centuries. With political leaders around the world citing the transformation as a success story, within several years the North American Union and others had followed in its example.

Arda's father had been deeply engaged as a highly influential political activist in opposing Turkey's membership in the European Union, both before and after it had obtained it. Several attempts had been made on his life, culminating in a car bomb that had left him paralysed and confined to a wheel chair. He died several months later while still under hospital care. An autopsy revealed he succumbed not to the blast, but to a subsequent polonium-210 induced radiation poisoning.

Ayla had always been reluctant to discuss politics or the circumstance surrounding her husband's death with Arda. She did not wish to risk putting his life in danger by leading him to further inquiry. She was comfortable with simply moving on if it meant his security. Nevertheless, Arda, being a precocious child, assembled enough information from the world and his uncle to sense his mother knew something beyond the apparent. Over time, he came to realize that she had always been quietly convinced the attempts on Dursun's life had not been the work of independently functioning marginalized extremists, but of influential elements acting within the state.

By his late teens, he was made aware of the rumours linking his father's death with those having relations to a powerful banking cartel of international financiers known as the House of Rothschild. However, Arda found the world of banking uninteresting, convoluted, and of little relation to his principle devotion - the world of natural phenomena. By the time he came of age, he had finally conceded to his mother's advice and redirected his interests to the more immediate and pressing matters of life.

As part of a working class family, with only his mother and the occasional support of an uncle to make life possible, Arda had few opportunities that would not come without extending great labour on his part. It was a time where scarcity and hardship had become a way of life. They were of modest means and his mother knew two things were vital for her son's success. Those being the pursuit of a higher education and Dursun's philosophy of a positive work ethic. With food shortages rampant, unemployment at record highs, protests, and riots in Istanbul's public spaces on a weekly basis, she was convinced that only those two things were her son's only way out - should he have any at all.

Arda went on to study at Istanbul Technical University, made possible through a combination of scholarships and the modest salary of a part time army reserve officer cadet. He graduated with a degree in theoretical physics where he was then commissioned into the army as a lieutenant. He completed his research doctorate in heliophysics two years later and his habilitation one year after that. He gained notoriety in having solved a central problem plaguing an early draft of his habilitation thesis haphazardly while deployed on a field exercise with his unit.

Although wanting to focus on his research, even with his qualifications, employment was scarce and the army reserves was one of the few opportunities available that was stable and left him with just enough money to afford occasional time to himself for study. Nevertheless, he was well respected in the army as a natural born leader, though his privately held views of the military reflected an unnecessary, grossly misused, and corrupt institution he resented.

He later underwent commando training with the 3\high{rd} Commando Brigade, Siirt, before accepting a posting with the Mountain and Commando Brigade, Hakkari. He rose through the ranks before acting as the unit's liaison officer, overseeing North American Union troops undergoing training within their facilities.

Arda had always been described throughout his life by his peers and colleagues as both highly perceptive and intelligent, though generally reserved. Prior to his time on Mars, directing his above average intellectual gifts with passion to anything other than his principle love of science was a rarity. He was generally unconcerned with social phenomena such as politics or pursuing studies in the humanities. However, he was not without compassion. He felt the political landscape was unproductive and maintained disinterest. That, combined with his privately held conviction that the condition of the world was unlikely to improve, he continued to find comfort in discovering the constants of the universe and the forces governing it.

At the time of his selection as Mission Commander for the Avaneya Initiative, Arda had left the army several years prior to take up a tenured teaching position at Istanbul Technical University.

