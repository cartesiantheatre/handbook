% This is part of the Avaneya™ Project Crew Handbook.
% Copyright (C) 2010-2017 Cartesian Theatre™ <info at cartesiantheatre dot com>.
% See the file Copying for details on copying conditions.

% Henrik Nørgaard section...
\StartSection{Henrik Nørgaard}

\placetable[nonumber,right,0*hang]{}
{
    \SetupCharacterTable

    \bTABLEbody

        \bTR 
            \bTD[nc=2] \externalfigure[Leading_Characters/Images/Henrik.png][][width=.3\textwidth] \eTD 
        \eTR

        \bTR 
            \bTD[nc=2] Henrik Nørgaard \eTD 
        \eTR

        \bTR
            \bTC Born \eTC
            \bTC 49 Aries, 32 B.R. \eTC
        \eTR

        \bTR
            \bTC Birthplace \eTC
            \bTC Copenhagen, European Union \eTC
        \eTR
        
        \bTR
            \bTC Gender \eTC
            \bTC Male \eTC
        \eTR
            
        \bTR
            \bTC Nationality \eTC
            \bTC European Union \eTC
        \eTR
        
        \bTR
            \bTC Ethnicity \eTC
            \bTC Danish \eTC
        \eTR
        
        \bTR
          \bTC Hair \eTC
          \bTC Brown \eTC
        \eTR
        
        \bTR
            \bTC Eyes \eTC
            \bTC Blue \eTC
        \eTR

        \bTR
            \bTC Age (Selection) \eTC
            \bTC 21 MYrs / 40 Yrs \eTC
        \eTR

        \bTR
            \bTC Age (Year Zero) \eTC
            \bTC 32 MYrs / 62 Yrs \eTC
        \eTR

        \bTR
            \bTC Education \eTC
            \bTC 
                \startitemize[4]
                \startpacked
                \item Cellular Microbiology
                \item Molecular Biology &\\Microbiology
                \item Orthomolecular Medicine
                \stoppacked
                \stopitemize
            \eTC
        \eTR
        
        \bTR
            \bTC Occupation \eTC
            \bTC 
                \startitemize[4]
                \startpacked
                \item Naturopathic Physician
                \item Scientist
                \stoppacked
                \stopitemize
            \eTC
        \eTR
        
        \bTR
            \bTC Mission Titles \eTC
            \bTC 
                \startitemize[4]
                \startpacked
                \item Chief Field Science\\Officer
                \item Chief Medical Officer
                \item Experimental Team Lead
                \stoppacked
                \stopitemize
            \eTC
        \eTR
    \eTABLEbody

\eTABLE
}

Henrik was born 49 Aries, 32 B.R., in Copenhagen, European Union. He attended Lund University\index{Lund University} where he completed his undergraduate education in cellular microbiology. Following that he went on to complete a Masters in Molecular Biology and Microbiology. After completing his doctoral studies he pursued research opportunities with the Max Planck Institute for Medical Research in Heidelberg\index{Max Planck Institute+for Medical Research}, European Union.

While in Heidelberg Henrik's principle area of investigation was in the institute's primary focus of alternative medicine seeking to promote recovery from illness through nutrition and lifestyle. During his time with the institute Henrik had more than 80 peer reviewed papers published on topics ranging from clinical trials involving the use of traditional herbal remedies, the biochemistry of fermented foods, and efficient strategies for the recycling of biomass.

After several years with the institute Henrik went on to acquire his status as a naturopathic physician at the University of Copenhagen while lecturing part time at the university. As a lecturer he was noted for his satirical world view, intellect, and an extensive set of mushrooms and herbs from virtually every corner of the Earth in his office.

Following his accreditation as a naturopath, Henrik went on to open a medical clinic in Copenhagen. Shortly after a period of lengthy evaluation he accepted a UNSA offer to venture to the Red Planet as Chief Medical Officer and Chief Field Science Officer (Experimental Team Lead). Henrik passed over the care of his clinic to the supervision of a close colleague in his absence.

