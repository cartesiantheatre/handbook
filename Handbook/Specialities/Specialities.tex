% This is part of the Avaneya Project Crew Handbook.
% Copyright (C) 2010-2013 Cartesian Theatre <kip@thevertigo.com>.
% See the file Copying for details on copying conditions.

% Specialities chapter...
\StartChapter{Specialities}
Avaneya is a fairly large, creative, and collaborative endeavour. It should be of little surprise then that it calls upon a variety of disciplines to produce. They are all important in their own way. It should be noted that there is nothing to stop someone from being involved in more than one capacity if they choose to. If you want to model, but also happen to be fluent in Italian, by all means consider translating as well.

In this chapter we will discuss the many different exciting ways members of our community can get involved in whatever capacity they are most comfortable with.

\StartSection{Audio Engineer}

These folks can take audio samples and creatively transform them into things usable for this game with the right mastering. An example would be sampling the squeak of a chair or a car driving by with a high end portable recorder. This sample could then be remixed into the low rumble of an idling diesel engine in the hands of a skilled audio engineer. 

These people perfect their craft with {\it libre} tools like Ardour, Rosegarden, Audacity, and portable high resolution hardware audio recorders whenever necessary.

\StartSection{Cinematic Artist}

Cinematics play an important role in games. They prepare the user and set the stage in ways that would be difficult to do during normal game play. These artists work with tools like Blender, Cinelerra, and Lombard, to name a few.

\StartSection{Conceptual Illustrator}

Conceptual illustrators prepare high quality artwork drawn from the game world. This can include environments, machines or other equipment, timeline scenes, or characters.

\StartSection{Engineer}

The engineers design, discuss, and implement the engine specification. They do not write just engine code, but sometimes write the Lua scripts that drive it. They work mostly in the languages of C++ and GLSL. 

The engineers rely on the GNU Autotools as though it were the \quote{construction scaffolding} around the project. It is useful for ensuring the software stays as versatile as possible. 

To get started as an engineer with this project, see \in{chapter}[Engineer Contributors].

\StartSection{Modeller}

Modellers produce the 3D game models the users see during game play. They work with the 2D artists to ensure models are properly textured and lit. They work with Blender, Wings 3D, or any other modelling program that supports standard patent free model formats.

\StartSection{Musician}

The musicians create either new or provide existing tracks for the game. The music falls into two categories. The first is in game ambient music that the user passively listens to. The second is music that is more actively listened to during navigation menus, cinematics, and possibly by way of a separately released game score soundtrack in redbook or some other format.

\StartSection{Package Maintainer}

Package maintainers are responsible for building and preparing Avaneya packages for the community. They need to be familiar with the standard methods of rolling out software for a given platform. In our case, we are mostly just concerned with Ubuntu. Other GNU platforms are important too, but only so far as demand and available resources allow. See \in{chapter}[Package Maintainer Contributors] for more information on this topic.

\StartSection{Quality Assurance}

Quality assurance stress tests our applications, looking for their weaknesses and trying to break them. They spot bugs as well as identify user interface and gameplay issues. They are very important for ensuring that nasty bugs are caught and helping the crew to triage resources so that we can address them before the general public suffers. They are comfortable using the Launchpad interface, with prejudice for the issue tracker, and know how to generate stack traces. This latter skill is very important because it provides critical information engineers need to replicate a problem.

\StartSection{Researcher}

These people provide the background information and attention to detail that makes the game rich. Perhaps they have an interest in areology\index{areology} (the study of Mars), terraformation, atmospheric geography, simulation and complex modelling, social, political, and economic issues (e.g. the GPI), or what have you. They have some valuable area of knowledge or expertise that they would like to bring to the table.

\StartSection{Script Writer}

Script writers produce code in Lua that drives and breathes life into the game engine. They work closely with the engineers to ensure that the API's the engine needs to expose are available to them. They also write the scripts that drive the CEGUI graphical user interface subsystem since its interface is Lua based.

\StartSection{SysOp}

System operators administer critical software that form the bridging point between what we produce and our interaction with the general public. A very large number of people are dependent on them to work properly. Sometimes they protect things from the nefarious. Other times they might be moderating forums, chatrooms, mailing lists, managing the issue tracker, and eventually monitoring and administering the Solnet cluster described in \in{section}[Multiplayer: Solnet].

\StartSection{Translator}

Translators are what makes Avaneya available to a wider audience. They ensure cinematic subtitles, graphical user interface, website, and other interactive elements are properly localized. They work with any tools that support standard GNU gettext and language catalogues, as covered in \in{section}[i18n & L10n]. They must be sensitive to the needs of a given locality which is usually, though not always, their own.

\StartSection{Typeface Designer}

Typeface designers are involved in all aspects of producing the high quality Avaneya Font Family\index{Avaneya Font Family}. This font family is used mainly at higher resolutions, the user interface, printed media, some documentation, DVD jewel case, and so on. They are familiar with the different font rasterization technologies that control the way fonts render on screen, as well as the {\it TrueType}\index{TrueType} font format and it's successor, {\it OpenType}\index{OpenType}. The fonts produced must ultimately be compatible with {\it SDL_ttf}\index{SDL_ttf}.

\StartSection{Voice Actor}

Cinematics and in game audio require real people to play a role at time. This should be fun. Voice actors would need a high quality professional microphone, such as the Zoom H2 or similar. They undoubtedly need to work closely with the audio engineers.

\StartSection{Web Developer}

Web developers are familiar with free standards. They work with technologies like HTML5, PHP, CSS2/3, jQuery, MySQL, JavaScript, and SVGA. These technologies are important because they are the best way to ensure a reliable and consistent user experience as far as one can hope for.

\StartSection{Writer}

Writers are important. They are so often overlooked, but are truly key to ensuring users get the best experience they can. They are needed wherever one must write primarily for humans and not machines. 

Writers can be either of the creative or technical kind, whatever they are most comfortable with, or both. Those more interested in creative writing are concerned more with elements of story, dialogue, characters, in game fictional literature, and other creative fictional aspects of the game. Naturally, they work closely with researchers.

Technical writers, on the other hand, are responsible for writing and maintaining technical documentation which is just as important. They work with powerful document engineering software like \BIBTEX\ and \CONTEXT\ for typesetting our literature into books like this one, editing wikis, producing man and info pages, and so on.

All writers are responsible for maintaining, editing, and producing most of the documentation related to Avaneya. They are very important for editing material for grammar, spelling, and clarity. They often work closely with researchers, script writers, and other artists as they make suggestions to each other. 

To get started as a writer with this project, see \in{chapter}[Writer Contributors].

\StopChapter
