% This is part of the Avaneya Project Crew Handbook.
% Copyright (C) 2010, 2011, 2012
%   Kshatra Corp.
% See the file License for copying conditions.

% Licensing Rationale chapter...
\StartChapter{Licensing Rationale}

Avaneya is composed of different types of files that fall into general categories. The game is licensed under multiple licenses, with a given license for each category. These categories are for {\it artwork}, {\it code}, {\it documentation}, {\it music}, and {\it typography}.

\startitemize[4]
\setupwhitespace[big]
\head {\em Artwork}

This includes all literary works, cinematics, models, audio and voice effects, textures, materials, concept art, screenshots, and other relevant non-executable creative data. 

These objects are released under the terms of the Creative Commons Attribution-NonCommercial-ShareAlike (BY-NC-SA) 3.0 Unported license, copyright \CopyrightDates\ \CopyrightHolder. See \in{section}[Creative Commons Attribution-NonCommercial-ShareAlike] for the full text of the license.

The license ensures that, under the default conditions, unless given permission, others cannot use it for commercial purposes (NC), must give attribution (BY), and if they alter, transform, or build upon it, they must distribute the resulting work only under the same or similar license to this one (SA). Everyone is free to copy, distribute, and transmit the work, as well as to adapt the work as they like. These are just the default conditions and can be waived with the permission of the copyright holder.

\head {\em Code}

This includes the AresEngine, shaders, Lua scripts, build environment scripts, and any other relevant executable data.

These objects are released under the terms of the GNU General Public License 3.0, copyright \CopyrightDates\ \CopyrightHolder. See \in{section}[GNU General Public License] for the full text of the license.

In a nutshell, this license ensures that users have four fundamental freedoms that are always protected. These are the freedom to use the software for any purpose, the freedom to change the software to suit your needs, the freedom to share the software with your friends and neighbors, and the freedom to share the changes you make. The license for most software and other practical works are designed to take away your freedom to share and change the works. By contrast, the GNU General Public License is intended to guarantee your freedom to share and change all versions of a program - to make sure it remains free software for all its users.

\head {\em Documentation}

This includes UML schematics and other design documents, doxygen output, man and info pages, this handbook, and other relevant data.

These objects are released under the terms of the GNU Free Documentation License 1.3, copyright \CopyrightDates\ \CopyrightHolder. See \in{section}[GNU Free Documentation License] for the full text of the license.

The Free Software Foundation explains the purpose of the license as being to make a manual, textbook, or other functional and useful document "free" in the sense of freedom: to assure everyone the effective freedom to copy and redistribute it, with or without modifying it, either commercially or noncommercially. Secondarily, this license preserves for the author and publisher a way to get credit for their work, while not being considered responsible for modifications made by others.

\head {\em Music}

This includes all music in Ogg Vorbis, FLAC, Speex, or other formats and associated project files.

These objects are released under the terms of their respective artists or publishers. We feel that music does not have to be free, but it is preferred that it be at least shareable. This is the position of Richard Stallman of the Free Software Foundation, along with many artists.

\head {\em Typography}

This includes the Avaneya Font Family and all source files. The Avaneya Font Family is currently still in design phase.

These objects are released under the terms of the SIL Open Font License version 1.1, copyright \CopyrightDates\ \CopyrightHolder. See \in{section}[SIL Open Font License] for the full text of the license.
\stopitemize

\StopChapter

