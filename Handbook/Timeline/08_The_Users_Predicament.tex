% This is part of the Avaneya Project Crew Handbook.
% Copyright (C) 2010, 2011, 2012 Cartesian Theatre <kip@thevertigo.com>.
% See the file Copying for details on copying conditions.

% The User's Predicament section...
\StartSection{The User's Predicament}

At this point, it is useful to step back and examine where all of this has left the user. Where the aforementioned fictional timeline ended, the user's experience carries on. The plains of {\it Arcadia Planitia} become home to a number of city--states, of which, depending on the mode of play, the user may function in any of a number of capacities -- head of state, principal among them.

This birth and erection of an abundance of city--states answered Arda's public proclamation of the dire need for an Arcadian diaspora. Arda's reasoning was the need for mutual protection from governmental, private, and corporate influences of Earth that would not rest until all of Arcadia's infrastructure, so costly to develop; rich human resources, expertise, and cultural demographics; material resources; and administrative facilities were under its control -- directly or indirectly.

Each of these cities shared a common strategic, economic, and military alliance, while simultaneously experiencing the full autonomy of independent states. Smaller outposts can exist, but they always fall within the governance of a given city--state.

The only codified legal framework these city--states share in common is the {\it Constitution of the Republic of Arcadia Planitia}. Thus, they are a federated republic of autonomous city--states, bound together by a common heritage, foundational legal framework and values, currency, and mutual strategic interests. Politically, the city is the state. Culturally, the federated republic is the nation.

Each city--state maintains its own unicameral legislative assembly; an executive branch consisting of a head of state, their cabinet; and an independent judicial branch. All public officers, judicial, executive, or legislative, since the time of confederation, are required that they be democratically elected. There is no electoral college and anyone may submit themselves directly for candidacy.

Although Arcadia is considered the capital of the republic, this is largely ceremonial. There are exceptions, however. The first of which is in carrying out the function of judicial arbitration between cities in disagreement. Secondly, its wisdom is frequently sought in an advisory capacity for historical reasons.

To understand the cultural fabric and mentality of the republic, one must understand the background of its constituents. With each {\it Tarikin} personnel ferry having shuttled five--hundred new immigrants twice every year, the first settlement, Arcadia, was replenished with additional manpower at regular intervals. This influx of people of disparate backgrounds came from every corner of the Earth, serving to enrich the settlement's cultural makeup. Since the {\it Tarikin's} arrived at regular intervals, this unintentionally prevented any already present social or ethnic group sufficient time to identify and assert itself as the settlement's natural cultural identity.

The immigrants were generally well educated progressive intellectuals, dedicated to research in their given fields -- hence UNSA's approval of their application for a visa. They were a people of botanists, geologists, engineers, physicists, and the like, and later to count among themselves, playwrites, musicians, artists, and more. They were not a people to consider themselves empire builders. Having a mind of no imperial inclinations, seeing such things as anachronistic and of pale comparison to the wonders of the frontier. However, they were convinced, though exercised with caution, of the reasonable need to seek mutual protection if they were to secure the life romanticized prior to arrival.

