% This is part of the Avaneya Project Crew Handbook.
% Copyright (C) 2010-2013 Cartesian Theatre <info@cartesiantheatre.com>.
% See the file Copying for details on copying conditions.

% Using Bzr section
\StartSection{Using Bzr}
As discussed already in \in{section}[Rationale for Bzr], we opted for Bzr as our revision control management software. We host the project's Bzr master branch on Launchpad. The branch contains the source code to Avaneya, and all relevant documentation -- including the source to the book you are reading now. Game media we are still trying to figure out.\footnotecite[launchpad2012_bzr_not_ready_digital_media]

This section contains instructions on how to interact with the repository. You can browse the master branch on Launchpad through a web browser at the following link.

\startnarrower[3*left]
\fullahref{https://code.launchpad.net/~avaneya/avaneya/trunk}
\stopnarrower

\StartSubSection{Installing Bzr Under Ubuntu}
To edit source code, you will need a local copy. A browser is fine for casual viewing, but not really a viable option as a contributor. To access the source locally, you will need to download and install Bzr. We will cover the installation on Ubuntu only, but other operating systems should have simple documentation available for them. 

Start by instructing your package manager to fetch and install the latest version of the Bzr software it has available to it.

\startCodeExample
$ sudo aptitude update
$ sudo aptitude install bzr 
\stopCodeExample

Bzr is a purely command line tool written in Python, however, there is a mature and attractive graphical user interface that integrates well into the popular Nautilus file manager for Gnome. If you would like to use it, run the following command to download and install it. 

Note that the second command will restart Nautilus to make it available without having to log out and back in again. Be careful, this will kill Nautilus so make sure it is not doing anything important like copying a file.

\startCodeExample
$ sudo aptitude install nautilus-bzr
$ killall -9 nautilus
\stopCodeExample

To {\tt identify} yourself to Bzr and in any future commits you make, run the following command. Try not to use handles, but use your actual name with proper spelling and punctuation if possible. The reason being is that we would like to maintain clean commit logs for the benefit of everyone.

\startCodeExample
$ bzr whoami "FirstName LastName <your@address.com>"
\stopCodeExample

\StartSubSection{Creation of a Local Branch}

If you do not know whether you have been granted write access to the master branch, you probably only have read access. You will know if you have write access because you are listed as a member of the \href{https://launchpad.net/~avaneya}{Avaneya Project Crew} on Launchpad.

Run the following command to retrieve a local copy of the Avaneya source to a folder called Avaneya in the current working directory. 

\startCodeExample
$ bzr branch lp:avaneya Avaneya
$ cd Avaneya
\stopCodeExample

This retrieves a local copy of the entire repository which is called a {\it branch} in the Bzr terminology. It can function entirely independent of a project's official repository ({\it upstream}). You can take it with you on the road, without internet connectivity, save changes incrementally as {\it revisions}, and even revert back to previous ones in the absence of internet connectivity.

However, if you happen to have write access to the repository, replace the word {\tt branch} in the previous example with {\tt checkout} instead. This would have looked as follows in this case.

\startCodeExample
$ bzr checkout lp:avaneya Avaneya
$ cd Avaneya
\stopCodeExample

The difference here between using {\tt branch} instead of {\tt checkout} is that the former is decentralized and not tied to our master branch. The latter, on the other hand, is centralized and tied to our master branch (write access). 

When saving changes in a branch that was retrieved using the {\tt branch} method, the changes are saved locally only. In the case of a {\tt checkout}, the changes will be saved on the remote project server if you have been given write permissions. We will discuss further how to do commits in \in{section}[Committing Changes].

\StartSubSection{Updating}
If you performed a {\tt checkout} instead of a {\tt branch} while following the instructions in \in{section}[Creation of a Local Branch], you can update your local branch to the latest revision available in Launchpad's master branch by running the following.

\startCodeExample
$ cd Avaneya
$ bzr update
\stopCodeExample

If instead you performed a {\tt branch} in \in{section}[Creation of a Local Branch], run the following substituting the {\tt branch} for {\tt pull}. Note that in either case internet connectivity is required.

\startCodeExample
$ cd Avaneya
$ bzr pull
\stopCodeExample

To view the latest {\it commit log} after either executing a {\tt pull} or an {\tt update}, run the following within the directory containing your local working copy. Hitting the {\tt q} key exits the log.

\startCodeExample
$ bzr log | less
\stopCodeExample

\StartSubSection{Committing Changes}
Go ahead and make any changes you like as you see fit. When you are done making those changes, you can then check to see a list of all the files that you added, modified, or removed that Bzr is tracking. Do this by running the following.

\startCodeExample
$ cd Avaneya/Documentation/Contributors/Handbook
$ bzr status .
added:
  Engineer_Contributors/Images/Fluid_Dynamics.png
modified:
  Makefile
  ReadMe
  Environment.tex
\stopCodeExample

The above example shows the user the status of their local branch within the current working directory. Bzr reports one file named {\tt Fluid_Dynamics.png} was added and three others were modified within the subdirectory containing the source to this handbook.

The next step is to send the changes either back {\it upstream} or locally into your own local branch. If the former, you want to make sure you update your local working copy first to verify there are no conflicts since you began your changes.

Depending on whether you have write access to the repository or not there are two methods to send your changes back upstream. If you have write access to the master branch, you can upload your changes directly into the master branch. Changes will be stored on the remote project server.

If you do not have write access, you generate what is called a {\it bundle} file which can be emailed to the public mailing list described in \in{section}[Public Discussion] for peer review. Your changes will commit locally only until a maintainer merges them into the master branch.

In either case, you need to first perform the commit like in the following example. Note that the {\tt --message} switch is used to specify the log message for the revision. If you need more than one line, omit this switch and you will be prompted to enter a multiline log entry in your default text editor.

\startCodeExample
$ bzr commit --message "Some commit log message..."
\stopCodeExample

At this point you are done if you had write access. If you did not, you need to inform upstream of a merge request so they can synchronize with your local branch's changes. You can do this by telling Bzr to generate the bundle which contains every commit you made locally, including log messages and any other book keeping details.

\startCodeExample
$ bzr bundle > ~/Desktop/MyChanges.bundle
\stopCodeExample

This will generate the bundle file on your desktop called {\tt MyChanges.bundle} which you can then email to the public mailing list described in \in{section}[Public Discussion] for peer review. The file is both machine and human readable making it mailing list friendly.

For the astute reader familiar already with unified diffs and other revision control systems, you may be wondering why the file has a {\tt .bundle} extension rather than a {\tt .patch} or {\tt .diff}. The reason for this is Bzr bundles are not vanilla unified diffs, though certainly similar. Bundles contain metadata specific to Bzr and so we do not recommend using any of the generic patch extensions to avoid confusion.


%\StartSubSection{Installing BZR in Windows}

%BZR can also be ran in a windows operating system using the command promt or a graphical user interface. The software is available from Canonical at http://bazaar.canonical.com/ and navigating to the downloads section. As Windows does not have SSH key automatically installed the software to connect your system to Avaneya's launchpad repository also requires downloading. The things are required, SSH key generation software and SSH key authentication software, both can be found in the download section at http://www.chiark.greenend.org.uk/~sgtatham/putty/.

%Depending on your needs and your hardware select the appropriate files. The first, putty.exe allows you to create a SSH key by following the steps. Make sure you save the file appropriately and securely and rememerb your passphrase for future use. Pageant.exe, the second file is the software you will requre to open before pushing commits to lp:avaneya.

%Describe key gen process
%

%Describe opening key in pageant
%Opening pageant.exe will bring up a window that will allow you to select the add key button. Using this you can navigate local files to find your saved SSH which you have either had from previous system/generation or had just generated using putty.


%BZR Launchpad Login
%To link BZR to you Launchpad account open comand prompt. Type BZR launchpad-login <username>


%Using BZR in command promt
%BZR in Windows command promt is similar to using BZR in terminal on GNU. Open command promt (search cmd) and type BZR will bring up an itial list of options allowing you to add files, make commits, and more.
%Pull files in cmd
%using command promt cd to appropraite branch. Cd to local file then first trunk of local cache.
%then just type bzr pull lp:avaneya to update all your local files.

%Commit files in cmd
%In command prompt, chang directory to changed file.
%CMD bzr add file.tex.
%CMD bzr commit -m "Notes you want noted about changes made to file go with these quotes."

%Push files in cmd
%Once you have done your commit note you are ready to push your local files to lp:avaneya.
%Simple type bzr push lp:avaneya to push all commits to the repository.
%CMD wil display its attempts to locate SSH key and appropraite software after an appropraite duratio it will display a revision number, for example 'Pushed up to revision 412.'


%Using BZR in GUI
%
%Pull files in GUI
%Commit files in GUI
%Push files in GUI


%Pageant is required to be opened and the key located each time you want to push a commit to lp:avaneya. Also don't forgot to login into launpad using BZR in command prompt. In addition your first use of BZR will require you to notify BZR and Launchpad of your launchpad ID, BZR and pageant will then link this to your Launchpad credentials contained in your SSH key stored locally and serverside. 



