% This is part of the Avaneya Project Crew Handbook.
% Copyright (C) 2010-2016 Cartesian Theatre <info@cartesiantheatre.com>.
% See the file Copying for details on copying conditions.

% Time section...
\StartSection{Time}

Time keeping in the game is different from what you are used to on Earth. This is useful to understand at least the story's chronology and the user interface. Although fictional Terrans encountered in the game would likely use their native system, for the sake of consistency and to minimize confusion to the user, all fictional future dates are represented in the Arcadian format described below.

\StartSubSection{Prime Meridians & Timezones}

% Timezones...
A prime meridian\index{prime meridian} is an imaginary longitudinal line of reference on a planet that is used as the basis for offsetting its timezones\index{timezones}. On Earth, this is the Coordinated Universal Time (UTC)\index{Coordinated Universal Time} overlying Greenwich, London. On Mars, the crater {\it Airy--0}\index{Airy--0} had been used as the prime meridian since 1969 A.D.. This latter convention was established based on early photographs from the {\it Mariner 6} and {\it 7} missions\index{Mariner missions}.

After asserting their independence, Arcadians moved the prime meridian they were using of {\it Airy--0} to their native city of Arcadia. This meant a shift from an uninhabited landmark roughly on the other side of the planet to one of greater cultural significance. Even with the change there still remained 24 timezones as before. The difference was that the zones would now have to be offset relative to Arcadia. Arcadian clocks are therefore set to {\it Coordinated Mars Time\index{Coordinated Mars Time}\index{MTC}} (MTC) since Arcadian independence.

\StartSubSection{Seconds, Minutes, Hours, Days, and Years}

% Explain seconds, minutes, and hours...
A Martian second, minute, and hour are still the same as their Terran counterparts. That is, a Martian hour is still the same as 3600 seconds on Earth.

% Explain solar day...
A {\it sol\index{sol}} is a solar day. This is the Martian analogue to a Terran day, but slightly longer by 2.7\%. This conversion means that there are 1.027491 Terran days for every Martial sol. Arcadian clocks \quote{time slip}\index{time slip} at midnight for 39 minutes, 40 seconds to makeup for the difference. This allows its residents to use the 24--hour clock they were accustomed to for centuries on Earth.

% Explain year...
A Martian year, abbreviated {\it MYr} and pronounced \quote{m--year}, contains 668.6 sols, or 689 days by a terrestrial metric. That means there are 88,775.245 seconds in a Martian year with 1.8876712 Martian years for every Terran year (1 year, 320 days, and 18.2 hours). These years are enumerated relative to a temporal datum of their choice, Arcadia's declaration of independence (\in{chapter}[Timeline]).

% Solstice and Equinox subsection...
\StartSubSection{Solstice and Equinox}

A solstice happens twice a year when the Sun's apparent position in the sky over Mars, like Earth, reaches its northernmost or southernmost extremes. The equinox also happens twice a year when the tilt of its axis is inclined neither away from nor towards the Sun, but laying directly within the plane of the planet's equator.

% Calendar subsection...
\StartSubSection{Calendar}

The colonists use a different calendar than on Earth for both cultural and practical reasons. The cultural reasons are two fold. The original Arcadians were predominantly secular and the Gregorian calendar's year zero has a theological significance to many. The other reason being that they desired a calendar which was meaningful in the context of their own history.

The practical reason is simple. The Gregorian calendar is useless in a world where the orbital period, seasons, and length of a solar day are different than those of Earth. The calendar's months, for instance, are not meaningful in the absence of the natural cycles of the Earth's Moon. As a consequence, they needed a calender that is meaningful to the physical idiosyncrasies of their own world.

Had an attempt to use the equipartitioned Terran months been made on Mars, one would realize they do not work because the Martian orbit has a high degree of orbital eccentricity\index{orbital eccentricity}. In orbital mechanics, this is the amount an orbital path deviates from a perfect circle to form an ellipse. Kepler's second law\index{Kepler's second law} states that the line adjoining a planet with the Sun will sweep through equal areas in equal time. This means its seasons cannot be of equal length because the planet speeds up as it gets close to the Sun and then slows down again as it becomes distant.

The Arcadian solution to the problem of the \index{Martian calendar}Martian calendar was inspired by 20th century aerospace engineer, Dr. \index{Robert Zubrin}Robert Zubrin.\footnote{See \in{section}[Mars Direct] for more on Zubrin's research.} He noted that since the Martian seasons, and therefore months, are not of equal length, to be useful, the calendar must be divided into equal segments that correspond to equal angles about the Sun. This enables Arcadians to not only predict the seasons, but to also keep the familiar concept of twelve months, each one still remaining as a \math{30^{\circ}} arc (\math{360^{\circ} / 12 = 30^{\circ}}). 

Where they departed from their Terran counterparts was in the frame of reference used. In ancient times, Terrans of a geocentric world--view named the months after whatever zodiac constellation the Sun, as seen from Earth, appeared to be in. The Arcadians, now having to reconcile the fact that Earth was no longer the only place in the solar system that has humans, instead relied upon a more universal \index{heliocentric perspective, calendar}heliocentric perspective. They used the location of Mars in the zodiac as seen {\it from} the Sun, as opposed to the Sun as seen from the Earth. This is reasonable because the zodiac is physically meaningful to any planet within our solar system since all of the planets it contains lie in the same orbital plane. 

The zodiac could therefore provide the Arcadian calendar with the names of its months. The results of these calculations are contained in \in{table}[table:Martian Months].
\crlf

\placetable[force,split][table:Martian Months]{The Martian calendar of twelve months.}
{
    \bTABLE[split=repeat,option=stretch]
    \setupTABLE[column][5]
        [width=.52\textwidth,
        align=yes]
    \setupTABLE[row][each][align=center]
    \setupTABLE[5][1][align=center]

\bTABLEhead
    \bTR[bottomframe=on]
      \bTH  Month \eTH
      \bTH  L\low{s}\footnote{This is the angle Mars sweeps through with respect to the Sun as seen from the latter.} \eTH
      \bTH  First\\Sol \eTH
      \bTH  Total\\Sols \eTH
      \bTH  Characterists \eTH
    \eTR
\eTABLEhead

\bTABLEbody
    \bTR
      \bTC Gemini \eTC
      \bTC \math{[0^{\circ}, 30^{\circ})}\eTC
      \bTC 1 \eTC
      \bTC 61 \eTC
      \bTC Beginning of the year begins with the vernal equinox on Gemini 1, the beginning of Spring.\eTC
    \eTR
    
    \bTR
      \bTC Cancer \eTC
      \bTC \math{[30^{\circ}, 60^{\circ})} \eTC
      \bTC 62 \eTC
      \bTC 65 \eTC
      \bTC  \eTC
    \eTR
    
    \bTR
      \bTC Leo \eTC
      \bTC \math{[60^{\circ}, 90^{\circ})} \eTC
      \bTC 127 \eTC
      \bTC 66 \eTC
      \bTC Aphelion on Leo 24 or when Mars is farthest from the Sun.\eTC
    \eTR
    
    \bTR
      \bTC Virgo \eTC
      \bTC \math{[90^{\circ}, 120^{\circ})} \eTC
      \bTC 193 \eTC
      \bTC 65 \eTC
      \bTC Northern hemisphere's summer solstice on Virgo 1. Sun is at its northernmost.\eTC
    \eTR
    
    \bTR
      \bTC Libra \eTC
      \bTC \math{[120^{\circ}, 150^{\circ})} \eTC
      \bTC 258 \eTC
      \bTC 60 \eTC
      \bTC  \eTC
    \eTR
    
    \bTR
      \bTC Scorpius \eTC
      \bTC \math{[150^{\circ}, 180^{\circ})} \eTC
      \bTC 318 \eTC
      \bTC 54 \eTC
      \bTC  \eTC
    \eTR
    
    \bTR
      \bTC Sagittarius \eTC
      \bTC \math{[180^{\circ}, 210^{\circ})} \eTC
      \bTC 372 \eTC
      \bTC 50 \eTC
      \bTC Autumnal equinox on Sagittarius 1. Sun is over the equator.\eTC
    \eTR
    
    \bTR
      \bTC Capricorn \eTC
      \bTC \math{[210^{\circ}, 240^{\circ})} \eTC
      \bTC 422 \eTC
      \bTC 47 \eTC
      \bTC Dust storm season starts. \eTC
    \eTR
    
    \bTR
      \bTC Aquarius \eTC
      \bTC \math{[240^{\circ}, 270^{\circ})} \eTC
      \bTC 469 \eTC
      \bTC 46 \eTC
      \bTC Perihelion on Aquarius 16. Mars is nearest to the Sun.\eTC
    \eTR
    
    \bTR
      \bTC Pisces \eTC
      \bTC \math{[270^{\circ}, 300^{\circ})} \eTC
      \bTC 515 \eTC
      \bTC 48 \eTC
      \bTC Northern hemisphere's winter solstice on Pisces 1. Sun is at its southernmost. \eTC
    \eTR
    
    \bTR
      \bTC Aries \eTC
      \bTC \math{[300^{\circ}, 330^{\circ})} \eTC
      \bTC 563 \eTC
      \bTC 51 \eTC
      \bTC Dust storm season ends. \eTC
    \eTR

    \bTR
      \bTC Taurus \eTC
      \bTC \math{[330^{\circ}, 360^{\circ})} \eTC
      \bTC 614 \eTC
      \bTC 56 \eTC
      \bTC End of year on Taurus 56. \eTC
    \eTR 
\eTABLEbody

\eTABLE
}

As an example, a date expressed by an Arcadian might be {\it 7 Virgo, 32 A.R.}. This is the seventh day of the fourth month in the Arcadian calendar in the thirty--second year following Arcadia's declaration of independence. The {\it A.R.} suffix following the year stands for {\it after the republic} (\in{chapter}[Timeline]).

% Conversion of Martian & Terran Years subsection...
\StartSubSection{Conversion Between Martian & Terran Years}

Whenever we make note of a length of time in years, we must be careful whenever it is not clear from the context the type of year that we are using. This is important since the length of a year is the time taken by a given planet to complete an orbit around our Sun. To convert Martian years (MYrs) to Terran years, use the simple equation in \in{formula}[formula:Convert to Terran Years].

\placeformula[formula:Convert to Terran Years]
\startformula
Y_t = Y_m \times 1.8876712
\stopformula
\startlegend
\leg Y_t \\ Terran years \\ \\
\leg Y_m \\ Martian years\\ \\
\stoplegend
\crlf

To go the other way around and express Terran years in Martian years (MYrs), just use \in{formula}[formula:Convert to Martian Years].

\placeformula[formula:Convert to Martian Years]
\startformula
Y_m = \frac{Y_t}{1.8876712}
\stopformula
\crlf

