% This is part of the Avaneya Project Crew Handbook.
% Copyright (C) 2010-2013 Cartesian Theatre <kip@thevertigo.com>.
% See the file Copying for details on copying conditions.

% Nayana Rai section...
\StartSection{Nayana Rai}

\placetable[nonumber,right,0*hang]{}
{
    \SetupCharacterTable

    \bTABLEbody

        \bTR 
            \bTD[nc=2] \midaligned{\rotate[rotation=42]{\color[red]{\bft Todo: Character's image goes here.}}} \eTD 
        \eTR

        \bTR 
            \bTD[nc=2] Nayana Rai \eTD 
        \eTR

        \bTR
            \bTC Born \eTC
            \bTC 32 Virgo, 25 B.R. \eTC
        \eTR

        \bTR
            \bTC Birthplace \eTC
            \bTC Amritsar, Punjab, India \eTC
        \eTR
        
        \bTR
            \bTC Gender \eTC
            \bTC Female \eTC
        \eTR
            
        \bTR
            \bTC Nationality \eTC
            \bTC Asian Union \eTC
        \eTR
        
        \bTR
            \bTC Ethnicity \eTC
            \bTC Punjabi \eTC
        \eTR
        
        \bTR
          \bTC Hair \eTC
          \bTC Black \eTC
        \eTR
        
        \bTR
            \bTC Eyes \eTC
            \bTC Green \eTC
        \eTR

        \bTR
            \bTC Age (Selection) \eTC % 19 Scorpius, 11 B.R.
            \bTC 13 MYrs / 26 Yrs \eTC % (11 + (18+318)/669)
        \eTR

        \bTR
            \bTC Age (Year Zero) \eTC
            \bTC 25 MYrs / 47 Yrs \eTC
        \eTR

        \bTR
            \bTC Education \eTC
            \bTC 
                \startitemize[4]
                \startpacked
                \item Cellular Microbiology
                \item Chemistry
                \item Computer Science
                \item Space Agronomy
                \item Terraformation
                \stoppacked
                \stopitemize
            \eTC
        \eTR
        
        \bTR
            \bTC Occupation \eTC
            \bTC
                \startitemize[4]
                \startpacked
                \item Professor of Space\\Agronomy
                \item Scientist
                \stoppacked
                \stopitemize
            \eTC
        \eTR
        
        \bTR
            \bTC Mission Titles \eTC
            \bTC
                \startitemize[4]
                \startpacked
                \item Greenhouse Team Lead
                \item Terraformation Team Lead
                \stoppacked
                \stopitemize
            \eTC
        \eTR
    \eTABLEbody

    \eTABLE
}

Nayana was born 32 Virgo, 25 B.R., in Amritsar\index{Amritsar}, Punjab\index{Punjab}. The state previously belonged to the Republic of India\index{India} until its amalgamation and absorption into the Asian Union\index{Asian Union} as the former became a signatory to the Tokyo Partnership Accord\index{Tokyo Partnership Accord} in 15 B.R..

Nayana's family was fairly well off with both of her parents working in education. Her mother was a professor at a local university where she taught mathematics. Her father was a primary school science teacher.

Despite coming from a family of means, Nayana nevertheless attended the Indian Institute of Technology Bombay\index{Indian Institute of Technology Bombay} entirely on scholarships. She completed her undergraduate education in cellular microbiology in only three years with dual minors in both chemistry and computer science.

While at IIT Bombay, she was an active and accomplished gymnast. She also took on the responsibilities of president of the Mountaineering Club. During her time with the latter, she frequently led mountaineering courses with the aid of Gurkhas deep into the Himalayas. This gave her the chance to supplement her studies by examining the effects of high altitude and lower ambient atmospheric pressure on the cellular metabolism of plants outside of a laboratory setting.

Coming out of IIT Bombay, Nayana was accepted for masters studies at the Indian Institute of Technology Kanpur\index{Indian Institute of Technology Kanpur}. Her research was focused on the young field of space agronomy and terraformation, still entirely theoretical and existing nowhere beyond the laboratory and the literature it produced. She received funding to carry out much of her work from NASA's Office of Biological and Physical Research.

Wrapping up her time with IIT Kanpur, Nayana went on to pursue doctoral studies with the University of Florida in Gainesville, North American Union. She published several seminal papers on the subject of terraformation and the prospects for a viable Martian agronomy. While conducting her research, she was an instructor for several graduate level courses in the Department of Horticultural Sciences.

Receiving her PhD, Nayana remained at the \index{University of Florida}University of Florida to work both in teaching and research. She was bestowed with the Distinguished Teaching Award from the university's College of Liberal Arts and Sciences. 

In 13 B.R., she publicly declined a lucrative offer from agricultural and biotechnology conglomerate Monsento. Several of her relatives in Punjab who had worked for generations as successful farmers witnessed their crops fail catastrophically after their having sowed Monsento genetically modified seed. As a result, the families were left bankrupt. Many of her relatives ended up committing suicide by drinking a popular herbicide also produced by the same vendor. 

With sentiments of resentment, Nayana instead remained at the University of Florida with her acceptance of a tenured professorship with the Department of Horticultural Sciences. She also assumed the title and responsibilities of Director of Space Agriculture Biotechnology Research and Education (SABRE).

Nayana's laboratory with SABRE gained international recognition when it was selected on 29 Aquarius, 12 B.R. as part of UNSA's Office of the Avaneya Initiative dissemination of more than ten--thousand contracts to provide several key technologies. These technologies were to be used within UNSA's ion--drive propelled impactor, {\it Don Quixote V}. The spacecraft's objective was to initiate a Martian terraformation strategy by intercepting the asteroid {\it 52048 Varuna}, which it did so successfully on 37 Gemini, 9 B.R., with a nuclear warhead. The detonation was calculated to deflect the asteroid into a Martian collision trajectory. The asteroid, being a C--type asteroid, contained high levels of hydrated (water--containing) minerals which, if released into the atmosphere as aerosols, had been calculated to greatly increase the total available cloud condensation nuclei. This would theoretically lead to an eventual thickening of the thin Martian atmosphere, a process taking an indeterminant length of time by even the best estimates.

Nayana, at UNSA's request, was requested to participate in the Avaneya Initiative's crew selection. She took advantage of her leave while on sabbatical to undergo a battery of interviews, followed by preparatory training and selection at both Antarctica and the European Union's Huelva. 

On 19 Scorpio, 11 B.R., with crew selection complete, Nayana received a recommendation from the selection committee in the form of an offer to accompany the crew on their journey to Mars as both Greenhouse Team Lead and Terraformation Team Lead. The latter group would have minimal responsibilities initially, her primary focus ensuring that the colony has a reliable food supply. She submitted her formal letters of resignation to the University of Florida's Department of Horticultural Sciences and SABRE the same day.

