% This is part of the Avaneya Project Crew Handbook.
% Copyright (C) 2010-2013 Cartesian Theatre <kip@thevertigo.com>.
% See the file Copying for details on copying conditions.

% Getting Involved section
\StartSection{Getting Involved}
We encourage everyone in the community intrigued with the project to get involved. There is something for everyone to do, regardless of their areas of expertise and strengths. We take pride in our professional work, as all {\it libre} projects should, but we discourage elitism. We believe knowledge must be shared.

This project does several positive things for its community contributors, besides just being incredibly entertaining for them. A contributor's experiences working on Avaneya can enrich other {\it libre} projects, and vice versa. The project encourages cooperation and team work across the planet. It provides difficult, albeit rewarding, challenges of countless types and thus, for those who value it, street credit as well. It adds to one's knowledge base through direct experience where many less fortunate, actually most, could not afford or have access to guidance.

But on a larger level, it contributes to society's technological and cultural wealth in ways that superficial proprietary software cannot. This is perhaps the greatest feature {\it libre} culture offers the world. These are just some of the reasons, but there are actually many.\footnotecite[free_software_motives]\footnotecite[lakhani2005_libre_motivators]

Contributions need not be large and complex to be useful. Not everyone needs to be a programmer either. Indeed, programming is just one of the many facets of this project. There must be story writers, documentation writers, 3D modellers, graphic designers, and more for the project to see itself through to fruition. 

Contributions can come in all different shapes and sizes. It might be one line of code altered deep within the engine to repair a serious bug, a few human readable strings of the user interface translated into another language, updated music, an enhanced story line, more voice overs, some corrected typos, new textures, or improved material shaders. But {\it everyone} that makes a noteworthy contribution is listed in the game's credits and rightly so.

Even if one does not contribute, everyone is certainly welcome to monitor the master branch; idle or converse with us on IRC; and to subscribe, read, and post on our mailing lists. We will expand on how to do all of this later in \in{chapter}[Communication].

This is an exciting project. For whatever reason, since it was first announced, there has been no shortage of people expressing interest and a desire to get involved -- and not just gamers and software {\it libre} advocates, but educators, artists, musicians, scientists, activists and culture jammers, writers, and many more. This should be a good thing. 

But we recommend that no one ask to join the project unless they are actually serious. We say this with the utmost respect and gratitude. If one wishes to become involved, they can. If they genuinely come with the best of intentions, but know that they lack the time to do whatever it is that they wanted to do, then this is not in anyone's interest. Consider that the project maintainers too have limited time and resources available for cleaning up potential incomplete messes people leave behind, or assisting new comers get \quote{plugged in} to our workflow. Remember that even with the best documentation in the world, people will still have questions.

With that in mind, it was realized that the best approach is to welcome anyone as a contributor, but only after their having actually done so. This philosophy protects both the project maintainers' time and the new comer's. It also helps to ensure that the project maintains its productivity and high standards.

If you would like to get involved, the best way to do so is to take the initiative and solve a practical problem, provide something needed, or propose a solution you have thought about and are willing to implement. To get an idea of the project's immediate needs, take a look at the issue tracker described in \in{section}[Issue Tracking] to see a list of some of the outstanding issues to date after reading \in{section}[Orientation].

