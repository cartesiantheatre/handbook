% This is part of the Avaneya Project Crew Handbook.
% Copyright (C) 2010, 2011, 2012 Cartesian Theatre <kip@thevertigo.com>.
% See the file Copying for details on copying conditions.

% Arda Baştürk section...
\StartSection{Arda Baştürk}

\placetable[nonumber,right,0*hang]{}
{
    \SetupCharacterTable

    \bTABLEbody

        \bTR 
            \bTD[nc=2] \midaligned{\rotate[rotation=42]{\color[red]{\bft Todo: Character's image goes here.}}} \eTD 
        \eTR

        \bTR 
            \bTD[nc=2] Arda Baştürk \eTD 
        \eTR

        \bTR
            \bTC Born \eTC
            \bTC 6 Capricorn, 33 B.R. \eTC
        \eTR

        \bTR
            \bTC Birthplace \eTC
            \bTC Istanbul, Turkey \eTC
        \eTR
        
        \bTR
            \bTC Gender \eTC
            \bTC Male \eTC
        \eTR
            
        \bTR
            \bTC Nationality \eTC
            \bTC European Union \eTC
        \eTR
        
        \bTR
            \bTC Ethnicity \eTC
            \bTC Turkish \eTC
        \eTR
        
        \bTR
          \bTC Hair \eTC
          \bTC Brown \eTC
        \eTR
        
        \bTR
            \bTC Eyes \eTC
            \bTC Green \eTC
        \eTR

        \bTR
            \bTC Age (Selection) \eTC
            \bTC 22 MYrs / 41 Yrs \eTC
        \eTR

        \bTR
            \bTC Age (Year Zero) \eTC
            \bTC 33 MYrs / 63 Yrs \eTC
        \eTR

        \bTR
            \bTC Education \eTC
            \bTC 
                \startitemize[4]
                \startpacked
                \item Theoretical Physics
                \item Heliophysics
                \stoppacked
                \stopitemize
            \eTC
        \eTR
        
        \bTR
            \bTC Occupation \eTC
            \bTC 
                \startitemize[4]
                \startpacked
                \item Professor of Physics
                \item Scientist
                \stoppacked
                \stopitemize
            \eTC
        \eTR
        
        \bTR
            \bTC Mission Titles \eTC
            \bTC 
                \startitemize[4]
                \startpacked
                \item Mission Commander
                \stoppacked
                \stopitemize
            \eTC
        \eTR
    \eTABLEbody

\eTABLE
}

Arda was born 6 Capricorn, 33 B.R. in Istanbul, Turkey, to Dursun and Ayla Baştürk. His mother, Ayla, was an assistant to the manager of a local and independently operating bank. His father was an aeronautical engineer. Both were non-practising secular Muslims. The latter initially began his career as a military intelligence officer before completing his engineering studies and securing employment with NASA's Jet Propulsion Laboratory.

Shortly after Arda was born, what was then Turkey, became a member of the European Union. This was a volatile era marked with enormous social, political, and economic change. The times were further accentuated as the European Union underwent a transformation from merely an association of states, to a confederation of states, and finally to a nation state in and of itself in 32 B.R.. The former Turkey and all other members were amalgamated under this transformation, dissolving the traditional geopolitical boundaries that had existed between them for centuries. With political leaders around the world citing the transformation as a success story, within several years the North American Union and others had followed in its example.

Arda's father had been deeply engaged and highly influential as a political activist in opposing Turkey's membership in the European Union, both before and after it had obtained it. Several attempts had been made on his life, culminating in a car bomb that had left him paralysed and confined to a wheel chair. He died several months later while still under hospital care. An autopsy revealed he succumbed not to the blast, but to a subsequent polonium--210 induced radiation poisoning.

Ayla had always been reluctant to discuss politics or the circumstance surrounding her husband's death with Arda. She did not wish to risk putting his life in danger by leading him to further inquiry. She was comfortable with simply moving on if it meant his security. Nevertheless, Arda, being a precocious child, assembled enough information from the world and his uncle to sense his mother knew something beyond the apparent. Over time, he came to realize that she had always been quietly convinced the attempts on Dursun's life had not been the work of independently functioning marginalized extremists, but of influential elements acting within the state.

By his late teens, he was made aware of the rumours linking his father's death with those having relations to a powerful banking cartel of international financiers known as the House of Rothschild. However, Arda found the world of banking uninteresting, convoluted, and of little relation to his principle devotion -- the world of natural phenomena. By the time he came of age, he had finally conceded to his mother's advice and redirected his interests to the more immediate and pressing matters of life.

As part of a working class family, with only his mother and the occasional support of an uncle to make life possible, Arda had few opportunities that would not come without the extension of great labour on his part. It was a time where scarcity and hardship had become a way of life. They were of modest means and his mother knew two things were vital for her son's success. Those being the pursuit of a higher education and Dursun's philosophy of a positive work ethic. With food shortages rampant, unemployment at record highs, protests, and riots in Istanbul's public spaces on a weekly basis, she was convinced that those two things were her son's only way out -- should he have any at all.

Arda went on to study at Istanbul Technical University, which would would have allowed him a military draft deferment. However, although it was made possible through a combination of scholarships, they were not enough and his modest salary as a part time army reserve officer cadet, offered conditional upon his a voluntary entry into the army, was necessary. 

He graduated with a degree in theoretical physics where he was then commissioned into the army as a lieutenant. He completed his research doctorate in heliophysics two years later and his habilitation one year after that. He gained notoriety in having solved a central problem plaguing an early draft of his habilitation thesis haphazardly while deployed on a field exercise with his unit.

Although wanting to focus on his research, even with his qualifications, employment was scarce and the army reserves was one of the few opportunities available that was stable and left him with just enough money to afford occasional time to himself for study. Nevertheless, he was well respected in the army as a natural born leader, though his privately held views of the military reflected an unnecessary, grossly misused, and corrupt institution he resented.

He later underwent commando training with the 3\high{rd} Commando Brigade, Siirt, before accepting a posting with the Mountain and Commando Brigade garrisoned in Hakkari, European Union. Rarely able to afford travel independently, this was the next best option. During his stay with the unit, he was deployed several times with to various locations within the Himalayas to undergo advanced, high altitude, and mountain warfare with 12 Gorkha Rifles, a regiment of the former Republic of India that was amalgamated during the Asian Union's unification in 15 B.R.. Upon his return, he rapidly rose through the ranks before acting as the unit's liaison officer to oversee North American Union troops undergoing training back in Hakkari.

Arda had always been described throughout his life by his peers and colleagues as both highly perceptive and intelligent, though generally reserved. His thoughtfulness and demeanour earned him the unofficial nickname of \quote{Zeno}, a title his peers had bestowed in reference to the Greek philosopher Zeno of Citium.

Prior to his time on Mars, directing his above average intellectual gifts with passion to anything other than his principle love of science was a rarity. He was generally unconcerned with social phenomena such as politics or pursuing studies in the humanities. However, he was neither without compassion nor inarticulate. He felt the political landscape was unproductive and maintained disinterest. That, combined with his privately held conviction that the condition of the world was unlikely to improve, he continued to find comfort in discovering the constants of the universe and the forces governing it.

At the time of his selection as Mission Commander for the Avaneya Initiative, Arda had left the army several years prior to take up a tenured teaching position at Istanbul Technical University where he lived there with his mother.
