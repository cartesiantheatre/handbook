% This is part of the Avaneya Project Crew Handbook.
% Copyright (C) 2010, 2011, 2012
%   Kshatra Corp.
% See the file License for copying conditions.

% Leitmotifs chapter...
\StartChapter{Leitmotifs}

This is probably the most important chapter in the entire book in helping you to decide whether this is a project for you to be involved in or not. Avaneya has several recurring ideas or {\it leitmotifs}\index{leitmotif} it adopts. The following form the core set the project feels are important to consider and reflect on, given in no particular order.

We would like to keep the leitmotif set as minimal as reasonably possible, with everything else that we felt relevant, it hopefully already being possible to derive when obvious from that which is already present. However, if you still feel the set should be modified in some way, please bring it up on any of the project crew communication methods related in \in{section}[Communication]. We are open minded, but, like us, you must also be prepared to rationally substantiate your argument.

\startitemize[4]
\setupwhitespace[big]
\item
People are never free. To be free is to be independent of influence, which no one is.\footnote{See {\it The Equality Trust} in \in{chapter}[Resources For Everyone].} We are all subject to influence, otherwise there would be no such thing as advertising. Everything is connected and nothing happens in a vacuum.

\item
All of the major axioms of neoclassical economics are wrong. People do not always have rational preference, do not always seek self maximization, and do not always act on the basis of full and relevant information.\footnote{See {\it Adbuster's} \href{http://anticap.wordpress.com/2010/10/25/jamming-neoclassical-economics/}{Jamming neoclassical economics} campaign. and especially \href{http://www.adbusters.org/cultureshop/backissues/85}{Issue \#85}.}

\item
Most of the worst acts of violence in history were carried out by governments.\footnote{See {\it Statistics of Democide: Genocide and Mass Murder since 1900 (Macht Und Gesellschaft, Bd. 2)} in \in{chapter}[Resources For Everyone].}

\item
Most acts of terrorism are staged by government.\footnote{See {\it Debunking 9/11 Debunking} and the {\it Open Letter to Minister of Public Safety} written by a member of this project in \in{chapter}[Resources For Everyone].}

\item
Actions of governments are generally to protect a handful of its stakeholders who are not elected.\footnote{See {\it Report From Iron Mountain: On The Possibility And Desirability Of Peace} in \in{chapter}[Resources For Everyone].}

\item
Most diseases that afflict mankind are rooted in diet.\footnote{See {\it The China Study} in \in{chapter}[Resources For Everyone].}

\item
The presence in society of the most dangerous of recreational narcotics are usually due to government, either overtly, as in the case of ethanol,\footnote{See {\it Drug Harms in the UK: a Multicriteria Decision Analysis} in \in{chapter}[Resources For Everyone]} or covertly, as in the case of cocaine.\footnote{\href{http://afp.google.com/article/ALeqM5j6QonBKKMo2gw1e3ql-xUcQEZbVg}{Mexico Drug Plane Used for CIA rendition Flights. }{\it Associated Free Press}. 4 Sep. 2008.}\footnote{\href{http://www.youtube.com/watch?v=UT5MY3C86bk}{Former LA Police Officer Mike Ruppert Confronts CIA Director John Deutch on Drug Trafficking}. {\it YouTube}. 15 Nov. 1996.}

\item
Agriculture is our principle means of interacting with the planet. Changing what we eat solves many global problems.\footnote{See {\it Food, Inc.} in \in{chapter}[Resources For Everyone].}

\item
Usury is slavery. This includes fractional reserve banking.\footnote{See {\it The Creature From Jekyll Island} in \in{chapter}[Resources For Everyone].}

\item
Corporations should not have the rights of humans.\footnote{See {\it The Corporation} in \in{chapter}[Resources For Everyone].}

\item
Very little of what we call human nature is actually genetic. Genes specify human needs, not behaviour.\footnote{See \about[Socioeconomic Modelling].}
\stopitemize

If you have gotten thus far, and you are still comfortable being involved in this project, then you will probably find it rewarding. Otherwise, there is no sense in being here. No one is forcing you to do anything. This is not a project for everyone and there are countless other community driven projects that could probably use your talents.

But regardless of whatever {\it you} choose to do, creativity is required whenever we present the aforementioned. We always try to require the users to do some thinking of their own to arrive at these conclusions themselves. As Socrates once said, {\it I cannot teach anybody anything, I can only make them think}.

\StopChapter

