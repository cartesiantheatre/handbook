% This is part of the Avaneya Project Crew Handbook.
% Copyright (C) 2010-2018 Cartesian Theatre™ <info@cartesiantheatre.com>.
% See the file Copying for details on copying conditions.

% Internationalization and Localization section...
\StartSection{i18n & L10n}

The acronyms {\it i18n} and {\it L10n} are old cryptic industry terms for \quote{internationalization} and \quote{localization} respectively. Internationalization is the process of adapting software so that it can accommodate different localities. This is implemented by engineers.

Localities, or sometimes just called locales, specify data specific to a given locality, region, or culture of the world. This usually includes the user's preferred language, writing direction, currency, and numeral system. This is provided by people with locale specific knowledge.

\startitemize[4]
\head {\em GNU gettext}

GNU gettext is used to implement {\it i18n} by providing C functions for string substitution or \index{translation}translation into any available human language. Strings in code are marked and substituted appropriately at runtime based on the user's preferred locale.

\head {\em lua--gettext}

The lua--gettext API is a Lua package that acts as a Lua wrapper for gettext bindings.

\stopitemize

