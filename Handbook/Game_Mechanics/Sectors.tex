% This is part of the Avaneya Project Crew Handbook.
% Copyright (C) 2010-2017 Cartesian Theatre™ <info@cartesiantheatre.com>.
% See the file Copying for details on copying conditions.

% Sectors section...
\StartSection{Sectors}

Depending on whether the user is playing single player or multiplayer, the game can take place in any one of a number of different {\it sectors}\index{Sector}. Sectors are physical geographical subdivisions of Mars that constrain most of the player's activities. Grouping the planet this way is useful for a number of different practical reasons.

For one, it allows us to highlight through isolation the different physical characteristics of the planet. Contrast a sector within {\it Valles Marineris} with one set in {\it Arcadia Planitia}. We would expect them to have very different appearances, geography, climate, lighting, and maybe even some resources not found in the other. This allows experts to apply their knowledge of specific regions of familiarity in their design.

A more practical reason is in the server architecture. Sectors allow a logical grouping of machine resources based on a virtual region on Mars. A server cluster could manage a single sector while a cluster of clusters the entire Red Planet with minimal disruption to the entire system if any one node goes down.

Server administrators would also have a useful virtual metric for scaling costly resources that make a large online multiplayer experience possible. When a sector within {\it Arcadia Planitia} approaches the maximum user activity it can host in the virtual space it encompasses or with the physical computing resources available it is time to add additional resources to keep the service running. This incremental approach allows us to gently scale the backend to meet user demand as necessary with more CPU cores, disk, RAM and cryptographic hardware accelerators.


