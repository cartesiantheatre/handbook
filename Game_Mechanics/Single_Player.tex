% This is part of the Avaneya Project Crew Handbook.
% Copyright (C) 2010-2018 Cartesian Theatre™ <info@cartesiantheatre.com>.
% See the file Copying for details on copying conditions.

% Single Player section...
\StartSection{Single Player}
We can start with an overview of the different modalities of play. These can be either single or multiplayer. Single player, as the name suggest, the player plays alone. Well actually with the computer and the challenges it presents. This includes tutorial, campaign, and custom modes. The player can also restore from a saved game in all three modes if they wish.

\StartSubSection{Tutorial}
In tutorial mode the player is guided through interaction with the elements of the game and the user interface by Khalid Zafar, one of the leading characters described in \in{chapter}[Leading Characters]. The aim is to train the player to the point where they are comfortable in taking on the campaign or multiplayer modes.

A tutorial which covers all of the game's aspects would be quite time consuming for the player. A balance needs to be sought between showing the player all the elements of the game and only what they need to begin having fun without continually hitting walls and becoming discouraged.

Taking into account that the tutorial is separate from the campaign, it does not need to contain very much of the storyline elements. More advanced skills can be taught through the campaign mode. Note that tutorial elements in the campaign need to be minimal since there is nothing worse than wanting to start a new game and going through all of the learning elements all over again.

Each tutorial can cover specific types of information. One tutorial could cover elements of zoning and the infrastructure that develops on it.

\StartSubSection{Campaign}
In campaign mode the player assumes the role of Arda Baştürk or any of the other leading characters as they journey through the campaign. We could either have the campaign begin from {\it Manu's} first landing on Mars or later during the Arcadian Diaspora described in \in{section}[Aftermath & Diaspora]. 

The advantage of starting from {\it Manu's} first landing is it allows the player to interact as though they were a part of the fictional timeline they may already be familiar with. The disadvantage is that this timeline is already fairly deterministic by comparison with a post--diaspora Mars.

\StartSubSection{Custom}

The custom mode allows players to select specific scenarios they would like to play without thought of commitment to a full campaign. This is useful for those who would like to explore nonlinear gameplay to focus on specific issues, such as reducing crime or optimizing public transportation. We will discuss scenarios further in \in{section}[Scenarios].

