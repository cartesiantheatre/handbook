% This is part of the Avaneya Project Crew Handbook.
% Copyright (C) 2010, 2011, 2012 Cartesian Theatre <kip@thevertigo.com>.
% See the file Copying for details on copying conditions.

% Militia & Combat section...
\StartSection{Militia & Combat}
The raising of militia and there use to protect a cities interest will form an aspect of Avaneya's gameplay. Keeping with a sandbox style of play the use of militia units will be a dynamic process involving varying degrees of player input at the players own descretion. In this section various types of militia units are discussed as well as their creation, movement, combat, and interaction. Aspects of these gameplay elements are also discussed in the {\it Interface_and_Interaction}\index{Interface & Interaction} section.

\StartSubSection{Militia Training and Units} 
If the history of strategy games has shown anything it is that people need enough diversity to keep a game interesting but having too many choices of units seems to turn people off. From another point of view the more units and options are put into a game the more it needs to be tested with there being greater room for error and user abuse. In my opinion what truelly makes massively expansive games a turnoff is that it is untested, making some units the no brainer choice and creating dead end paths which inhibit experimentation. Expansive games need to be well tested and while they can offer experimentation the fine line between creativity and pointlessness needs to be tread carefully. The idea of being able to create city options, defending it and researching however the player wants can be innovative, but paths players can take or the options they can mould, more precisely where they pour resources, needs to be kept in check, at leat initially.

This is most true for the training and use of military units which may change the tide of a games balace, especially on solnet. Militia units can be trained at there prospective academy of military building with (perhaps) the underlying social condition effecting how frequently they can be trained and the degree to which they can be trained. It may be very hard for cities with a famring/labour workforce to produce officers with knowledge in high tech areas or able to operate compelx military machinery. Militia units such as this may need to be supplied by an ally or bought/imported.

Machinery on the other hand can be produced by a cities population or imported/bought from another population. Housing and maintain this machinery may be another thing altogether and require a labour force of a certain type. **Will this machinery be useable on its own or require trained militia?




\StartSubSection{Attacking and Defending}
When a unit deemed an enemy is within line of sight and range of a unit they will engage provided they have the ability. They will also alert other units in there vicinity and through radio where applicable causing other units to engage in group tactics. Units can also be told to attack units by selecting the unit then right clicking the enemy unit, this can also be done by selecting the unit, left clicking the attack function on the user interface then left clicking the enemy unit or unit you want attacked. Depending on the units current stance the manner in which they engage the enemy will vary. As described above, various movement styles cause units to act differently towards the enemy. Also relevant is defensive or combat stances. Units can be put into aggressive, defensive, or general combat stances.

Units with certain weapons or abilites can also be given unique commands. Depending on situations they may utilize these abilites automatically but players can choose to use these at a time and in a manner that they see fit. 

Examples of unique commands may be:
\Startitemize[16]
 \item[suppressive fire]
 \item[draw fire]
 \item[ignore enemy fire]
 \item[grenade or various explosives]
 \item[charge]
 \item[hide]
 \item[take cover]
 \item[destroy cover]
 \item[assist]
 \item[mantle]
 \item[dismantle]
 \item[force reload]
 \item[regroup]
 \item[repair]
 \item[medical attention]
 \item[call fire support]
\stopitemize



\StartSubSection{Attacking and Defending Cities}
While units movement in general will be the same, moving through hostile terrain can be a time consuming process. This is multiplied by moving through enemy cities or cities held by another player. City streets hold lots of hiding places and spots for cover, but also hide ambushes, militant civilians, and multiple story buildings. Fast movement through city scapes leaves flanks open and areas unturned. 

On Solnet, due to its MMO gameplay, this is very relevant. While players cities can be attacked when they are offline they will not be able to be comepletely destroyed/captured due to the nature of street by street warfare. The aim is to make attacking and defending cities a gradual and expensive process (as it is) so that a player can log onto Solnet once per week and still defend his city or contueing the attack on someone else. 

With this in mind infantry movement through city streets will take a gradual, street by street clearence approach unless being ordered otherwise. The aim is to make attackign cities a long, drawn out affair with little strategic room for error by the attacker, or by defender reclaiming lost land/streets/buildings. Some methods in which this can be accomplished may be:
 
Multiple rooms and stories in buildings which require clearing and where fighting can get bogged down.
 Civilians left behind or angry which will fight back or cause ambushes.
 Media and civil unrest at attackers homeland due to civilian casualties.
 The need to import resource for attacking units.
 Ease of defender sending in resource and units due to proximinty.
 Large city scape extrapolates issues. 
 Day night cycle effecting combat conditions and strategic styles.
 Line of sight conditions making long city streets vulnerbale, requiring slow movement.
 General expseniveness straining alliances and creating vulnerabilities in attackers cities. 




\StartSubSection{Artillery}
%This section will deal with mechanics of explosive warfare and its balance on Solnet.

\StartSubSection{Unit Noise Exposure}
When units move, shoot, and generally perform actions they make noise, the degree of this noise may alert other units to their presence. A method of display noise is a sound wave indicator at the locations vicinity and on the minimap. Focusing on the location will play the sound for players attention.
