% This is part of the Avaneya Project Crew Handbook.
% Copyright (C) 2010-2017 Cartesian Theatre™ <info@cartesiantheatre.com>.
% See the file Copying for details on copying conditions.

% Communication section...
\StartSection{Communication}
\StartSubSection{Satellite Uplink With Earth}
Communicating with Earth is unfortunately not without high latency. It takes between 4 to 20 minutes to communicate between the two planets, depending on the position of their orbits. When communication is brokered by satellite the connection will be lost when its orbit takes it to the far side of the planet. Therefore (transluminal) realtime communication between the two planets is not possible.

\StartSubSection{Mars Enhanced Telecommunications Orbiter}
Personal communication across the surface of Mars is best done by relaying through the {\it Mars Enhanced Telecommunications Orbiter} (METO), or perhaps another satellite. Provided line of sight with the satellite is available, high bandwidth realtime applications over UHF are possible.

\StartSubSection{Personal Citizen Band Radio}
When METO is not available, personal communication on the surface can be done with a 2 m antenna working in the 144 MHz band (VHF). Line--of--sight communication is required. Note that this becomes useless at ranges in excess of 40 km due to the planet's curvature because Mars is smaller than Earth.

\StartSubSection{Shortwave DX Radio}
In the event that neither METO nor line--of--sight communication is possible, shortwave radio reflected off of the ionosphere ({\it DX radio}) is a cheap and viable backup. DX radio works by reflecting radio signals off of the ionosphere in the shortwave radio band (3 to 30 MHz). 

The {\it Mariner 9} and {\it Viking Mission} orbiters and landers provided us with a great deal of information on the Martian ionosphere so we are able to predict reasonably how well this kind of communications equipment would function on Mars. We know its ionosphere consists of about 90\% \chemical{O_2} cations, so we this should be possible to do.

The effectiveness of DX radio is dependent on the electron density peak in the ionosphere. This determines the total usable frequencies. This is directly proportional to the square root of the electron density. On Earth this is as high as 20 MHz which provides a great deal of bandwidth. On Mars this will vary based on the time of day. During the day the peak electron density of 200,000 \math{e/cm^{3}} at an altitude of 135 km. This provides 4 MHz of bandwidth. At night this density is reduced down to only 5,000 \math{e/cm^{3}} at an altitude of 120 km, affording only 700 KHz of bandwidth. That is pretty low for video but sufficient for voice and any small engineering telemetry that compresses well.

One of the added benefits of DX radio on Mars is that, unlike on Earth, signals suffer very little attenuation. This is because there is less interference on a world where there are no distant thunderstorms and far fewer radio stations.

