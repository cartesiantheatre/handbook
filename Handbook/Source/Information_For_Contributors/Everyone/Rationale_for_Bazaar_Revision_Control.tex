% This is part of the Avaneya Project Crew Handbook.
% Copyright (C) 2010, 2011, 2012
%   Kshatra Corp.
% See the file License for copying conditions.

% Rationale for Bazaar Revision Control subsection...
\StartSubSection{Rationale for Bazaar Revision Control}
Some ask why we had not initially chosen a distributed revision control system (DRCS), one class of source control management systems (SCM). SCMs are used to allow multiple people to collaborate over a shared set of files, track revisions and logs, and a number of other things. There are many free programs that allow this, but they can generally be grouped into two categories or paradigms based on how they are expected to be used.

DRCS are akin to peer-to-peer software where they can be used in the absence of a central canonical server. Proponents argue users are better able to work productively when not connected to a network, most operations are much faster since no network is involved, and more. Probably the strongest point raised is it allows participation in projects without requiring permissions from project authorities, and thus arguably better fosters a culture of meritocracy instead of requiring {\it committer} status. Software that implements these include Mercurial, Git, Bazaar, Monotone, Darcs, and others. This approach has been popularized by the open source movement in recent years, as it captures the {\it bazaar} approach to software development (think of the Persian marketplace).

CRCS, centralized revision control systems, are akin to peer-to-server model. They have a single canonical repository on a single server. Proponents argue it is more straightforward to contribute to, work is better coordinated, has a more approachable learning curve, backups are more straightforward, and has been around longer. CVS, Subversion, and many others implement this approach. This approach has been popularized by the free software movement, as it captures the {\it cathedral} approach to software development (think of a central coordinator).

Many people had suggested we use Bazaar because it has a feature that Subversion, what many are use to, does not, DRCS. It is clear that it is perfectly capable of the distributed approach, but that should not be characterized as a feature any more than the specific colour of a car is a feature. It is not a feature, but a preference. Nevertheless, we ended up settling with Bazaar because it can function in the DRCS approach, is the only SCM supported by Launchpad, integrates well with modern graphical GNU desktop environments,\footnote{As an example, check to see if your distribution carries the {\tt nautilus-bzr} package.} and improves on the features Subversion supports. Besides, it was time to try something new and learning something new is usually not a bad idea.

