% Terminology used chapter...
\startchapter{Terminology Used}

The following is a collection of terminology used throughout the handbook and the game world. They range from the everyday colloquial to the technical. Some of it is factual and some of it is fictional (e.g. the @i{jenya}). Getting familiar with the terminology before you read on might be useful.
@sp 1

% Albedo...
\startsection{Albedo}
The proportion of incoming light reflected. It is the ratio of reflected light to total incident light. That is, a material with an alebdo of 1 is a perfect mirror whereas a material with an albedo of 0 absorbs all incident light.
@sp 1

% Bancor...
\startsection{Bancor}
A global currency proposed in the early 1940s by the British economist John Maynard Keynes. On 13 April 2010, the @dfn{Strategy, Policy and Review Department of the International Monetary Fund} published a paper @dfn{Reserve Accumulation and International Monetary Stability}@footnote{@url{http://www.imf.org/external/np/pp/eng/2010/041310.pdf}} recommending the world adopt the bancor and create an international central bank to administer it.
@sp 1

% Endothermic reaction...
\startsection{Endothermic Reaction}
A chemical reaction that requires energy to be put into it in order to react. An example would be cooking an egg. The egg doesn't change unless you apply heat to it. This is contrasted with an @emph{exothermic reaction}.
@sp 1

% Exothermic reaction...
\startsection{Exothermic Reaction}
A chemical reaction that releases energy as part of its reaction. An example would be burning wood. As the wood changes, it releases energy in the form of mostly heat.
@sp 1

% Genuine Progress Indicator...
\startsection{Genuine Progress Indicator}
The GPI is a system of true cost accounting which is intended to be a replacement to the GDP, gross domestic product, sometimes called the gross national product, GNP. The GPI works by taking into account all costs of an activity to society and provides a net balance sheet. The latter, on the other hand, only functions as an income sheet by tabulating the total amount of goods and services produced in a year. 

An example where the difference between the two is well illustrated is in society's relation with the tobacco industry. The GDP accounts for the value of all cigarettes sold, a dollar figure greater than zero. Its functionality is limited more or less to that of an income sheet.

Conversely, the GPI, like the GDP, would also account for the total value of all cigarettes sold. Where it differs is it then goes on to subtract the dollar figures socialized to everyone in the form of health costs, fires, garbage collection, environmental toxicity, and the deaths of approximately 5,000,000@footnote{See @i{Merchants of Doubt: How a Handful of Scientists Obscured the Truth on Issues from Tobacco Smoke to Global Warming} in @ref{Resources For Everyone}.} people every year. The GDP calculated a gain. The GPI revealed it was actually a deficit.
@sp 1

% Jenya...
\startsection{Jenya}
@dfn{Article VII} of Arcadia's @i{Rubicon Act} superannuated the bancor fiat currency with the @dfn{jenya}. It is the first currency to be backed by a predominantly rhodium standard, with the remainder by other precious metals, such as gold and silver. The jenya became the exclusive legal tender within the Republic at the time the Act was passed. This constrained all Terran interests to acquire Arcadian goods in jenyas only.

The word @i{jenya} is Sanskrit. It means @i{"of noble origin, genuine, or true wealth"}. The idea being that since rhodium is considered precious, indeed, more so than gold on both Earth and Mars, distribution of jenyas across the populace implied the distribution of real wealth to the populace.
@sp 1

% Mars Direct...
\startsection{Mars Direct}
Mars Direct is a $50 billion dollar plan proposed by an American aerospace engineer named Robert Zubrin (born April 19, 1952) as an alternative to the prohibitively costly $450 billion dollar mission to Mars proposed by NASA in consultation with its government.

The then incumbent President of the United States, George H. W. Bush, announced the government's proposal in 1989 as the @dfn{Space Exploration Initiative}. It called for the creation of the @dfn{Space Station Freedom} and a permanent Lunar base as intermediate steps for an ultimate destination to the Red Planet. If implemented, it was to be rolled out over the process of several decades.

Zubrin reasoned that it is totally unnecessary to construct giant space stations in low earth orbit, useless Lunar bases on a barren moon, and massive spacecraft carrying hundreds of people to achieve a manned mission to Mars. That, along with transporting all that is necessary to get there, survive there, and return safely. He argued that the government prefers an intentionally bloated approach because it creates the illusion of progress and productivity through countless jobs, contracts, bureaucratic expansion, and so on. But it comes at the cost of enormous waste, misdirected resources, and through increased complexity, an increased likelihood of disaster.

Zubrin compared their approach to the failed Arctic explorer, Sir John Franklin, who, with government assistance in 1845 took two ships, the @i{Erebus} and @i{Terror}, each displacing more than 300 tonnes in an effort to navigate through the Northwest Passage. His ships carried all manner of useless items, including heavy English silverware, but spared many of the critical items necessary for survival. 

The crew met a bitter end as they dragged heavy iron and oak sleds across the Arctic ice, having abandoned their ships that were stranded. With shotguns useless in the Arctic and other heavy and inappropriate equipment, all 127 men ended up perishing to the combined efforts of the elements and scurvy. It never occurred to them to take advantage of @i{in situ} resources, like fur coats, seals, and fish.

Zubrin argued that the Space Exploration Initiative's mission architecture is an absolute inverse of a sound engineering approach. He outlined cogently in his book @i{The Case For Mars} for a very reasonable, well thought out, minimalistic approach of travelling light, living off the land, and manufacturing the necessary rocket propellant for the return trip @i{in situ}. This is akin to the efforts of early Terran settlers, like those who pushed through the American Western Frontier, or virtually ever other civilization's successful effort at settling a distant land. Going to another planet is, according to him, no different. Indeed, the travel time to Mars is comparable to that of navigating the Northwest Passage.

This trans-planetary travel to Mars is possible because Mars is so opulent. It has an abundance of natural resources necessary for creating rocket fuels, water, plastic polymers, alloyed metals, glass, gasses like oxygen, semi-conductors, ceramics, and just about everything else one might require. All this, he calculated, at a fraction of the cost of NASA's proposal, and using technology that has been around since the mid-@math{19^{th}} century.@footnote{See the @i{Reverse Water Gas Shift} and @i{Sebatier} reactors in the glossary, for instance.} His plan could allegedly be realized in less than a decade with current technology, as opposed to requiring several decades.
@sp 1

% Pressure rating...
\startsection{Pressure Rating}

The mean sea level atmospheric pressure on Mars ranges from 0.3 millibars to 11.35 millibars, which is about the same as one would find at 36 km above the Earth's surface. The mean sea level atmospheric pressure on Earth, by contrast is 1,013 millibars (1.013 bars). This means that the surface pressure on Mars is only about 1% that experienced on Earth.

This has an impact on the way buildings must be engineered on Mars. The main difference between inflatable buildings is their pressure rating. The lower the pressure rating, the fabric will be thinner, the building lighter, and therefore lower in cost to manufacture. The pressure rating also determines whether you need to where a pressure suit, merely a respirator, or nothing at all.

The @dfn{Armstrong Limit} of 62.62 millibars is the lowest the human body can survive before the vapour pressure of all exposed liquids (but not liquids like blood within your skin's pressure barrier), such as tears, saliva and the liquid wetting the alveoli within the lungs exceeds that of its surrounding atmospheric pressure. They will begin to boil away at this point. On Earth, the Armstrong Limit begins at about 19 km above the surface. On Mars, it is already well exceeded at the surface.

@multitable @columnfractions .10 .15 .15 .60
@headitem Pressure

@tab Respirator

@tab EVA Suit

@tab Description

\item68 mb
@tab Needed
@tab Needed
@tab These buildings are attractive because they are economical and very light to pack, requiring fabric only 0.2 mm in thickness. For plants, they are fine since plants require only 50 mb of pressure. But for humans, they need at least 170 mb to be able to live.

\item170 mb
@tab Needed
@tab Unneeded
@tab These buildings cost a little bit more, but you can work in them without wearing a pressure suit. You still need to wear a respirator in order for the gas exchange taking place in your lungs to still work, otherwise you will quickly pass out.

\item340 mb
@tab Unneeded
@tab Unneeded
@tab These buildings cost a little bit more, but you can work in them without wearing a pressure suit or respirator, although the O₂ partial pressure levels still need to be enriched. The other main advantage is that the pressure can also be equalized with a habitat making movement easier. As an added bonus, bees can polinate better at this pressure coupled with the lower gravity which makes it excellent for greenhouses.

\item1 bar
@tab Unneeded
@tab Unneeded
@tab These buildings cost the most, but they offer at least the same pressure as on Earth. Since everything needs to be three times as heavy as it needs to be, it is a waste of resources, too costly, and unnecessary.

@end multitable
@sp 1

@sp 1

% Railgun...
\startsection{Railgun}
A means of accelerating mass to supersonic velocities by applying a magnetic field to conductive objects. The acceleration, though high enough to crush a man's skull, can be used for material exports at speeds exceeding the minimum Martian escape velocity.

Arcadians use railguns to export deuterium, geochemically rare elements, among other materials, back to Earth. It only requires them to expend @math{{1/4}^{th}} the energy to lift off of Mars as it does Earth.
@sp 1

% Regolith...
\startsection{Regolith}
What most refer to as dirt. More technically, it is the the loose heterogeneous mixture of material that blankets the solid rock of a planet.
@sp 1

% Rhodium
\startsection{Rhodium}
An elemental chemical with the symbol Rh and atomic number 45. It is a member of the platinum family and considered to be the most precious metal of that family, even exceeding the value of gold. It is also one of rarist. 

Usually the only way of getting any kind of high quantity mineral is through high-grade ore. This only happens when complex hydrological and volcanic processes have happened, which in our solar system, has only occured on Mars and Earth - hence why the Moon is barren. But unlike the Earth, Martian deposits have remained untapped.
@sp 1

% RWGS reactor...
\startsection{RWGS reactor}
The reverse-water-gas-shift reactor is a method of producing oxygen (@math{O_2}) from carbon dioxide (@math{CO_2}). This is useful because the latter is plentiful in the Martian atmosphere at 95 %.

@sp 1
@math{CO_2(g) + H_2(g) \rightarrow O_2(g) + CO(g)}
@sp 1

The process has been known since the mid 1800s and works by reacting carbon dioxide and hydrogen gasses together over a copper-on-alumina catalyst. Aqua (liquid water) and carbon monoxide gas are produced as byproducts. The aqua is split via electrolysis to produce hydrogen and oxygen gasses. The hydrogen can then be recycled back into the reactor and the carbon monoxide purged out into the atmosphere.

The reactor needs to be at @math{400\,^{\circ}{\rm C}} and at low pressure. It requires about 180 watts of power, or about 3 @math{m^2} of solar panels on a fully sunny day's average solar flux. At that energy rate, you can expect to produce about 1 kg per day of oxygen, which is sufficient for a single person. The reactor requires power because it is an @emph{endothermic reaction}. However, it is possible to use a @emph{Sebatier reactor} in tandem, which is an exothermic process, to provide the heat required to drive the RWGS reaction.

To start the process, only a small amount of water is required which acts as a reagent. By importing hydrogen from Earth, it acts to the colonists' advantage in allowing it to be leveraged in the creation of water, or hydrogen gas if needed.
@sp 1

% Sebatier reactor...
\startsection{Sebatier Reactor}
A chemical process for creating methane @math{CH_4} from @math{CO_2} and hydrogen. This is useful because carbon dioxide gas is plentiful in the Martian atmosphere at 95 %.

@sp 1
@math{CO_2(g) + 4H_2(g) \rightarrow CH_4(g) + 2H_2O(g) + heat}
@sp 1

The reactor needs to be at @math{400\,^{\circ}{\rm C}} and at low pressure. This makes it almost the same as the @emph{RWGS reactor} except that it uses a different catalyst to make methane instead of carbon monoxide. You can either use nickel, which is cheap, or ruthenium-on-alumina, which is safer, but more expensive.
@sp 1

% Sol...
\startsection{Sol}
Short for solar day, the length of time a planet takes to rotate completely on its polar axis with respect to the sun. Terrans call this a day, Martians a sol. See also @i{yestersol}.
@sp 1

% Specific impulse...
\startsection{Specific Impulse}
Written @math{I_{sp}}, the specific impulse is a useful metric for comparing rocket efficiency. Whenever you see the word "specific" in a physics context, it means something per unit of mass. The units are in seconds. It measures the amount of time that one pound of fuel will burn for, producing one pound of thrust (higher being better). This can be calculated using either SI or Imperial units, but the end result is usually expressed in seconds. 

As an example, compare the specific impulse of some of the different types of rockets.

@multitable @columnfractions .10 .45 .30 .5
@headitem @tab Rocket Type @tab Fuel @tab @math{I_{sp}}

\item
@tab Ancient Chinese Rocket
@tab Gunpowder
@tab 80
\item
@tab Modern Rocket (e.g. ICBM)
@tab Solid
@tab 250
\item
@tab Saturn V
@tab LOx / kerosene
@tab 260
\item
@tab Space Shuttle Main Engine
@tab LOx / @math{H_2}
@tab 400
\item
@tab Nuclear Thermal
@tab Solid
@tab 800
\item
@tab Nuclear Thermal 
@tab Liquid
@tab 1300
\item
@tab Jet Engine 
@tab Compressed Air
@tab 3000
@end multitable
@sp 1

Note how high the specific impulse a jet engine offers. This is because it is has an unlimited supply of free air from the atmosphere to feed the air compressor so it does not have to carry its own supply.
@sp 1

% Yestersol...
\startsection{Yestersol}
The sol preceding the current one. This is the Mars analogue to the Terran yesterday, but different since the length of a sol on both worlds is different.
@sp 1

