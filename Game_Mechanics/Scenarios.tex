% This is part of the Avaneya Project Crew Handbook.
% Copyright (C) 2010-2018 Cartesian Theatre™ <info@cartesiantheatre.com>.
% See the file Copying for details on copying conditions.

% Scenarios section...
\StartSection{Scenarios}

Scenarios are at the heart of Avaneya. The player can select specific ones through the single player's custom mode or incorporated in creative ways in the single player's campaign mode -- which is itself just a series of well scripted scenarios.

These scenarios seek not only to entertain but also to inform by drawing upon issues in the environmental, social, political, economic, agricultural, and other realms.

The user may not be the only source of influence over what happens, despite being a mayor. There are many other influences that will attempt to reshape the world. These might be transplanetary corporations or even the citizens themselves.

Scenarios can incorporate any number of goals from zero to multiple. These goals play a role in conditionally determining whether the player has successfully completed a scenario or not. In the case of the single player's campaign mode a scenario must be completed before the player can advance to the next.

Let's look at some of the general types of scenarios. This is only a subset of all the possibilities since new ones can and will be written. You will note that the Genuine Progress Indicator (GPI) described \in{section}[Genuine Progress Indicator] is a recurring theme in several of them.

\StartSubSection{No Goal} 

A scenario may be goalless. This was the most popular mode in a classic city builder simulation of the 1990s. In this scenario the player erected their city and was responsible for its management without thought of any end state. As with any other scenarios the player can still focus on whatever they like, such as maximizing their GPI, reducing crime, or responding to natural disasters.

\StartSubSection{Achieve Minimal GPI}

The player must achieve a minimal GPI. This may or may not involve time or other constraints. Although the GPI is a recurring theme in Avaneya, scenarios that require this goal focus on it more than others.

\StartSubSection{Combat Globalization}

In previous empires, whether they were Persian, Greek, Roman, or Macedonian, it was always clear to all that regardless of the rationale given for their projection, they knew they were doing it. There was no existential question of whether an empire was being created. Anyone could physically point to the Roman legion garrisoned in Germania or on some distant frontier of Britannia.

There is a process that is not particularly complex, seldom known to the general public, and yet incredibly effective in the creation of modern empires. This is done through clandestine means of manipulating the nations subject to its control. It is at least as effective as the cavalry and bayonet approach exercised in previous centuries. It works through a series of escalating measures that involve combinations of usury, bribery, political manipulation, assassination, and as a last resort, not always predictable military force. 

This process is sometimes referred to euphemistically as {\it globalization}. Through globalization it is not always necessary that this creation and expansion of empire be conspicuous. Nor is it even necessary that its benefactors, such as the people of the world's most influential nations, are even made aware that their way of life is sustained through the fruits of an empire socializing the true costs of what it reaps in the form of slavery and exploitation abroad.

\StartSubSubSection{Step One: The Economic Hit Men}
There are three steps at work in the formula. First economic hit men start by identifying a nation with the desired resources. They attempt to corrupt the nation's government officials into accepting very large loans by any means necessary. If successful, arrangements are made for the transfer by working through large international financial organizations such as the World Bank or International Monetary Fund.

The money is not actually intended for public infrastructure but is instead destined for large private corporations with close relations to the economic hit men. It is true that the loans are sometimes used for useful things such as power plants, water treatment, and industrial facilities. But in such a way so as to be principally beneficial to the corporations. Secondarily to a handful of people within the target nation. The public is then left with the responsibility of debt repayment. Unfortunately the debt is so large that it can never be repaid. This was by design.

Eventually the economic hit men are sent in again to remind the government of its debt obligations. Since the debt is so large now that it cannot be repaid, the debt is used as leverage to control them since they are now beholden to the money lenders. Under the guise of {\it quid pro quo},\footnote{From the Latin, {\it a favour for a favour}.} various remedial measures are recommended to address the debt. These measures go by any number of euphemisms, such as \quote{conditionalities}, \quote{structural adjustments}, or \quote{good governance policies}. Large public assets may have to be sold off. These may include water, utility companies, oil, education, penal and insurance systems, social services, or what have you. It may involve new policies, such as supporting a specific UN resolution, permitting the presence of foreign military bases, or possibly even deploying their own forces abroad in support of the lender's.

\StartSubSubSection{Step Two: Jackals}
If the government is still non--compliant the next step is to send in the jackals. The jackals are tasked with either assassinating or overthrowing a (usually) democratically elected head of state through whatever means necessary. This may involve staging populist uprisings, giving them the appearance of grass--roots activism. This can draw on the exploitation of well meaning individuals with the best of intentions by misleading them. The staging of popular uprisings was well illustrated in the \quote{Green Revolution} during the 2009 Iranian presidential election.\footnotecite[hersh2008]\footnotecite[ross2007]

\StartSubSubSection{Step Three: Foreign Military Intervention}
In some cases both measures fail. When this happens the third and last resort is in the use of military force. This was the case in Iraq in 1991 when Saddam refused to accept large loans and all attempts at assassination repeatedly failed. This was probably due at least in part to the proficiency he gained in conducting similar affairs while having previously worked for the CIA to assassinate the Iraqi president who preceded him.

This formula has been well documented by those directly involved in its use.\footnotecite[perkins2005] It was used successfully in Guatemala (1954), Ecuador (1981), Panama (1981), Venezuala (1981), Iraq (2002), and elsewhere.\footnotecite[zeitgeist_addendum]

\StartSubSubSection{Rationale For Private Consultants}
Private consultants are usually used as the economic hit men instead of directly employing intelligence agency personnel. This was realized as the future model in Washington after the successful overthrow in 1953 of Iran's democratically elected Prime Minister Mohammad Mossadegh by British and American intelligence. Mossadegh had attempted to remove the British monopoly of the nation's oil assets. Instrumental in the affair was a personal relative of Teddy Roosevelt who was also a CIA field agent. Had he been exposed while he was in Iran it would have been disastrous. Had a private consultant been exposed it would have been easier to contain the political fallout.

\StartSubSubSection{Game Adaptations}
One such game scenario could involve Arcadian water. Water is even more precious on Mars than on Earth since a greater expenditure of resources is necessary to acquire potable water. A corporation such as Bechtel--Biwater\index{Bechtel--Biwater} could attempt to privatize the artesian aquifers or water ice sites your city depends on. Without water one cannot survive. If one can control what everyone needs to survive they can influence the destiny of all those dependent on it. The opportunity for abuse is actually much greater on Mars than on Earth.

The player will have to use their head and be creative as they exercise their attempts at water reclamation. Some options might involve reliance on a judicial system or possibly even the use of force if the former proves corrupt.

Although readers may balk at the prospects for such a scenario, this actually played out in Bolivia in 1999. Economic hit men were so successful in corrupting the government, they actually made it illegal for citizens to collect rain water for drinking as they had for centuries. This led to a successful popular uprising, expulsion of the economic hit men, and ultimately the reclamation of their water.\footnotecite[the_corporation]

\StartSubSection{Improve Energy Supply}

Your city needs energy and it may not have enough. The player must adjust factors in their city that will improve the supply to the grid. This may mean building a nuclear plant, digging a geothermal well, or erecting more photovoltaic panels. Alternatively the player may look for ways to increase efficiency of existing infrastructure. This might mean examining how much energy is required to produce the different types of things that people eat and exploring alternatives.

\StartSubSection{Improve Public Health}

Many factors effect public health, among the most important are what the public consumes. But other factors contribute as well. Education, annual take home income, available free time, and many others are factors. Scenarios that depend on this goal require the player to improve the public's health. A scenario incorporating this goal may begin with the player's city with an over extended presence of the Clown Food franchise.

\StartSubSection{Improve Transportation Network}

As a city grows efficiency of movement becomes a problem. This goal requires the player to increase in efficiency their city's transportation network.

\StartSubSection{Natural Disasters}

The player must rebuild their city in response to damage sustained from solar flares, micro meteorite strikes, land slides, marsquakes, or other natural disasters. Yes, Mars experiences all of these. But some \quote{natural} disasters can be caused through human interaction with its surrounding natural environment. If an entire city block is swallowed whole into the ground, chances are the aquifer it was sitting on was depleted. As in reality, we are given a chance to avert a disaster entirely when we pay attention.

\StartSubSection{Prevent Staged Terror}

Staged terrorism is as common on Mars as it is on Earth. Foreign intelligence agencies will work tirelessly to stage acts of terrorism to advance whatever the agenda of the day may happen to be. They may blame their actions on arbitrary groups, including on fronts of their own creation. The player will use whatever means necessary to prevent acts of staged terrorism. If they fail to do this the NAU could claim sabotage of a mineral extraction site leased from your city as justification for military intervention to install a despot.

\StartSubSection{Protect Another City}

The player must come to the assistance of another Arcadian city in distress. This could happen when a neighbouring city comes under siege at the hands of an armed foreign aggressor, such as North American Union troops, United Nations peacekeepers, foreign intelligence backed contras, or private security contractors. Constraints may vary. Preventing the city from exceeding a certain threshold of material loss or the protection of its inhabitants are examples.

\StartSubSection{Protect From Junk Science & Media}

Democracies are only so useful as the people are informed. An uninformed people will be less capable of self determination than an informed one. We make decisions that are influenced by what we know and other environmental influences. If our information sources have conflicts of interest we can become encumbered.

The player may need to reduce the influence corporations, nefarious foreign states, and special interest groups wield in socially engineering public opinion. Their influence may be overt, such as through advertising. Alternatively it may be more subvert, such as through funding junk science,\index{Junk science} corrupting universities, or otherwise reputable scientific bodies.\footnotecite[corbett_nist_satire]\footnotecite[merchants_of_doubt]\footnotecite[the_china_study]

\StartSubSection{Protect Human Rights}

Articles XI --- XXVI of the Rubicon Act, as described in \in{section}[Civil Disobedience & False Flags], enumerate the fundamental freedoms and human rights Arcadians enjoy with respect to their new republic. The player is responsible for ensuring that these are upheld.

\StartSubSection{Recover From GMO Terminator Gene}

People that have nothing to eat will eventually die. If the city's greenhouses produce nothing after having adopted new genetically modified seed stock, chances are the greenhouse did what farmers had been doing for centuries. They stored seed and planted it, only to find they had been genetically programmed by the manufacturer to become sterile. This way the player's city's food supply is held hostage since it will have to buy new seeds annually, even though there are still plenty held in stock. Buying new stock may be a condition of having existing public debt rolled out.

Scenarios that depend on this goal require the player to restore their annual agricultural yield to a variable minimum and to ensure that the food is safe to eat.

\StartSubSection{Reduce Crime}

The player must reduce crime by addressing factors like education, affordable living, economic disparity, pollution,\footnotecite[drum2013] health, and other fundamental factors that give rise to crime.

