% This is part of the Avaneya Project Crew Handbook.
% Copyright (C) 2010-2016 Cartesian Theatre <info@cartesiantheatre.com>.
% See the file Copying for details on copying conditions.

% Game Mechanics chapter...
\StartSection{Workforce Interaction}


%--------------> WORKFORCE UNIT DECISION MAKING <------------------------------
\StartSubSection{Workforce Unit Decision Making}


%--------------> GROUP DYNAMICS <---------------
\StartSubSection{Group Dynamics}

A group is 2 or more units together within a set vicinity undertaking the same task/action. Groups can be longer term, called static, or brief and fleeting, called fluid.

%Static vs Fluid
Static groups are groups a unit will be in over an extended period of time, they will either meet with members or this group regularaly or seek them out. Examples of static groups are families, co-workers at work, and friends. 

Fluid groups are groups formed spontaneiously and that only last for a short duration or single event. Examples of fluid groups are people meeting in the street and chatting, or large group events such as protests or concerts.


%properties of a group
-Id number [group location and time]
-Units in group.
-Interests that group shares
-Variables that group shares

The group identity number is the key unique identifier of the group that is used to tag it to units associated with it.

The units in the group is a list of the unit id variable of each unit tagged into this group.

Interests and variables are properties of a group in order for the group to be compared by a unit against any other group to determine if it is dropped from the permenant list.



%Forming friendship groups / fluid groups on like interests.
Friendship groups form when units meet by chance or as part of another group and have ... number of workforce variables in common, more variables leeds to a stronger friendship. Same interests increase the friendship further. 
%Figure

%Example
%Figure

%Example
%Figure

%Example
%Figure



%Keeping friendship groups.
Each unit is only allowed ... number of static friendship groups. When this number is reached units will not have additional friendship groups unless the new group is of higher quality. The quality of friendship groups is determined by the number of interests and variables that units share, with more being of greater quality and interests being of greater quality that variables.
%Example
%Figure

%Permanent friendship groups
Units will make or attempt to make contact with well liked units they are in a friendship group in order to meet belonging need. 

The number of interests/variabls shared by the unit with others in the group determine the order in which units seek to meet up. Better quality friendship groups are sought after in descending order.
%Figure
%Example

This is the order in which groups are sought to meet belonging needs, however they do not effect how fast the belonging need is met. It is thought that while this makes people feel like they belong more, it also meens they would want to be with these people more. And since the belonging need is the key factor in determining when units seek other units as company, both attributes of a good quality group cancle eachother out to make the unit continue to seek this group out. 


%Organising group meetups and how this happens.
%Also relevant for rallies, concerts, etc.
Units can make contact with other units to meet up. They can attempt to contact 1 or more other units in the group to meet up. Methods of doing this vary and depend on what is available to each unit as well as technology that is available. There are ... broad means of communicating with units at a distance in order to meet up, these are:

Vicinity communication: this entails organising to meet up where units are in the immediate vicinity, either face to face or within ear shot.

One way communication: these methods require the organising unit to send a message, then wait for a reply. The reply usually takes some time. Examples include notice boards, email, snail mail. These can be sent to multiple or single units.   

Two way communication: these methods utilize an originating unit who sends the message to a unit or several units who recieve the message and are able to immediately reply. Examples of this type of service include phones, messenger apps, and radios (2 way radio). 

While in all three categories of communication their is a sender and receiver, the main differences between the groups are the distance the sender can be from the receiver, and the time it takes the receiver to get a message back to the sender. 

These broad groups leave the simulator open enough to allow for changes in technology but clear enough on inital construction to create the obvious, already established norms of communication being things like: phones, email, and post mail.
%Figures
%Examples
%x3



%Groups or socializing at a distance
In the same way that units can organise to meet up, they can also use technology to socialise as a group. The same drive that causes a unit to seek contact with another unit to meetup can lead to socialising over a phone or similar. The diffence is in many cases, interests shared will not be enjoyed or engaged in alongside the socialising. If two units meet up and go bowling, when the same two units communicate over the phone they will obviously not be bowling at the same time. Therefore by communicating via distance means units may not be fulfilling interests or needs while socializing.  

The aspect of socializing, that is the meeting of the belonging need, is met when the unit receives communication from another unit as a part of a group. 


%Units forming new groups from larger groups and friendships denied because of other friendship groups present.




%Family Groups


%Work Groups
%Work group and interests meeting things like friendship groups, ie if people at work have some variables/interests then its as if they are with friends.

Static groups things like work and family, remain for longer periods of time.

multiple interests alike, the units will be more frienyl and therefore make permient friend groups.
Perma group eating and another unit attends and has same interests, this unit will join group

?Larger groups require more interests the same?
?Do interests, degree or amounts of importance or involvment that draws groups?

some groups will have members come and go, like companies. 
Some groups will disolve.

can multiple groups cross over or are groups exclusive? Probably need to have cross over, or can groups be focused on current activity.

%How families are formed
%How groups are formed
%Groups being companies
%Businesses as groups


%Notes so far:
Units together doing things form group bubbles (bubbles is just a descriptor at this stage).
Units move between groups throughout day/time. 
They enter groups to fulfil needs, act on interests, or out of routine (whatever routine is).
Group 'bubbles' grow and shrink as units come and go.
Units who enter work group/business group contribute to productivity. 

Group types - as in family, work, etc:
Family, living, support
Work, company, productivity
Friends, interests, fun, belonging

%Group dynamics in emergencies or scale events.


%--------------> WORKFORCE UNIT INTERACTION <----------------------------------
\StartSubSection{Workforce Unit Interaction}
