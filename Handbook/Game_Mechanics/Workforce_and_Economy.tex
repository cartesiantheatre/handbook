% This is part of the Avaneya Project Crew Handbook.
% Copyright (C) 2010, 2011, 2012 Cartesian Theatre <kip@thevertigo.com>.
% See the file Copying for details on copying conditions.

% Game Mechanics chapter...
\StartSection{Workforce & Economy}
What a city can produce, how much it can produce, and what it is capable of producing are largely byproduct of its workforce, the production, import and export ot goods, and internal use of the goods make up the cities economy. This section discusses the variety of workforces, what they represent, how they are created and maintained, and what they are capable of. It also deals with a cities income and expenditures and how these are managed by the player.  

\StartSubSection{Wokrforce}
All objects and buildings are produced by a form of labour, what this is may change with technology or player choice, but this labour force must be able to a) exist, and b) travel to production location.

A workforce is essentially another resource of either population of technology that needs to be either educated or built on a variety of levels and scales and therefore its creation, maintainence, and control needs to be carefully managed by the player. A labour workfoce requires certain factors to bring it into fruition which may include a certain level of education, access to work, and perhaps even restrictions. An intellectual workforce requires different factors to develop then manual labour workforce. While the two are not mutually exclusive, it becomes more difficult to create multiple classes in a single city. Workforce includes all labour factors driving the economy and extends to consumers. How people are raised, educated, and conditioned through experience largely determines their consumer patterns and may be reflected in a population's consumer choices.

Workforce is a dynamic resource that can move globally. If your workforce requires cheap housing and this becomes easier to gain in another city, many may migrate to this city seeking better opportunities. Likewise, they can move within your city to different opportunities. The movement within a city can be controlled to an extent to enable better or worse access to certain locations, increasing and decreasing travel time, or diverting travel for other means (likewise a workforce of machines can be moved with expense or stationary as in a plant). Whether workfoce movement can be depicted visually be small people objects moving around the city is yet to be determined, though would be cool. 

Basicly there are different fundamental types of workforce. There is a robotic/mechanical workforce and a human/biological/social workforce. The mechanical workforce requires research and development into avenues such a robotics and mechanical engineering with a social system in place requiring either the development of these to the eventual production and acceptance or the import of these and the acceptance by the population. The human workforce alternative is made up of human labour performing the tasks required by the society. This may be accepted to varying degrees also. Cities in Avaneya may also have a mix of these two workforce systems, indeed most will. 

There will be underlying conditions that will determing what a cities workforce is comprised of and players will have certain degress of influence over the direction the workforce takes but if certain conditions aren't met the citizenry may not accept or appreciate the current workforce type with the vary consequences of innaction being varried and obvious. How players make these conditions and the direction they take with there workforce is a choice made in the sandbox style of play, it is a creation. They can create anything from human/animal labour pool similar to the Ahmish to complete nanocracies with robotic workforces and human indulging in deccandent pleasure palaces. The survival and logevity rating of their choice is to be contended.



\StartSubSection{Economy}
%All to be expanded upon massively. So much work to do here. Economy may require own section?
To begin we will discuss income streams for a players cities. These are primarily made up of taxation, exports and imports, and private enterprise. Within each of these categories are several methods which players can use to generate income.

\StartSubsection{Taxation}
Income is gained through taxation by taking a portion of what is owned by a cities people and/or businesses.

\StartSubSection{Exports and Imports}
Through exports and imports levies, fees, charges or taxation income can be gained for the city from its businesses and industries exporting goods or from other cities importing goods to the players cities. Additionally, exports increase the income of a city by bringing in wealth from other cities. Trade in general may increase the income of a city through established trade routes, larger markets, and peaceful diplomacy backed by financial gains. 

\StartSubSection{Private Enterprise}
Through private enterprise players can have government run businesses which directly generate profit or loss. If ran correctly these business are a potential income stream.

Another form of private enterprise is selling government owned resources, infrastruture, or commodities to businesses or other players.

And a third form of private enterprise is leasing or selling land to business or other cities.
