% Socioeconomic modelling section...
\section{Socioeconomic Modelling}

We attribute a person or group's destructive behaviour to their nature (genetic) when we say that there will always be some people like that in society. We make these claims with ease as if we have dedicated years of our lives to having studied and reflected on the matter.

The truth is, research strongly shows that @i{there is no such thing as human nature, only human needs}. Very little behaviour is genetic, if any. All organisms have needs, such as oxygen, water, security, a sense of belonging, community, a meaningful existence, nutrients, or what have you. Human behaviour manifests from whether and how fundamental needs are met.

This is why there is an asymmetrical spatial distribution of the problems that plague mankind. Where there is poverty, there is despair. Where there is despair, there is crime. Provide for all fundamental human needs and the problems disappear. It is neither wishful thinking, nor magic.

The evidence @dfn{The Equality Trust} has agregated powerfully reasserts this idea.@footnote{See @i{Zeitgeist: Moving Forward} in @ref{Resources For Everyone} for an excellent introduction.} It is an organization that aims to reduce income inequality through a programme of public and political education designed to achieve a widespread understanding of the harm caused by income inequality, public support for policy measures to reduce income inequality, and the political commitment to implementing such policy measures. 

Since Avaneya relies heavily on actual scientific research to substantiate the quantitative models used, @i{The Equality Trust} provides an invaluable repository of research.
@sp 1

@smallexample
@url{http://www.equalitytrust.org.uk/resources/publications}
@end smallexample

