% This is part of the Avaneya Project Crew Handbook.
% Copyright (C) 2010-2013 Cartesian Theatre <info@cartesiantheatre.com>.
% See the file Copying for details on copying conditions.

% Ubuntu Packages section...
\StartSection{Ubuntu Packages}
Ubuntu package maintainers are responsible for building and preparing Avaneya packages for the Ubuntu community. They are comfortable with standard Debian packaging conventions, their requisite structure, key signing, and any helper tools like CDBS, debhelper, and lintian that can aid with the tasks at hand. They integrate any hooks that need to execute at various times during user installation to perform required actions, such as registering a file extension or desktop launchers in a freedesktop.org compliant manner. They know how to maintain an apt repository which we may use to roll out our packages directly. When they do not know on how to do something, they know where to ask.

\StartSubSection{Dependencies}
Package maintainers must be careful to ensure our packages describe the appropriate dependencies the game actually requires to function. Without that, a user's package manager cannot make automated intelligent decisions to reconcile dependencies and the game will fail to launch when executed. 

Sometimes the Debian packaging tools can do this automatically, but other times they need help. This might mean being very careful to avoid known bugs in specific versions of some dependency. If we have to do that, we may need to include our own build prefixed within our package. This also allows us to take advantage of a bleeding edge features of an experimental API not found in a stable package.

\StartSubSection{Package List}
There are many packages that need to be maintained, such as the game's main binaries; media files, such as textures, models, audio, and cinematics; user extras, such as maybe one day a screensaver or Plymouth boot theme; documentation; debugging symbols; developer related files, and so on. Each of these different types of files should be contained within a distinct package representing that logical type. It is useful to plan these things in advance.

Take a look at \in{table}[table:ubuntu_packages] for a list of all of the proposed packages. Some packages do not actually contain anything, but are just {\it metapackages}. That is, they are just aliases for other packages.
\crlf

\placetable[force,split][table:ubuntu_packages]{List of all Ubuntu packages.}
{
    \bTABLE[split=repeat,option=stretch]
    \setupTABLE[column][2]
        [width=.65\textwidth,
        align=yes]
    \setupTABLE[row][each][align=center]
    \setupTABLE[7][1][align=center]

\bTABLEhead
    \bTR[bottomframe=on]
      \bTH  Name \eTH
      \bTH  Description \eTH
    \eTR
\eTABLEhead

\bTABLEbody
    \bTR
      \bTC {\tt avaneya-repository} \eTC
      \bTC This package should be the only one a user needs to get plugged into our repository. It provides our keyring, an {\tt avaneya.list} for {\tt /etc/apt/sources.list.d/}, Apport hooks, and Ubuntu Software Centre integration. This should make all other tasks easy for the layman if they only need this one package to get started. \eTC
    \eTR

    \bTR
      \bTC {\tt avaneya-bundle} \eTC
      \bTC Metapackage that pulls all packages the hardcore enthusiast might want. \eTC
    \eTR

    \bTR
      \bTC {\tt avaneya} \eTC
      \bTC Stripped architecture dependent binaries, such as the game engine. This will also depend on {\tt avaneya-repository}. \eTC
    \eTR

    \bTR
      \bTC {\tt avaneya-data} \eTC
      \bTC Architecture independent data files, such as cinematics, models, 
textures, icons, audio effects, and other artwork. This also includes executable scripts 
that drive the engine. \eTC
    \eTR

    \bTR
      \bTC {\tt avaneya-music} \eTC
      \bTC High definition music for the game score. \eTC
    \eTR

    \bTR
      \bTC {\tt avaneya-dev} \eTC
      \bTC Development related tools for writing scenarios. \eTC
    \eTR

    \bTR
      \bTC {\tt avaneya-dbg} \eTC
      \bTC Debugging symbols that were stripped out of the main binaries. \eTC
    \eTR

    \bTR
      \bTC {\tt avaneya-doc} \eTC
      \bTC All game documentation, including this handbook. \eTC
    \eTR

    \bTR
      \bTC {\tt avaneya-extras} \eTC
      \bTC Metapackage which pulls a collection of non--essential binaries, such as the Plymouth boot theme and screensaver. \eTC
    \eTR

    \bTR
      \bTC {\tt plymouth-theme-avaneya} \eTC
      \bTC Avaneya stylized Plymouth boot theme. \eTC
    \eTR
    
    \bTR
      \bTC {\tt avaneya-screensaver} \eTC
      \bTC Avaneya screensaver for various desktop environments. \eTC
    \eTR

    \bTR
      \bTC {\tt avaneya-restricted} \eTC
      \bTC Contains anything whose redistribution may be restricted in some localities, such as Dolby Digital surround sound encoding. See \in{section}[Restricted Data Considerations] for more on this topic. \eTC
    \eTR
\eTABLEbody
\eTABLE
}

\StartSubSection{Restricted Data Considerations}

The {\tt avaneya-restricted} package contains anything potentially useful to an Avaneya user that may be restricted in some localities. This includes the patent encumbered Dolby Digital surround sound technology.

Currently there are only two ways of getting surround sound from a computer to any kind of home theatre setup. One way is the \index{surround sound:+analog}analog method with a wire for every speaker connected to the back of the user's sound card. This makes a mess, but works for many people. The programmer does not really have to do much work to get this working because OpenAL will do most of the heavy lifting in the case of Avaneya.

A better way is the \index{surround sound:+digital}digital method. This works by passing a compressed digital stream unmodified from the source medium out to specialized dedicated hardware, usually in the form of a digital home theatre receiver (e.g. Logitech Z--5500). The receiver decodes a superior signal compared to the analog method. It then amplifies the signal before sending it out on its way to the user's speakers. 

But there is a problem to take advantage of this latter digital method with Avaneya. At present, nearly every digital home theatre receiver supports only two digital surround sound compressed formats. One of these is Dolby Digital, sometimes called AC--3 or A52, and the other is the Digital Theatre System or DTS. Both are patent encumbered. We need to be careful how we encode to these bitstreams.

The {\tt libasound_module_pcm_a52.so} library is a plugin for ALSA and therefore only available on supported platforms (usually GNU/Linux). It is a frontend for for an encoder in {\tt libav}. The latter provides a way to encode a digital surround sound stream in A52 format to a pass--through device (e.g S/PDIF). The user's home theatre receiver then takes over from there. The difficult is in this {\tt libasound_module_pcm_a52.so} library seldom being available in most distributions that provide pre--compiled packages of ALSA. This may be due to the aforementioned licensing issue. 

What we need to do is several things in order to make A52 output work. First we need to ensure at runtime that the user has the {\tt libasound2-plugins} package installed. This provides the needed {\tt libasound_module_pcm_a52.so} library. Secondly we need to either verify that the user's libav was built with the A52 encoder enabled, or fallback to one we ship ourselves somehow.

Whatever we decide to do, we need to make sure that this feature is disabled by default, due to Dolby's licensing requirement. We can leave it as an option for the user to enable if they know that software patents are not valid in their locality, but we still need to balance several constraints. These include legalities, reasonable expectations and ease of use for the user, and ease of maintenance on our part. Probably the best way to do this is to provide all of this functionality separately within an optional and properly disclaimered {\tt avaneya-restricted} package.

If we can get A52 working, there is no reason to spend time on redundant DTS support as well since they both do the same thing. Where one is supported on a given receiver, nearly without exception, so is the other.

