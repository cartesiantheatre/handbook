% This is part of the Avaneya Project Crew Handbook.
% Copyright (C) 2010-2018 Cartesian Theatre™ <info@cartesiantheatre.com>.
% See the file Copying for details on copying conditions.

% Rationale for Bzr...
\StartSection{Rationale for Bzr}
Source control management systems (SCMs) are needed to allow multiple people to collaborate over a shared set of files, track revisions, keep logs, revert to previous states, and a number of other things. There are many free programs that allow this, but they can generally be grouped into two categories or paradigms based on how they are expected to be used.

Distributed revision control systems (DRCS) are one class of source control management system. They are akin to peer--to--peer software. They can be used in the absence of a central canonical server. Proponents argue users are better able to work productively when not connected to a network, most operations are much faster since no network is involved, and more. Probably the strongest point raised is that it allows participation in projects without requiring permission from project authorities, and thus arguably better fosters a culture of meritocracy as opposed to requiring {\it committer} status. 

DRCS software includes Mercurial, Git, Bzr, Monotone, Darcs, and others. This approach has been popularized by the open source movement in recent years, as it captures the {\it bazaar} approach to software development (think of the Persian marketplace).

CRCS, or centralized revision control systems, are akin to the peer--to--server model. They have a single canonical repository on a single server. Proponents argue it is more straightforward to contribute to, work is better coordinated, has a more approachable learning curve, backups are more straightforward, and has been around longer. 

CVS, Subversion, and many others implement this approach. This approach has been popularized by the software {\it libre} movement. It captures the {\it cathedral} approach to software development (think of a central coordinator).

Many people had suggested we use Bzr because it has a feature that some centralized SCMs do not. Bzr allows both a distributed and centralized approach. Although we would argue that that really is not a feature any more than the colour of a car is a feature. It would be more fair to characterize it as a preference. But having the distributed functionality available to the community will be appreciated by some even though we believe a cathedral approach is probably more appropriate for a project like this. 

\startregister[index][bzr]{bzr}Bzr is also the only SCM currently supported by Launchpad at the time of writing. It is also easier to use than some of its popular counterparts\footnotecite[git_sucks]\footnotecite[i_hate_git] and builds on the features traditional centralized tools like Subversion support.

