% This is part of the Avaneya Project Crew Handbook.
% Copyright (C) 2010-2017 Cartesian Theatre™ <info@cartesiantheatre.com>.
% See the file Copying for details on copying conditions.

% Getting Involved section
\StartSection{Getting Involved}
We encourage everyone in the community intrigued with the project to get involved. There is something for everyone to do, regardless of their area of expertise and strengths. We take pride in our professional work, as all {\it libre} projects should, but we discourage elitism. We believe that information wants to be free.

\placefigure
    [right, 0*hang]
    [figure:UVLC_Documentation_Jam]
    {{\it Ubuntu Vancouver's} Avaneya Documentation Jam. The community bright and early 23 June 2012 working on the document you are reading now}
    {\externalfigure[All_Contributors/Images/UVLC_Documentation_Jam.png][][width=.55\textwidth]}

This project does several positive things for its community contributors, besides at least being incredibly entertaining. A contributor's experiences working on Avaneya can enrich other {\it libre} projects, and vice versa. The project encourages cooperation and team work across the planet. It provides difficult, albeit rewarding, challenges of countless types and thus, for those who value it, street credit as well. 

For many less privileged, that is most on this planet, working directly on creating new technology whenever possible can sometimes be one of the few options they have to gain new skills.\footnotecite[moss2013_homeless_coder]

But on a larger level, it contributes to society's technological and cultural wealth in ways that superficial proprietary software cannot. This is perhaps the greatest feature {\it libre} culture offers the world. These are just some of the reasons, but there are actually many motivations.\footnotecite[free_software_motives]\footnotecite[lakhani2005_libre_motivators]

%Contributions need not be large and complex to be useful. Not everyone needs to be a programmer either. Indeed, programming is just one of the many facets of this project. There must be writers, musicians, artists, and engineers. 

Contributions can come in different forms. It might be one line of code altered deep within the engine to repair a serious bug, a few human readable strings of the user interface translated into another language, updated music, an enhanced story line, more voice overs, some corrected typos, new textures, or improved material shaders. But {\it everyone} that makes a noteworthy contribution will be listed in the game's credits and rightly so.

Even if one does not contribute, everyone is certainly welcome to monitor the master branch; idle or converse with us on IRC; and to subscribe, read, and post on our mailing lists. We will expand on how to do all of this later in \in{chapter}[Communication].

For whatever reason since it was first announced, there has been no shortage of people expressing interest and a desire to get involved -- and not just gamers and software {\it libre} advocates, but educators, artists, musicians, scientists, activists, culture jammers, writers, and more. This is a good thing. 

%If you would like to get involved, the best way to do so is to take the initiative and solve a practical problem, provide something needed, or propose a solution you have thought about and are willing to implement. To get an idea of the project's immediate needs, take a look at the issue tracker described in \in{section}[Issue Tracking] to see a list of some of the outstanding issues to date after reading \in{section}[Orientation].

