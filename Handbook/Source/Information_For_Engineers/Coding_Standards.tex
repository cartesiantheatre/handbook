% This is part of the Avaneya Crew Handbook.
% Copyright (C) 2010, 2011
%   Kshatra Corp.
% See the file License for copying conditions.

% Coding standards section...
\StartSection{Coding Standards}

Try to abide by the project coding standards whenever possible. It is easier to adapt to a new coding standard in a project when that standard is uniformly applied to it, as opposed to where everyone applies their own. Use the following conventions whenever working in C++, as well as whatever is practical to carry over to shaders, scripts, and elsewhere. These are guidelines, not written in stone, so use common sense.

\StartSubSection{Legal}
Headers (*.h), implementations (*.cpp), and other code, as defined in \in{chapter}[Licensing Rationale], should have prefixed the following legal notice for engine related code. Adapt the syntax for comments as necessary for the given machine environment (e.g. shader, makefile, etc.).

\StartCodeExample
\starttyping
/*
    AresEngine, a 3D game engine.
    Copyright (C) 2010, 2011 Kshatra Corp <kip@thevertigo.com>.

    Public discussion on IRC available at #avaneya (irc.freenode.net)
    or on the mailing list <avaneya@lists.avaneya.com>.

    This program is free software: you can redistribute it and/or modify
    it under the terms of the GNU General Public License as published by
    the Free Software Foundation, either version 3 of the License, or
    (at your option) any later version.

    This program is distributed in the hope that it will be useful,
    but WITHOUT ANY WARRANTY; without even the implied warranty of
    MERCHANTABILITY or FITNESS FOR A PARTICULAR PURPOSE.  See the
    GNU General Public License for more details.

    You should have received a copy of the GNU General Public License
    along with this program.  If not, see <http://www.gnu.org/licenses/>.
*/
\stoptyping
\StopCodeExample

Change the description line from referencing the engine to referencing the mod itself, Avaneya, if it is for Avaneya specific code. This should look as such.

\StartCodeExample
\starttyping
/*
    Avaneya, a cerebral science fiction game for GNU.
    Copyright (C) 2010, 2011 <kip@thevertigo.com>.

    Public discussion on IRC available at #avaneya (irc.freenode.net)
    ...
\stoptyping
\StopCodeExample

\StartSubSection{Mathematical}

\startitemize[3]
\setupwhitespace[big]
\item
All angles should be worked with in degrees, not radians, whenever possible.

\item
All floating point calculations should be done in single precision whenever possible.
\stopitemize

\StartSubSection{Formatting}

\startitemize[3]
\setupwhitespace[big]
\item
English (Canadian) should be used for everything non-localized and saved in UTF-8 whenever possible.

\item
Always use spaces and not tabs. You can use the latter, if you like, but just as long as your editor converts them to spaces when saving.

\item
Each level of nesting should be indented by four spaces. The exception to this is when a prefix operator precedes an identifier, in which case it should be indented so that the identifier begins at a four space interval.

Right:
\StartCodeExample
\starttyping
float SomeFunction()
{
    int i = 0;
  ++i;
    return 1.0f;
}
\stoptyping
\StopCodeExample

Wrong:
\StartCodeExample
\starttyping
float SomeFunction()
{
    int i = 0;
    ++i;
        return 1.0f;
}
\stoptyping
\StopCodeExample

\item
You do not need to indent to accommodate a namespace when an entire source file is enclosed within one. If only part of the source is enclosed in a namespace, then those parts affected should be indented.

\item
An opening brace appears on the next line as preceding code.

Right:
\StartCodeExample
\starttyping
float SomeFunction()
{
    for(int x = 0; x < 100; ++x)
    {
        ...
    }
}
\stoptyping
\StopCodeExample

Wrong:
\StartCodeExample
\starttyping
float SomeFunction() {
    for(int x = 0; x < 100; ++x) {
        ...
    }
}
\stoptyping
\StopCodeExample

\item
Put spaces around binary operands

Right:
\StartCodeExample
\starttyping
A = X + Y;
if(X == Y)
    ...
\stoptyping
\StopCodeExample

Wrong:
\StartCodeExample
\starttyping
A = X+Y;
if(X==Y)
    ...
\stoptyping
\StopCodeExample

\item
There is no space between unary operators and the affected variable's name.

\item
When accessing an array, there is no space between the array name and the opening left bracket. 

Right:
\StartCodeExample
\starttyping
int n[100];
y = n[3];
\stoptyping
\StopCodeExample

Wrong:
\StartCodeExample
\starttyping
int n [100];
y = n [3];
\stoptyping
\StopCodeExample

\item
The if, for, and while keywords are not followed by a space separating them and the left parenthesis. 

Right:
\StartCodeExample
\starttyping
if(x == 4)
    return;
\stoptyping
\StopCodeExample

Wrong:
\StartCodeExample
\starttyping
if (x == 4)
    return;
\stoptyping
\StopCodeExample

\item
When declaring variables, align the variable names on the same column evenly divisible by four.

Right:
\StartCodeExample
\starttyping
int     Count;
bool    Complete;
Window  MainWindow;
\stoptyping
\StopCodeExample

Wrong:
\StartCodeExample
\starttyping
int Count;
bool Complete;
Window MainWindow;
\stoptyping
\StopCodeExample

\item
When calling a function, definitions and function declarations should have no space between the function name and the following left parenthesis.

Right:
\StartCodeExample
\starttyping
int Add(int a, int b)
{ 
    ... 
}

int i = Add(4, 5);
\stoptyping
\StopCodeExample

Wrong:
\StartCodeExample
\starttyping
int Add (int a, int b)
{
    ...
}

int i = Add (4, 5);
\stoptyping
\StopCodeExample

\item
When calling a function or making a function declaration, no space appears after the left parenthesis or before the right parenthesis.

Right:
\StartCodeExample
\starttyping
foo(x, y);
bar(z);
baz();
\stoptyping
\StopCodeExample

Wrong:
\StartCodeExample
\starttyping
foo( x, y );
bar( z );
baz( );
\stoptyping
\StopCodeExample

\item
A brace preceding or following an else keyword appears on the same line as the else. A statement following an else keyword appears on the same line as the else. 

Right:
\StartCodeExample
\starttyping
if(x == 4)
{
    ...
}
else
  ++y;

if(x == 4)
{
    ...
}
else
{
    ...
}
\stoptyping
\StopCodeExample

Wrong:
\StartCodeExample
\starttyping
if(x == 4) {
    ...
}
else ++y;

if(x == 4) {
    ...
} else ++y;

if(x == 4) {
    ...
}
else {
    ...
}
\stoptyping
\StopCodeExample

\item
A brace preceding a catch keyword appears on a separate line as the catch. 

Right:
\StartCodeExample
\starttyping
try
{
    ...
}

catch(Error SomeError)
{
    ...
}
\stoptyping
\StopCodeExample

\item
A value in a return statement is parenthesized where it contains more than one term.

Right:
\StartCodeExample
\starttyping
return x;
return (a + b);
\stoptyping
\StopCodeExample

Wrong:
\StartCodeExample
\starttyping
return (x);
return a + b;
\stoptyping
\StopCodeExample

\item
If the body of an if, for, while or similar statement consists of a single statement, the statement does not need to be surrounded by braces.


Right:
\StartCodeExample
\starttyping
if(x == 3)
  ++x;
\stoptyping
\StopCodeExample

\stopitemize

\StartSubSection{Naming}


\startitemize[3]
\setupwhitespace[big]
\item
Do not use Hungarian notation\index{Hungarian notation}. We prefix objects to denote scope only.


Member of a global namespace:
\StartCodeExample
\starttyping
g_Wheels
\stoptyping
\StopCodeExample

Member of a structure or class:
\StartCodeExample
\starttyping
m_Wheels
\stoptyping
\StopCodeExample

Static member of a structure or class:
\StartCodeExample
\starttyping
ms_Wheels
\stoptyping
\StopCodeExample

\item
Class and object names should be intuitive, try to avoid abbreviations, and each word should begin with a capital letter. Modern storage mediums can afford to spare brevity, allowing for greater clarity.


Right:
\StartCodeExample
\starttyping
// Class for abstracting a camera interface...
class Camera
{
    ...
};

// Create a camera...
Camera LogitechCamera;
\stoptyping
\StopCodeExample

Wrong:
\StartCodeExample
\starttyping
class cam
{
    ...
};

cam log;
\stoptyping
\StopCodeExample

\stopitemize

\StartSubSection{Structure}

\startitemize[3]
\setupwhitespace[big]
\item
Source lines may be up to 100 characters long. (You can configure gedit to display a margin at 100 characters; that may help you follow this convention.)

\item
Functions or methods should be broken down into other functions or methods if they get too long and this is reasonable to do.

\item
Use {\tt assert()} to check your assumptions for things that ought to always be true. Do not abuse it for situations where it is reasonable for a condition to not be true, such as a socket connection failure or a file that could not be opened.

\item
Follow the {\it GNU Coding Standards} as much as reasonably possible, save the code formatting points made in this handbook. There is a great deal of wisdom in it.

\item
A comment which indicates task which needs to be done at some point should look like this: 

\StartCodeExample
\starttyping
// TODO: Check portability here...
\stoptyping
\StopCodeExample

\stopitemize

\StartSubSection{Comments}

\startitemize[3]
\setupwhitespace[big]
\item
All comments should be written in Canadian English since nearly all programmers in all countries can read that. If you cannot do that, write them as best you can and have someone help you rewrite them.

\item
Comments should begin with a single space, then a capital letter and end with a trailing ellipses. 

Right:
\StartCodeExample
\starttyping
// Load the image...
GrayImage = cvLoadImage(Path.mb_str(), CV_LOAD_IMAGE_GRAYSCALE);
\stoptyping
\StopCodeExample

Wrong:
\StartCodeExample
\starttyping
GrayImage = cvLoadImage(Path.mb_str(), CV_LOAD_IMAGE_GRAYSCALE);//load the image.
\stoptyping
\StopCodeExample

\item
The farther left the comment, the higher level and abstract what you are trying to do is. The farther right, the more detailed they are. You can think of a given level of indentation as elaborating on how to carry out what was described at a higher (less indented) level. It should be possible in many cases to strip away all the code, except the comments, and still understand what it is that you were trying to do.

Before:
\StartCodeExample
\starttyping
// Analyze single image...
void AnalysisThread::AnalyzeImage(wxString Path)
{
    // Variables...
    IplImage   *GrayImage   = NULL;
    wxString    TempString;

    // Reset the tracker, if not already...
    Frame.Tracker.Reset(0);

    // Load the image...
    GrayImage = cvLoadImage(Path.mb_str(), CV_LOAD_IMAGE_GRAYSCALE);

        // Failed to load media...
        if(!GrayImage)
        {
            // Alert...
            wxLogError(wxT("Unable to load image."));
            
            // Abort...
            return;
        }

    // Feed into tracker...
    Frame.Tracker.Advance(pGrayImage);
    
    // Cleanup gray image...
    cvReleaseImage(&GrayImage);
}
\stoptyping
\StopCodeExample

Stripped:
\StartCodeExample
\starttyping
// Analyze single image...

    // Variables...

    // Reset the tracker, if not already...

    // Load the image...

        // Failed to load media...

            // Alert...

            // Abort...

    // Feed into tracker...
    
    // Cleanup gray image...
\stoptyping
\StopCodeExample

\item
A single blank line should appear between each pair of functions or methods. 

\item
Do not use a blank line after an opening brace or before a closing brace.

\item
Do not use duplicate blank lines.
\stopitemize

\StartSubSection{Integrated Development Environment (IDE)}

You are welcome to use any editor or IDE you like, provided it does not require non-standard IDE-specific project files to pollute the repository's source tree. See \in{section}[Developer Tools] for a list of all other developer tools required for this project.

