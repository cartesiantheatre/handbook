% This is part of the Avaneya Project Crew Handbook.
% Copyright (C) 2010, 2011, 2012
%   Kshatra Corp.
% See the file License for copying conditions.

% Food section...
\StartSection{Food}

What people eat is more than an arbitrary preference. It effects quality of life and has lasting economic, environmental, and political consequences.\footnotecite[food_inc] This section will discuss some of the considerations for a Martian diet and what is necessary for growing food. This is important for at least modellers, scripters, and researchers.

\StartSubSection{Interplanetary Travel}
During the flight to Mars, it is critical that the crew eat properly. Nutrition is an extremely important factor in determining health. This is even more so in space where traditional allopathic approaches to illness are even less effective\footnotecite[drugs_ineffective_in_space] than they already are on Earth.\footnotecite[lazarou1998_adr]

\StartSubSection{Politics}
Depending on the political circumstance and the most influential elements in policy making, some foods will be considered acceptable by some and possibly even illegal by others. For example, influential elements acting through UNSA may attempt to pass legislation that bans any number of herbs, such as St. John’s Wort or Echinacea. Approval might need to go before the NAU Congress or EU Parliament.

\StartSubSection{Dietary Considerations}
Raising of warm--blooded herbivores is pointless in space or on Mars. There are reasons why one does not see poultry on the International Space Station. They are extremely energy inefficient, at least strictly in the perspective of being a resource for human consumption. Most of the energy that goes in to warm--blooded herbivores is bled off or radiated into space long before it reaches a human anyways. 

In addition, tillable land is not free. Adapting the land for the large agricultural purposes necessary for the raising of warm--blooded herbivores is energy intensive on Mars. This and a long list of other reasons ensure that it is too expensive to responsibly prepare animals for human consumption. Besides, and more to the point, our physiology does not require that we eat them in the first place.\footnotecite[the_china_study]

There are other inherent advantages to a plant--based diet on Mars. The lower Martian gravity means plants need less energy to transport water and other nutrients up from their stems than they would on Earth. The planting of orchards are useful not only for providing fruit, but in doubling in providing a colony with a source of wood for furniture and other uses.

\StartSubSection{Gasses}
The gas pressure and constituents in the greenhouses are critical for supporting the growth of plant life. Thankfully plants only require a minimum of 5.0 kPa to grow, although this is too low for direct human exposure. This has a number of advantages in the engineering of the building as discussed in \in{section}[Buildings].

Assuming the greenhouse is rated to only 6.8 kPa, which is both economical and very light, the partial pressures should include \chemical{N_2} at 3.0 kPa, \chemical{O_2} at 2.5 kPa, \chemical{H_2O} vapour at 1.2 kPa, and \chemical{CO_2} at 0.1 kPa. The \chemical{CO_2} levels in the greenhouse limit plant life since it is their primary nutrient. 

By adding water to Martian regolith, it should react with the peroxide in it to evolve \chemical{O_2}. This could be useful as one way of obtaining an initial supply.

\StartSubSection{Hydroponics}
Hydroponics are probably not a good idea at first because they require a lot of water which takes a lot of work to get in the beginning of the settlement. After infrastructure is built up, it is an option.

\StartSubSection{Lighting}
Plants require light for photosynthesis to function. Indoor artificial lighting is too energy inefficient for crops. Greenhouse domes are a better option because they are open to natural sunlight. Although the Martian solar flux is only 40\% of what is received on Earth, it is still enough for plants.

\StartSubSection{Mushrooms}
Mushroom ranches have a very practical application on Mars. As with any crop, not all of it is actually consumed and there is always some plant waste. If the plant waste is fed to mushrooms, they can turn up to 70\% of it into a high quality source of protein. They do not require any sunlight, but only a dark, warm, room with a little bit of oxygen.

\StartSubSection{Regolith}

Martian regolith is actually more nutrient rich than Earth. The only nutrient it is lacking is potassium which can easily be acquired from any dry shore or former water body.

The regolith is loosely packed, porous, and mechanically supportive of plants. 

The pH is about 8.3, which is too alkaline for most plants which prefer a pH between 6 -- 7.5. Smectite clays which are found to be naturally occurring in the regolith can be useful for buffering and stabilizing the pH into a slightly acidic range. They double in being able to store exchangeable nutrient ions within them as well.

In terms of fertilizer, nitrates are needed. Unfortunately they are not in all Martian regolith. Raw molecular nitrogen could be extracted from the atmosphere of 3\% if need be to produce the nitrates. This can be done by combining it with hydrogen, \chemical{N_2, +, 3H_2, ->, 2NH_3}. If that is not a viable option, an alternative is to look for existing natural nitrate beds which can provide truck loads of fertilizer.

There is a concern, however, that the top layer is too saturated with antioxidants (oxygen ions) from the intense ultraviolet radiation it receives from the Sun. These atoms will destroy organic molecules, such as plants, which are the building blocks of life.

\StartSubSection{Merits of Kelp Fertilizer}
It is important that soils are not nutritionally deficient. Deficient soils make for deficient plants. Deficient plants make for inadequate self--defence of pests, giving agribusiness an excuse to provide disgruntled farmers with toxic pesticides, herbicides, fungicides, larvicides, and so on. These deficient plants make for deficient foods. Deficient foods make for deficient human health -- even if the latter happens to be eating deficient animals that eat deficient plants. Deficient human health means sickness.

Despite what the agricultural industry would have us believe and what many of us were taught in school by the orthodoxy, plants actually require far more than just nitrogen, phosphorus, and potassium (\chemical{N}, \chemical{P}, and \chemical{K}). They need more like 52 different minerals, depending on how one counts them.\footnotecite[food_matters] Three is just the bare minimum necessary to enable commercialization of fake foods that only appear convincing enough to a buyer. Thus, we are in need of a viable solution for healthy soils in a Martian environment.

Various forms of algae have been considered many times for off--world use, such as in the design of the first American Lunar military base as a food.\footnotecite[extras={ p.~50.}][project_horizon_volume_2] But phytoplanktons, such as kelp, are just as valuable as a food for plants as they are for people. That is, they make for an excellent fertilizer. 

Kelp fertilizer is a powerful natural technology that is rich in every trace element that plants need only a little of. It also contains the usual full spectrum of soil nutrients at 0.3\% nitrogen, 0.1\% phosphorus, and 1.0\% potassium, along with a full range of trace elements and amino acids. Further, it provides these things in ideal ratios.

Besides its contents, kelp fertilizer has many other benefits. It does not introduce foreign seeds of weeds, unlike plant composts. It breaks down quickly since it contains very little cellulose. It contains natural hormones which improve cellular development of plant leaves, flowers, and fruit. It has a higher photosynthetic efficiency than terrestrial plants which means that it produces more of its own biomass from the same amount of light. Its jelly like alginate helps to bind together loose soil. Seed germination is improved. Resulting plants will develop more extensive root systems, which means they will have healthier foliage, flowers, fruit, and therefore food. Lastly, they will have greater natural resistance to nematodes, disease, and pests.

Delivery is simple. Once the kelp is ready, it is not necessary to rinse it before use in either soil or directly through with a foliar dispersal method. The latter works by spraying the liquid fertilizer directly on to the leaves. This is up to 20 times more effective in supplying nutrients when the soil is of a poor quality or the roots are stressed from transplant shock, extreme heat, or drought conditions.

\StartSubSection{Photobioreactors}
To produce kelp on Mars, a {\it photobioreactor} can be used. It consists of a tank made of either glass, polyethylene, or some other suitable material. Various equipment to support the growth of the phytoplankton inside of it will be needed in addition.

A computer is necessary to regulate and provide the temperature, lighting, pH, sterilized water, air, \chemical{CO_2}, and all other nutrients at the correct rates. The nutrients primarily include nitrogen, phosphorus, and potassium, as well as silica, iron, and many other trace minerals which are not difficult to find on Mars. Photosynthesis provides the organism with energy it requires to grow, \chemical{CO_2, +, H_2O, +, photons, ->, C_6H_12_O6, +, O_2, +, H_2O}.

\placefigure
    [right,2*hang]
    {A photobioreactor. Image courtesy of {\it Ralf Reski} of Universität Freiburg.}
    {\externalfigure[Game_World/Images/Photobioreactors.png][][]}

At the bottom of the tank, a dead coral bed provides detritus for the kelp roots to take hold of. This could be dehydrated coral plankton. Only a small seed piece is necessary to get it started which could be imported. For deeper ponds, the bottom of the tank will also need to be agitated, either with bubbles or with some kind of a paddle wheel.

Fluorescent lighting could be used, but only with the right bulbs. Direct sunlight, however, is too strong for kelp and most other algae. This is because these species evolved in an underwater environment where they could only expect one--tenth of what a typical plant exposed to direct sunlight would receive. This is a useful property to take advantage of because Mars receives only 40\% of the solar flux of Earth anyways. 

This lighting system could be embedded directly into the tank by using glow plates. During periods of little or no solar energy, the computer can activate the fluorescent lighting elements embedded in the glow plates. When there is sufficient solar energy, it can save energy by routing the sunlight to the glow plates by using fibre optics. This would make for a beautiful lighting effect for the photobioreactor farm's model during the solar day and night cycles.

