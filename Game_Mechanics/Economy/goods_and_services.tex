\StartSubSection{Goods and Services}
%\StartSubSection{City Resources and Infrastructure}
Utilizing resources and land available to the city and the cities infrasture, through privitization, the city can generate revenue by lease agreements or similar. In addition players can have government run busineess which directly generate profit or loss. If ran correctly these business are a potential income stream where the entire profit enter city coffers, rather than a taxable percentage or similar. 
%\StartSubSection{City Business}
The creation of business interests setup, controlled, and owned by the player/city allows for profit from these projects to return to the city player, however it also requires constant outlays and potential for massive loss. Players may wish to setup player/government owned mining companies, housing estates, industries, or agriculture. 

After enacting the (ORDINANCE_NAMES), players can then implement city ran business. These buildings are selected from the appropriate building tool and are placed anywhere at the player's descretion.





%EXPEDNITURE
\StartSubSection{Goods and Services}
Goods and services can also cost a city. While they cover things such as city ran businesses they also cover things such as medical and emergecy services, education and various others. Vast metropoleis require healthcare and advanced workforces require thorough and lengthy education, these things may, depending on their implementation, require funding from city coffers.





