% This is part of the Avaneya Project Crew Handbook.
% Copyright (C) 2010-2016 Cartesian Theatre <info@cartesiantheatre.com>.
% See the file Copying for details on copying conditions.

% Editing This Handbook section
\StartSection{Editing This Handbook}

If you would like to edit this book, you should obtain a copy of the project's source using the method described in \in{section}[Using Bzr]. You will need to do this because the book's source is contained within it.

In writing this book we used a programmable typesetting language instead of the traditional word processor approach many are familiar with. Why? The reasons are many and already written many times outside of this book, as the general advantages are the same as with any programmable typesetting language, such as \LATEX. In a nutshell, these include everything from enabling simpler collaboration to many sophisticated features not found in traditional word processors.\footnotecite[taraborelli2011]

In typesetting this book we used the powerful document engineering system \CONTEXT, \contextmark, and version \contextversion. \CONTEXT\ is a decedent of the very popular \LATEX. The rendering backend used at the time the copy you are reading was typeset on \currentdate\ was \texjitenginename, version \texengineversion. It was based on Bzr revision \BzrRevisionClickable. This information might be useful in the event you observe unexpected differences between a local copy you typeset yourself and another. 

We worked closely with the \CONTEXT\ developers from time to time while writing this book. Together we identified many bugs that the \CONTEXT\ developers were able to patch. Hopefully the patched versions are not too bleeding edge and lagging behind what is available through your distribution's package manager.

To get you started building this document from source, you will need a number of tools installed, including a working \CONTEXT\ installation. If you would like to be hard on yourself and build \CONTEXT\ from source, instructions are available on the \CONTEXT\ \href{http://wiki.contextgarden.net/ConTeXt_Minimals}{wiki} for most platforms. 

Otherwise, if you are running some flavour of Ubuntu, you are in luck. \MailTo{adam.reviczky@kcl.ac.uk}{Adam Reviczky} was kind enough to provide us with his Personal Package Archives (PPA) containing the latest pre--compiled packages for both \CONTEXT\ and \texenginename. Consider yourself warned though, Adam's PPA automatically pulls from the latest upstream for both on a daily basis. That means either may work fine one day, only to break on some other.

The remaining support packages should already be available from within Ubuntu's official package repositories, including those for generating the various illustrations within this book.

To prepare your build environment on Ubuntu 13.04 Raring, run the following commands with internet connectivity. Note that the \quote{\type{\}} character is just there for readability to denote a line break. You do not need to actually type it. In your terminal, you can collapse lines containing this trailing character into a single command before hitting enter.

\startCodeExample
$ sudo add-apt-repository ppa:reviczky/context-daily
$ sudo add-apt-repository ppa:reviczky/luatex-daily
$ sudo aptitude update
$ sudo aptitude install     \
    context                 \
    context-modules         \
    fonts-larabie-deco      \
    graphviz                \
    inkscape                \
    pdftk                   \
    texlive-math-extra      \
    ttf-ubuntu-font-family  \
    xvfb                    \
    umbrello
\stopCodeExample

Note that if you are using a version of Ubuntu prior to {\it Quantal}, such as {\it Precise} or older, you will need to download the \href{http://distribution.contextgarden.net/current/fonts/new/fonts/opentype/public/lm-math/latinmodernmath-regular.otf}{latinmodernmath-regular.otf} font directly into your {\tt \type{~}/.fonts} path. It is needed to typeset some mathematical formulae.

The above commands will add Adam's PPA, update your package list, download, install, and automatically configure the necessary programs you need. This may take a while, depending on the speed of your connection and computer, but thankfully you only need to do this once.

Be warned that the version of Graphviz included in Ubuntu's repositories may not be new enough. You will want to make sure you are pulling at least version 2.29. If you are not (check {\tt dot -V}), you can get the latest packages from \href{http://www.graphviz.org/Download_linux_ubuntu.php}{upstream}. If you use an older Graphviz anyways, some of the diagrams will come out wrong. If you are not concerned with the diagrams and merely editing other parts of the book, then this will not matter.

%Please also note that the latest version of \CONTEXT\ requires at least Ubuntu version 11.10 or later.

If either your \CONTEXT\ or \texenginename\ packages are ever upgraded, which is probably inevitable, Adam wisely recommends that you do several things to avoid any issues that might arise from residual artifacts.

\startCodeExample
$ cd ~/
$ rm -Rfv ~/.texmf-var
$ luatools --generate
$ mtxrun --generate
$ context --make cont-en
$ mtxrun --script fonts --reload
\stopCodeExample

Now that you have everything installed, change your working directory to the handbook's source root directory and run the following to build it.

\startCodeExample
$ cd Avaneya/Documentation/Contributors/Handbook
$ make
\stopCodeExample

If everything went well, you should have the handbook PDF sitting within your working directory. Note that this process may take a few minutes, depending on your computer. If at any time you would like to clean out intermediate build files, including the resulting PDF, run the following.

\startCodeExample
$ make clean
\stopCodeExample

Now that you are able to build the handbook, you can edit the source and see the result of your work just by running {\tt make}. To commit your changes back upstream to the official master branch, follow the instructions in \in{section}[Committing Changes].

