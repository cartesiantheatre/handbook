% This is part of the Avaneya Project Crew Handbook.
% Copyright (C) 2010, 2011
%   Kshatra Corp.
% See the file License for copying conditions.

% Contact section...
\StartSection{Contact}

\StartTimelineDate{A.R.}
A squadron of Arcadian Dragoons while on patrol detect a non-Arcadian nav-aid transponder signal within their demarcation.
\StopTimelineDate

\StartTimelineDate{A.R.}
Arcadia comes under fire from orbital shelling directed by UNMEPF forward observers\footnote{A forward observer is an artilleryman who is in a forward position directly overlooking the area to be shelled. By communicating with an artillery unit, he directs and corrects their fire. He is needed because sometimes the actual artillery does not have direct line of sight with the target.} on the ground. The event marks the first shots ever fired in anger on Mars. The original {\it Arcadia Hall} is successfully targeted and destroyed, though most of Arcadia is deliberately left intact. 

Outside the city across {\it Arcadia Planitia}, any location identified by MARSCOM's {\it Bhadra I} photographic reconnaissance satellite as a suspected location of harbouring the emergency underground settler are hammered on the surface from orbit. This continues for the next month. With the city already evacuated, and all military personnel garrisoned underground in the city, the settlers experience no casualties.

Between the shelling, Arda has a team of combat support engineers erect a rudimentary high-gain directional antenna which circumvents the need for a satellite communication relay to transmit to Earth. Relying on S-band microwaves, he explains to anyone able to receive on Earth what is happening on Mars. Suspecting that the average Terran has probably been left in the dark until now, Arda speaks of the UN's shelling campaign as an attack on peaceful and defenceless civilians levied from a manned orbital installation on Phobos that is capable of functioning as an artillery battery.

Arda refers to a document hacked from a UNSA intranet he received prior to having lost solnet connectivity. The document was part of the dossier he received previously by the anonymous informant that tipped him off regarding the UN's plan for landing a military force. The detailed white paper is complete with schematics. It describes the first orbital artillery battery in existence. The classified document, {\it Project Horizon: A UNSA Study for the Establishment of a Phobos Artillery Battery}, details an ambitious covert UNSA project to construct {\it Thor Battery}. According to the document, the outpost would be manned by a garrison of a single platoon of thirty soldiers, powered by three nuclear reactors, and located on the Martian moon Phobos. The mostly porous composition of the moon allows for a less energy intensive excavation. This means that the majority of the installation remains subterranean. If the document is genuine, then the installation had been kept from being public knowledge, as it is clear it has nothing at all to do with sustainability research on Mars. 

Arda provides high resolution imagery taken from the Martian surface a month prior of the installation under construction with the transmission as evidence.

Arda explains that the outpost was not really intended for use yet, but rather it had been part of a long term UNSA insurance policy for dealing with future \quote{unruly settlement elements}, should the need ever arise. This was option was on the table, but only taken seriously well after, as MARSCOM had presumed, a straightforward capture of Arcadia.

Arda's transmission is picked up on Earth and rapidly disseminated over peer-to-peer networks. Terran citizenry are best described as being predominantly a mixture of fury and disgust at the actions of their government's involvement. Amnesty International calls for a public inquiry with powers of subpoena as well as the presence of Red Cross observers.

\startTimelineGeneralDocument
...The Phobos outpost is required to develop and protect UNSA interests on Mars; to develop techniques in Phobos and general orbital-based surveillance of Mars, in communications relay, and in operations on the surface of Mars; to serve as a base for exploration of Phobos, for further exploration into space and for military operations on Phobos if required; and to support scientific investigations on Phobos...

\hskip 1.5cm {\it - Project Horizon: A UNSA Study for the Establishment of a Phobos Military Outpost}
\stopTimelineGeneralDocument

Arda explains that by exploiting a legal loophole in the {\it Outer Space Treaty}, the {\it Anti-Ballistic Missile Treaty}, and the {\it Mars Treaty} whose drafting predated the advent of kinetic bombardment weapons, UNSA believes it is justified in targeting the surface with hypersonic tungsten rods with surgical precision at downward velocities of 10 km/s. The treaties do not observe kinetic bombardment as weapons of mass destruction since they require no explosive payload, though they deliver sufficient kinetic energy in the order of a small tactical nuclear weapon.

The transmission is picked up on Earth by amateur radio operators and quickly disseminated over the solnet. UNSA dismisses Arda's claims as {\it "baseless allegations propagated by Terran conspiracy theorists, intertwined with a misunderstanding of the nature of meteorite strikes on Mars, and used to justify the theft and reckless destruction of the costly off-world public asset of the Arcadian Settlement"}. A UNSA spokesman refuses to respond to questions on the existence of the controversial outpost.
\StopTimelineDate

\StartTimelineDate{A.R.}
A subsonic aircraft crashes 40 km south of RAP Starport carrying 3600 kg of cocaine in the plains of {\it Arcadia Planitia}. The transportation secretary's investigators confirm that the aircraft's tailnumber was registered to the CIA for alleged UNSA diplomatic flights. Chemical analysis suggests it to be the first known synthetic cocaine and of Martian origin.
\StopTimelineDate

UAV shot down

UNSA destroy food stocks and then provides aid with terminator gene gmo

photobioreactors converted for growing biofuels

red cross

