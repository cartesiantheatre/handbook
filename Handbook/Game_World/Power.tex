% This is part of the Avaneya Project Crew Handbook.
% Copyright (C) 2010, 2011, 2012
%   Kshatra Corp.
% See the file License for copying conditions.

% Power section...
\StartSection{Power}

There are several different options for generating electrical energy on Mars. Some are more practical than others. We will explore these options in this section.

\StartSubSection{Microwave Power}
Microwave power works by having a nuclear fission reactor on the ground somewhere generating electricity. This energy would be transmitted by microwave from the surface to a satellite in areostationary orbit at an altitude of 16,600 km. The satellite would then transmit this energy back to a receiving station on the surface for local redistribution.

It may sound appealing, but it has its issues. Every time the microwave energy is transmitted, it loses half of its total available energy. Since it goes up and then comes down before reaching its destination, it is reduced twice. This means it has only 25\% efficiency.

But even if it had 100\% efficiency, if the satellite ever veered slightly out of alignment due to any orbital or attitude corrections that it makes, the resulting inaccuracy could be very dangerous. Microwave radiation can kill. Considering the distance is some 16,600 km away, it would take only a very modest adjustment to veer as much as several kilometres off course.

\StartSubSection{Dynamic Isotope Power System}
A dynamic isotope power system (DIPS), sometimes called a radioisotope thermoelectric generator, works by converting the thermal radiation of a piece of nuclear material into electrical energy via a process known as the {\it Seebeck effect}. This process requires a temperature differential to work.

Many existing spacecrafts, such as the Viking landers and orbiters, carried these generators to supply themselves with modest amounts of long--term, reliable, electrical power. It requires no moving parts, but its power output will degrade as a function of time.

This is a useful device, but it does not provide much energy. Sometimes an accompanying deep--cycle battery can be used to buffer sufficient energy to handle peak loads, such as during a transmission. This was the case with both Viking landers.

The choice of radioactive materials used have both longevity and economic considerations. Plutonium--238 (\chemical{\high{238}Pu}) has a half--life of 88 years, but is very expensive. Strontium--90 (\chemical{\high{90}Sr}) or caesium--135 (\chemical{\high{135}Cs}) are cheaper alternatives that have half--lifes of 30 years. These latter two could be acquired from nuclear waste.

\StartSubSection{Hydrazine}

Hydrazine (\chemical{N_2H_4}) is a popular monopropellant that sees use frequently in the small thrusters responsible for attitude control on spacecraft. It is chemically attractive because it allows for long term storability and simplicity of use since all that it needs for combustion can be contained within a single storage vessel. That is, it does not require separate tanks for both a fuel and oxidizer. This is why it is called a monopropellant. 

Hydrazine reserves can be used to supply a spacecraft, aircraft, or ground vehicle with emergency power for short periods of time. The {\it Avaneya} spacecraft has several of these hydrazine powered auxiliary power units (APUs).

Hydrazine can be dangerous if mishandled because its combustion is extremely exothermic, meaning it releases a very large amount of energy in a very short time.

More work needs to be done to determine how it could be synthesized {\it in situ}.

\StartSubSection{Nuclear Fission}
Nuclear fission reactors are common on Earth. They have unlimited mileage and are very efficient for large scale power use. Thus, they could be useful for the first settlers backed with substantial government funding. An initial 4,000 kg reactor might produce 100 kWe of electrical energy and 2 mW of thermal process heat. This thermal energy could be used to drive other endothermic reactions used to synthesize materials on Mars, heat buildings, or for other purposes.

The drawbacks are of course well known. They can be politically unpopular, are very expensive, and can be catastrophic in the event of a meltdown or other disaster.

\StartSubSection{Geothermal}
We know that Mars is still volcanically active, and therefore still has a molten core. We know that Mars contains enormous stores of water. We also know that some liquid saline actually makes it to the surface from time to time. The {\it Mars Global Surveyor} has even identified signs of flowing water that appear to be recent in the {\it Cerberus} region near the equator of Mars. Even if this happened 10 million years ago, this is still considered \quote{recent} by geologists because in geological time this is an extremely short period. Therefore, we know that hot thermal wells are almost guaranteed to exist on the planet. These facts make geothermal energy an attractive option.

Geothermal energy works by tapping into hot artesian aquifers and extracting heat from them or possibly even converting it into electricity by having it drive a turbine. Besides energy, geothermal wells could also supply a base with liquid water. It would still need to undergo desalination, however.

On Earth, our cities were erected first, knowledge of geothermal came after and sometimes even by thousands of years. On Mars, we have the advantage and convenience of the inverse. We can build settlements where we already know there to be hot thermal wells.

There is a relation between well depth, power output, and consequently, cost of installation. Locations with viable aquifers will cost different amounts, depending on how far down a drilling rig needs to bore. But keep in mind that it is still easier to drill deeper on Mars than on Earth because the lower gravity will have compressed the regolith less.

For more on the relation between well depth and temperature on Mars, see Zubrin's work.\footnotecite[extras={ p.~211.}][case_for_mars] For more on the research that has gone into considering the utility of geothermal energy on Mars, see Fogg's work.\footnotecite[fogg1996]

Geothermal typically has a very low cost per kW hour to operate. By a comparison done in 1996, geothermal electricity cost between 3--10¢, fossil fuels at 4--6¢, hydropower at about 3¢, burning of biomass at about 5¢, nuclear fission at about 5¢, tidal at least at 8¢, solar thermal at about 9¢, photovoltaic between 25--35¢, and wind between 6--15¢.\footnote{{\it Ibid} p.~404.}

\StartSubSection{Solar}
The process of converting solar energy into electricity through photovoltaic panels has a pretty good reputation on Earth. Unfortunately, these panels are not as useful on Mars as they are on Earth. The latter receives about two and a half times the solar flux Mars does.

But even if it got the same, dust storms reduce what is received on the surface to only a tenth of its total potential for weeks or even months at a time. These dust storms usually occur around the time of the planet's perihelion -- when it should be getting the most sunlight. This is the point in its orbit when it is closest to the Sun when the planet receives 45\% more solar flux than when it is at its furthest, the aphelion. 

Further, even if dust storms did not occur, Martian winds will reduce the panels effectiveness through the gradual deposition of very fine iron oxide dust. The panels would need to be brushed off regularly if they are to be of any use, even on a clear and bright afternoon.

At night time, photovoltaic panels are obviously useless. A settlement would need another supply of power during the night, such as batteries.

The panels are expensive both to build as well as to replace. Even the best panels available on the market today still do not produce very much power. Nevertheless, they are useful for low power applications, mounted on vehicles, or in emergencies.

