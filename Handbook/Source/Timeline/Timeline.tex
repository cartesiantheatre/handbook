% This is part of the Avaneya Project Crew Handbook.
% Copyright (C) 2010, 2011
%   Kshatra Corp.
% See the file License for copying conditions.

%c Timeline chapter...
\StartChapter{Timeline}
\startlines

The world Avaneya takes place in is highly self-referential. A fictional timeline is positioned relative to a {\it year zero\index{year zero}}, with years preceding suffixed {\it B.R.\index{B.R.}}, before the republic; while those following are suffixed {\it A.R.\index{A.R.}}, after the republic. This is the method adopted by \goto{Arda}[Arda Baştürk] and his settlers around the time they declare their independence.

Note that a year is defined as a Martian year (MYr) and is not the same as a Terran year. You may want to refer to \in{section}[Time] for a detailed explanation on how the settlers keep track of time on Mars.

Year zero's placement relative to the Gregorian calendar\index{Gregorian calendar} is deliberately never explicitly given. But inferentially, one can reason it is proximate enough for a contemporary audience to feel relevant, yet distant enough to afford some creative license. As said earlier, Avaneya may take place in the future, yet it deals with contemporary issues.
\crlf

% Provides hint to astute reader of rough order of magnitude of time between now and story...
\StartTimelineDate{(date here) B.R.}
The last of the {\it International Space Station} crew return to Earth with the station subsequently decommissioned, deorbited, and incinerated during atmospheric entry.
\StopTimelineDate

\StartTimelineDate{B.R.}
NASA's {\it Zubrin} spacecraft lifts off atop a 140 tonne capable Enigma-D heavy-lift booster from its launchpad at Cape Canaveral, Florida, North American Union. The manned module separates from the booster in low Earth orbit. It carries a crew of four with two flight engineers and two mechanics with combined specialities in geology and biogeochemistry. This is the first attempted piloted mission to Mars, implementing the Mars Direct mission architecture as a direct launch. Its objective is to determine primarily if the Red Planet ever harboured life, as well as to survey and explore.
\StopTimelineDate

% This should be 180 days after departing Earth for a fast conjunction class manoeuvre...
\StartTimelineDate{B.R.}
{\it Zubrin} performs a deceleration manoeuvre by aerobreaking through the Martian atmosphere and entering orbit to prepare the lander stage which carries a 25 tonnes payload of a habitat module and Earth return vehicle (ERV).
\StopTimelineDate

\StartTimelineDate{B.R.}
Mars Direct crew report ERV's chemical processing plant successfully brought online, producing oxygen, liquid water, and the methane/oxygen return propellant {\it in situ}.
\StopTimelineDate

% Constraint: Should be 910 days total mission time since departure, with 550 days of Mars stay time...
\StartTimelineDate{B.R.}
{\it Zubrin ERV} completes its mission objectives, including extensive mineralogical data and an unsuccessful search for life. It lands at the Kennedy Space Center, Florida, North American Union.
\StopTimelineDate

\StartTimelineDate{B.R.}
The bancor is adopted as the new global currency and implemented via the respective central banks of the North American Union, European Union, African Union, and Asian Union.
\StopTimelineDate

% Don't use provisional designation for asteroid since includes fixed Gregorian date in name...
\StartTimelineDate{B.R.}
Professor Ramraj discovers a near-Earth, C-type Amor II asteroid, with a size comparable to Phobos from the Lincoln Near-Earth Asteroid Research laboratory, Socorro, North American Union. The discovery is rapidly assigned the designation {\it 52048 Varuna} by the {\it International Astronomical Union}. His findings are chronicled in the {\it Minor Planet Circular} where he calculates a near collision trajectory with Mars.
\StopTimelineDate

\StartTimelineDate{B.R.}
Wikileaks publishes a leaked list of 163 purported attendees of the annual Bilderberg conference held three days prior at the Hotel de Crillon, Paris, European Union. Dr. Samuel Lieberman, director of the National Institute of Standards and Technology, and Adriaan Janssen, the Secretary-General of the United Nations, are among those listed.
\StopTimelineDate

\StartTimelineDate{B.R.}
The United Nations holds the Second International Mars Summit in Geneva, North American Union, with the purpose of discussing the options on the table for the second manned mission to Mars. The attendees number in the thousands, representing the states of the African Union, North American Union, European Union, and the Asian Union. In addition, thousands of scientists, engineers, and philosophers attend as either independent presenters or as part of any number of associations ranging from former Case For Mars conference members to The American Astronautical Society, NASA, the RAND Corporation, and others.

The minimalistic Mars Direct\footnote{See \about[Mars Direct] for more information on Mars Direct.} approach of travelling light, living off of the land, and using indigenous materials to produce the fuel necessary for the journey home is again a central theme, but with more emphasis on the nature of the duration on Mars and material infrastructure required.
\StopTimelineDate

\StartTimelineDate{B.R.}
Leonard Kissinger, president of the {\it Council of Foreign Relations}, holds a symposium at the Harold I. Pratt House in New York City, North American Union.
\StopTimelineDate

\StartTimelineDate{B.R.}
The {\it Trilateral Commission} holds its biannual meeting several months in advance, Tokyo, Asian Union.
\StopTimelineDate

\StartTimelineDate{B.R.}
The International Mars Summit concludes and publishes its findings in the {\it Report on the International Proceedings of the Mars Summit}. The report contains the following:
\crlf

\startTimelineDocument
Zero capability missions have no value. The Mars Direct mission, having already demonstrated successfully the feasibility of sending humans to Mars, the bar now can and ought to be raised. The attendees concluded nearly unanimously that it is not prudent to return merely for the sake of demonstrating that a safe return is possible, that having already been established, but to create a permanent settlement on Mars. 

A suitable upper Amazonian geologically classified region where the presence of underground liquid water reservoirs has already been observed is a strong candidate location. 

Such a mission architecture would primarily provide unique, cutting edge, off world, research facilities to address major global issues afflicting all nations. Secondarily, it would also serve to bootstrap the first non-terrestrial expansion of human civilization. Both of these goals are in the service of all mankind.
\stopTimelineDocument
\crlf

The report's abstract outline for a new mission architecture is as follows:

\startitemize[n]
\item Deploy a space station in low Earth orbit to act as a construction berth for trans-Martian spacecraft supplied by efficient single-stage-to-orbit (SSTO) vehicles.

\item Launch a spacecraft constructed within the space station that would transport only the minimum necessary to bootstrap a settlement, such as cryogenic \chemical{H_2} needed to initiate the \goto{Sabatier}[Sabatier Reactor] and \goto{RWGS}[Reverse Water Gas Shift Reactor] reactors to produce fuel and oxygen, {\it in situ}, respectively, a nuclear reactor, excavation equipment, and additional equipment.

\item Use a chemical propulsion system of \chemical{H_2/O_2} to get to Mars, then use indigenous resources to supply the return propellant of \chemical{CH_4/O_2}. This requires the mission payload to include propellant for the outbound trans-Martian injection stage only.

\item The crew would deploy construction equipment and machinery to process indigenous materials to create the {\it Arcadian Settlement}, a permanent outpost located in the Arcadia Planitia region and falling within a United Nations extraterritorial mandate.

\item The spacecraft would return to Earth unmanned, but fully fuelled with {\it in situ} produced methane/oxygen propellant. This would allow subsequent shipments to depart Earth bearing spare parts, machinery, additional excavation equipment, and whatever else the outpost could not have determined necessary {\it a priori}.

\item Personnel ferries departing Earth at regular intervals would continue to augment the settlement capability through providing additional manpower. This would increase the number of available habitats, greenhouses, and other vital infrastructure necessary to support the research being conducted.
\stopitemize

A preliminary draft of the {\it Mars Treaty} is produced using a revised {\it Outer Space Treaty} of 1967 A.D. as an adaptable boilerplate. The treaty provides the contractual basis for all signatories, outlining the road map, responsibilities, and party resource allocations necessary to bring about the terraformation and first manned mission to Mars operating under a United Nations mandate. The treaty also provides an agreement governing the activities of states on Mars and its two moons, Phobos and Deimos.

The terraformation strategy is to be initiated by intercepting {\it 52048 Varuna} with a nuclear warhead and deflecting it into a Martian collision trajectory. Being a C-type asteroid, it contains high levels of hydrated (water-containing) minerals which, when released into the atmosphere as aerosols, have been calculated to greatly increase the total available cloud condensation nuclei.

Data was presented at the summit predicting that the direct effect of the introduction of these new, dark, carbonaceous, aerosols would be a decrease in \goto{albedo}[Albedo] since they absorb more solar radiation. This would cause a cascading series of reactions, resulting in increases in the mean surface temperature and atmospheric pressure.

The deflection strategy, if executed as calculated, would have the asteroid impact directly over the Martian south pole. The south pole contains a large sheet of buried \chemical{CO_2} (dry ice) approximately 8 metres thick which, when sublimated, would theoretically act as a powerful greenhouse gas, thickening the atmosphere, along with the underlying water ice beneath it.

While the legalities of the project are defined in the {\it Mars Treaty}, the international collaborative project it implicates is known as the {\it Avaneya Initiative}.
\StopTimelineDate

\StartTimelineDate{B.R.}
{\it United Nations General Assembly Resolution 5571} is adopted, ratifying the Mars Treaty. The resolution augments {\it Chapter III} of the {\it United Nations Charter} to beget its seventh principle organ, the {\it Space Agency} (UNSA). Its mandate reifies the treaty, overseeing the implementation of the Avaneya Initiative.

The United Nations Office for Outer Space Affairs\index{United Nations Office for Outer Space Affairs} (UNOOSA\index{UNOOSA}) is dissolved. The former responsibilities that it held since its formation in 1958 A.D. are amalgamated under the new UNSA.

The resolution requires members to accede state space agencies, such as the North American Union Aeronautical and Space Agency (NAUSA), European Space Agency (ESA), Asian Aerospace Exploration Agency (AAXA), and others under a centralized UNSA administration. The public rationale given that space exploration and settlement ought to be an endeavour for all mankind, best realized through the cooperative aggregation of resources.

Lieberman is nominated to the Office of the President of the UNSA, resigning his position as director for the National Institute of Standards and Technology.
\StopTimelineDate

\StartTimelineDate{B.R.}
The UNSA's Office of the Avaneya Initiative disseminates more than ten-thousand requests for quotations.
\StopTimelineDate

\StartTimelineDate{B.R.}
UNSA's Office of the Avaneya Initiative announces successful bids. The {\it Iterum Shipyard} contract is jointly awarded to Lockheed Martin-Boeing and Mercedes-Pratt & Whitney to robotically build a station remotely in low Earth orbit at a cost of 2.2 billion bancors. The station is to be used as a general purpose interplanetary space vehicle construction platform.

% Lockheed Martin's failed X-33 design would be a good starting point for a model...
% TODO: Get the Sanskrit devangari characters to display some how...
Lockheed Martin-Boeing is awarded a contract to provide three, leased, single-stage-to-orbit (SSTO), manned, re-usable, {\it VentureStar VII} class suborbital spaceplanes. These vehicles, known as {\it Aravans},\footnote{Sanskrit अरवन् for a steed or horse.} provide UNSA with heavy lift workhorses for ferrying personnel and materials to the {\it Iterum Shipyard}. They require no external solid booster rockets or external fuel tanks and each operate at less than a fifth of the cost of the retired Space Shuttles.

Lockheed Martin-Boeing is also awarded 21 billion bancors worth of contracts to incrementally provide four direct-launch to Mars {\it Tarikin}\footnote{Sanskrit तरीकिन for a ferry-man.} ferries capable of carrying payloads of 150 personnel each, plus provisions. The {\it Tarikins} will shuttle settlers to Mars using a chemical \chemical{H_2/O_2} propulsion system for the outbound trajectory, but then provided in Martian orbit with the return \chemical{CH_4/O_2} propellant manufactured {\it in situ}. The {\it Tarkins} will depart Earth every 780 day minimum-energy launch window, taking 180 days to arrive at Mars in constant rotation with each other.

Mitsubishi-Saab's is awarded a contract to robotically construct {\it Avaneya} remotely within the {\it Iterum Shipyard} orbital berth at a cost of 7.8 billion bancors. {\it Avaneya} will carry the bootstrap crew and large amounts of construction equipment.

Volvo-John Deere is awarded a contract to provide chemical propulsion systems for {\it Avaneya}, {\it Aravans}, and {\it Tarikins} at a cost of 920 million bancors. It is used for the outbound trans-Martian trajectory, with the return methane/oxygen (\chemical{CH_4/O_2}) propellant manufactured {\it in situ}.

Carlyle Holding's Soros-Murdoch Aerospace is awarded a contract to provide four {\it Mars Positioning System} satellites and the {\it Mars Enhanced Telecommunications Orbiter} carrying an optical relay at a cost of 750 million bancors. The five satellites are carried as part of the {\it Avaneya} mission payload
\StopTimelineDate

% Huelva pronounced \ˈwel-vä, ˈhwel-\
\StartTimelineDate{B.R.}
Construction of the last of UNSA personnel training facilities in Antarctica and Huelva, European Union, are completed.
\StopTimelineDate

\StartTimelineDate{B.R.}
The first crew candidates arrive and report to UNSA training facilities. Courses begin the following day.
\StopTimelineDate

\StartTimelineDate{B.R.}
The {\it Iterum Shipyard} is completed remotely in low Earth orbit. The station orbits Earth at an altitude of 340 km, travelling at a speed of 27,400 km/h, and taking one and a half hours to complete one orbital revolution about the Earth.
\StopTimelineDate

\StartTimelineDate{B.R.}
The crew selection is completed with a broad range of personnel including flight engineers, artificial intelligence specialists with knowledge engineers among them, astrogeophysicists, cyberneticists, chemical and civil engineers, cold weather construction experts, mechanics, biogeochemists, geologists, areobotanists, and a xenobiologist, and many others. 

\goto{Arda Baştürk}[Arda Baştürk] is named Mission Commander. 

Leonard Kissinger will accompany the crew as the newly appointed United Nations Deputy Secretary-General.
\StopTimelineDate

\StartTimelineDate{B.R.}
{\it Mars Science Laboratory Curiosity XI}, an unmanned autonomous aerial vehicle, surveys potential landing sites in Arcadia Planitia. The integrated onboard artificial intelligence is instructed to evaluate sites based on average available sunlight, underground water ice revealed by airborne shallow radar, mineralogy, surface geography, and other simultaneously weighted factors.
\StopTimelineDate

\StartTimelineDate{B.R.}
The {\it Internet Assigned Numbers Authority} allocates {\tt A001:CA7:3134::/48} IPv6 address block for general Martian use. 
\StopTimelineDate

\StartTimelineDate{B.R.}
{\it Aravan III} departs UNSA's Cape Canaveral launchpad en-route to the {\it Iterum Shipyard}, low Earth orbit, with a mission payload of materials for {\it Avaneya}. This includes the disassembled aeroshield, whole food, potable water, lab equipment, and its nuclear reactor.
\StopTimelineDate

\StartTimelineDate{B.R.}
{\it Avaneya} construction is completed remotely in orbit at the {\it Iterum Shipyard}, low Earth orbit.
\StopTimelineDate

\StartTimelineDate{B.R.}
Launched from the Kennedy Space Center, Florida, North American Union, the {\it Aravan II} transporting all {\it Avaneya} crew docks successfully with the {\it Iterum Shipyard}, low Earth orbit.
\StopTimelineDate

\StartTimelineDate{B.R.}
{\it Avaneya} completes all system checks and disembarks the {\it Iterum Shipyard} berth. It commences its six month journey with a delta-v from low Earth orbit injecting itself into a trans-Martian conjunction class orbital manoeuvre.
\StopTimelineDate

\StartTimelineDate{B.R.}
The Avaneya crew commence scheduled in-transit experiments in space medicine, life sciences, astronomy, physical sciences, meteorology, and human research.
\StopTimelineDate

\StartTimelineDate{B.R.}
{\it Avaneya} engages a short burn of its manoeuvring thrusters to performs a debris collision avoidance in response to a warning emanating from the {\it Solar and Heliospheric Observatory} stationed at the Lagrange \math{L_1} gravity well stationed between the Sun and the Earth.
\StopTimelineDate

\StartTimelineDate{B.R.}
The Avaneya crew complete all scheduled in-transit experiments in space medicine, life sciences, astronomy, physical sciences, meteorology, and human research.
\StopTimelineDate

\StartTimelineDate{B.R.}
{\it Avaneya} deploys all four {\it Mars Positioning Satellites} (MPS) into high Martian orbit.
\StopTimelineDate

% Constraint: 180 days after launch with a departure velocity of 3.34 km/s...
\StartTimelineDate{B.R.}
Assisted by retrorockets, {\it Avaneya} performs orbital capture by aerobraking into Martian orbit. The ship's various instrumentation subsystems update the onboard areology database on the most recently available surface geography, weather dynamics, mineralogy, and other available telemetry.

The {\it Mars Enhanced Telecommunications Orbiter} is released and deploys itself into areosynchronous orbit, 17,065 kilometers above the equator.

\StartTimelineDate{B.R.}
The {\it Mars Enhanced Telecommunications Orbiter} successfully completes all self diagnostics and registers itself into UNSA's {\it Interplanetary Internet} as an available communications node, giving rise to what becomes colloquially known as {\it solnet}. Terran downlink passes through a ground station at UNSA's Jet Propulsion Laboratory staffed by Soros-Murdoch Aerospace contractors and provides tier 1 network access.

Brokered by a satellite uplink with METO, the systems team establish contact with Mission Control routed over solnet. Solnet becomes the defacto standard for Earth-Mars intercommunication.
\StopTimelineDate

\StartTimelineDate{15 Leo 8 B.R.}
Arda issues coordinates to the loadmaster for a preselected drop site in {\it Arcadia Planitia}, one of the several candidates previously catalogued by Curiosity XI. The cargo of mostly construction equipment, cryogenic liquid hydrogen, water, and other provisions, are jettisoned and parachuted to the surface with no material loss, save one asset due to an attitude control computer malfunction.
\StopTimelineDate

% Leo 13, about the peak of the Martian Spring with clear skies and low winds with the 
%  weather at its finest...
\StartTimelineDate{B.R.}
All crew alight {\it Avaneya}, boarding the {\it Manu} landing craft. No one is left onboard exposed to further solar flares and cosmic radiation.

The {\it Mars Positioning Satellites} provide a guided landing by tracking the critical manoeuvres of {\it Manu's} entry, descent, and soft landing. The communications uplink with Mission Control is maintained throughout over a 4 minute delayed solnet connection.

{\it Manu} makes contact and reports a successful soft landing where they are met with a temperature of \math{-70^{\circ}}C and an atmospheric pressure of 30 Pa. The time is local noon, allowing for maximum photovoltaic use.

Team briefings are conducted within {\it Manu} at the drop site. Concurrently, the recovery team is dispatched to begin asset recovery.

The base's nuclear reactor is brought online, along with several temporary mobile portable dynamic isotope power systems.

The construction team begin minor excavation for anchoring and erecting temporary 34.0 kPa \goto{rated}[Pressure Rating] aluminium strut reinforced inflatable polypropylene tents for the Command Centre and habitats. The greenhouse tent is rated 6.8 kPa, sufficient for plant life, but requiring personnel to don EVA suits.
\StopTimelineDate

\StartTimelineDate{B.R.}
Gas extractors are brought online and run at full capacity capturing liquid oxygen, liquid nitrogen, argon, and carbon dioxide. 

The \goto{Sabatier}[Sabatier Reactor], \goto{RWGS}[Reverse Water Gas Shift Reactor], and methanol reactors successfully create methane, oxygen, hydrogen, methanol, and aqua successfully.
\StopTimelineDate

\StartTimelineDate{B.R.}
The recovery team's rovers directed by scouts on methanol motorbikes complete asset recovery of all undamaged parachuted cargo within a 92 kilometre radius from {\it Manu's} landing site.
\StopTimelineDate

\StartTimelineDate{B.R.}
Vehicular onboard artificial intelligence and system firmware is upgraded from UNSA's Jet Propulsion Laboratory over solnet.

The mechanics team complete the necessary preparation of the backhoes, front loaders, bulldozers, tractors, graders, water ice processors, dump trucks, and other vehicles rendering them available for construction team use.

Mining and excavation operations begin through a mixture of directly manned, remotely manned, and autonomous operation.
\StopTimelineDate

\StartTimelineDate{B.R.}
The Mars Ascent Vehicles' Alpha and Bravo alternate launches to perform rendezvous and dockings with {\it Avaneya} in low Martian orbit. They provide the ship with the methane/oxygen return propellant it requires over the next several weeks as it is manufactured.
\StopTimelineDate

% Constraint: 550 days after first arrival...
\StartTimelineDate{B.R.}
Arda directs the flight engineers to remotely issue the command sequence necessary for the fully refuelled {\it Avaneya} to begin its unmanned journey back to the {\it Iterum Shipyard}.
\StopTimelineDate

\StartTimelineDate{B.R.}
{\it 52048 Varuna} is intercepted by the ion-drive propelled impactor {\it Don Quixote V}. The spacecraft's nuclear warhead payload is detonated successfully above the surface.
\StopTimelineDate

\StartTimelineDate{B.R.}
{\it Tarikin I} and its crew begin preparation to leave Cape Canaveral, Florida, North American Union, in response to Arda's UNSA advisory report of the completion of additional settlement infrastructure. This marks the beginning of the {\it Tarikins'} continual supply of 150 new colonists every 780 day launch window.
\StopTimelineDate

% At 300 new arrivals every year, each Aravan capable of transporting
%  150 persons twice every year, this would take about ten years...
\StartTimelineDate{B.R.}
The {\it Arcadian Settlement's} annual census reports a population exceeding 3,000 inhabitants and growing.
\StopTimelineDate

\StartTimelineDate{B.R.}
Arda wins a landslide election and accepts executive office as Executor of an interim government. Kissinger strongly cautions him against {\it "forming a redundant administration inconsistent with UNSA interests"}. 

The Security Council holds an emergency meeting in New York, North American Union, on the settlement situation.
\StopTimelineDate

\StartTimelineDate{B.R.}
Promoted from United Nations Deputy Secretary-General, Security Council Resolution 12661 is adopted appointing Kissinger to the newly created {\it Office of the Governor of the Arcadian Settlement} in response to the settlement's self initiated election. Arda does not make a public statement.
\StopTimelineDate

\StartTimelineDate{A.R.}
{\it 52048 Varuna} aerobreaks into the Martian atmosphere and disintegrates as predicted with meteor showers over the south pole's dry ice sheets.
\StopTimelineDate

% Year zero...
\StartTimelineDate{Year Zero, 15 Virgo 0 A.R.}
Arda formally addresses the Secretary-General of the United Nations through an internationally broadcast speech over solnet in which he likens his settlers predicament to that of...
\crlf

\startTimelineDocument
...the latest revision of the banana republic, now \goto{\it railgun}[Railgun] catapulting deuterium across the solar system in an endless effort to entertain an insatiable Terran appetite which now limits itself but to the absurdities of usury and \goto{\it cornucopian}[Cornucopianism] amidst all the beauty, splendour, and knowledge the Red Planet has to offer humanity.

Arcadia's greatest achievement occurred today with its realization that it can do better, that Earth has nothing that it needs, and that it will not follow in its example. The era of plundering the fruits of the Red Horn of Plenty to the detriment of those who laboured tirelessly and assumed all of the greatest dangers is over. 

We arrived on Mars under the impression that we had left Earth, only to find ourselves still steeped in the very worst of it. Today we have made plans to truly depart Earth, and with no plans of return.
\stopTimelineDocument
\crlf

Arda concludes his speech transmitted across both planets of the passage that morning of the {\it Rubicon Act}, initiating steps to secure the settlement's independence and begetting the first non-terrestrial, autonomous, city-state, through self-declared legislation. 

Notable paraphrased portions of the Act containing the {\it Constitution of the Republic of Arcadia Planitia} are as follows:
\crlf

\startTimelineDocument
\startitemize[5]
\setupwhitespace[big]
\item {\it Article I} declares the colony an independent, sovereign, self-governed, constitutional republic with a right to self determination. It is self-styled the {\it Republic of Arcadia Planitia (RAP)}, colloquially known by its capital and sole city, {\it Arcadia}.

\item {\it Article II} defines the head of state, the {\it Executor}. The Executor and his cabinet ministers are democratically elected and serve for as long as the electorate permit.

%\item {\it Article III} defines the state's rule as predicated upon natural law and not positive law, limiting the state's mandate exclusively to the preservation of life, liberty, property, and rights.
\item {\it Article VII} superannuates the Terran \goto{\it bancor}[Bancor] fiat currency with the \goto{\it jenya}[Jenya]. The jenya is declared the exclusive legal tender within the Republic, necessitating all Terran interests to acquire Arcadian goods and services in jenyas only.

\item {\it Article VIII} defines the state's relation to a military. Militias are permitted, but conditional upon exclusive executive authority vested in none other than the people. The creation of a permanent standing army at the government's disposal is strictly forbidden.

The militia's three restrictions are that it is forbidden from deployment outside of Arcadia's territorial regions, that its purpose is exclusively defensive - of people, not government, and that it cannot be deployed domestically as an aid to civil power under any circumstance.

\item {\it Article IX} denies the deployment of munitions of war suborbital, orbital, in outer space, or anywhere outside of Arcadia's territorial region.

\item {\it Article X} discharges all public debt held by the {\it International Monetary Fund} and the {\it World Bank} effective immediately, ending the Terran central banks' use of the settlers' registered biological property, their birth certificates, as collateral against the debt.

\item {\it Article XI} prohibits the state from providing itself with a not-withstanding clause, preventing circumvention of this Act.
\stopitemize
\stopTimelineDocument
\crlf

In 2001 A.D., the former NASA salvaged aluminium from the ruins of the former World Trade Centers in lower Manhattan for reuse in the Mars rover {\it Spirit's} cable shield. It was a symbol of human perseverance in the face of evil. 

On Arda's directive, and to remind future Arcadians of the reason for independence, the {\it Rubicon Act} is engraved in aluminium recovered from the derelict {\it Spirit} as "a symbol of human perseverance in the face of government".
\StopTimelineDate

\StartTimelineDate{A.R.}
{\it Arcadia} passes the {\it Humanoid Act}, amending Article III of the Rubicon Act, stripping corporate legal fictions of the rights of human beings.
\StopTimelineDate

% Coke pulled out of Vietnam the year before the United States military did...
\StartTimelineDate{A.R.}
{\it Coke} quietly closes down its production and bottling facilities and makes preparations to return to Earth. The fast food franchise, {\it Clown Food}, also follows in suit. No public statements are made.
\StopTimelineDate


\StartTimelineDate{A.R.}
No. 2 \goto{Railgun}[Railgun] is taken offline for unscheduled rail and insulator surface repair by the Terran UNSA contractor, {\it Solar Urban Moving Systems}.
\StopTimelineDate

\StartTimelineDate{A.R.}
No. 2 Railgun's blockhouse reports an explosion that has taken out its compulsator power supply. Two-hundred and fifty three people are killed and seven injured as the {\it Fourth Precinct's} adjacent residential district's pressure dome is depressurized from shrapnel.

Arda convenes an emergency cabinet meeting at {\it Arcadian Hall} where forensic evidence of a cryobomb's (\chemical{CH_4/O_2}) chemical residue is presented.
\StopTimelineDate

\StartTimelineDate{A.R.}
Lockheed Martin-Boeing's Advanced Development Programs (Skunk Works) begins refitting the {\it Tarikin III} into the {\it Yama} in response to a UNSA contract for a classified mission payload.
\StopTimelineDate

\StartTimelineDate{A.R.}
Arda has a security detail escort United Nations Governor Leonard Kissinger to {\it ERV Bravo}, Earth-bound, taking advantage of the minimum-energy launch window.
\StopTimelineDate

\StartTimelineDate{B.R.}
The terraformation team at the south pole report findings of dry ice sublimation in response to the asteroid impact.
\StopTimelineDate

\StartTimelineDate{A.R.}
United Nations Security Council resolution 12664 is unanimously adopted, beginning with:

\startTimelineDocument
The Security Council,

Reaffirming the principles and purposes of the Charter of the United Nations,

Determined to combat by all means threats to transplanetary peace and security caused by terrorist acts,

Deploring the gross and systematic violation of human rights, including the repression of peaceful demonstrators, expressing deep concern at the deaths of civilians, and rejecting unequivocally the incitement to hostility and violence against the civilian population made from the highest level of local administration within the United Nations Arcadian Settlement,

Recognizing the inherent right of individual or collective self-defence in accordance with the Charter,

\startitemize[n]
\setupwhitespace[big]
\item Unequivocally condemns in the strongest terms the horrifying terrorist attacks which took place recently in the United Nations Arcadian Settlement, Mars, and regards such acts, like any act of international terrorism, as a threat to international and transplanetary peace and security;

\item Expresses its deepest sympathy and condolences to the victims and their families and of the United Nations Arcadian Settlement;

\item Calls on all States to work together urgently to bring to justice the perpetrators, organizers and sponsors of these terrorist attacks and stresses that those responsible for aiding, supporting or harbouring the perpetrators, organizers and sponsors of these acts will be held accountable;

\item Calls also on the international community to redouble their efforts to prevent and suppress terrorist acts including by increased cooperation and full implementation of the relevant international anti-terrorist conventions and Security Council resolutions;

\item Expresses its readiness to take all necessary steps to respond to the terrorist attacks of 11 September 2001, and to combat all forms of terrorism, in accordance with its responsibilities under the Charter of the United Nations;

\item Authorizes Member States that have notified the Secretary-General, acting nationally or through regional organizations or arrangements, and acting in cooperation with the Secretary-General and the Office of the President of the Space Agency, to take all necessary measures to protect civilians and civilian populated areas under threat of attack in the United Nations Arcadian Settlement, and requests the Member States concerned to inform the Secretary-General immediately of the measures they take pursuant to the authorization conferred by this paragraph which shall be immediately reported to the Security Council;

\item Decides to remain actively seized of the matter.
\stopitemize
\stopTimelineDocument

The resolution recommends to the {\it United Nations Department of Peacekeeping Operations} (UNDPKO) that it place all four permanent standing {\it Rapid Reaction Force} battalions on high alert.

The announcement is made one day prior to the Superbowl. There is no significant public reaction.
\StopTimelineDate

\StartTimelineDate{A.R.}
Footage of riots erupting across Arcadia with security forces responding disproportionately are aired across Earth. Concurrently, Arcadia remains tranquil and undisturbed.
\StopTimelineDate

\StartTimelineDate{A.R.}
Selected Rapid Reaction Force personnel are assigned to UNSA training facilities in Antarctica and Huelva, Spain.
\StopTimelineDate

% Need a strategic transport...
\StartTimelineDate{A.R.}
{\it Yama's} launch window is missed due to low-Earth orbit overly saturated with space debris. UNSA engineers fear they may be approaching the Kessler effect of causing a runaway chain reaction, reducing all objects in orbit. This would threaten the {\it Iterum Shipyard} and all Terran space exploration. Departure is rescheduled for the next launch window of 25 months.
\StopTimelineDate

\StartTimelineDate{A.R.}
Arcadia's {\it ERV Bravo} carrying Kissinger lands at Edwards Air Force Base, California, NAU. The spacecraft becomes a UNSA asset.
\StopTimelineDate

% Yama takes a free-return trajectory taking 180 days and drops off personnel and material
\StartTimelineDate{A.R.}
The {\it Yama}, a manned and remotely piloted spacecraft, completes trans-Martian journey, aerobreaking into Martian geostationary orbit. 

{\it Yama} carries a payload of remotely operated equipment destined for Phobos, a photographic reconnaissance satellite for sun-synchronous orbit, as well as a single Rapid Reaction Force battalion. The battalion is to be deployed as the {\it United Nations Emergency Assistance Peacekeeping Force}. Its numbers are drawn principally from North American Union and European Union airborne light infantry units.
\StopTimelineDate

\StartTimelineDate{A.R.}
{\it Yama's} commanding officer Lieutenant Colonel Dragov issues warning orders down the chain of command to prepare for insertion, geostationary orbit, Mars.
\StopTimelineDate

\StartTimelineDate{A.R.}
Arcadia's 3rd militia battalion's attached Signal Corps detects a non-RAP transponder signal.
\StopTimelineDate

\StartTimelineDate{A.R.}
An aircraft crashes 40 km south of RAP Starport carrying 3600 kg of cocaine. The Ministry of Transportation's investigators confirm that the aircraft's tailnumber was registered to NAU CIA. Chemical analysis suggests it to be the first known synthetic cocaine, as well as having been synthesized {\it in situ}.
\StopTimelineDate

\stoplines

\StopChapter

