% This is part of the Avaneya Project Crew Handbook.
% Copyright (C) 2010, 2011, 2012
%   Kshatra Corp.
% See the file License for copying conditions.

% Getting Involved section
\StartSection{Getting Involved}
This is an exciting project. For whatever reason, since this project was first announced, there has never been any shortage of people wanting to get involved. This is a good thing. 

We also welcome and encourage spectators. There is nothing wrong with that. They are most welcome to monitor the master branch (\in{section}[Using Bzr]); idle on IRC with us(\in{section}[Internet Relay Chat]); subscribe, read, and post on the mailing lists (\in{section}[Mailing Lists]); or what have you. But no one should request to be added to the project crew unless they are actually serious.

Just about everyone is busy with a life of their own. No one is required to involve themselves with this project. If they wish to become involved, they can. But if they simply wish to associate themselves merely for the sake of association, or genuinely come with the best of intentions, but lack the time, it is not in any of the project maintainers' interests. Consider that they too have limited time available for helping new comers to become acquainted with the project.

With this in mind, it was realized that the best approach is to accept new comers based on their having actually contributed. That protects both project maintainers' time, the new comer's, as well as ensuring the project maintains its productivity.

All contributions need not be large and complex to be useful. It might be as simple as one line of code changed buried deep in the engine to repair a serious crash, a few human readable strings of the user interface translated into French, some typos, or perhaps a whole set of complex shaders.

If you would like to get involved, the best way to do so is to solve a problem, provide something needed, or propose a solution you have thought about and are willing to implement. Everyone that contributes is given full attribution. To get an idea of the project's immediate needs, take a look at the bug tracker described in \in{section}[Bug Tracker] which keeps track of not just software bugs, but any issue that needs to be tracked.

We will discuss some of the different ways of being involved later in \in{section}[Specialties].

