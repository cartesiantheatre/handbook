% This is part of the Avaneya Project Crew Handbook.
% Copyright (C) 2010, 2011, 2012
%   Kshatra Corp.
% See the file License for copying conditions.

% Aviation section...
\StartSection{Aviation}

Aviation is as viable as a means of transportation and exploration on Mars as it is on Earth. This section raises some of the considerations that must be taken into account in the design of all vehicles capable of flight.

\StartSubSection{Aerostats}
Aerostats are a class of aerial vehicles that are kept alift through the boyant force provided by an inflatable bag. They can lift the weight equivalent to the volume of the fluid they displace. This is known as the {\it Archimedes' principle}. Since air is a fluid, the size of the bag indicates how much it can lift. Further, since the Martian atmosphere is less dense than on Earth, they need to be larger on Mars to lift the same mass. This works out to about 10 grams of effective lift for every cubic metre of balloon. So a bag of 30,000 m\high{3} could lift a 300 kg mass, which is about 40 metres in diameter. This is the same as on Earth, but up at 30,000 metres. This would have a mass of about 50 kg if using the standard plastic film which is only 10 microns thick.

\crlf
\placeformula[formula:aerostat diameter to payload]
\startformula
\math{M_{payload}(d) = \frac{{\rho}_a\pi d^{3}}{8}}
\stopformula
\startlegend
\leg M_{payload} \\ total payload \\ kg \\
\leg {\rho}_a \\ atmospheric density \\ \frac{16 g}{1 m^{3}} \\
\leg d \\ diameter of bag \\ m \\
\stoplegend
\crlf

A useful formula for modellers is presented in \in{formula}[formula:aerostat diameter to payload]. It will determine how much an aerostat, such as any dirigible, can carry as a function of the bag's diameter. Just plug in a diameter and you will know how much it will be able to carry. Bear in mind that the payload must also account for any kind of gondola or basket area for passengers or instruments, if necessary as well, along with the mass of the bag's fabric.

A comprehensive evaluation of the different fabrics to choose from for the bag is provided in {\it A Realistic Comparison of Biaxial Performance of Nylon 6,6 and Nylon 6 Fabrics Used in Passive Restraints--Airbags}\footnotecite[keshavaraj1996] by Keshavaraj {\it et al.}. Biaxial nylon 6 appears to be a good choice. It is only a third the thickness of a plastic bag. Like all nylons, it also ages well. It does not leak and has a high specific strength.\footnote{A material's strength which is the force per unit of area at the time of failure divided by the material's density. Sometimes also called the strength--to--weight ratio} It is abrasion resistant, which is especially important when it comes in contact with the sometimes rocky surfaces of Mars. If that happens, it also has good energy--absorption properties, but the dangers of a puncture on a nearby rock will still be a concern.

In terms of the flight characteristics, aerostats should be easier to fly on Mars than on Earth. This is because the molecular weight of the Martian atmosphere's mostly \chemical{CO_2} is 44 amu, but Earth's mostly nitrogen and oxygen mix is 29 amu. The atmospheric circulation is simpler and more consistent than on Earth which turns on the possibility of aerial highways for trade routes. The circulation has been very well mapped out for a century. You could even fly half way around the planet in less than a week.

But what to fill the bag with? There are several options. Using \chemical{H_2} is dangerous on Earth because it is explosive in the presence of the oxygen in our atmosphere, but it is perfectly fine on Mars. Unfortunately it could be expensive in terms of electricity to produce. 

Another option is just plain and simple \chemical{H_2O} water vapour. This is possible because a water molecule is only 18 amu, whereas atmospheric \chemical{CO_2} is heavier at 44 amu. It is cheap to produce, but must be heated to at least \math{5^{\circ}}C or the pressure will be too low to fly. Remember that the gas's temperature, pressure, and volume are all interrelated through the {\it ideal gas law}, and therefore the amount it can lift. If we paint the bag black, it would work during day only. 

Yet another option is even simpler, \chemical{CO_2}. Although it may be the poor man's gas, it can still work. It would, however, require a bag larger than the previous options and would also have to be solar heated.

\StartSubSection{Fixed Wing}
Fixed wing aircraft are possible on Mars, either unpowered in the case of gliders, or powered. The latter are possible in the Red Planet's thin atmosphere for the same reasons as aircraft at high altitude in Earth's stratosphere.

Subsonic aircraft travelling at about 700 km/h can use long straight wings for a better lift--to--weight ratio. A rocket is not necessarily required and a propeller could do the job. If the aircraft's form--factor incorporates a tilt--rotor, then vertical take off and landings would be possible (VTOL). They are also efficient for long range flights, and even more efficient in the case of ultralights.

Supersonic rocket--planes are possible. Small delta wings would work better than the straight wings in the subsonic profile. Like the subsonic aircraft, VTOL would be possible if designed with ventral jets.

Another supersonic alternative is the combustion ramjet. One could use an engine that burns silane (\chemical{SiH_4}) fuel directly in the \chemical{CO_2}. This is excellent because it means that we would not need to carry an oxidizer. See \in{section}[Specific Impulse] for more on this.

\StartSubSection{Ascent & Descent Rockets}
Mars ascent vehicles (MAV) are capable of leaving the surface of Mars and entering low Martian orbit. Once landed, they can refuel themselves automatically from the Martian atmosphere using the \goto{Sabatier}[Sabatier Reactor] technology. They are typically lighter and carry a much smaller payload than \goto{ERVs}[Earth Return Vehicle (ERV)] because they are not required to support the requirements of interplanetary travel. They eventually become superannuated with the adoption of the MADV.

Mars Ascent--Descent Vehicle (MADV) vehicles combine the features of the MAV and a lander into one. They are capable of leaving the surface of Mars to enter low Martian orbit and vice versa. This is useful for carrying personnel and materials in either direction. Some interplanetary spacecraft are not designed for atmospheric entry and landing, so they may transport one or more MADVs with them. Once landed, they can begin refuelling themselves automatically from the Martian atmosphere using the \goto{Sabatier}[Sabatier Reactor] technology.

\StartSubSection{Trans--orbital Rockets}
A rocket \quote{hopper} is yet another type of rocket vehicle with suborbital ballistics. It is useful for trans--global suborbital flight where the atmosphere is so thin, it is almost drag free. One could travel half way around the planet in less than an hour. It has its disadvantages though. Unlike aircraft with wings, any course corrections that need to be made require it to burn more fuel. It is not very manoeuvrable either. For short and medium distances for travel, it is not very efficient either.

The hopper has several options far as propellants go. Using methane (\chemical{CH_4 / O_2}) and oxygen is a good propellant, but depending on how it is acquired, it could be very expensive.

Using a nuclear--thermal rocket engine (NTR) is another possibility. It does not require a chemical propellant, but instead operates more like a \quote{flying steam kettle}. A solid nuclear fission reactor heats up some kind of fluid, vaporizes it, and then pushes off of it by ejecting it out the nozzle to produce thrust. Any liquid will do, such as \chemical{H_2} which has an exhaust speed of 9 km/s, but is energy intensive to produce. 

Cheap \chemical{CO_2} could be used instead, but its exhaust speed is at best only 2.3 km/s. If we did this, we would have a nuclear rocket that uses indigenous Martian fuel (NIMF). Although the fuel is not very exotic, the advantage is that the rocket could refuel itself anywhere and it would not drain a base's \chemical{CH_4 / O_2} stocks. This gives surveyors complete mobility over the entire planet.

The reactor would require bomb--grade fissionable material, such as 93\% enriched uranium--235 or its plutonium equivalent. A containment breach would result in a dispersal of radioactive material all over the place. If the rocket was at a high altitude when an accident occurred, it would mean that an area very large would be affected.

