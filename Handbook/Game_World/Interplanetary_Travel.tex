% This is part of the Avaneya Project Crew Handbook.
% Copyright (C) 2010, 2011, 2012
%   Kshatra Corp.
% See the file License for copying conditions.

% Interplanetary Travel section...
\StartSection{Interplanetary Travel}

This section will discuss interplanetary travel between Earth and Mars. This includes both from the former to the latter and vice versa.

%\StartSubSection{Trajectory to Mars}

A minimum--energy launch window is a time when the least amount of energy is required to send an object from one orbit in the solar system to another. It can involve taking advantage of the orbital positions of both the start and destination, useful gravitational fields {\it en route}, and so on. In the case of a trajectory from Earth to Mars, it occurs at intervals of approximately 2.135 years or 780 days. Depending on the constraints one optimizes for, such as energy, time, safety, and whether the cargo is piloted or unpiloted, different manoeuvres are available.

For piloted cargo there are at least three options to consider. A Hohmann type--I orbital manoeuvre has a transit time of only 6 months, but requires a lot of propellant.

A Hohmann type--II orbital manoeuvre is slow, but efficient. It has a small propulsion requirement and is therefore considered a {\it minimum energy trajectory}. It requires high thrust engines firing in two impulses. The trip time is approximately 8.5 months one way.

A third method takes advantage of the Sun and is called a {\it close--to--minimum energy trajectory}. Assume one departs Earth with an escape speed of 3.34 km/s. Since Earth is travelling at about 30 km/s already, plus the spacecraft's propulsion speed of 3.34 km/s, the latter would have a great deal of kinetic energy that it needs to dump in order to safely enter Martian orbit. It can do this by transferring some of its kinetic energy into gravitational potential energy {\it en route} by decelerating or \quote{climbing out} of the Sun's very large gravity well. Since Mars is travelling at roughly 24 km/s in the same direction as the spacecraft's trajectory, the spacecraft would only need to reduce 9 km/s (\math{33 km/s - 24 km/s}) worth of kinetic energy to a safe level. 

With the energy transferred to the Sun, the spacecraft can be safely captured into Martian orbit by performing a deceleration manoeuvre made possibly with the planet's substantial gravity well and by aerobraking through its atmosphere. This method takes only 6 months one way. To come back to Earth it would take 550 days. If, for whatever reason, the crew decide to abort before their arrival, they can continue on a free return trajectory ride back home that swings by Mars without stopping. If they did this, the total trip would take about 2 years before it returned to Earth.

Unpiloted cargo would be best to take this third method of the 3.34 km/s departure which provides the free-return trajectory option. It offers a good compromise between energy, time, and safety.

%\StartSubSection{Trajectory to Earth}

For returning back to Earth, like the journey to Mars, the same minimum--energy launch window intervals of approximately every 2.135 years or 780 days still applies. If the cargo is piloted, a departure velocity of 4 km/s is necessary.

