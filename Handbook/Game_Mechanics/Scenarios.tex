% This is part of the Avaneya Project Crew Handbook.
% Copyright (C) 2010, 2011, 2012 Cartesian Theatre <kip@thevertigo.com>.
% See the file Copying for details on copying conditions.

% Scenarios section...
\StartSection{Scenarios}

Scenarios are at the heart of Avaneya. They can be selected specifically through the single player's custom mode, enjoyed with your friends over a LAN, or incorporated in creative ways in the single player's campaign mode -- which is itself just a series of well scripted scenarios. But not all scenarios can be played in all game modes. Some are available only in the single player campaign mode, while others only for multiplayer over LAN.

These scenarios seek not only to entertain, but also to inform by drawing upon issues in the environmental, social, political, economic, agricultural, and other realms. But each scenario should never draw strictly from only one of the aforementioned. Remember that, as we discussed back in \in{section}[Leitmotifs], everything is connected -- even if not always apparent. As an example, the issue of water privatization is relevant to discussions in at least all of the aforementioned.

In all scenarios, the user is responsible for directly controlling, working with, or maintaining Arcadians and their infrastructure. At other times, however, they may be micromanaging only a handful of units perform a task, such as assisting a remote survey team locate water ice as opposed to erecting or managing an entire city.

Avaneya is different from other strategy games in that you are not the only source of influence over the units you think you are in control of. Even though you at times may be functioning like a mayor of an Arcadian city, there are other influences that will attempt to reshape the world you are immersed in. These could be, depending on the scenario, transplanetary corporations, the United Nations, Terran intelligence agencies, and others, with the lines frequently blurring between them. At times you will feel powerless. At other times you will know exactly what you need to do.

Scenarios can incorporate multiple goals. These goals, depending on how the scenarios are scripted, play a role in conditionally determining whether the user has successfully completed a scenario or not. In the case of the single player's campaign mode, a scenario must be completed before the user can advance to the next one.

Let us now take a look at some of the general types of scenarios. Bear in mind that this is only a subset of all the possibilities, since new ones can and will be written to drive the engine in novel ways. You will note that the Genuine Progress Indicator (GPI) described \in{section}[Genuine Progress Indicator] is a recurring theme.

\StartSubSection{No Goal} 

A scenario may be goalless. In this scenario, the user erects their city and is responsible for its management without thought of any terminal condition for success, such as a time limit. But as with any other scenarios, the user can still strive to maximize their GPI.

\StartSubSection{Achieve Minimal GPI}

The user must achieve a minimal GPI. This may or may not involve a time constraint. Although the GPI is a recurring theme in Avaneya, scenarios that require this goal bring greater attention to it than others.

%\StartSubSection{Free Trade}

%The North American Union would like Arcadia to become a signatory to the North American Union Free Trade Agreement.

\StartSubSection{Combat Globalization}

In previous empires, whether they were Persian, Greek, Roman, Macedonian, or what have you, it was always clear to everyone that, regardless of the rationale given for their creation, they knew they were doing it -- that an empire {\it existed}. There was no existential question raised since anyone could physically point to, or see, a Roman legion at home, or garrisoned abroad on some distant frontier of Britannia, or where have you.

There is a process that is not particularly complex, seldom known by the general public, and yet incredibly effective in the creation of modern empires through the clandestine means of manipulation of the nations subject to its control. It is at least as effective as the cavalry and bayonet approach exercised in previous centuries. It works through a series of escalating measures that involve combinations of usury, bribery, political manipulation, assassination, and as a last resort, not always predictable military force. 

This process is sometimes referred to euphemistically as {\it globalization}. Through globalization it is not always necessary that this creation and expansion of empire be so conspicuous. Nor is it even necessary that its benefactors, such as the American people, are even made to be aware that their way of life is enabled through the fruits of an empire which socializes the true costs of all that it reaps in the form of slavery and exploitation abroad.

\StartSubSubSection{Step One: The Economic Hit Men}
There are three steps at work in the formula. First, economic hit men start by identifying a nation with the desired resources. They attempt to corrupt the nation's government officials into accepting very large loans by any means necessary. If successful, arrangements are made for the transfer by working through large international financial organizations, such as the World Bank or International Monetary Fund.

The money is not actually intended for public infrastructure, but is instead destined for large private corporations with close relations to the economic hit men. It is true that the loans are sometimes used for useful things, such as power plants, water treatment and industrial facilities, or what have you, but in such a way so as to be beneficial primarily to the corporations, secondarily to a handful of people within the target nation. Nevertheless, the public is left with the responsibility of debt repayment. Unfortunately the debt is so large that it can never be re--payed. This was intentional.

Eventually the economic hit men are sent in again to remind the government of its responsibility to repay the debt. Since the debt is so large that it cannot be repaid, the debt is used as leverage to control them since they are now beholden to the money lenders. Under the guise of {\it quid pro quo}, various remedial measures are advised to address their debt. These measures go by various euphemisms, such as \quote{conditionalities}, \quote{structural adjustments}, or \quote{good governance policies}. Large assets may have to be sold off. These may include water, utility companies, oil, education, penal and insurance systems, social services, or what have. It may involve new policies, such as supporting a specific UN resolution, permitting the presence of foreign military bases, or possibly even deploying their own forces abroad in support of the lender's. Virtually anything goes.

\StartSubSubSection{Step Two: Jackals}
If the government is non--compliant, the second step is to send in the jackals. The jackals are tasked with either assassinating or overthrowing an often democratically elected head of state through whatever means necessary. This may involve staging populist uprisings, giving them the appearance of grass--roots activism. This frequently draws on the exploitation of well meaning individuals with the best of intentions who are being misled. This second step, the staging of popular uprisings, was well illustrated by the \quote{Green Revolution} during the 2009 Iranian presidential election.\footnotecite[hersh2008]\footnotecite[ross2007]

\StartSubSubSection{Step Three: Foreign Military Intervention}
In some cases, both measures fail. The third and last resort to be undertaken after the failure of the jackals is in the application of military force. Such was the case in Iraq in 1991 when Saddam refused to accept large loans and all attempts at assassination repeatedly failed. This was probably due, at least in part, to the proficiency he gained in conducting similar affairs while having previously worked for the CIA to assassinate another Iraqi president that had preceded him.

This formula has been well documented by those directly involved in its use.\footnotecite[perkins2005] It was used successfully in Guatemala (1954), Ecuador (1981), Panama (1981), Venezuala (1981), and Iraq (2002), to name a few among the innumerable.\footnotecite[zeitgeist_addendum] One might see it as an example of a political \quote{design pattern}, akin to those used in architecture and software engineering.

\StartSubSubSection{Rationale For Private Consultants}
Private consultants are usually used as the economic hit men instead of directly employing intelligence agency personnel. This was realized as the future model in Washington after the successful overthrow in 1953 of Iran's democratically elected Prime Minister Mohammad Mossadegh by British and American intelligence. Mossadegh had attempted to remove the British monopoly of the nation's oil assets. Instrumental in the affair was a personal relative of Teddy Roosevelt who was also a CIA field agent. Had he been exposed while he was in Iran, it would have been disastrous. Had a private consultant been exposed, there would have been less chance of adverse consequences for foreign governments.

\StartSubSubSection{Game Adaptations}
One such game scenario could involve Arcadian water. Water is even more precious on Mars than on Earth since a greater expenditure of resources is necessary in attaining potable water. A corporation such as Bechtel--Biwater\index{Bechtel--Biwater} may attempt to privatize the artesian aquifers or water ice sites your city depends on. Without water, one cannot survive. If one can control what everyone needs to survive, they can control the destiny of all those dependent on it. Thus, the opportunity for abuse is possibly even greater on Mars than on Earth.

The user will have to use their head and be creative in exercising any attempt at water reclamation. Some options might involve reliance on a judicial system, or possibly even the use of force if the latter proves corrupt.

Indeed, a similar scenario actually played out in Bolivia in 1999. Economic hit men were so successful in corrupting the government that it actually became illegal for citizens to even collect rain water for drinking as they had for centuries. The people knew they were being exploited, threw the corporation responsible out, and successfully reclaimed their water.

\StartSubSection{Improve Energy Supply}

Your city needs energy and it may not have enough to power everything on the grid. The user must adjust factors in their city that will improve the situation. This may mean building a nuclear plant, a geothermal well, erecting photovoltaic panels, or perhaps looking for ways to increase efficiency, such as examining how much energy is required to produce the different types of things that people eat and exploring alternatives.

\StartSubSection{Improve Public Health}

Many factors effect public health, among the most important are what the public consumes. But other factors contribute as well, such as education, annual take home income, available free time, and many others. Scenarios that depend on this goal require the user to improve the public's health.

\StartSubSection{Improve Transportation Network}

As a city grows, efficiency of movement can become a problem. This goal requires the user to have their city's transportation network increase in efficiency.

\StartSubSection{Natural Disasters}

The user must rebuild their city in response to damage sustained from solar flares, micro meteorite strikes, land slides, marsquakes, or other natural disasters. Yes, Mars experiences all of the aforementioned. But some \quote{natural} disasters can be caused through human mishandling of the natural environment. For example, if an entire city block is swallowed whole into the ground, chances are the aquifer it was sitting on was depleted. In many situations, the user may be given a chance to avert the disaster entirely.

\StartSubSection{Prevent Staged Terror}

Staged terrorism is as common on Mars as it is on Earth. The NAU--CIA will work tirelessly to stage acts of terror to be blamed on arbitrary groups which are usually just fronts for itself. The user must use whatever means necessary to prevent acts of staged terror. If they fail to do this, as an example, the NAU could claim sabotage of a mineral extraction site they are leasing from your city as justification for a troop deployment.

\StartSubSection{Protect Another City}

The user must come to the assistance of another Arcadian city as it comes under siege at the hands of an armed foreign aggressor, such as North American Union troops, the United Nations peacekeepers, CIA backed contras, private security contractors, or what have you. Constraints may vary, such as preventing the city from exceeding a certain threshold of material loss, the protection of its inhabitants, and so on.

\StartSubSection{Protect From Junk Science & Media}

People can never exercise any greater degree of control over their own lives and the world around them than they are uninformed. We make decisions that are influenced by what we know and other environmental influences. If the source of the information we rely on has nefarious conflicts of interest, we will operate bearing information that does not consider our best interests.

You may need to reduce the influence corporations, nefarious states, and special interest groups wield in socially engineering public opinion. Their influence may be overt, such as through advertising, or it may be more subvert, such as through funding junk science\index{Junk science} or corrupting a university\footnotecite[annett2006] or scientific body.\footnotecite[corbett_nist_satire]\footnotecite[merchants_of_doubt]\footnotecite[the_china_study]

\StartSubSection{Protect Human Rights}

Articles XI --- XXVI of the Rubicon Act, as described on \at{page}[item:rubicon_act_fundamental_rights_first], enumerate the fundamental freedoms and human rights Arcadians enjoy with respect to their new republic. The user is responsible for ensuring that they are upheld.

\StartSubSection{Recover From GMO Terminator Gene}

People that have nothing to eat will eventually die. If the city's greenhouses produce nothing after having changed little other than adopting new genetically modified seed stock, chances are the greenhouse did what farmers had been doing for centuries. They stored seed and planted it with the expectation that life would grow, only to find that the seeds had been genetically programmed by the vendor to be non--reusable. This way, your city's food supply is held hostage to the vendor since it will have to buy new seeds annually, even though there are still plenty in stock.

Scenarios that depend on this goal require the user to restore their annual agricultural yield to a variable minimum and to ensure that the food is safe to eat.

\StartSubSection{Reduce Crime}

The user must reduce crime by addressing factors, such as education, affordable living, health, and other fundamental factors and environmental conditions that give rise to crime since crime is not random.

