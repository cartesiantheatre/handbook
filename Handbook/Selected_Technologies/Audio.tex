% This is part of the Avaneya Project Crew Handbook.
% Copyright (C) 2010, 2011, 2012 Cartesian Theatre <kip@thevertigo.com>.
% See the file Copying for details on copying conditions.

% Audio section...
\StartSection{Audio}

\startitemize[4]
\head {\em Audacity}

We use Audacity for audio recording, editing, post--processing, effects, and more. It can take advantage of any of the multitude of the LADSPA and VST plugins compatible with it. It can also export to surround sound formats.

\head {\em libasound_module_pcm_a52}

The {\tt libasound_module_pcm_a52} library is a plugin for ALSA and therefore only available on supported platforms (usually GNU/Linux). It provides a way to encode a digital surround sound stream in A52 format to a pass--through device (e.g S/PDIF). The user's home theatre receiver then takes over from there. We have to be careful here. See \in{section}[{Restricted Data Considerations}] for more information on why.

\head {\em libmikmod}

We use libmikmod for playback of tracker music modules. Some people were requesting tracker support. Since it should not take very much effort to integrate into our engine's audio subsystem, as described in \in{section}[Audio], we considered this a reasonable request. One caveat is that we need to make sure the library is compiled thread safe.

\head {\em OpenAL}

OpenAL is used for 3D spatial audio rendering. Actual decoding of audio data is done through other APIs.

\head {\em SDL_audio}

This SDL module is used to provide audio decoding. It provides a standardized interface to various decoder backends, such as Ogg Vorbis and others.
\stopitemize

