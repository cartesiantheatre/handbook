% This is part of the Avaneya Project Crew Handbook.
% Copyright (C) 2010, 2011, 2012
%   Kshatra Corp.
% See the file License for copying conditions.

% The Game chapter...
\StartChapter{The Game}

\StartSection{Why?}
Although the answers are many, they are best brought to light when the question is further refined. {\it Can a game be useful to society?}

Too often, people have come out of experiencing great dystopian science fiction only to say to themselves, Thank goodness we don't live in {\it that} world. That needed to change. Perhaps nothing could have better deserved a finger pointed scornfully at it for having prompted this difficult undertaking that that attitude.

Dystopian science fiction may be among the most honest kind of story telling when examined in the context of history. But unless people can see the pedagogical relevance, analogies will remain vague and metaphors too cryptic for it to be most useful. Perhaps it could be said then that {\it constructive} dystopian science fiction is more reluctant in withholding its relations with reality.

\placefigure
    [right,2*hang]
    {Captured by Viking orbiter I on 22 February 1980. {\it Valles Marineris} is the scar on Mars over 3000 km long and up to 8 km deep.}
    {\externalfigure[The_Game/Images/Valles_Marineris.png][][width=.5\textwidth]}

Still, constructive dystopian science fiction, or call it what you will, ought to strive to be well received. One of the ways it does this is in being only partially fictional, because it is the plausible scientific dimension that usually attracts people to it in the first place.

But science, by its nature, is a thinking and knowledge oriented enterprise. It is the study of reality, and yet fiction, by definition, is outside of reality. So science fiction must be a mediator between the two, often acting as an {\it avant-garde} for both, with the dystopian variety debatable which end it is more proximate.

This can set the bar high for what its patrons expect and requires designers to pay a great deal of attention to detail. A certain degree of creative license is expected, but people still prefer it to be as consistent with what we already know to be true, or what is at least reasonably plausible.

Another important reason why this project is necessary is that there is very little, if any, {\it free}, commercial, games for the GNU operating system. When people use GNU, they are treated as second class citizens in many respects -- not least of which is the availability of good games. Not only are there very few higher production titles that are available for GNU, those that exist are usually proprietary, and even then, generally bad ports using deprecated APIs, poorly packaged,\footnote{Assuming the platform's native package manager was even used at all.} and integrating horribly into the user's desktop -- ignoring useful community driven conventions like {\it freedesktop.org}. 

A user should be able to simply play the game without having to worry about manually setting environment variables or using the right version of ALSA\footnote{{\it Advanced Linux Sound Architecture}, a component of the Linux kernel responsible for providing audio capabilities. Reliable audio support under GNU/Linux has historically been a nightmare.} to get it working. The user should be treated as a first class citizen as much as reasonably possible. But just as the users can enjoy the game, the developers can enjoy building on the hard work and thoughtfulness that went into the community driven standards and technologies that it is driven by.

In some sense, Avaneya is part of an overall effort to complete the GNU operating system. When the GNU project set out to replace the body of non-free software required to use a computer, Richard Stallman once remarked that it would also require games since a complete operating system needed games too.\footnotecite[linux_and_gnu] One of the aims of this project then is to lend a hand by providing the many good people who commit themselves to free software for ethical reasons with a game they might enjoy.

Another reason is to present a case for the {\it responsible} human settlement of Mars. This can be done by drawing on the existing research to and adapting it creatively for use in a non--casual game. This doubles by encouraging a greater interest in science by showing how exciting and useful it can be to people of all ages.

A final reason for Avaneya is found in the existing proprietary games themselves. They tend to appeal more to the mainstream proprietary user. They do not share the {\it software libre} community's values of freedom, and not merely in how they are licensed. The authors of these games tend to be comfortable with the assumption that their users are unconcerned with learning about the world they live in while playing. All games certainly need not offer a remedy for this. That would deprive many of us of the escapism they can offer which is harmless in moderation. But when all of the most prevalent share that assumption, and when one considers the amount of time contemporary society loses to games, they become social liabilities. Of all the literally hundreds of contemporary video games that carry a theme of terrorism, of what proportion staged terror? It may be said that while a good game departs from reality, a better one can complement it. We must be free not only in our licensing, but free to examine even the most sacred of assumptions.

Many of these proprietary games are meaningless and have no real value.\footnotecite[blackwater_game] Their authors are able to get away with this because they know that many proprietary users see efforts at computing with a conscience as pointless since machines are just tools capable of existence external of any social context. Such a superficial perception of machines is well illustrated in the open source school of thought in contrast to its cousin, {\it software libre}.\footnotecite[open_source_misses_the_point]

The truth is, all technology we create always has a social context. It can never exist suspended independently in a vacuum.\footnotecite[black2012] Free games can take a different approach by not ignoring the social context, by not authoring software as though such an ideal were possible, and by not treating the user as though they were reductive abstract economic units, consumers, there to be exploited. Instead, it recognizes that they are living human beings. 

Proprietary technology developers create technology with the {\it profit motive}\index{profit motive} in mind. While it is important that free software is sustainable, the {\it purpose motive}\index{purpose motive} plays at least as fundamental a role. This is why it usually tends to be of higher craftsmanship and does not ignore the social ramifications caused when the superficial values that produce proprietary technology are left to run their course.\footnotecite[rotten_fruit2011_death_toll]\footnotecite[rotten_fruit2011_alarm_over_suicides]\footnotecite[rotten_fruit2011_stand_24_hours]\footnotecite[rotten_fruit2011_refused_benefits]

We assert that the choice between using free and proprietary software is not arbitrary in the same way one selects chocolate over vanilla, since such a choice says nothing of the user's values and constitution. When users are users of software {\it libre} for the purpose of freedom, they are different from proprietary users in terms of their values, not merely just in the software that they happen to have installed on their computer. Therefore, a game for such an audience ought to strive beyond merely a distinction in licensing and also reflect the progressive thinking and themes of {\it libre} culture in spirit as well. If you understand this, you will understand this game.

\StartSection{Classification}
People tend to struggle to classify Avaneya. It is what it is, but the closest near minimal traditional descriptive categories would be the classic city builder and management simulations and the real time strategy. But upon closer examination, you will find it a political thriller, drama, educational, science fiction, and more.

\StartSection{Likely Users}
The game so far has attracted a fairly large base of followers. From what can be observed at this time, the game appears to appeal to those with an interest in

\startitemize[4]
\item
{\it games that are not only fun, but useful;}
\item
{\it challenging the consensus of reality;}
\item
{\it software libre;}
\item
{\it a social conscience;}
\item
{\it science fiction;}
\item
{\it exploring the interconnectedness of everything.}
\stopitemize

The game may take place in the future, but it deals with current problems. The best way to get an idea of the intended audience is to quickly scan both \in{chapter}[Leitmotifs] and \in{section}[Resources For Everyone]. You will be in a better position to try and gauge the type of audience that this game resonates with after doing that. 

\StartSection{Unlikely Users}
Avaneya is a {\it sui generis}.\footnote{{\it "Literally meaning of its own kind / genus or unique in its characteristics. The expression is often used in analytic philosophy to indicate an idea, an entity, or a reality which cannot be included in a wider concept,"} (Wikipedia).} It is not like other games, and thus it is not for all people. It does not try to be, nor will it ever be.

Those with a brief attention span, limiting themselves to seek only immediate gratification; those that believe everything they consume was spawned into existence, beginning its life in a can with little appreciation for process; and accidentalists, to name a few, will probably not enjoy this game. There are already many such games that appeal to that type of audience, so that need not be our aim here.

This game will challenge you to think, and possibly even offend you. It challenges the consensus of reality, and therefore, potentially, your world view. Consequently, some have accused Avaneya of being a vehicle for culture jamming and political commentary. This project is shamelessly guilty as charged---like the newspapers, film, television, games, and other mainstream media that saturate us. 

The only difference is that, unlike those mediums, the very presence of a normative bias in Avaneya is not subject to dispute and is self evident. Other mediums sometimes pretend to not have one. In any case, you would be very hard pressed to try to find any classical work of science fiction, or really any kind of fiction for that matter, that did not. Moreover, that in itself is not necessarily a bad thing.

\StartSection{Availability}
We do not believe that deliberately {\it restricting} users to a specific platform or hardware is ethical for any other reason than practical technical limitations of the alternatives, such as the absence of a programmable shader interface under {\it any} license. We do, however, strongly {\it promote} a specific platform, that being GNU and any available hardware it can run on capable of providing for the technical requirements of Avaneya.

Since we consider it unethical to encourage people to use non-free software, we will not ourselves ever be the primary maintainers undertaking such an endeavour. However, it would also be unethical to deliberately design it in such a way so as to cripple the efforts of others from doing so. They have the freedom to disagree. Thus, since Avaneya relies on portable libraries, it should not be unreasonable for someone to do this if they do not share our values.

Other GNU distributions that provide alternatives to Linux as the kernel are also considered fair game. These include GNU/kFreeBSD, GNU/Hurd, GNU/NetBSD, and possibly others, depending on their level of maturity, technical feasibility, and demand from the community. As Linux continues to deviate\footnotecite[linux_non_free] from the free (and even open source) philosophy, it becomes increasingly important to encourage and support alternative choices as they become viable.

Some have suggested that maintaining for free platforms can be rather self limiting in terms of user outreach. There was a time when this view had validity. Now, with tens of millions of users worldwide running some flavour of GNU, such as Ubuntu, the perception of a non-existent or marginal demographic  is rather antiquated. Consider that gaming under non-free operating systems such as OS X is still active and lively today, and yet Ubuntu alone (not including other GNU distributions) has arguably surpassed it in terms of popularity with a user base in the tens of millions.\footnotecite[ubuntu_surpasses_osx] GNU long ago stopped being merely an operating system for hobbyists. It is so ubiquitous, you will find it anywhere from the Parliaments of the world to up in orbit.

\StopChapter

