% This is part of the Avaneya Project Crew Handbook.
% Copyright (C) 2010, 2011, 2012
%   Kshatra Corp.
% See the file License for copying conditions.

% Avaneya Code of Conduct. This file is intended to be included 
%  within another document, hence no sectioning command...
\midaligned{Version 0.1}
\blank
\midaligned{\tt Copyright \copyright\ 2010, 2011, 2012 Kshatra Corp.}
\blank

This Code of Conduct covers our behaviour as members of the Avaneya Community, in any forum, mailing list, wiki, web site, IRC channel, install-fest, public meeting or private correspondence. Avaneya governance bodies are ultimately accountable to the project lead and will arbitrate in any dispute over the conduct of a member of the community.

\startitemize[R]
\item
{\bf Be transparent, but don't be reckless.} This is a free software project and so we must work as transparently as possible. Having said that, no one likes to have the ending of a good story spoiled before they have had a chance to see it for themselves. Sometimes when we withhold things from the public, it is merely for artistic purposes and has nothing to do with legalities. Consequently, do not reveal any elements of the storyline or game until we have discussed it and you have been given permission to do so by a project officer who has been appointed by the project lead.

\item
{\bf Be considerate.} Our work will be used by other people, and we in turn will depend on the work of others. Any decision we take will affect users and colleagues, and we should take those consequences into account when making decisions. Avaneya may one day have millions of users and thousands of contributors. Even if it's not obvious at the time, our contributions to Avaneya will impact the work of others. For example, changes to code, infrastructure, policy, documentation, and translations during a release may negatively impact others' work.

\item
{\bf Be respectful.} The Avaneya community and its members treat one another with respect. Everyone can make a valuable contribution to Avaneya. We may not always agree, but disagreement is no excuse for poor behaviour and poor manners. We might all experience some frustration now and then, but we cannot allow that frustration to turn into a personal attack. It's important to remember that a community where people feel uncomfortable or threatened is not a productive one. We expect members of the Avaneya community to be respectful when dealing with other contributors as well as with people outside the Avaneya project and with users of Avaneya.

\item
{\bf Be collaborative.} Collaboration is central to Avaneya and to the larger free software community. This collaboration involves individuals working with others in teams within Avaneya, teams working with each other within Avaneya, and individuals and teams within Avaneya working with other projects outside. This collaboration reduces redundancy, and improves the quality of our work. Internally and externally, we should always be open to collaboration. Wherever possible, we should work closely with dependent upstream projects and others in the free software community to coordinate our technical, advocacy, documentation, and other work. Our work should be done as transparently as possible, without ruining the elements of storytelling, and we should involve as many interested parties as early as possible. If we decide to take a different approach than others, we will let them know early, document our work and inform others regularly of our progress.

\item
{\bf When we disagree, we consult others.} Disagreements, both social and technical, happen all the time and the Avaneya community is no exception. It is important that we resolve disagreements and differing views constructively and with the help of the community and community processes when necessary.

\item
{\bf When we are unsure, we ask for help.} Nobody knows everything, and nobody is expected to be perfect in the Avaneya community. Asking questions avoids many problems down the road, and so questions are encouraged. Those who are asked questions should be responsive and helpful. However, when asking a question, care must be taken to do so in an appropriate forum.

\item
{\bf Step down considerately.} Members of every project come and go and Avaneya is no different. When somebody leaves or disengages from the project, in whole or in part, we ask that they do so in a way that minimises disruption to the project. This means they should tell people they are leaving and take the proper steps to ensure that others can pick up where they left off.
\stopitemize

The Avaneya Code of Conduct is an adaptation of the Ubuntu Project's Code of Conduct, also licensed under the \href{http://creativecommons.org/licenses/by-sa/3.0/}{Creative Commons Attribution-ShareAlike 3.0 Unported} licence (CC BY-SA 3.0). You may re-use it for your own project and modify it as you wish, just please allow others to use your modifications. Remember to give credit to the Ubuntu Project.

