% This is part of the Avaneya Project Crew Handbook.
% Copyright (C) 2010, 2011, 2012 Cartesian Theatre <kip@thevertigo.com>.
% See the file Copying for details on copying conditions.

% Weapons & Related section...
\StartSection{Weapons & Related}

As you probably already noticed, a variety of weaponry surfaced at a number of times during the fictional timeline as well as in the unit tree. It is neither our desire to see their presence one day on the Red Planet, nor to have them as a dominant theme within this game. Nevertheless, Avaneya does not depict an ideal world. If that was the case, there would be nothing for the user to do. 

This section will discuss some of the considerations to take into account in the general design of weapons and similar equipment suited for use on Mars.

\StartSubSection{Explosives & Pyrotechnics}

Explosives and pyrotechnics have many uses on Mars. To wit, excavation during construction, excavation of ice deposits, communications, mining, and military applications to name a few.

Explosives are not difficult to manufacture on Mars since we already know how to produce {\it in situ} both fuels, or what actually burns, and oxidizers, or what supplies the exothermic reaction with electrons. As an example, nano--thermite has already seen creative military applications on Earth\footnotecite[harrit2009] and requires a material which is already plentiful on the Red Planet, haematite.

We should consider the properties of explosives on Mars and how they are different from those on Earth. We would expect them to be much more dangerous on the former body because fragmentation and shrapnel will be airborne for a longer period due to the weaker gravitational field acting on them. This means their effective range would be about 2.66 times greater than on Earth. That means you would require fewer explosives on Mars than on Earth to accomplish the same task.

This also has implications for many different types of weapons, including portable infantry anti--armour weapons that fire explosive projectiles. This class of weapon generally requires direct line of site to be established with a target, but is frequently limited by its maximum effective range. 

Consider that whatever need their projectiles may have had on Earth for a propellant to reach its target, it would be much less on Mars. As an example, the {\it Projector, Infantry, Anti Tank} (PIAT) weapon successfully used by Allied forces in occupied Italy during the Second World War against German armour\footnotecite[extras={ p.~351.}][roy1965] was spring loaded with a maximum range of 320 metres. Under Martian gravity, that is the equivalent of 850 metres -- nearly a kilometre, using only a rudimentary compressed spring's potential energy, and still operating under the assumption of Earth's more energy dampening atmosphere.

But things are even more impressive when we consider the advantage of the thin Martian atmosphere. Penetration capability is related to the amount of kinetic energy (\math{E_k}) the projectile has as it moves through space. This is expressed in the simple \in{formula}[formula:kinetic energy]. The greater the kinetic energy, the greater its penetrating power.

\crlf
\placeformula[formula:kinetic energy]
\startformula
E_k = \frac{1}{2} m|\vec{V}|^{2}
\stopformula
\startlegend
\leg E_k \\ kinetic energy \\ j \\
\leg m \\ mass \\ kg \\
\leg |\vec{V}| \\ speed \\ m/s \\
\stoplegend
\crlf

You will have observed that kinetic energy varies with the square of the magnitude of the velocity. This means that the speed an object travels is a much greater contributor to its kinetic energy than its mass. Since there is less atmospheric attenuation than on Earth, there being only a one--hundredth the atmospheric density, a fragment would maintain its speed, and therefore kinetic energy, for longer.\footnotecite[extras={ {\it Vid.} p.~18 to see how this was considered in a Lunar environment.}][project_horizon_volume_2]

In terms of the actual method of delivery, there are many possibilities, but we will discuss the more likely ones. On Mars, methane, or \chemical{CH_4}, is easily manufactured with the aid of the Sabatier reactor described in \in{section}[Sabatier Reactor]. Methane is also useful in explosives. It can be combined with molecular oxygen, \chemical{O_2}, when both are in either gaseous or liquid states. When both reactants are combined in gaseous states, a powerful explosive is created just out of a simple pressurized container with a detonator. A 35,000 kPa bottle could pack 23,000 kPa of \chemical{O_2} and 12,000 kPa of \chemical{CH_4}, for example.

When the reactants are combined in their liquid states, \chemical{CH_4 / LOx}, even greater quantities can be packed in the same container volume by a factor of three. This is possible because fluid densities are much greater than gasses. This gives an explosive yield roughly twice that of TNT, \chemical{C_6H_2(NO_2)_3CH_3}. One draw back, however, is that liquid states require that the reactants be stored at cryogenic temperatures to reduce boil--off.

One application of the liquid form is in explosive devices for anti--personnel (soft) or anti--vehicle (hard) targets where operational conditions require that munitions be small. The Arcadian RGA1 anti--personnel cryonade launcher found in \in{section}[Countdown] is one such for at least soft targets. Other uses are in claymores and mines for either type of target. When used against soft targets, they need not be directly fatal, but only have sufficient capability to puncture a pressure suit at short range for the atmospheric conditions to finish the job.

Whether one uses liquid or gaseous forms, it is important that the reactants are stored in distinct chambers until they are allowed to mix when the device is armed. Both the liquid and gaseous combination methods just described are highly volatile and this provides an additional degree of safety for the operator. Arming a cryonade in this manner would be akin to pulling the pin on a traditional fragmentation grenade.

For pyrotechnic applications, \index{magnesium, pyrotechnic}magnesium is plentiful on Mars and can be used for signal and illumination flares. Since there is virtually no free molecular oxygen in the atmosphere for it to react with, it is much safer to handle on Mars than on Earth.

\StartSubSection{Firearms}

We consulted with the professional weapons designers of {\it Fabrique Nationale d'Herstal} who were generous with both their time as well as their enthusiasm for science fiction. They provided us with a great deal of advice in the design of the Arcadian R1A1 service rifle first mentioned in \in{section}[Countdown]. It is a fictional adaptation of the FN FAL. They were also kind enough to provide us with a set of useful general considerations to take into account in the design of any firearm suited for use in Martian theatre.

A weapon's design and the application of good metallurgical science will ensure that it is able to withstand the harsh Martian operating temperatures. This means a temperature range of a minimal \math{-90^{\circ}}C to a maximum of \math{20^{\circ}}C and with an average of only \math{-63^{\circ}}C. Even the worst high altitude and arctic conditions Earth has to offer generally do not come anywhere near these figures. 

Frosting would be a recurring problem on Mars for the reason just described. Special chemical lubricants would have to be engineered. These are necessary not only for keeping all load bearing and moving surfaces well lubricated, but also for reducing frosting. Again, good metallurgical science will aid in reducing the likelihood of frost induced misfires and fouling. If possible, battery powered heating elements could be integrated within weapons such that the receiver can maintain a minimal operating temperature.

Since gravity is only \math{38\%} on Mars what it is on Earth, heavier materials are more acceptable. It should be noted however that while the Martian gravitational field is weaker, mass is still mass and a weapon that is massive still requires a greater expenditure of human energy to accelerate than a weapon less massive. An example would be repeatedly adopting and recovering from a prone firing position.

A weapon's action must be tolerant of fine iron oxide dust which is roughly only 1.5 μm on average. One approach is to consider strategies of preventing dust from entering the action in the first place. This could be done with the aid of ejection port covers as found on some existing Terran service weapons.

Since visibility at night is very difficult on Mars, night vision or thermal imaging optics are useful. When using only iron sights, the glow of tritium engraved front and rear posts would be helpful in low light conditions.

In terms of ballistics, bullet projectiles would accelerate much faster on Mars than on Earth because they have only a one--hundredth the atmospheric resistance attenuating their kinetic energy from the moment they start moving down the barrel. This means that point blank ranges would be greatly increased. Bullets would also exhibit much flatter trajectories. Besides these characteristics, the near vacuum atmosphere should not affect the weapon in any other way. On the down side, dust storms would degrade accuracy. 

\StartSubSection{Orbital Kinetic Bombardment}

As related in \in{section}[Contact], UNSA has a manned facility on the Martian moon Phobos. {\it Thor Battery} contains a garrison of a single platoon of thirty soldiers. The installation is powered by three nuclear reactors and is embedded within the moon's porous rock.

{\it Thor Battery} is an orbital weapon's delivery platform. That is, it drops tungsten shells from Phobos on targets on the surface of Mars. Although its shells do not contain an explosive payload, and they are not particularly massive, they reach speeds of at least 9 km/s before hitting their terminal velocity. As already illustrated in \in{formula}[formula:kinetic energy], they are capable of delivering sufficient kinetic energy to a target on the surface in the order of a small tactical or even strategic nuclear warhead.

The battery is clearly a formidable weapon and does not provide its opponents on the surface with much redress. Its launches are difficult to detect because its shells have very small radar cross signatures. With the right equipment, however, an infra--red launch signature could be detected. Unfortunately it is still difficult to localize, meaning a defender might know only that an orbital salvo is {\it en route}, but not its actual target until the strike. 

One other tactic a defender on the surface could employ is to take advantage of the fact that the plasma sheath surrounding a shell during its atmospheric reentry will render its guidance sensors blind. Any attitude correction the shell attempts to apply if tracking a mobile target would be impossible for most of its downward journey of several minutes.

Lastly, the battery's launch window is limited by the position of Phobos with respect to its desired target. This means that a defender will always know what constitutes its potential targets of opportunity based on the time of day.

