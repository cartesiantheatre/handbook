% This is part of the Avaneya Project Crew Handbook.
% Copyright (C) 2010-2018 Cartesian Theatre™ <info@cartesiantheatre.com>.
% See the file Copying for details on copying conditions.

% Environmental section...
\StartSection{Environmental}

We should discuss some of the relevant environmental characteristics of Mars. This is important for shader writers and visual designers in particular.

\StartSubSection{Atmosphere}
As already discussed the Martian atmosphere is 95\% \chemical{CO_2}. Its mean sea level (MSL) pressure ranges from 30 Pa to 1135 Pa. By contrast, Earth's MSL is 101,325 Pa. To compare the two you would need to be 36 km above the Earth's surface to experience the Martian MSL pressure. Plant life requires at least 5,000 Pa.

Because there is very little atmospheric buffering, the temperature will rise and fall in rapid synchronicity with its proximity to the Sun. This varies with its position in its orbit. The temperature ranges from a minimal \math{-90^{\circ}}C to a maximum of \math{35^{\circ}}C and with an average of only \math{-63^{\circ}}C. But despite the extreme temperatures the greatest threat to an exposed human is actually the low atmospheric pressure.

The {\it Armstrong Limit\index{Armstrong Limit}} of 6,262 Pa is the lowest the human body can survive before the \index{vapour pressure}vapour pressure of all exposed liquids, such as tears, saliva, and the liquid wetting the lung's alveolar membrane exceed that of its surrounding atmosphere. They will begin to boil away at this point, but not fluids like blood contained within the skin's pressure barrier. On Earth the Armstrong Limit begins at about 19 km above the surface. On Mars it is already well exceeded at MSL.

It should be noted that locations where the altitude is much lower than MSL, the atmospheric pressure would be the greatest if it were increased through terraformation. {\it Valles Marineris}, as an example, is up to 7 km deep.

\StartSubSection{Desert Threats}
Deserts are dangerous on Earth and the vast ones of Mars are no exception. It is very easy for one to get lost for any number of reasons. The Mars Positioning System (MPS) could stop working due to a hardware failure, a solar flare affecting a relay satellite, a cosmic ray disrupting a receiver or even defects in proprietary software.

Compounding things it is virtually pitch black at night on Mars in the absence of artificial lighting. This is because Mars' two moons Phobos and Deimos do not reflect much solar light. Vehicles and dismounted persons would need proper outdoor lighting.

But there are other threats. One can also freeze to death in the \math{-90^{\circ}}C weather if their suit loses power or the integrated heating elements fail for any reason. Or if they are out for long enough they will run out of oxygen.

One could also run out of water. The average person requires about 1 kg per day under normal circumstance. To extract the water from the soil, perhaps in an emergency, one could setup a greenhouse--tent to condense the water out of the regolith. The regolith contains anywhere from 3\% to 60\% water by weight. If a photovoltaic oven is available, it could be used to melt the water in the soil. Whatever the process of extraction used, water from the soil still needs to undergo distillation due to its high salinity.

\StartSubSection{Regional Characteristics}

The planet's north and south poles are different. The north is mostly water ice and contains about 821,000 km\high{3} of it. It has a more moderate \index{climate}climate than the south and is wettest in the Spring. The south is mostly dry ice (\chemical{CO_2}) and colder. Both locations have gale strength winds and thick permafrost mantles.

Climate varies throughout Mars like on Earth, and not just at its poles. The \index{northern hemisphere}northern hemisphere is advantageous with less seasonal variation in temperature. This is because the planet is closer to the Sun in the northern winter and farthest during Summer. It was for this reason that our fictional UNSA sought a mid--latitude location in {\it Arcadia Planitia}, roughly half way between the north pole and the equator. 

The centre of {\it Arcadia Planitia} is mostly uniform in appearance with its centre at roughly \math{46.7^{\circ}}N \math{192.0^{\circ}}E. The name of the region was borrowed from its original in Ancient Greece, so named after the Greek legend of Arcas\index{Arcas}.

The Martian seasons are not only more predictable, but there is a great deal of water ice and sunlight for year round greenhouse use. Images of a fresh meteorite crater 12 metres across taken in 2008 revealed under the surface a massive blanket of water ice.\footnotecite[water_ice_on_mars] It also turns out fortuitously that this ice is almost completely pure, with only about one percent of it being other matter.

Even though there is a great deal of \index{ice}ice on Mars, it is unstable in the thin Martian atmosphere and you rarely ever see it exposed anywhere other than in the polar regions. This is because the ice rapidly \index{sublimate}sublimates as soon as it is exposed. A solid material is said to sublimate when it skips melting to a liquid and turns directly into a vapour state.

The region has experienced recent lava flows. By recent, in a geological time scale, that means within the last few hundred--million years. Mars most likely is still volcanically active and must have a molten core. It does not however appear to spin around its core which is why there is no magnetic field. This is why \index{magnetic compasses}magnetic compasses are useless on Mars as described in \in{section}[Navigation].

{\it Arcadia Planitia's} windswept landscape consists of a vast, mostly flat, pale tan coloured plain. It has sand dunes of modest height, never approaching anything higher than a few feet with small uniformly sized rocks littering the surface. As the prevailing theory goes these rocks are remnants of some of the underlying bedrock which is an older layer of solidified lava. Every time an asteroid impacts penetrating the younger upper layer the underlying bedrock ejecta is scattered everywhere.

Like all other explored regions it undergoes constant sterilization through intense ultraviolet radiation. This is why there is no {\it known} life on the {\it surface} of Mars.

Visual designers and shader writers are encouraged to draw on the resources provided by our {\it Avaneya: Viking Lander Remastered DVD} described in detail in \in{chapter}[Viking Lander Remastered]. No amount of writing really can account for the value of actual high resolution imagery captured at various times of day and across all seasons from this region.

\StartSubSection{Radiation}

Radiation is always a concern on Mars. Levels are 50 times those of Earth. It is usually measured in either rems or Sieverts (sv in SI). One rem is the equivalent of \math{1/100} sv. The amount or dosage received, combined with various weighting factors, determine how effective the radiation is likely to cause biological damage.

Exposure can be grouped into two categories, acute and extended. Acute or prompt exposure is shorter than the time of cellular regeneration. Anything less than 50 rems is typically sub--clinical and should not produce anything other than blood changes. 50 to 200 rems can cause illness but is rarely fatal. 200 to 1,000 rems will cause serious illness. Anything greater than 450 would probably kill half of all those exposed. Anything greater than 1,000 rems and it would almost certainly be fatal. An extended exposure is any exposure that took longer than the normal time of cellular regeneration. A low exposure will not make you sick, but a medium dose will increase your likelihood of cancer.

Sources of natural radiation on Mars are solar flares and cosmic rays. Solar flares\index{solar flares} are of no danger to people on Mars, but only to space travellers. This is because the planet's thin atmosphere is still thick enough to protect those on the surface. 

Even then, although mostly unpredictable, solar flares average only about one per terrestrial year. They tend to happen more frequently during the planet's solar maximum and less frequently during the solar minimum. They can last for several hours at several thousand rems and consist mostly of protons of a few million volts. Shielding with 12 centimetres of water or comparable mass of \math{12g/cm^{2}} or greater is enough protection. For underground buildings at a non--shallow depth this is not issue.

\index{cosmic rays}Cosmic rays are another significant source of radiation and of concern both to Martians and space travellers. Unlike solar radiation these particles have energies in the billions of volts and no one is certain where they come from -- although we are certain they do not come from the Sun. Interstellar crew would be exposed to about 30 rems per year. That is about 21 rems for an 8.5 month type--II Hohmann transfer.

Despite all of the emphasis NASA gives the dangers of cosmic radiation on Mars, there are several reasons to not be alarmed. Half of the sky is always blocked out to anyone anywhere on the surface. The remaining radiation is reduced by roughly a factor of half to only about 10 rems per year actually reaching the surface. Proper shielding of buildings, vehicles, and EVA suits therefore render this a moot point.

Consider for a moment that with all the concern raised about the dangers of radiation, we know now that the cancer we see every day is nearly always induced by diet.\footnotecite[the_china_study] Even if nutrition was given more than a head nod, it should be noted that proper nutrition can even assist in the repair of cellular DNA damage caused by radiation.\footnotecite[cavusoglu2009]

\StartSubSection{Sky}
Mars has a beautiful sky. At night time it may be possible to see auroras. During the day one can see \chemical{CO_2} and water clouds. Visual designers are encouraged to examine the images provided by our {\it Avaneya: Viking Lander Remastered DVD} described in detail in \in{chapter}[Viking Lander Remastered] to get an idea of the sky's appearance.

