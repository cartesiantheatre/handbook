% Time section...
\section{Time}

The colonists use a different calendar than on Earth for both practical and political reasons.

Practical, because the Gregorian calendar is useless in a world where the orbital period, seasons, and length of a solar day are different than on Earth. The calendar's months, for instance, are not meaningful in the absence of the natural cycles of the Earth's Moon.

Political, partly because the original Avaneya settlers were predominantly secular, the Gregorian calendar's year zero being incompatible with that; and partly that they desired a calendar which was meaningful in the context of @i{their own} collective memory.

% Explain hours...
But as the details go, a Martian hour is the same as a Terran hour, composed of 3600 Terran seconds.

% Explain solar day...
A @dfn{sol} is a solar day.@footnote{A @dfn{yestersol} being the sol preceding the current one.} This is the Martian analogue to a Terran day, but 2.7 % longer. The conversion of 1.027491 Terran days to Martial sols means that a sol is slightly longer than on Earth. Martian clocks are therefore designed to @dfn{time slip} at midnight for 39 minutes, 40 seconds. This allows Martians to use a 24 hour clock which they are accustomed to.

% Explain year...
A Martian year, abbreviated @dfn{MYr}, contains 668.6 sols, or 689 days by a terrestrial metric. That means there are 88775.245 seconds in a Martian year with 1.8876712 Martian years for every Terran year (1 year, 320 days, and 18.2 hours). 

Three Martian years would be pronounced, "three m-years".

The @math{L_s} system is used to designate the solar longitude, the angle Mars makes with respect to the Sun. This is measured from the northern hemisphere with the vernal or spring equinox (@math{L_s=0^{\circ}}). Therefore @math{L_s=90^{\circ}} is the summer solstice; @math{L_s=180^{\circ}}, the autumn equinox; and @math{L_s=270^{\circ}} the winter solstice. 

There are no timezones. All clocks are set to those of the first settlement location.


