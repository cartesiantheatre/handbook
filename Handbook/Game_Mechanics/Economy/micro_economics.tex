% This is part of the Avaneya Project Crew Handbook.
% Copyright (C) 2010-2015 Cartesian Theatre <info@cartesiantheatre.com>.
% See the file Copying for details on copying conditions.

% Game Mechanics chapter...

\StartSection{Micro Economics}

Micro econommics deals with the actions of workforce units that impact the various markets in a city, from financials to food stuffs. This section deals with the how and why each individual unit will be influencing each of the cities markets and the broader economy, in the first instance locally.

For further informantion about workforce units refer to section [workforce.tex]. Also of relevance is section [companies???].


\StartSubsection{Consumption}
Through processes described in section [workforce.tex] each workforce unit will have set traits that determine it consumption patters. Each unit has preferences of consumption that will compete with its own resources - such as money - to be purchased. Through the course of a day an individual unit will spend resources to obtain food and water, keep its shelter (house/unit/etc), and purchase luxuries like scented candles.


\StartSubsection{Budget Constraints and Preferences}
%Consumption options will need to be in classes or organised and given heirarchies. %Perfect Substitutes.
%Income categories for consumerables or similar?


For a given amount of resources (money) for a given span of time, both of which are predicted by the unit to be consistent and regular, a unit will cover its needs. A unit with more money will purchase either more of each thing or more expensive versions of these things, being limited only by its interests and aversions (section[workforce.tex]).

A quick, lets compare the consumption choices of three units Bob, Roger, and Ian. In this example the time span and monetary amout are arbitrary. 

Bob makes 100 a week. With this he purchases his basics, enough to eat, drink, have a house, and get to work. He eats rice and beans and cheap takeaway; lives in a trailer park and catches public transport to work. 

Roger makes 200 a week. He purchases every that Bob does but more of it and at a better quality. He eats steak, potatoes and vegetables as well as cheap takeaway; lives in a townhouse and catches public transport to work. In this example Roger does not make quite enough to purcahse a vehicle. Because Roger makes more than Bob, in this example we will say the Roger eats significantly more takeaway than Bob causing health problems.

Ian makes  1000 a week. This amount allows him to eat well and a lot more. He lives in a house and drives his own car to work. Ians high consumption ability might cause him to eat a lot more food, however he happens to have an interest of "health", making him avoid junk food and over eating. His left over income is spent of other things.

This example demonstrates that a units income will immediately cause it to purchase more and better things, with this being hindered only by an interest or other variable as defined in the section on workforce units [workforce.tex].


When a unit first gets its allocated resources (money) at the start of each time period, it will automatically map or plan the spending for that time period. While this is what units expect to happen things may change, affecting the units financial status unexpected and causing him to recalculate.

Lets look at another example. Take Ian again, part way through his weeks pay of 1000, his car is stolen, and he is mugged and bashed. Ian is left with only 250 for the remainder of the week, with no car and medical fees. This unit will then relcalculate its ability to live by meetings its basic needs while attempting to stay within its interests. From this we can see that Ian may live somewhat closes to how Roger lives for the rest of the week.

In this example it is important to note that: basic needs outweigh interests.



\StartSubsection{Optimal Choices: Mapping Consumption Patterns}
%Units needs to spend more on consumerables that offer them more, or in some cases at least appear to offer them more.
The complex of consumerables is a categorized mix of a priori value with value added by the producer cross referenced to the units consumption interests.

All goods products available in Avaneya for workforce units to consume are categorized then sub-categorized, for instance food is a category with a sub-category of vegetables. Within the sub-category of vegetables, each vegetable is organized in a hierarchy of goodness or value. Further, producers of vegetables (maybe farmers) have an impact on quality making zuchinis produced by some farmers better (an therfore worth more) than others. Finally some units may desire zuchinis more than others making them will to pay more, and the more units that like and buy zuchinis the greater the demand and price.

Also of note here is that some consumerables may not actually be better but are marketted as better. Marketting both increases the value of a consumerable and - if timed right - creates a consumption interest within the workforce unit. This is as true for cars as it is for zuchinis.   

From this and previous sections it can be seen that what a unit ultimately purcahses will be the best goods and services that it can most afford within a pay period after making a calculation at the start of that pay period so as not to leave it starving and homeless. And also as mentioned before, situations will arise causing units to forgo luxuries or prior determined purchases for less desirable options, or perhaps cause units to take drastic actions to survive (crime). Likewise, goods and services may become cheaper unexpectedly or from month to month, effectively making the unit wealthier within the domain of said purchases.

\StartSubsection{Investing, Savings, and Finances}
Each workforce unit will save an amount of its resources by default. This amount will be a set standard for all units except for those with either an interest that results in saving or with education variables that lead them to value saving or investing. Savings amounts will also be one of the las value placed on the use of a units resources for a given time frame, therefore for some periods of time certain workforce units will not save or may use some of their savings.

Saved resources will be stroed depending on the resource, of primary note is money. Money can be stored in a financial institution such as a bank or in a units place of residence (or perhaps other place such as behind a rock). Primarily units will store monetary resources in banks, however some units may have variables causing them to distrust such institutions, in this case they will store money physically. 


In addition to savings, units with the correct variables and in advanced cities will invest their resources into business' via stocks (and perhaps bonds). Financial investments are described in greater detail in section [...] but we will go briefly into investments as far as they interact with workforce units in the micro economy.

Similar to savings, workforce units with the appropriate variables and enough resources, will set aside an amount of money each week (on top of their savings) that will be invested in corporate stocks  (and or corporate and government bonds). Units will be able to invest money by reaching an appropriate financial instution or communicating with another workforce unit whose current job/occupation variable is label as "broker", this communication can take place directly in person or via distant communitcation such as phone or email. The act of investing has a fee associated with it for units that are not brokers. After this act is undertaken this unit will then need to be tagged with a variable that specifies what and how much of each investment it holds that remain open.     

What financials a unit invests in also varies with its degree of education and interests. Section [...] describes corporate entities. In this section it is noted that corporate entites will have numerical and hierarchical determinations of success and the better companies score of these the better they are to invest in. The better a workforce units level of education the higher up on a this hierarchy they will pick to invest in thier resource in. Further to this, units with only the most basic interest in investment will invest in companies assocaited with things they have other interests in and companies that have a media presence, such as advertisements.

 
