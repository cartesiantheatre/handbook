% This is part of the Avaneya Project Crew Handbook.
% Copyright (C) 2010, 2011, 2012
%   Kshatra Corp.
% See the file License for copying conditions.

% Weaponry section...
\StartSection{Weaponry}

As you probably already noticed, weapons come up at a number of times during the fictional timeline as well as in the unit tree. This section will discuss the general theory of weapons on Mars and design considerations to take into account.

\StartSubSection{Explosives}

\StartSubSection{Firearms}

We consulted with the professional weapons designers of {\it Fabrique Nationale d'Herstal} who were generous with their time as well in their enthusiasm for science fiction. They provided us with a great deal of advice in the design of the Arcadian R1A1 battle rifle first mentioned in \in{section}[Countdown]. It is a fictional adaptation of the FN FAL. They were also kind enough to provide us with a set of useful general considerations to take into account in the design of any firearm suited for use in Martian theatre.

A weapon's design and the application of good metallurgy science will ensure that it is able to withstand the harsh operating temperatures of Mars. This means a temperature range of a minimal \math{-90^{\circ}}C to a maximum of \math{20^{\circ}}C and with an average of only \math{-63^{\circ}}C. Even the worst high altitude and arctic conditions Earth has to offer generally do not come anywhere near these figures. 

Frosting would be a recurring problem on Mars for the reason just described. Special chemical lubricants would have to be engineered. These are necessary not only for keeping all load bearing and moving surfaces well lubricated, but also for reducing frosting. Again, good metallurgical science will aid in reducing the likelihood of frost induced misfires and fouling. If possible, battery powered heating elements could be integrated within weapons such that the receiver can maintain a minimal operating temperature.

Since gravity is only \math{38\%} on Mars what it is on Earth, heavier materials are more acceptable. It should be noted however that while the Martian gravitational field is weaker, mass is still mass and a weapon that is massive still requires a greater expenditure of human energy to accelerate than a weapon less massive.

A weapon's action must be tolerant of fine iron oxide dust which is roughly only 1.5 μm on average. One approach is to consider strategies of preventing dust from entering the action in the first place. This could be done with the aid of ejection port covers.

Since visibility at night is very difficult on Mars, night vision or thermal imaging optics are useful. When using only iron sights, the glow of tritium engraved front and rear posts would be useful in low light conditions.

In terms of ballistics, bullet projectiles would accelerate much faster on Mars than on Earth because they have only a one--hundredth the atmospheric resistance attenuating their kinetic energy from the moment they start moving down the barrel. This means that point blank ranges would be greatly increased. Bullets would also exhibit much flatter trajectories. Besides these characteristics, the near vacuum atmosphere should not affect the weapon in any other way. On the down side, dust storms would degrade accuracy. 

