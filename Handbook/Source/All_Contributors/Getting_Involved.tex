% This is part of the Avaneya Project Crew Handbook.
% Copyright (C) 2010, 2011, 2012
%   Kshatra Corp.
% See the file License for copying conditions.

% Getting Involved section
\StartSection{Getting Involved}
We encourage anyone in the community intrigued with the project to get involved. There is something for everyone to do, regardless of their areas of expertise and strengths. We take pride in our professional work, as all {\it libre} projects should, but we refrain from elitism. We also recognize that this project can help many in strengthening various skills as well as providing them with valuable experience that they can take to enrich other {\it libre} projects and vice versa.

Contributions need not be large and complex to be useful. Not everyone needs to be a programmer either. Indeed, programming is just one of many aspects of this project. There must be story writers, documentation writers, 3D modellers, graphic designers, and more for the project to see itself through to fruition. Contributions might be as simple as one line of code altered deep within the engine to repair a serious bug, a few human readable strings of the user interface translated into another language, new music, an enhanced story line, more voice overs, some corrected typos, new textures, or perhaps some improved material shaders. But {\it everyone} that makes a noteworthy contribution is listed in the game's credits.

Even if one does not contribute, everyone is certainly welcome to monitor the master branch (\in{section}[Using Bzr]); idle or converse with us on IRC (\in{section}[Internet Relay Chat]); and to subscribe, read, and post on our mailing lists (\in{section}[Mailing Lists]).

This is an exciting project. For whatever reason, since it was first announced, there has been no shortage of people expressing interest and a desire to get involved - and not just gamers and {\it libre} culture advocates, but educators, artists, musicians, scientists, activists, writers, and the list goes on. This should be a good thing. 

But we recommend that no one ask to join the project unless they are actually serious. If they wish to become involved, they can. But if it is simply to associate merely for the sake of association, or they genuinely come with the best of intentions, but lack the time to do whatever it is that they wanted to do, then this is not in any project maintainer's interest. Consider that they too have limited resources available for assisting new comers become acquainted, or \quote{plugged in}, to our workflow.

With that in mind, {\it it was realized that perhaps the best approach is to accept new comers based on actual contributions}. This philosophy protects both the project maintainers' time and the new comer's. It also helps to ensure that the project maintains its productivity and high standards. 

If you would like to get involved, the best way to do so is to solve a problem, provide something needed, or propose a solution you have thought about and are willing to implement. To get an idea of the project's immediate needs, take a look at the issue tracker described in \in{section}[Issue Tracking] to see a list of some of the outstanding issues to date after reading the \about[Orientation] in \in{section}[Orientation].

