% This is part of the Avaneya Project Crew Handbook.
% Copyright (C) 2010, 2011, 2012
%   Kshatra Corp.
% See the file License for copying conditions.

% Avaneya Spacecraft section...
\StartSection{Avaneya Spacecraft}

The {\it Avaneya} spacecraft is used to transport the bootstrap crew to their destination of Mars, as related in the fictional narrative on 29 Aquarius, 12 B.R. of \in{section}[Promises Made]. We will discuss some of the features of this spacecraft in this section.

{\it Avaneya} would come equipped with a large variety of instruments. We can discuss some of them and always add more as necessary.

A helium magnetometer would be used to measure the characteristics of the Martian magnetic fields. An ionization chamber or Geiger counter would be used to measure the charged--particle intensity and distribution in the space between Earth and Mars, as well as in the vicinity of the latter. A cosmic dust detector to measure the momentum, distribution, density, and direction of any it may find. A set of high resolution cameras to acquire detailed surface imagery taken while in orbit. A solar plasma probe sensitive enough to measure very low--energy charged particle flux emanating from the Sun.

Radiation is always a problem in space and so there are a number of different considerations for the safety of the crew. A radiation assessment detector (RAD) is needed for automatically triggering an alarm. It can detect high--energy atomic and subatomic particles from the Sun or any other cosmic radiation from distant supernovas or other sources. 

Further, all of {\it Avaneya's} avionics and instrumentation must be protected from a variety of hazards, including freezing {\it and} radiation. Radioisotope heater units, similar to the Dynamic Isotope Power System (DIPS) found in \in{figure}[figure:Units_User_Basic_Energy], but used only for thermal energy, are used for heating critical electronic systems. Electronic systems must also be hardened to protect them from radiation and any soft errors they may cause.

Human factor considerations require that all provisions, waste, water, and so on, should surround a solar storm shelter that the crew can take refuge in when the RAD triggers an alarm.

{\it Avaneya's} aeroshield is large and can be constructed in low Earth orbit at the {\it Iterum Shipyard} described in \in{section}[Promises Made]. The aeroshield allows the spacecraft to aerobreak safely.

An observation post should allow the crew not only a survey of the surround space and planetary bodies, but also of the spacecraft itself.

Life support is another obvious necessity. \chemical{O_2} can be provided en route with the aid of a RWGS reactor, described further in \in{section}[Reverse-Water-Gas-Shift Reactor].

Propulsion is required to move the spacecraft from one planet to another. To get to Mars, a chemical propulsion system of \chemical{H_2/O_2} is used. The spacecraft need not bear the return propellant required of the flight back to Earth, but instead rely on the indigenous resources of Mars to supply the return propellant of \chemical{CH_4/O_2} once there. This requires the mission payload to include propellant for the outbound trans--Martian injection stage only and begin manufacturing the methane and liquid oxygen propellant on arrival. 

For small attitude control corrections, the monopropellant of hydrazine (\chemical{N_2H_4}) can be used. It is chemically attractive because it allows for long term storability and similicity of use. For larger impulses, such as the large deceleration necessary to enter Martian orbit, retrorockets will be needed. This is necessary since aerobreaking alone is insufficient for a spacecraft of great mass, such as {\it Avaneya}, to reduce its kinetic energy.

Attitude control is important not just for minor course corrections, but also for avoiding high velocity debris and micrometeorites. All spacecraft slowly wear down from this kind of \quote{sandblasting}. By shielding as much of the spacecraft as possible in foil, the particles are vaporized into plasma with the energy dissipating sufficiently over an equal area that it is less likely to cause serious damage to the inner wall. Unfortunately not all components can be shielded in this manner because they require being exposed to the element in order to function, such as solar panels, telescopes, star trackers, and other optical devices. This equipment will slowly wear down.

Finally, we must describe the spacecraft's payload. Crew is a given. Instrumentation is required. Construction vehicles, lighting, and equipment are necessary for all excavation and building that will take place on the surface. Providing energy for all of this, an initial nuclear fission reactor and several temporary mobile portable dynamic isotope power systems like those described in \in{figure}[figure:Units_User_Basic_Energy].

{\it Avaneya} must also transport large amounts of \chemical{H_2} in tanks in liquid form (sometimes called \chemical{LH_2}). This is very important for starting the Sebatier and RWGS reactors, described in \in{section}[Reverse-Water-Gas-Shift Reactor] and \in{section}[Sabatier Reactor] respectively, once the crew arrive on Mars. Without hydrogen, they cannot bootstrap their own self--sufficiency. Fortunately hydrogen is the lightest known element in the universe. It can be kept in cryogenic storage tanks with multilayer insulation to reduce in--flight boil--off. The tanks can be gelled with small amounts of methane to further reduce loss. Contamination is not important since the \chemical{LH_2} is used for {\it in situ} propellant production and not as raw fuel for an internal combustion engine. Even then, we we can still expect a modest 1\% boil--ff per month.

