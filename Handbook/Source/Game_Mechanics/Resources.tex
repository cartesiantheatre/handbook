% This is part of the Avaneya Project Crew Handbook.
% Copyright (C) 2010, 2011, 2012
%   Kshatra Corp.
% See the file License for copying conditions.

% Resources section...
\StartSection{Resources}
The set of diagrams in this section show how resources can be transformed through a processing pipeline model. This relationship between {\it sources}, {\it mutators}, {\it resources}, and sinks is illustrated in the minimalistic \in{figure}[figure:Resource_Legend].

\placefigure
    [force,here]
    [figure:Resource_Legend]
    {A resource pipeline.}
    {\externalfigure[Source/Game_Mechanics/Images/Legend.pdf][][width=1.0\textwidth]}
    
A {\it source} could be the atmosphere, the ground, solar energy, or perhaps imported from another city or possibly even off world. But except in the case of imports, sources usually contain resources in an unrefined state. Think of them as simply sets of resources. We denote them on our diagrams as triangles.

A {\it resource} can be anything that is, was, or could be of use to someone living on Mars. It could be a kind of energy, such as thermal, chemical potential, electrical, or what have you. Or it could be human waste. Resources are denoted on our diagrams as octagons.

A {\it mutator} is an abstract entity that transforms one type of resource into another. They are denoted on our diagrams as three dimensional boxes. Usually these mutators are physical buildings, but they do not have to be.

A {\it sink} is akin to a {\it source} in that it is a set of resources, only they are the terminal end of a resource's journey. The source and sink may in some cases be the same and in others they may not be. We denote them on our diagrams as inverted triangles.

Whenever the source and the sink are the same, we call this recycling. Whenever the source and sink are not the same, and the source is non-renewable, it will eventually become depleted. This is one of the many reasons why the models adopted by neoclassical economist are dangerously broken since they presuppose the indefinite feasibility of a linear transformation pipeline.

As an example, consider the Martian atmosphere as a source. This source contains many resources, of which is plenty of \chemical{CO_2}. This \chemical{CO_2} can be fed into the \about[definition:Reverse-Water-Gas-Shift Reactor] which is described on \at{page}[definition:Reverse-Water-Gas-Shift Reactor] to produce new resources, \chemical{CO}, or carbon monoxide, hydrogen, and oxygen. The carbon monoxide could, in turn, undergo further refinement when combined with other resources and mutators to produce yet additional resource such as methanol, plastics, or whatever is possible. Let us assume for the sake of example that the hydrogen was cycled back into the reactor with perfect efficiency as a reagent and the oxygen was used to breath. The lungs, a resource mutator, produce \chemical{CO_2} from the oxygen and the food you ate, with the \chemical{CO_2} vented back out into the atmosphere.

Note that not all resources require processing to be considered a resource. For instance, although the atmosphere is mostly \chemical{CO_2}, that gas is already a resource in and of itself since it has many applications, even though it could still undergo further refinements.

Since all of Avaneya's sources, resources, mutators, and sinks are data driven, to add new ones one would do so simply by providing its definition through the Lua interface. Although there is no theoretical limit to the diversity, the game comes with a core set. Let us take a look at how they work as illustrated by \in{figure}[figure:Resources_Pipeline].

\FullPageDiagram
    {figure:Resources_Pipeline}
    {The resource pipeline of all the core sources, resources, mutators, and sinks.}
    {Source/Game_Mechanics/Images/Resources.pdf}

