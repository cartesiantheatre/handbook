% This is part of the Avaneya Project Crew Handbook.
% Copyright (C) 2010, 2011, 2012 Cartesian Theatre <kip@thevertigo.com>.
% See the file Copying for details on copying conditions.

% Erecting Home section...
\StartSection{Erecting Home}

\StartTimelineDate{12 Leo 10 B.R.}
Arda consults with Senka and Nayana before issuing coordinates to the loadmaster for the preselected drop site within {\it Arcadia Planitia}. The location is one of several candidates previously catalogued by {\it Curiosity XI} and is located inside the impact crater of a relatively recent meteorite within the Diacria quadrangle. Senka and Nayana advise Arda that {\it Curiosity} found large amounts of invaluable water ice just below the surface. 

Senka is cleared to begin jettisoning cargo. The cargo consists of mostly construction equipment, cryogenic liquid hydrogen, water, and other provisions the crew will need once on the surface. The equipment parachutes to the surface with no material loss, save one asset due to an attitude control computer malfunction.
\StopTimelineDate

% Leo 13, about the peak of the Martian Spring with clear skies and low winds with the 
%  weather at its finest...
\StartTimelineDate{13 Leo, 10 B.R.}
All crew alight {\it Avaneya}, boarding the {\it Manu} landing craft. No one is left onboard to avoid being exposed further to solar flares and cosmic radiation.

The {\it Mars Positioning Satellites} provide a guided landing by tracking the critical manoeuvres of {\it Manu's} terminal descent. The communications uplink with Mission Control is maintained throughout over a 4 minute delayed solnet connection.

{\it Manu} makes contact and reports a successful soft landing. Nayana reports an outside temperature of \math{-70^{\circ}}C and an atmospheric pressure of 30 Pa. The time is local noon, allowing for maximum photovoltaic use.

Team briefings are conducted by team leads within {\it Manu} at the drop site. Concurrently, the Recovery Team's rovers guided by scouts on \index{methanol motorbikes}methanol motorbikes are dispatched to begin asset recovery across {\it Arcadia Planitia}. The region is also colloquially known simply as {\it the planitia}.

The base's nuclear reactor is brought online, along with several temporary mobile portable dynamic isotope power systems.

The Construction Team begin minor excavation for anchoring and erecting temporary 34.0 kPa rated, aluminium strut reinforced, inflatable, polypropylene tents for the Command Centre and habitats. Nayana's greenhouse tent is rated 6.8 kPa, sufficient for plant life, but requiring personnel to still don EVA suits.
\StopTimelineDate

\StartTimelineDate{14 Leo, 10 B.R.}
\index{Gas extractors}Gas extractors are brought online and run at full capacity capturing liquid oxygen, liquid nitrogen, argon, and carbon dioxide. 

The Sabatier, RWGS, and methanol reactors successfully produce methane, oxygen, hydrogen, methanol, and aqua.
\StopTimelineDate

\StartTimelineDate{15 Leo, 10 B.R.}
The Recovery Team complete their asset recovery of all undamaged parachuted cargo within a 92 kilometre radius from {\it Manu's} landing site.
\StopTimelineDate

\StartTimelineDate{40 Leo, 10 B.R.}
Vehicular onboard artificial intelligence and system firmware upgrades are remotely uploaded by UNSA's Jet Propulsion Laboratory over the solnet with Khalid's assistance.

Senka's Mechanics Team reports having completed the necessary preparations for the backhoes, front end loaders, bulldozers, tractors, graders, water ice processors, dump trucks, and other vehicles. They are now available for Construction Team use.

Mining and excavation operations begin through a mixture of directly manned, remotely manned, and autonomous operation.
\StopTimelineDate

\StartTimelineDate{42 Leo, 10 B.R.}
The {\it Mars Ascent Vehicles' Alpha} and {\it Bravo} alternate launches to perform rendezvous dockings with {\it Avaneya} in low Martian orbit. As methane/oxygen fuel stocks are manufactured {\it in situ}, they provide the ship with the propellant required for its return journey back to {\it Iterum Shipyard}.
\StopTimelineDate

% Constraint: 550 sols after first arrival...
\StartTimelineDate{21 Gemini, 9 B.R.}
Arda directs Senka to remotely issue the necessary command sequence for the fully refuelled {\it Avaneya} to begin its unmanned return.
\StopTimelineDate

\StartTimelineDate{37 Gemini, 9 B.R.}
{\it 52048 Varuna} is intercepted by UNSA's ion--drive propelled impactor, {\it Don Quixote V}. The spacecraft's payload of a nuclear warhead is detonated above the asteroid's surface, successfully altering its trajectory into a collision course with the Martian south pole.
\StopTimelineDate

\StartTimelineDate{45 Taurus, 9 B.R.}
Arda attends a ceremony commissioning the first settlement fire brigade.
\StopTimelineDate

\StartTimelineDate{55 Taurus, 9 B.R.}
{\it Tarikin I} and its crew begin preparation to depart Cape Canaveral, North American Union, in response to Arda and Henrik's UNSA advisory report of the completion of additional settlement infrastructure. This marks the beginning of the {\it Tarikins'} continual supply of 400 new civilian colonists every 780 day launch window.
\StopTimelineDate

