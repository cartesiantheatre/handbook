% This is part of the Avaneya Project Crew Handbook.
% Copyright (C) 2010, 2011, 2012 Cartesian Theatre <kip@thevertigo.com>.
% See the file Copying for details on copying conditions.

% Single Player section...
\StartSection{Single Player}
Let us start with an overview of the different modalities of play. These can be either single or multiplayer. Single player, as the name suggest, the user plays alone. Well, actually, with the computer and the challenges it presents. This includes a tutorial, campaign, and custom mode. The user can also restore from a saved game in all three modes if they wish.



\StartSubsection{Tutorial}
In tutorial mode, the user is guided through interaction with the elements of the game and its user interface by Khalid Zafar, one of the leading characters described in \in{chapter}[Leading Characters]. The aim is to train the user to the point where they are comfortable in taking on the campaign or multiplayer modes.

Due to the extesnsive nature of Avaneya a tutorial which covers all of the games aspects would be quite time consuming for the player. A balance needs to be sought between showing the players all the elements of the game and what they need to begin playing the game without continually hitting walls and becoming discouraged.   

Taking into account that the tutorial is separate from the campaign it should contain little of the essential storyline elements and be designed to give players the knowledge they need to jump into Solnet multiplayer and leave out tutorial elements that may be able to be addressed in the campaign.  

With this in mind the tutorial should cover elements of zoning lands and placing and changing buildings, introduce agents and their influence on population and how they may limit player control, and bring in a small range of military units to show the diverse capablities of different units and how they interact. %This could be done in a series of "chapters" or as a 20 to 30 minute scenario. 

In addition to the tutorial learning how to play Avaneya can occur in the campaign which may become progressively intricate. A similar method may be taken on Solnet. The idea of this is to get the players into the 'real deal' as quickly as possible.

Another option is the opposite of that just proposed which is to have an intricate and time consuming tutorial which allows players to learn a majority of the games elements leaving all other play options open to be developed for a player to jump in and play with little or no help needed. Taking such an approach, the tutorial would need to be approached as a pseudo-campaign, perhaps with a short side story, or take place at events leading up to where the campaign takes off.  	 



\StartSubsection{Campaign}
In campaign mode, the user assumes the role of Arda Baştürk, or any of the other leading characters, as they journey through the campaign which takes place near and following the end of the fictional timeline described in \in{section}[Aftermath & Diaspora]. That is, around the time of the Arcadian Diaspora.

%Tutorial elements in the campaign need to be minimal as there is nothing worse then wanting to start a new game and going through all the learning elements again. 

\StartSubsection{Single Player Custom}

The custom mode allows users to specifically select which scenarios they would like to play without thought of commitment to a full campaign. This is useful for those who would like to explore nonlinear gameplay. We will discuss scenarios further in \in{section}[Scenarios].
