% This is part of the Avaneya Project Crew Handbook.
% Copyright (C) 2010, 2011, 2012 Cartesian Theatre <kip@thevertigo.com>.
% See the file Copying for details on copying conditions.

% Leitmotifs chapter...
\StartChapter{Leitmotifs}

This is probably the most important chapter in the entire book in helping readers to determine whether this is the project for them to be involved in or not. Avaneya has several recurring ideas, or {\it leitmotifs}\index{leitmotifs}, within it that will help the reader to determine how the project resonates with them. While they may seem unrelated, they are useful cues in reconstructing a virtual world that is illustrative of some of our most defining contemporary problems.

We would like to keep this \quote{leitmotif set} as minimal as reasonably possible. Anything else that we wanted to express, it hopefully already possible to derive, where obvious, from that already present. However, if you still feel the set should be modified in some way, please bring it up on any of the communication mediums described in \in{section}[Communication]. We are open minded but, like us, you too must be reasonably prepared to substantiate your thoughts.

The following form the core leitmotif set the project felt were important, given in no particular order. 

\startitemize[4]
\head {\em Staged terrorism is in vogue.}

Most major acts of terrorism are staged. This is a fad that has been gaining in popularity. Despite their synthetic nature, this concept remains either unknown or an unregistered interest treated as anecdotal at best by the mainstream media. There is an overwhelming body of evidence implicating government involvement in staging major acts of terrorism.\footnotecite[extras={ p.~193.}][shirer1960]\footnote{{\it Ibid.} pp.~518--520.}\footnotecite[harrit2009]\footnotecite[debunking_911_debunking]\footnotecite[letter_to_minister_regarding_911]\footnotecite[turbeville2010_staged_terror]\footnotecite[griffin_andrew2010_murrah_bombing_survivor]\footnotecite[berger2009_mcveigh]\footnotecite[watson2006_mcveigh]\footnotecite[northwoods]

\head {\em Drug policies do not reflect reality.}

Most illicit narcotics originate covertly from governments.\footnotecite[cia_drug_plane_crash]\footnotecite[top_mexican_drug_lord]\footnotecite[afghan_opium_kingpin]\footnotecite[cia_drugs_list]\footnotecite[vicente_rule]\footnotecite[ruppert_confronts_deutch] Thus, common criticisms of the public safety policies that are implemented as being ineffective because they were drafted with the best of intentions, albeit simply misinformed, cannot be considered valid if the former claim is true.

Interestingly, alcohol, a socially validated, recreational narcotic, is usually legal, yet still more dangerous overall than all of the most common illicit narcotics anyways.\footnotecite[extras={ See figure 2.}][nutt2010]

\head {\em No one is free.}

To be free is to be independent of influence, which no one is. We are all subject to influence, otherwise there would be no such thing as advertising and information could never have become a corporate commodity.\footnotecite[santoso2008] Everything is connected and nothing happens in a vacuum. \footnote{See {\it The Equality Trust} in \in{chapter}[Resources For Everyone].}

\head {\em Neoclassical economics is junk science.}

The major axioms of neoclassical economics, the backbone of modern economic theory, are wrong. People do not always have rational preference, seek self maximization, always act on full and relevant information, or stand the most to gain via competition over cooperation. Markets are not natural systems akin to the laws of thermodynamics, do not always correct themselves, and their growth is sometimes destructive. The assumptions are many in number and they go on, and on, and on.\footnotecite[lasn2012occupy]\footnotecite[adbusters_jamming_neoclassical_economics]\footnotecite[raj2010]\footnotecite[cobb1999]\footnotecite[adbusters_issue_85] The field is not a science, but a pseudo--science rife with fraud.

\head {\em The most violent repeat offenders work in government.}

Government is among the leading causes of death.\footnotecite[statistics_of_democide] It is generally there to protect the interests of a few unelected principle benefactors and not to make changes in the public interest.\footnotecite[report_from_iron_mountain] Governments tolerate democracy only so long as it remains merely cosmetic and is not permitted to actually function.\footnotecite[extras={ Select the {\it Votes} column to sort demand by vote.}][occupy_poll]

\head {\em Disease begins on the plate.}

Most health related deaths are rooted in diet.\footnotecite[the_china_study]\footnotecite[extras={ According to Campbell, all of the top three causes of death in Canada in 2007 identified by StatsCan are entirely dietary.}][statscan_leading_causes_of_death] This is the reason for the disease state that we live in since nutrition and exercise are not particularly lucrative enterprises.

\head {\em Conventional medicine is fraud.}

Conventional medicine is allopathic medicine. Allopathic medicine is pharmaceutical medicine. Pharmaceutical medicine culminated in 106,000 fatalities in 1994 due to adverse drug reactions (ADR), just in the United States, not counting overdoses, and under the direction of a credentialed allopath. This places the paradigm between the nation's fourth and sixth leading causes of death in that year.\footnotecite[lazarou1998_adr] By estimation then, approximately 2,700,000 fatalities can be attributed to the use of allopathic medications between the years of 1983--2010.\footnotecite[prousky2012_toxicology] 

But fatalities aside, even just the management of disease is at least as damning.\footnotecite[lindsey2012_vitamins_and_diabetes]\footnotecite[leape2000]

By contrast, in considering the dangers of some alternative treatments, but 12 fatalities were allegedly attributable to vitamin supplements during that same time span.\footnotecite[aapcc_annual_reports] Note that these are allegations only and remain unproven.

\head {\em Agribusiness is destroying the planet.}

Agriculture is our principle means of interacting with the planet. Changing what we eat solves many global problems.\footnotecite[food_inc]\footnotecite[food_matters]

\head {\em Today's conquests are made through debt, not the sword.}

Usury, including fractional reserve banking, is slavery.\footnotecite[the_creature_from_jekyll_island]\footnotecite[minutes_bank_of_canada] Debt is the tool of choice in an era of globalization.\footnotecite[perkins2005] This will be elaborated on in \in{section}[Combat Globalization].

\head {\em Corporations are not people.}

The notion that corporations deserve the same rights as living human beings has proved itself catastrophic.\footnotecite[the_corporation]\footnotecite[coca_cola_case] This will be elaborated on in sections \in{}[Corporate Personhood] and \in{}[Money As Free Speech].

\head {\em Genes are a cop out.}

Very little of what we call human nature is actually genetic. Genes specify human needs, not behaviour. There is no such thing as {\it default} human behaviour. This will be elaborated on in \in{section}[Socioeconomic Simulation].
\stopitemize

If you have gotten this far and you are still comfortable being involved in this project, you will probably find it rewarding. Otherwise it may not be the best place for you, as it is neither a casual game, nor free of the well justified accusation of having courted controversy. There are no requirements that anyone believe or do anything here. Everyone comes by having made a personal choice and through their own research. But rest assured, there are many other {\it libre} projects that could use the talents of those readers already infuriated or still with serious reservations.

But regardless of whatever readers choose to do, creativity is required in the game's pedagogy. We always try to encourage its users to do some thinking of their own. To paraphrase Socrates, {\it we cannot teach anyone anything. We can only make them think}.

\StopChapter
