% This is part of the Avaneya Project Crew Handbook.
% Copyright (C) 2010, 2011, 2012
%   Kshatra Corp.
% See the file License for copying conditions.

% Sectors section...
\StartSection{Sectors}

Depending on whether the user is playing single player or multiplayer, the game can take place in any one of a number of different {\it sectors}\index{Sector}. Sectors are geographical subdivisions of Mars that the user is emersed in. Grouping the planet as such is useful for a number of reasons.

For one, it allows us to highlight the different physical idiosyncrasies of the planet. As an example, contrast a sector within {\it Valles Marineris} with one set in {\it Arcadia Planitia}. We would expect them to have very different appearances, geography, and bear some different resources. This allows experts to apply their knowledge to the specific regions of their familiarity.

Another reason is that sectors allow the project crew a useful virtual metric for managing costly resources that make a large online multiplayer experience possible. When a sector within, say, {\it Arcadia Planitia}, approaches the maximum user activity it can host in the virtual space it encompasses with the physical computing resources available for it, it indicates to us that additional resources are necessary to keep the server going. This incremental approach allows us to adapt the backend to scale to user demand as necessary. Perhaps the server just needs more CPU cores, or RAM. 

On the other hand, perhaps growing user demand has necessitated that a new sector be allocated somewhere new on the Red Planet, such as {\it Olympus Mons}\index{Olympus Mons}. In that case, we can efficiently host different sectors using a distributed approach with each sector potentially running on a distinct physical server, offloading what would have been a single saturated machine into an efficient {\it Solnet cluster}\index{Solnet+Cluster}. We will discuss Solnet further in \in{section}[Multiplayer: Solnet].

