% This is part of the Avaneya Project Crew Handbook.
% Copyright (C) 2010, 2011, 2012 Cartesian Theatre <kip@thevertigo.com>.
% See the file Copying for details on copying conditions.

% Attire & Personal Equipment section...
\StartSection{Attire & Personal Equipment}

Obviously the colonists cannot wear what their Terran counterparts would wear when exposed to the Martian extremes. This section raises some of the considerations that must be taken into account when designing their attire and other personal equipment. Valuable cues for artists can be found here.

Any fabrics or other personal items exposed to Martian dust would benefit from being a reddish--brown hue. This minimizes the need for frequent washing. The materials must also be easy to sterilize. Hemp has so many uses that no one should be surprised if it finds yet another one here.

Whenever one performs an EVA (extra--vehicular activity), they will need special equipment. Despite what is typically portrayed in films, you do not need a bulky pressurized suit to survive on Mars -- nor even in the complete vacuum of space for that matter. Nevertheless, it is an option. But for the world Avaneya takes place in, we figured we would try something less explored and with other advantages -- or at least for Arcadians, but not necessarily of UNSA military personnel which could don militarized pressure suits.

An Arcadian requires a pressurized helmet, but can get by with only a special \index{Arcadian+elasticized mesh suit}unpressurized elasticized mesh around the rest of their body. This mesh is tight enough to contain the human body at the same average atmospheric pressure it evolved under on Earth at a mean sea level pressure of 1 atm. This is possible because our skin does not need to be protected from a vacuum since it is already gas--tight. It only needs to be mechanically compressed to retain its normal shape. As long as a minimum pressure of 29.6 kPa is maintained, the occupant will be fine. Although it would be very tight when first put on, as soon as the occupant left the airlock and went outside, the pressure differential with the outside environment should relieve the discomfort by pulling the mesh back out.

One of the advantages of an elasticized mesh suit is that it is very light and easily stowed away within a small space -- possibly even within a settler's helmet. This also means it is much easier to work in. It can even suffer a minor puncture without permanent damage to the occupant in an absolute vacuum, but the exposed tissue would still be painfully sucked out into a brutal environment which would result in at least bruising. Prolonged exposure, however, would probably require that the dead tissue be removed.

Temperature regulation is necessary to ensure the occupant neither freezes nor overheats. The body's natural sweat is used to keep the suit cool while an integrated electrical heating element is used to prevent freezing. A radioisotope heating unit could be used for this latter function, but they are very expensive.

The respirator provides the operator with a supply of breathable oxygen. There are at least two approaches. Compressed oxygen is cheap and provides about a 12 hour supply. Cryogenic liquid oxygen, on the other hand, provides about 36 hours. Although the latter is more expensive and susceptible to slow losses through boil--off, it can be siphoned out of a vehicle's propulsion system in an emergency.

For electrical power, there are at least two options. A combined methanol--oxygen (\chemical{CH_3OH / O_2}) fuel cell is one route. It is reliable, efficient, and requires the operator to carry only a modest amount of methanol if they already have a supply of cryogenic liquid oxygen available for their respirator. One drawback is that this kind of technology is quite expensive.

% If anyone has a citation for NASA's power unit, please feel free to provide here...
An alternative power supply could be provided with batteries instead. NASA currently has a model that could be used as a starting point, the \index{HOTSS--FuCU}{\it \chemical{H_2/O_2} Technology Space Suit Fuel Cell Unit (HOTSS--FuCU)} system. It is efficient, cheap, and lighter than the aforementioned approach since molecular hydrogen is less massive than methanol. Unfortunately it can also be very dangerous. The boil--off of the unit's hydrogen supply when combined with the boil--off of the cryogenic liquid oxygen vented from the respirator's supply within a confined space for a prolonged period of time could be disastrous. The mixture could explode if ignited.

