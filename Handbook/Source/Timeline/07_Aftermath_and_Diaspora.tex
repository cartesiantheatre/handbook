% This is part of the Avaneya Project Crew Handbook.
% Copyright (C) 2010, 2011, 2012
%   Kshatra Corp.
% See the file License for copying conditions.

% Aftermath & Diaspora section...
\StartSection{Aftermath & Diaspora}

\StartTimelineDate{42 Leo, 2 A.R.}
A disarmed and dismounted UNMEPF brigade is escorted on both its flanks by dual columns of Arcadian dragoons and infantry. The critically wounded, of either side, are rapidly transferred to Arcadian hospitals. The dead temporarily remain behind frozen. The prisoners of war are destined for Arcadia's underground emergency shelter. The brigade's vehicles and ordnance are dispersed to various Arcadian units.

Arcadia Hall is repopulated with staff overseeing the transfer of settlers out of the shelter back to their homes in the city. The shelter, whose means of movement in and out are few in number, becomes an easily guarded and practical prisoner of war camp for the quartering of the UNMEPF brigade.

Arda reassures Nilhara that, although he is confident the UN brigade would not have abided by the Geneva Conventions, Arcadia considers itself a civilized state and will tend to the sick and wounded of both sides without reproach.
\StopTimelineDate

\StartTimelineDate{45 Leo, 2 A.R.}
Nilhara agrees to provide Arcadia Hall with access credentials to the {\it Bhadra I/II} satellites. Arcadia's solnet connectivity with Earth is re-established.

Arda contacts the Secretary General of the United Nations to establish a back door diplomatic communication channel. From this sol and the eight that succeed it, the following terms are successfully negotiated over diplomatic cables. The concessions of the United Nations are as follows:

\startitemize[R]
\item The General Assembly shall acknowledge the Republic of Arcadia Planitia as a legitimate and independent state and shall not infringe upon its sovereignty.

\item The Republic of Arcadia Planitia shall be represented in the General Assembly on Earth with a representative of its own choosing. Arda is reassured that he will be given amnesty if he chooses to fulfil that office himself, but declines.

\item The United Nations shall cease peacekeeping operations in the proximity of, and within, Arcadia. The term \quote{hostilities} is not used. Further, the Secretary General on behalf of the Security Council refuses to unambiguously reaffirm his commitment to the principles of prohibition of the deployment of munitions of war, suborbital, orbital, or in outer space, by neither acknowledging nor denying the existence of {\it Thor Outpost}. Questions of its use, therefore, remain quietly unresolved.
\stopitemize

The concessions of the Republic of Arcadia Planitia are as follows:

\startitemize[R]
\item All UNMEPF prisoners of war are to be treated humanely in accordance with the Geneva Conventions.

\item All UNMEPF prisoners of war are to be returned to Earth as soon as logistically feasible.

\item The Republic of Arcadia Planitia shall become a signatory to the Vienna Convention on Diplomatic Relations, allowing the United Nations to be represented with a permanent diplomatic mission in Arcadia. This shall include a single embassy, but possibly one or more consulates as agreed upon in the future.

\item A limited number of Terran corporations are permitted to resume commercial activity in Arcadia. With their registered offices on Earth, the status of corporate personhood\footnote{See \in{section}[Corporate Personhood]  for more on the concepts of corporate personhood.} within Arcadia remains unofficially unresolved.
\stopitemize

Arcadia Hall prepares formal requests to various Terran nations for the opening of embassies in the North American Union, European Union, and others. In turn, it receives a submission of Leonard's credentials for entry into the United Nations' diplomatic mission in Arcadia. Arcadia Hall rejects Leonard's application as a {\it persona non grata}, as permitted under Article 9 of the Vienna Convention.
\StopTimelineDate

\StartTimelineDate{55 Leo, 2 A.R.}
Arcadian forces transport UNMEPF personnel, including the dead, to the dropsite for Kali's MADVs. Senka assists them with refuelling and performing their preflight checks. Arda returns Nilhara's sidearm with a bottle of Arcadian Manowar. The pair salute prior to the MADVs lifting off.
\StopTimelineDate

% Assume 180 days to return after launch...
\StartTimelineDate{40 Scorpius, 2 A.R.}
Kali's crew return to Earth. Nilhara is taken into custody. He is tried by a military court-martial on charges of having violated Article 104 of the Uniform Code of Military Justice, aiding enemy combatants, in having provided Arcadia with access credentials to the {\it Bhadra I/II} satellites. He is found guilty and sentenced to life imprisonment without parole at the Midwest Joint Regional Correctional Facility, Fort Leavenworth, North American Union.
\StopTimelineDate

\StartTimelineDate{6 Capricorn, 2 A.R.}
Arda holds a public forum at Arcadia Hall on the future of the republic. The forum ends with a public vote carrying a motion to initiate an Arcadian diaspora\index{Arcadian diaspora} across Mars, beginning with the rest of Arcadia Planitia. The rationale being that a republic of a sole city has little chance of security and independence in an era dominated by powerful Terran interests.
\StopTimelineDate

\StartTimelineDate{9 Capricorn, 2 A.R.}
An inbound subsonic aircraft crashes in the desert of {\it Arcadia Planitia} 40 km south of Starport Arcadia. Emergency responders arrive on scene to find the pilot dead, the first officer missing, and 3600 kg of cocaine dislodged from false bottom coffins with UNMEPF emblems on them. 

Arcadia Hall's transportation secretary informs Arda that his investigators have determined that the tail number of the aircraft was registered to NAU-CIA for UN diplomatic flights. While investigators analyze the substance to determine its origin, it is inadvertently discovered that the material is not only of Martian origin, but of a synthetic form more addictive and dangerous than the genuine substance.
\StopTimelineDate

