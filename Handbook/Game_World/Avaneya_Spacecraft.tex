% This is part of the Avaneya Project Crew Handbook.
% Copyright (C) 2010, 2011, 2012
%   Kshatra Corp.
% See the file License for copying conditions.

% Avaneya Spacecraft section...
\StartSection{Avaneya Spacecraft}

The {\it Avaneya} spacecraft is used to transport the initial bootstrap crew to their destination of Mars. This was related in the fictional narrative found in \in{section}[Promises Made]. We will discuss some of the features of the spacecraft in this section.

\StartSubSection{Avionics, Instrumentation, and Other Equipment}
{\it Avaneya} is equipped with a large variety of equipment. This is necessary to aid the crew in their journey, to allow for various scientific studies {\it en route}, or at their final destination.

A helium magnetometer can be used to measure the characteristics of the Martian magnetic field. An ionization chamber or Geiger counter can be used to measure the charged--particle intensity and distribution in the space between Earth and Mars, especially in the vicinity of the latter. A cosmic dust detector can be used to measure the momentum, distribution, density, and direction of any that it can find. A solar plasma probe can be used that is sensitive enough to measure very low--energy charged particle flux emanating from the Sun.

A set of high resolution cameras can be used to acquire detailed surface imagery taken while in orbit. These cameras are needed for last minute adjustments in the selection of a landing site on arrival and for monitoring the point of rest of any cargo jettisoned down to the surface.

Radiation is always a problem in space and so there are a number of different considerations for the safety of the crew. A radiation assessment detector (RAD) is needed for automatically triggering an alarm. It can detect high--energy atomic and subatomic particles emanating from the Sun, cosmic radiation from distant supernovas, or other sources. Further, all of {\it Avaneya's} sensitive electronics must be protected from this radiation. This is done through hardening of these components to avoid radiation induced soft errors.

Freezing of delicate electronic systems is also a concern. To remedy this, {\it Avaneya} is furnished with multiple radioisotope heater units. These are similar to the Dynamic Isotope Power System (DIPS) found in \in{figure}[figure:Units_User_Basic_Energy], but used only for their thermal energy. They maintain all critical electronic systems within their optimal operating temperature range.

\StartSubSection{Construction & Form Factor}

{\it Avaneya's} aeroshield is large and can be constructed in low Earth orbit at the {\it Iterum Shipyard} described in \in{section}[Promises Made]. The aeroshield allows the spacecraft to aerobreak safely through the Martian atmosphere. Without it, the atmospheric friction or drag would rip an otherwise exposed {\it Avaneya} apart.

\StartSubSection{Crew Facilities}
In the event that the RAD triggers an alarm, human factor considerations dictate that all provisions, waste, water, and so on, are arranged in such a way that they are stored by surrounding a solar storm shelter. The crew can take refuge within the shelter with virtually no danger to themselves, but only with ample forewarning.

An observation post or solarium allows the crew to not only survey the surrounding space and planetary bodies, but also the spacecraft itself. This is important for assessing any damage that might have been caused by micrometeorites and determining what requires repair.

The importance of life support is a given. A steady supply of \chemical{O_2} is provided {\it en route} with the aid of a RWGS reactor. This device was described in \in{section}[Reverse-Water-Gas-Shift Reactor].

\StartSubSection{Payload}
We must describe what it is that the spacecraft carries, besides its crew, their provisions, and any electronics and instrumentation systems that may be required. This would at least include, for what the crew expect to take place once on the surface, cold weather adapted construction vehicles, night lighting, and any other equipment for the excavation and erection of various structures. 

Providing the initial supply of energy for all activity that takes place on the surface, the crew are equipped with a nuclear fission reactor and several portable dynamic isotope power systems. The DIPS are listed in the unit tree of \in{figure}[figure:Units_User_Basic_Energy]. For higher peak energy uses, like for bulldozers and drilling rigs, the crew will use dimethyl ether \chemical{(CH_3)_2O} (DME). The latter can be manufactured {\it in situ}.

Hydrogen is really the only major material resource that must be imported from Earth, or at least initially.  It is critical for initiating the Sabatier and RWGS reactors upon arrival on the surface, as described in \in{section}[Reverse-Water-Gas-Shift Reactor] and \in{section}[Sabatier Reactor] respectively. Without hydrogen, the bootstrap crew cannot maintain their own self--sufficiency. 

Fortunately hydrogen is the lightest known element in the universe. {\it Avaneya} can transport a large amount of \chemical{H_2} in liquid form (sometimes called \chemical{LH_2}). It can be stored within cryogenic storage tanks. A multilayer insulation on the tanks can be used to reduce in--flight boil--off. In addition, the tanks can also be gelled with small amounts of methane to reduce losses even further. This \chemical{CH_4} contamination is not a concern since the \chemical{LH_2} it is helping to preserve is used only for the {\it in situ} propellant production, not as a raw fuel for internal combustion engines. Even with these measures, the crew will still expect a modest 1\% boil--ff per month.

\StartSubSection{Propulsion}
Propulsion is required to move the spacecraft from one planet to another. To get to Mars, a chemical propulsion system of \chemical{H_2/O_2} is used. The spacecraft need not bear the return propellant required of the flight back to Earth, but instead rely on the indigenous resources of Mars to supply the return propellant of \chemical{CH_4/O_2} once there. This requires the mission payload to include propellant for the outbound trans--Martian injection stage only and begin manufacturing the methane and liquid oxygen propellant on arrival. 

For small attitude control corrections, a monopropellant of hydrazine (\chemical{N_2H_4}) can be used. It is chemically attractive because it allows for long term storability and simplicity of use. For larger impulses, such as the large deceleration necessary to enter Martian orbit safely, retrorockets will be needed. This is necessary since aerobraking alone is insufficient for a spacecraft of great mass, such as {\it Avaneya}, to reduce its kinetic energy. In addition, hydrazine reserves can be used to power {\it Avaneya's} auxiliary power units (APU) for short periods of time in the event of an emergency.

Attitude control is important not just for minor course corrections, but also for avoiding high velocity debris and micrometeorites. All spacecraft slowly wear down from this kind of \quote{sandblasting}. By shielding as much of the spacecraft as possible in foil, the particles are vaporized into plasma with the energy dissipating sufficiently over an equal area that it is less likely to cause serious damage to the inner wall. Unfortunately not all components can be shielded in this manner because they require being exposed to the element in order to function, such as solar panels, telescopes, star trackers, and other optical devices. This equipment will slowly wear down.

