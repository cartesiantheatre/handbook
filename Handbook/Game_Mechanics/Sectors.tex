% This is part of the Avaneya Project Crew Handbook.
% Copyright (C) 2010-2016 Cartesian Theatre <info@cartesiantheatre.com>.
% See the file Copying for details on copying conditions.

% Sectors section...
\StartSection{Sectors}

Depending on whether the user is playing single player or multiplayer, the game can take place in any one of a number of different {\it sectors}\index{Sector}. Sectors are physical geographical subdivisions of Mars that constrain most of the player's activities. Grouping the planet this way is useful for a number of different practical reasons.

For one, it allows us to highlight through isolation the different physical characteristics of the planet. As an example, contrast a sector within {\it Valles Marineris} with one set in {\it Arcadia Planitia}. We would expect them to have very different appearances, geography, climate, lighting, and maybe even some resources not found in the other. This allows experts to apply their knowledge of specific regions of familiarity in their design.

Another reason is that sectors allow server administrators a useful virtual metric for managing costly resources that make a large online multiplayer experience possible. When a sector within, say, {\it Arcadia Planitia}, approaches the maximum user activity it can host in the virtual space it encompasses or with the physical computing resources available, it indicates that additional resources are necessary to keep the server going. This incremental approach allows us to gently scale the backend to meet user demand as necessary with more CPU cores, disk, RAM, cryptographic accelerators, or what have you. 

Another benefit of modularizing most of a user's activity into sectors is in the server architecture itself. Perhaps growing user demand has necessitated that a new sector be created somewhere new on Mars, such as {\it Olympus Mons}\index{Olympus Mons}. We could efficiently host this by using a distributed architecture with each sector potentially running on its own dedicated physical server. This allows us to offload what would have been a single saturated machine across an efficient {\it Solnet cluster}\index{Solnet+Cluster}. We will discuss Solnet further in \in{section}[Multiplayer: Solnet].

