% This is part of the Avaneya Project Crew Handbook.
% Copyright (C) 2010-2013 Cartesian Theatre <kip@thevertigo.com>.
% See the file Copying for details on copying conditions.

% Terms & Concepts chapter...
\StartChapter{Terms & Concepts}

The following is a collection of terminology used in various places in this book and the game world. They range from the everyday colloquial to the technical. Some of it is factual and some fictional.

% Albedo...
\StartSection{Albedo}
The proportion of incoming light reflected by a surface. It is the ratio of reflected light to total incident light. That is, a material with an albedo of 1 is a perfect mirror whereas a material with an albedo of 0 absorbs all incident light.

% ATATÜRK
\StartSection{ATATÜRK}
NAU-CIA's cryptonym for Arda Baştürk. See \in{section}[Arda Baştürk] for more on Arda. The name is drawn from Mustafa Kemal Atatürk (1881--1938 A.D.), an Ottoman and Turkish intellectual, army officer, writer, statesman, and first president of the former Republic of Turkey.

% Bancor...
\StartSection{Bancor}
A global currency proposed in the early 1940s by the British economist John Maynard Keynes. On 13 April 2010, the {\it Strategy, Policy and Review Department of the International Monetary Fund} published a paper\footnotecite[imf_bancor] recommending the world adopt the bancor and create an international central bank to administer it. The Avaneya game universe adopts the small Greek beta character \quote{β} as the bancor's currency symbol.

% Cornucopianism...
\StartSection{Cornucopianism}
The term {\it cornucopia} comes from the Greek legend of the magical horn of plenty, which was capable of providing the bearer with an abundance of whatever they desired.

Cornucopianism is a school of thought that argues that increased demand for resources as a function of time can always be met with increases in the means used to acquire them. As an example, advocates for cornucopianism sometimes argue that Terran peek oil is not a problem because either it will never run out, new innovations will reveal new reserves, or future technological advances will provide alternative forms of energy in time. 

The doctrine asserts that systems of infinite growth aspirations will never meet any limiting constraints, either because the environment it operates within is believed to be infinite, or for some other reason. This is often contrasted with the doctrine of Malthusianism (\in{section}[Malthusianism]).

% Corporate personhood...
\StartSection{Corporate Personhood}
A corporate charter creates a legal fiction with many of the basic rights of an actual human being. These may include the rights to own and sell assets, to sue and be sued, and so on. Critics argue that this ensures the actual physical human beings responsible for its decisions are buffered from ever having to be held personally accountable. Advocates for corporate personhood argue it promotes economic prosperity which, in turn, generally increases the quality of life for everyone.

% ERV...
\StartSection{Earth Return Vehicle (ERV)}
Earth return vehicles are capable of ascending from the surface of Mars and returning back to Earth directly. Once they have landed, they can refuel themselves automatically from the Martian atmosphere using the Sabatier reactor technology (\in{section}[Sabatier Reactor]).

% Endothermic reaction...
\StartSection{Endothermic Reaction}
A chemical reaction that requires energy to be put into it in order to react. An example would be cooking an egg. The egg does not change unless you apply heat to it. This is in contrast with an exothermic reaction (\in{section}[Exothermic Reaction]).

% Exothermic reaction...
\StartSection{Exothermic Reaction}
A chemical reaction that releases energy as part of its reaction. An example would be burning wood. As the wood changes, it releases energy in the form of mostly heat.

% FAP...
\StartSection{Free Arcadia Partisans (FAP)}
Free Arcadia partisans are the political and ideological movement in the Arcadian Settlement in the years leading up to its independence. They were instrumental in the drafting of the Arcadian Constitution and in encouraging an Arcadia free of Terran political and economic control. Prior to year zero, it was mostly clandestine in nature and out of UNSA sight.

% Fifth column...
\StartSection{Fifth Column}
A clandestine force that undermines a greater power from within. The term originated during the Spanish Civil War on Earth in reference to a \quote{fifth column} of supporters within the city of Madrid that would support the four columns of military forces attempting to place the city under siege.

% In situ...
\StartSection{In Situ}
From the Latin, in position. When used in space--related contexts, it denotes a situation where something is not done or not available until in the actual field and not prior.

% Malthusianism...
\StartSection{Malthusianism}
The term {\it Malthusianism} is used to denote the doctrines of the 18th century English political economist, Reverend Thomas Malthus (1766--1834). His school of thought argued that populations tend to increase at a rate that exceeds their environmental carrying capacity. When this occurs, he expected that human suffering would inevitably follow.\footnotecite[malthus1798]

The 19\high{th} century French mathematician, Pierre-François Verhulst, after having read Malthus' seminal works, {\it An Essay on the Principle of Population}, produced his famous Verhulst equation shown here (\in[formula:Verhulst equation]) solved for population as a function of time.
\crlf

\placeformula[formula:Verhulst equation]
\startformula
P(t) = \frac{K P_0 e^{rt}}
             {K + P_0(e^{r t} - 1)}
\stopformula
\startlegend
\leg P \\ population at time {\it t} \\ \\
\leg t \\ time \\ \\
\leg K \\ maximum carrying capacity \\ \\
\leg P_0 \\ initial population \\ \\
\leg r \\ growth rate \\ \\
\stoplegend
\crlf

Observe in \in{formula}[formula:Limit of Verhulst equation] how as time goes on, the population will always asymptotically approach {\it K}. That is, it keeps getting closer to {\it K}, but never actually surpasses it. This is known as the {\it law of population growth}. 
\crlf

\placeformula[formula:Limit of Verhulst equation]
\startformula
\lim_{t\to\infty} P(t) = K.
\stopformula
\crlf

When the Verhulst equation is graphed out, it looks like \in{figure}[figure:Verhulst equation]. The population's maximum carrying capacity is denoted by {\it K}. It starts off with near exponential growth, but then rapidly decelerates as resources are consumed.

\placefigure
    [here, force]
    [figure:Verhulst equation]
    {The Verhulst equation illustrating the law of population growth.}
    {\externalfigure[Terms_and_Concepts/Images/Law_of_Population_Growth.svg][][width=.5\textwidth]}

Malthusians acknowledges the inherent dangers of systems with aspirations for infinite growth that operate within the constraints of a finite environment. This is often contrasted with the Cornucopian school of thought (\in{section}[Cornucopianism]). 

On the surface, although Malthus' observations seem obvious at first glance to most, it sometimes requires unlearning deep seated convictions held since times of great antiquity of which we will explore further momentarily. Further, it is important to explore how these two schools of thought actually form the basis for much of the contemporary thought which underlies innumerable heated debates across the social, political, economic, and even technological strata.

Although the law of population growth is not a political philosophy, but a basic tenet of ecology, as previously mentioned, it is not without its Cornucopian critics. These critics are usually influenced, sometimes without realizing it, by Abrahamic theology or the economic philosophy espoused by Milton Friedman (1912--2006) and his derivatives.

The latter was a prominent Nobel prize winning economist and proponent of an economic philosophy advocating a market system of unlimited, unbound, unregulated, markets. A common theme of all neoclassical economists, like Friedman, is the implicit requirement of an environment distinct from the economy. The economy is not thought of as a utility, that which facilitates life on Earth in the provision of roads, waterworks, educational facilities, and maximizing the quality of life, but as yet another distinct system. It is not thought of as a subset of the environment, but as something in and of itself. The economy and the environment are logically distinct systems having no causal relationship with one another. The language used in its descriptions and models might leave one convinced that an economy could still remain even in the absence of people and a planet to put them on.

This is why whenever neoclassical economists are forced to admit that every single critical life support system on our planet is in a state of decay, they are fast to salvage their dangerous theoretical framework by relegating all such phenomena to {\it externality}. In other words, this euphemism, doubling as a subtle cop out, prevents them from ever having to accept some form of responsibility in never having accounted for the environment, that is, reality, within their models. The models, of course, are entirely self referential with no life coordinate ever present among them.

But not all critics come from a secular neoclassical position. The Abrahamic theologians, as critics, reason the planet's resources cannot ever be used irresponsibly, given that they are under the custody of an omniscient, benevolent,  supernatural deity -- though the more intelligent among them are seldom to risk revealing ghostly metaphysics explicit in their apologetics. Man, nature, and God are logically distinct from one another with God exercising custodianship over nature. Man, having been made in God's image and endowed with freedom of the will, is free to provide for himself through God's opulent kingdom, their to be subdued under his paternalistic reign. Although having been born with freedom of the will, he is also born sick and punished for his sins.

It is also interesting to note that the term {\it conservative} in this latter context is taken to mean something other than conservation of God's kingdom, given that it is self--replenishing under his benevolent tutelage such that it never requires that his subjects accept responsibility for any environmental disasters. The prospect that God will clean it up is interesting to considers when one realizes that freedom of the will has been a foundational axiom of social conservative thinking for centuries and is a recurring necessary condition in virtually every position advocated by them. Freedom of the will, but not freedom to be held accountable.

As one thoughtful satirist aptly characterized critics of Malthusianism, \quote{\it Some commentators are now on their feet saying Climate change is a plot by the Power elite. It's tempting to cry victim when the system curbs behaviour, but are we the victims or are we the perpetrators?} \footnotecite[juicemedia_climate_change] 

In some parts of the world, great expenditures are made in convincing millions that there are but two ways of understanding every problem. Well understood as they may be,\footnotecite[goodnight1980] neither present viable substitutes for the role science, like ecology, can play in public policy decisions since the environment has never been known to take a political stance. As an example, the hydrological cycle preceded the development of every political philosophy ever conceived of.

As a last act of desperation, many critics are drawn to attack Malthus himself by claiming Malthus had advocated extreme measures of population control, such as executions of surplus population. However, nearly always and without exception, it is evident that Malthus' original writings on the subject, of which no such claims were ever made, were ever consulted.

Whether critics are schooled in the pseudo science of secular contemporary neoclassical economic theory or the cookie--cutter cartoon--like world view social conservatives inherited from the fearful Bronze Age superstitions of antiquity, or some hybrid of the two, the important observation to take away is that they all have found creative ways of presupposing an Earth of infinite capacity, and therefore of accommodating an infinite margin of error.

% Mars Direct...
\StartSection{Mars Direct}
Mars Direct is a \dollar 50 billion dollar plan proposed by American aerospace engineer, Robert Zubrin\index{Robert Zubrin} (born April 19, 1952). It is an alternative to the prohibitively costly \dollar 450 billion dollar mission to Mars proposed by NASA in consultation with its government.

The then incumbent President of the United States President of the United States\index{President of the United States}, George H. W. Bush\index{George H. W. Bush}, announced the government's proposal in 1989 as the {\it Space Exploration Initiative\index{Space Exploration Initiative}}. It called for the creation of the {\it Space Station Freedom\index{Space Station Freedom}} and a permanent Lunar base\index{Lunar base} as intermediate steps for an ultimate destination to the Red Planet. If implemented, it was to be rolled out over the process of several decades.

Zubrin reasoned that it is totally unnecessary to construct giant space stations in low earth orbit, useless Lunar bases on a barren moon, and massive spacecraft carrying hundreds of people to achieve a manned mission to Mars. That, along with transporting all that is necessary to get there, survive there, and return safely. He argued that the government prefers an intentionally bloated approach because it creates the illusion of progress and productivity through countless jobs, contracts, bureaucratic expansion, and so on. But it comes at the cost of enormous waste, misdirected resources, and, through increased complexity, an increased likelihood of disaster.

Zubrin compared their approach to the failed Arctic explorer, Sir John Franklin\index{Sir John Franklin}, who, with government assistance in 1845 took two ships, the {\it Erebus} and {\it Terror}, each displacing more than 300 tonnes in an effort to navigate through the Northwest Passage\index{Northwest Passage}. His ships carried all manner of useless items, including heavy English silverware, but spared many of the critical items necessary for survival. 

The crew met a bitter end as they dragged heavy iron and oak sleds across the Arctic ice, having abandoned their vessels frozen in the ice. With shotguns useless in the Arctic and other heavy and inappropriate equipment, all 127 men ended up perishing to a combination of the elements and scurvy. It never occurred to them to take advantage of {\it in situ} resources, like fur coats, seals, and fish.

Zubrin argued that the Space Exploration Initiative's mission architecture is an absolute inverse of a sound engineering approach. He outlined cogently in his book {\it The Case For Mars} for a very reasonable, well thought out, minimalistic approach of travelling light, living off the land, and manufacturing the necessary rocket propellant for the return trip {\it in situ}. This is akin to the efforts of early Terran settlers, like those who pushed through the American Western Frontier, or virtually ever other civilization's successful effort at settling a distant land. Going to another planet is, according to him, no different. Indeed, the travel time to Mars is comparable to that of navigating the Northwest Passage.

This trans--planetary travel to Mars is possible because of the wealth Mars already contains. It has an abundance of natural resources necessary for creating rocket fuels, water, plastic polymers, alloyed metals, glass, gasses like oxygen, semi--conductors, ceramics, and just about everything else one might require. All this, he calculated, at a fraction of the cost of NASA's proposal, and using technology that has been around since the mid--19th century.\footnote{See the RWGS and Sabatier reactors in sections \in{}[Reverse-Water-Gas-Shift Reactor] and \in{}[Sabatier Reactor] respectively.} He claims his plan could be realized in less than a decade by leveraging existing technology, as opposed to the several decades required of plans like the Space Exploration Initiative.

% Money as free speech...
\StartSection{Money As Free Speech}
In 1976 A.D., the Supreme Court of the former United States ruled in {\it Buckley v. Valeo} that, while there must be limits on contributions to political campaigns, spending money to influence elections without limit is permissible. The logic being that since corporations are considered legal persons, and since the rights of persons are constitutionally protected under the 14th amendment, and since corporate media has the right to freedom of the press (speech), corporations, therefore, ought to have freedom of speech as well. Since they are not capable of speech in the physical sense, the closest analogue is money.

% Regolith...
\StartSection{Regolith}
What most refer to as dirt. More technically, it is the the loose heterogeneous mixture of material that blankets the solid rock of a planet. Although the term may seem redundant, note that \quote{earth} refers to regolith from the planet Earth, while dirt can sometimes mean regolith that contains organic materials.

% RWGS reactor...
\StartSection{Reverse-Water-Gas-Shift Reactor}
The RWGS reactor is a method of producing oxygen (\chemical{O_2}) from carbon dioxide (\chemical{CO_2}). This is useful because the latter is plentiful in the Martian atmosphere at 95 \% .

\placeformula[reaction:Reverse Water Gas Shift Reaction]
\startformula
\inlinechemical{CO_2,+,H_2,+,->,H_2O,+,CO}
\stopformula

The process has been known since the mid 1800s and works by reacting carbon dioxide and hydrogen gasses together over a copper-on-alumina catalyst. Aqua (liquid water) and carbon monoxide gas are produced as byproducts. The is illustrated in \in{reaction}[reaction:Reverse Water Gas Shift Reaction].

The aqua is then split via electrolysis to produce molecular hydrogen and oxygen. The hydrogen can be recycled back into the reactor and the carbon monoxide purged back out into the atmosphere.

The reactor needs to be at \math{400\,^{\circ}{\rm C}} and at low pressure. It requires about 180 watts of power, or about 3 \math{m^2} of solar panels on a fully sunny day's average solar flux. At that energy rate, you can expect to produce about 1 kg per day of oxygen, which is sufficient for a single person. The reactor requires power because it is an endothermic reaction. However, it is possible to use a Sabatier reactor in tandem which is an exothermic reaction to provide the heat required to drive the RWGS reaction.

To start the process, only a small amount of water is required which acts as a reagent. Otherwise, one needs to import some hydrogen from Earth, but this only needs to be done once.

The reactor featured in Avaneya is an augmented form of that just described because it incorporates the electrolysis directly into the machine to produce \chemical{O_2} from the water produced, cycling the \chemical{H_2} back into the reactor.

% Sabatier reactor...
\StartSection{Sabatier Reactor}
A chemical process for creating methane \chemical{CH_4} from \chemical{CO_2} and hydrogen. This is useful because carbon dioxide gas is plentiful in the Martian atmosphere at 95 \%.

\placeformula[reaction:Sabatier Reaction]
\startformula
\inlinechemical{CO_2,+,4H_2,->,CH_4,+,2H_2O,+,heat}
\stopformula

The reactor needs to be at \math{400\,^{\circ}{\rm C}} and at low pressure. This makes it almost the same as the RWGS reactor except that it uses a different catalyst to make methane instead of carbon monoxide. You can either use nickel, which is cheap, or ruthenium--on--alumina, which is safer, but more expensive. This process is described in \in{reaction}[reaction:Sabatier Reaction].

% Sierra
\StartSection{Sierra}
Mars Command (MARSCOM) military phonetic alphabet designation for a \quote{settler} of the Arcadian Settlement.

% Sol...
\StartSection{Sol}
Short for solar day, the length of time a planet takes to rotate completely on its polar axis with respect to the Sun. Terrans call this a day, Martians a sol. The Arcadian method of timekeeping is described in detail in \in{section}[Time]. See also {\it yestersol} in \in{section}[Yestersol]. 

% Specific impulse...
\StartSection{Specific Impulse}
Written \math{I_{sp}}, the specific impulse is a useful metric for comparing rocket efficiency. Whenever you see the word \quote{specific} in a physics context, it means something per unit of mass. The units here are in seconds. Specific impulse measures the amount of time that one pound of fuel will burn for, producing one pound of thrust (higher being better). This can be calculated using either SI or Imperial units, but the end result is usually expressed in seconds (SI).

As an example, compare the specific impulse of some of the different types of rockets.
\crlf

\placetable[here,split][table:Specific Impulse Comparison]{Comparison of different rocket specific impulses.}
{
    \bTABLE[split=repeat,option=stretch]% head on every page, stretch columns
    \setupTABLE[row][each][align=center]

    \bTABLEhead
    \bTR
      \bTH Rocket Type \eTH
      \bTH Fuel \eTH
      \bTH \math{\bf I_{sp}} \eTH
    \eTR
    \eTABLEhead
    \bTABLEbody
    %
    \bTR
      \bTC Ancient Chinese Rocket \eTC
      \bTC Gunpowder \eTC
      \bTC 80 \eTC
    \eTR
    \bTR
      \bTC Modern Rocket (e.g. ICBM) \eTC
      \bTC Solid \eTC
      \bTC 250 \eTC
    \eTR
    \bTR
      \bTC Saturn V \eTC
      \bTC LOx / kerosene \eTC
      \bTC 260 \eTC
    \eTR
    \bTR
      \bTC Space Shuttle Main Engine \eTC
      \bTC LOx / \chemical{H_2} \eTC
      \bTC 400 \eTC
    \eTR
    \bTR
      \bTC Nuclear Thermal \eTC
      \bTC Solid \eTC
      \bTC 800 \eTC
    \eTR
    \bTR
      \bTC Nuclear Thermal \eTC
      \bTC Liquid \eTC
      \bTC 1300 \eTC
    \eTR
    \bTR
      \bTC Jet Engine \eTC
      \bTC Compressed Air \eTC
      \bTC 3000 \eTC
    \eTR
\eTABLEbody
\eTABLE
}

Note how high the specific impulse a jet engine offers. This is because it is has an unlimited supply of free air from the atmosphere to feed the air compressor so it does not have to carry its own supply.

% Yestersol...
\StartSection{Yestersol}
The sol preceding the current one. This is the Martian analogue to the Terran yesterday, but different since the length of a sol on both worlds is different. The Arcadian method of timekeeping is described in detail in \in{section}[Time]. See also {\it sol} in \in{section}[Sol].

\StopChapter

