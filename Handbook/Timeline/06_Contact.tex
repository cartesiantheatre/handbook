% This is part of the Avaneya Project Crew Handbook.
% Copyright (C) 2010-2013 Cartesian Theatre <info@cartesiantheatre.com>.
% See the file Copying for details on copying conditions.

% Contact section...
\StartSection{Contact}

\StartTimelineDate{12 Leo, 2 A.R.}
Elements of the {\it 1\high{st} People's Own Arcadian Mechanized Infantry} inform their Chain of Command that they have detected an unlocalizable nav--aid transponder signal bearing a non--Arcadian signature within their demarcation while conducting a routine dismounted patrol.
\StopTimelineDate

\StartTimelineDate{20 Leo, 2 A.R.}
UNMEPF's forward observers establish visual contact with Arcadia. The personnel are closer to the target area than on Earth due to the smaller line of sight on a smaller planet where the horizon is closer. The gunners relay precision bearings, ranging, and ordnance selection over a secure uplink with an artillery battery. The latter is instructed to fire for effect with the gunners prepared to direct and correct fire in realtime as necessary.

Arcadia comes under fire from what appears to be orbital shelling. The event marks the first shots ever fired in anger on Mars. The original {\it Arcadia Hall} is destroyed, though most of the city is deliberately left intact. 

Having realized that the city had been evacuated, MARSCOM's Intelligence Company prepares the necessary command sequence for the {\it Bhadra I} photographic reconnaissance satellite in orbit to identify all potential locations suspected of harbouring the settlers' emergency underground shelter. Phantom targets across the surface of {\it Arcadia Planitia} are hammered from what appears to be the fast moving Phobos for more than an hour every sol over the next twenty. There are no casualties reported.

Between shellings, Arda has personnel from the {\it 3\high{rd} People's Own Arcadian Combat Support Engineers} erect a rudimentary high--gain directional antenna. This circumvents the need for access to a satellite communication relay normally necessary to transmit back to Earth by instead relying on S--band microwaves, allowing settlers, along with embedded journalists, to maintain communication with Earth. Several journalists liken the hostilities between the belligerents to the scale and asymmetry of the late 20\high{th}--century Falklands War.

Arda explains to anyone able to receive him on Earth what is happening on Mars. He suspects that the average Terran has probably been left in the dark by their governments and mainstream media. He describes the United Nations' shelling campaign as an attack on peaceful and defenceless civilians levied from a manned orbital installation on Phobos. The installation apparently functioning as an orbital artillery battery. He claims the UN's intentions are to force the settlers to surrender so that they will resume applying their extensive expertise in the resumption of material exports under a newly restructured administration.

During his transmission, Arda references a document hacked from a UNSA intranet as evidence of the orbital artillery battery's existence. The document was part of the dossier received from the anonymous tip--off that had revealed the United Nations' intentions in the use of force, received prior to Arcadia having lost solnet connectivity when UNSA's {\it METO} communications satellite was taken out. The document is titled {\it Project New Horizon: A UNSA Study for the Establishment of a Phobos Artillery Battery}.

The detailed white paper, complete with schematics, describes the first orbital artillery battery ever put into service. The classified document details an ambitious covert UNSA project to construct {\it Thor Battery}. The outpost is apparently manned by a garrison of a single platoon of thirty soldiers, powered by three nuclear reactors, and located on and within the Martian moon Phobos. 

\startTimelineGeneralDocument
...The Phobos outpost is required to develop and protect strategic UNSA administered interests on Mars; to develop techniques in Phobos and general orbital--based surveillance of Mars, in communications relay, and in operations on the surface of Mars; to serve as a base for the exploration of Phobos, for further exploration into space and for military operations on Phobos if required; and to support scientific investigations on Phobos...

\hskip 1.5cm {\it --- Project New Horizon: A UNSA Study for the Establishment of a Phobos Military Outpost}
\stopTimelineGeneralDocument

Construction was feasible due to the moon's composition. From the surface down to the first one--hundred metres, the moon is covered with a very fine regolith. The remaining solid material beneath it is of a mostly porous composition. These two characteristics allowed for a straightforward excavation in ensuring that most of the installation could remain subterranean.

Arda provides high resolution imagery taken by Senka from the Martian surface a month prior with the transmission as evidence. The images clearly depict an outpost still under construction within the moon's Stickney crater\index{Stickney crater}. He explains that the outpost was originally not intended for active service yet, but rather as part of a long term United Nations' insurance policy for dealing with \quote{\it the distinct possibility of future settlement anomalies} (Project New Horizon), should the need ever arise. The option was on the table, but only to be taken seriously well after, as MARSCOM had expected, a straightforward capture of Arcadia with the assumption that its inhabitants would not be capable of defending themselves.

Arda illustrates that by exploiting a legal loophole in the {\it Outer Space Treaty}, the {\it Anti--Ballistic Missile Treaty}, and the {\it Mars Treaty}, whose draftings predated the advent of kinetic bombardment weapons, the United Nation's Office of Legal Affairs probably advised UNSA that it was legal to target the surface using hypersonic shells. The shells contain tungsten cores and are capable of hitting targets with surgical precision at downward velocities approaching 10 km/s. These treaties did not observe kinetic bombardment as weapons of mass destruction since they require no explosive payload, though the shells containing more massive cores deliver sufficient kinetic energy in the order of a small tactical nuclear weapon upon impact.

The transmission is picked up on Earth by amateur radio operators and quickly disseminated over solnet peer--to--peer networks. Though mainstream media do not acknowledge the transmission, dissemination is sufficiently rapid and widespread that Terran citizenry are best described as mostly of a mixture of fury and disgust at the actions of their respective government's involvement. Amnesty International calls for a public inquiry with powers of subpoena. The Red Cross appeals to UNSA to allow them to provide observers and medical personnel on the ground.

UNSA dismisses Arda's claims as \quote{\it baseless allegations propagated by Terran conspiracy theorists, intertwined with a misunderstanding of the nature of meteorite strikes on Mars, used to justify the theft and reckless destruction of a costly set of off--world publicly funded assets}. A UNSA spokesman refuses to respond to existential questions over the now controversial Phobos based outpost. 

The subject is controversial since the exposure of a clandestine orbital installation, if the sensitive document Arda cited is genuine, provides strong evidence that there were ulterior motives behind the {\it Avaneya Initiative} since its inception, given that the purpose of the {\it Thor Outpost} clearly has little or nothing to do with sustainability research. This reasoning increases in popularity on Earth and helps to promote Terran support towards Arcadian independence.
\StopTimelineDate

% Constraint: 20 sols after Thor Outpost commenced shelling...
\StartTimelineDate{40 Leo, 2 A.R.}
A column of UNMEPF armour and a battalion's worth of mechanized infantry are spotted approaching the city by an Arcadian Dragoons armoured reconnaissance platoon commanded by a single Armoured Officer, Capt Henrik Nørgaard. The Dragoons have strict rules of engagement and {\it no move before} orders to not engage UNMEPF forces unless they cross a perimeter extending two kilometres outward from Arcadia's surrounding impact crater's elevated rim.

The UNMEPF column halts in response to Henrik's order to the Dragoon's lead tank to fire a single warning shot. The shell lands several hundred metres directly ahead of the column's line of advance. Arda addresses UNMEPF's Commanding Officer over the tactical radio's common band. He advises that UNMEPF forces will be met with a measured response if they continue with their advance through the DMZ and do not withdraw immediately outside the RAP's demarcation. 

Brigadier General Dragov responds by identifying himself as the brigade's Commanding Officer responsible for the formation. He assures Arda that, if the settlers' forces surrender themselves immediately and disclose the location of their emergency shelter, they will be given quarter. Arda repeats the original directive and breaks radio contact.

Having not anticipated the presence of the militia, Dragov liaises over satellite with MARSCOM's Joint Chiefs of Staff at Peterson Air Force Base, North American Union, for further orders. He is advised that MARSCOM's photographic reconnaissance satellite {\it Bhadra I} shows only a single platoon of Dragoons of four light tanks on the inside lip of the impact crater's rim. Dragov orders a temporary tactical retreat of the brigade to withdraw eight--hundred meters backward and establishes a communications uplink with a UNMEPF Weapons Control Officer operating from Kali in orbit who is in contact with {\it Thor Outpost}.
\StopTimelineDate

\StartTimelineDate{41 Leo, 2 A.R.}
Dragov takes advantage of {\it Thor Outpost}'s two and a half hour launch window. The UNMEPF Weapons Control Officer requests forward observers to issue grid references for Henrik's Dragoons. The orbital artillery battery is instructed to load smaller calibre precision tactical tungsten shells to minimize damage to adjacent UNSA assets, Arcadia. Dragov gives the fire command to release the salvo.

The {\it 3\high{rd} People's Own Arcadian Combat Support Engineers} alert the Dragoons that an infrared signature in orbit has been detected. The engineers know through previous experience observing {\it Thor Outpost} that, though unlocalizable, the signature is characteristic of an orbital kinetic launch which has always preceded past bombardments.

Henrik instructs the crews to load and fire smoke canisters containing an infrared reflective additive through their vehicles' grenade--launchers. The canisters fire vertically allowing the temporary screening of movement from orbital surveillance. Dismounting the vehicles, Henrik leads the platoon to seek refuge within tunnels previously excavated for the occasion. The platoon rendezvous two--hundred metres below the surface with an anti--tank platoon of the {\it 1\high{st} People's Own Arcadian Mechanized Infantry}.

Eleven minutes later the salvo completes its atmospheric entry, destroying the abandoned tanks with surgical precision. Falling rocks rolling down the slope of Arcadia's impact crater and chunks of shredded airborne permafrost moving with high speed depressurize an evacuated habitat cluster within Arcadia's Third Precinct. There are no casualties.

Arda contacts Dragov over the tactical radio's common band. In a tone of distress, he offers the Commander Arcadia's conditional surrender, contingent on quarter being granted to all of Arcadia's civilian and military personnel. Dragov is agreeable and advises a jubilant Joint Chiefs of Staff over satellite of Arda's disposition. In response to Dragov's request for the emergency shelter's location, Arda provides a grid reference within the city and advises Dragov to proceed via the most expeditious means leading him into the city through its main entrance set within the impact crater rim surrounding the city.

Though darkness rapidly descends, Dragov finds Arda suspect and reasons it would not be prudent to hold fast until dawn. However, he is weary of taking the brigade through Arcadia's main entrance through the crater's rim, suspecting it mined. He leads his brigade without contest from eight--hundred metres outside the two kilometre perimeter previously withdrawn to, forward to meet with the base of the slope of the impact crater's ridge leading into the city. Dragov briefs his officers that this is necessary to feign committing themselves to the city's main entrance. The column performs a ninety degree turn at the base of the slope, altering its course without warning at the mouth of the city to carry on clockwise along the base of the ridge rapidly for two kilometres where the apex of the base is the site of the recently destroyed dragoons.

Dragov, having earlier studied {\it Bhadra I's} photographic reconnaissance imagery, observed the site of the destroyed Dragoons on the impact crater's ridge and calculated that the shells had cleared enough of the outer ridge away to allow armour and the rest of his units to penetrate through with an element of surprise. The route would lead them along the inside of the ridge for about eight--hundred metres, then down the route the Arcadian Dragoons probably had taken to ascend where they were destroyed. He bases this on what appears to have been recent armoured rovers' tank tracks leading from somewhere within the city.

Having arrived at the base of the ridge that leads up to the destroyed Dragoons, UNMEPF forces rapidly negotiate the incline of the ridge. The first armoured squadron penetrates through the ridge with three battalions of mechanized infantry and combat support closing in the rear. By the time the last of the brigade passes through the blasted rim, the fore of the column reaches the end of the path along the inner ridge as it prepares to descend down into the city. Moving with haste, the column fires illumination flares laterally over various unlit areas of the city. There are no visible signs of settlers.

The convoy passes under an anti--tank hunter platoon of the {\it 1\high{st} People's Own Arcadian Mechanized Infantry} concealed within thermal reflective blankets. The lead tank at the fore of the UNMEPF column is rendered immobile as it makes contact with a non--magnetic mine housed within composite plastic and bamboo. The tank unravels itself of its tracks. The next vehicle in the column panics and attempts to negotiate around but is hit from an anti--materiel rifle, breached, and rendered immobile -- blocking the full width of their route.

Simultaneously the rearmost combat support vehicle in the column is hit from an elevated firing position. The platoon hammer the target with several rounds from each section's recoilless shoulder fired rifles. The lead tank's main gun is knocked out. 

With the lead tank destroyed and the engine block blown out of the rear vehicle, the UNMEPF column is immobilized between the fore and aft with no room to negotiate out of the column laterally. The column's mechanized infantry immediately dismount armoured personnel carriers as they attempt to secure cover at the path's flanks behind the only available, the vehicles themselves, activating RAP trip flares in the process.

With the halted UNMEPF column now in enfilade, its longest axis exposed down the line of Arcadian infantry arcs of fire, the latter's Section Commanders open fire on well sighted targets, signalling to their fireteams to engage all targets of opportunity. UNMEPF issue contact reports throughout the column as personnel and armour come under fire from multiple unlocalized elevated firing positions.

Several salvos of RGA1 anti--personnel cryonades are released down the ridge at the halted column, dispatching dismounted soldiers in number. The explosive ordnance, having a fragmentation range more than twice that on Earth, suffering minimal atmospheric attenuation, send an array of UN armour rolling down the incline -- including Dragov's staff rover. Many of those having sought protection behind vehicles are knocked down the ridge and pinned under.

UNMEPF forces attempt to return fire, but have difficulty sighting well dug in targets in the low Martian night light. Illumination flares and tactical thermal imaging equipment prove useless as the brigade becomes immersed in its own thermal gas clouds as pressure suits are systematically punctured, degas, and the occupants rapidly bleed out in the near vacuum of \math{-72^{\circ}}C.

MARSCOM receives UNMEPF's first contact report and responds rapidly with only satellite assisted target identification -- the belligerents now too proximate for Phobos fire support. In the six minutes that have passed from the moment of issuance of the brigade's first contact report and its reception on Earth, a wounded Second in Command in the second to last vehicle in the column, Major General Nilhara, already has his updated and final situation report {\it en route} back to Earth.

Nilhara's final situation report informs the Joint Chiefs of Staff that Dragov is missing in action, presumed dead; that they have sustained heavy casualties; that, with the exception of isolated exchanges of small arms fire, hostilities have discontinued; that they are in the process of being relieved of their weapons; that some of their communications equipment may have been seized before all cryto codes could be zeroed; that personnel are being separated from one another; and what remains still in communication are already negotiating their unconditional surrender over the tactical common band.

Having been monitoring the tactical common band since the onset of Arcadia's orbital shelling, embedded journalists in the settlement's emergency underground shelter are authorized from an RAP liaison officer to broadcast the news to Earth over S--band microwave.
\StopTimelineDate

