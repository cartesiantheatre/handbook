% This is part of the Avaneya Project Crew Handbook.
% Copyright (C) 2010-2013 Cartesian Theatre <kip@thevertigo.com>.
% See the file Copying for details on copying conditions.

% Interface & Interaction section...
\StartSection{Interface & Interaction}
The section of the handbook deals with the user interface and the users interaction with various elements of the game including selecting, placing, and controlling units.

\StartSubSection{Scrolling & Unit Focus}
Scrolling around the map can be done by moving the mouse to the sides of the screen, using arrow keys, or clicking a minimap to instantly center on that location and then move by dragging the mouse. Focus can also be bought on an area of interest by pressing a key relevant to that event. There may be several "event types" or "categories" of events happening annd players can choose to quick jump to the latest of each event type by hitting the corresponding key. For instance the latest building to finish may be jumped to by hitting "J", the latest unit to finish moving jumped to by hitting "K", and the latest resource depletion may be jumped to by hitting "L". The list of such events would be quite exhaustive and events given allocation keys chosen on necessity.

Unit movement will not be focused on a single unit unless this is selected by the player or is being dictated so by cinematics or other game events. Players can choose to focus on unit movement by clicking a button on the user interface or pressing the corresponding key if there is any.

Depending on the size of battles and the number of controllable units in Avaneya, players will be able to have as many units selected as needed at any one time. If battles end up being of a smaller scale then this aspect of the game can be modified.

Units can be ordered into permanent squads and by selecting one unit of the squad the entire squad is selected and will move together as the selected unti is moved and commanded. Units can also be grouped into command groups by selecting the units and allocating them to a number key, then to select all of those units again the player need only push the number.

Unit damage can be displayed on a unit and toggeled on and off by pressing a key. Unit damage may be more or less relevant depending on the type of unit. General infantry based units may be either taken out or injured unless wearing an armor type that recieves damage. Mechanical units will recieve damage to various areas of the vehicle creating vulnerablities that can be exploited and that will need to be protected. If units have health, how the health of each unit is tracked will be important. This can be done by health bars over the heads or underneath of units, colour overlays or outlines, or limited to a separate window on the screen. Damage or injuries can be done in a similar manner with affected areas highlighted in manners depicting the extent to injury or damage.




\StartSubSection{Placing Objects}
Objects are placed onto the terrain by first selecting the object to be purchased/placed then moving to a desired location on the screen. Selection is done by either clicking once on the object then moving the mouse to where the object is to be placed then clicking again, or by clicking the object and holding, then moving the mouse to the location and letting go.

Depending on certain situations and game mechanics, terrain may need to be discovered or line of sight may be required. Other factors may also come into play such as needing a nearby unit, resources, and other purchase costs. Terrain may be in the way of placing an objects and so to ease this, a terrain clearance mechanic could be added to the cost of object placement (if mechanics and other factors permit) and the terrain removed instantly as the object is placed with an increased cost. An alternative is to make terrain clearence a time consuming and costly process requiring units to place charges etc and then leveling terrain etc. The use of each method would depend on how time is played out in the game and how important terrain is through the game.

When an object is selected to be placed, moving the mouse/object to the side of the screen should move the screen as normal. Also deselecting the object should be as easy of hitting the right mouse button or esc or similar key. In addition, to ease misplaced buildings it is advantageous if building costs are established slowly as the buidling is built, with a cancelled building only incurring costs up to the moment it has ben cancelled. This is of course provided that players have not entered contracts with corporations or terran bodies forcing up front payments, in these instances players may be forced to brunt all or some of the cost. Also, certain planning conditions may require large initial cost outlays. All this considered, forcing players to spend hard earned money and resources on an accidental placement may be frustrating, so balance of realism and playability is required with this mechanic.

Objects can be placed anywhere on the landscape, within reason. Some objects may be add-ons, spawned units, or need certain requirements and are therefore built within their certain constraints. In general, provided workers and materials can get to a location and that location has the means to build the specific object, objects and buildings can be built anywhere. How these buildings and objects are run and by whom may change as a players sphere of influence is changed overall. Player control of distant buildings can become controlled by corporations or taken over by other players more easily unless the means of protection and continual control are put in place.



\StartSubSection{Overlays}
As mentioned previously the use of overlays is a valuable tool to display information to the player quickly, allowing them to make remedies and adjustmusts to the city as they see fit. The use of overlays to display information such as crime rates, education levels/types, or resource availability allows the informaiton to be shown in multiple levels/degrees. 

Overlays need to be loaded quickly and turned off just a quick.

Overlays need to be chosen both through a menu and through quick keys.

Display overlays is reducing the players ability to manage other affairs such as diplomacy, combat, exploration, and research and development. Therefore the information overlays divulge, if overlays are implemented, will often neeed to be of a non critical, micro managing nature.

If overlays are going to be used extensively they need to look pretty.



\StartSubSection{Unit Movement and Control}
As described in unit focus multiple units may be selected in multiple ways. Moving units to a location is done by right clicking on this location or selecting the move icon then left clicking. Units will move to this location using the most direct route unless advised to move in another manner.

Militia may also be commanded to move using a variety of techniques which include tactics such as move a shoot, as well as hiding and taking cover. These other commands will mean militia will often move in a less direct route to the selected location and may change course if certain situation arrise such as coming under fire.

Moving militia units by right clicking will cause those selected to move at an average pace, not exhausting themselves but not walking, they will also take cover if fired at and return fire and perform tactics within their selectedd group and with other units in order to get to the chosen location with as little casualties as possible.

Units can also be told to move fast/sprint/double time/pace or some term which will cause them to move at as fast a pace as terrain conditions will allow, ignoring combat conditions to a degree, and largely ignoring return fire and tactics with other units. This fast movement option will allow units to get places quickly but exposes them to the enemy in a number of ways a lot more.

To help avoid exposure to the enemy and for units to be vigilant of enemy movement they can be told to move slowly/sneak. In this type of movement units will move through terrain at such a pace as to avoid as much as possible making noise, taking casualties, and being aware enemy movement and sounds. Units taking fire in this stance will disengage from combat, attempt to hide or remove themselves from contact with the enemy in a manner which keeps casualites low.

These are the three basic movement types. More may be added or needed depending on scenario/campaign conditions or perhaps specialy trained militia units such as special forces or snipers may have extra or different movement types. Special forces may have a power through movement allowed them to run a long distance at a fast pace or there cautions movement may be at a fast pace then others. They may also be able to move through tough or dense terrain without making noises. 

In addition to movement units can be told to patrol locations, being kept on this until told otherwise reducing management require by the player.

All movement can be halted and used told to remain stationary by hitting the stop button or right clicking very close to the unit, causing them to move that short distance and then stop.


