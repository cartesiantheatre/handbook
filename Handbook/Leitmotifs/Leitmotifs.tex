% This is part of the Avaneya Project Crew Handbook.
% Copyright (C) 2010, 2011, 2012
%   Kshatra Corp.
% See the file License for copying conditions.

% Leitmotifs chapter...
\StartChapter{Leitmotifs}

This is probably the most important chapter in the entire book in helping readers to decide whether this is a project for them to be involved in or not. Avaneya has several recurring ideas or {\it leitmotifs}\index{leitmotif} within it that will help you to determine how the project resonates with you.

We would like to keep the leitmotif set as minimal as reasonably possible, with everything else that we felt relevant, it hopefully already being possible to derive when obvious from that which is already present. However, if you still feel the set should be modified in some way, please bring it up on any of the project crew communication methods related in \in{section}[Communication]. We are open minded but, like us, you must also be prepared to reasonably substantiate your argument.

The following form the core leitmotif set the project felt were important, given in no particular order.

\startitemize[4]
\head {\em The world's most successful drug dealers work for government.}

Most illegal narcotics on the street originate from government.\footnotecite[cia_drug_plane_crash]\footnotecite[top_mexican_drug_lord]\footnotecite[afghan_opium_kingpin]\footnotecite[cia_drugs_list]\footnotecite[vicente_rule]\footnotecite[ruppert_confronts_deutch]

\head {\em No one is free.}

To be free is to be independent of influence, which no one is. We are all subject to influence, otherwise there would be no such thing as advertising. Everything is connected and nothing happens in a vacuum. \footnote{See {\it The Equality Trust} in \in{chapter}[Resources For Everyone].}

\head {\em Neoclassical economics is a pseudo-science rife with fraud.}

The major axioms of neoclassical economics, the backbone of modern economic theory, are wrong. People do not always have rational preference, seek self maximization, always act on full and relevant information, or stand the most to gain via competition over cooperation. Markets are not natural systems akin to the laws of thermodynamics, do not always correct themselves, and their growth is sometimes destructive. The assumptions are many in number and they go on, and on, and on.\footnotecite[lasn2012occupy]\footnote{See {\it Adbuster's} \href{http://anticap.wordpress.com/2010/10/25/jamming-neoclassical-economics/}{Jamming neoclassical economics} campaign, with prejudice for \href{http://www.adbusters.org/cultureshop/backissues/85}{Issue \#85}.}

\head {\em Governments kill.}

Government is among the leading causes of death.\footnotecite[statistics_of_democide]

\head {\em Terrorism is usually synthetic.}

Most acts of terrorism are usually staged by governments for any number of reasons.\footnotecite[northwoods]\footnotecite[harrit2009]\footnotecite[debunking_911_debunking]\footnotecite[letter_to_minister_regarding_911]

\head {\em Government is there to keep things the same.}

Government is generally there to protect the interests of a few unelected principle benefactors and not to make changes in the interest of society.\footnotecite[report_from_iron_mountain]

\head {\em Disease begins on our plate.}

Most health related deaths are rooted in diet.\footnotecite[the_china_study]\footnotecite[extras={ According to the study, the top three causes are entirely dietary.}][statscan_leading_causes_of_death]

\head {\em Agribusiness is destroying the planet.}

Agriculture is our principle means of interacting with the planet. Changing what we eat solves many global problems.

\head {\em Today's conquests are brought about through debt, not the sword.}

Usury, including fractional reserve banking, is slavery.\footnotecite[the_creature_from_jekyll_island]

\head {\em Corporations are not people.}

Giving corporations the rights of human beings was catastrophic.\footnotecite[the_corporation]

\head {\em Genes are a cop out.}

Very little of what we call human nature is actually genetic. Genes specify human needs, not behaviour. We will elaborate on this in \in{section}[Socioeconomic Simulation].
\stopitemize

If you have gotten this far and you are still comfortable being involved in this project, you will probably find it rewarding. Otherwise it may not be the best place for you, as it is neither a casual game, nor free of courting controversy. There are no requirements that anyone believe or do anything here. Everyone comes by having made a personal choice. But rest assured, there are many other {\it libre} projects that could use the talents of those still with serious reservations.

But regardless of whatever readers choose to do, creativity is required whenever we present the aforementioned. We always try to encourage users to do some thinking of their own. As Socrates once said, {\it I cannot teach anyone anything. I can only make them think}.

\StopChapter

