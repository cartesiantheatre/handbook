% This is part of the Avaneya Project Crew Handbook.
% Copyright (C) 2010, 2011, 2012 Cartesian Theatre <kip@thevertigo.com>.
% See the file Copying for details on copying conditions.

% Specialities chapter...
\StartChapter{Specialities}
Avaneya is a fairly large, creative, and collaborative project. It goes without saying that it calls upon a variety of disciplines. In this chapter we will discuss the many different exciting ways members of our community can get involved.

There is nothing to stop someone from being involved in more than one capacity. If you want to model, but also happen to be fluent in Italian, by all means consider being a translator as well.

\StartSection{2D Artist}

2D artists produce images that are usually flat and do not move, though there are exceptions. One exception would be the media that forms the basis of the graphical user interface which is managed by CEGUI. It can contain dynamic elements, such as small animations. They are key for producing high quality textures and materials for models, vector graphics, bump and normal maps, conceptual artwork, logos, and storyboarding images. Some of them might also be experts in typeface design. They can work with a range of software, such as GIMP, Inkscape, Blender, FontForge, and others.

\StartSection{Audio Engineer}

These folks creatively can take unassuming sounds and transform them into things usable for science fiction game. An example would be sampling the squeak of a chair or a car driving by with a high end portable recorder and remixing it into the sound of a nuclear electric ion-drive propulsion system. You might find them working with software like Ardour, Rosegarden, and portable high resolution audio recorders.

\StartSection{Cinematic Artist}

Cinematics play an important role in games. They prepare the user and set the stage in ways that would be difficult to do during normal game play. They work with tools like Blender, Cinelerra, and Lombard.

\StartSection{Engineer}

The engineers design, discuss, and implement the engine specification. They do not write just engine code, but sometimes write the Lua scripts that drive it. They work mostly in the languages of C++ and GLSL. 

They use the GNU Autotools as the \quote{construction scaffolding} around the project. It is important for ensuring the software stays as versatile as possible. 

To get started as an engineer with this project, see \in{chapter}[Engineer Contributors].

\StartSection{Modeller}

Modellers produce the 3D game models the user sees during game play. They also work with the 2D artists to ensure models are properly textured. They work with Blender, Wings 3D, or any other modelling program that supports standard patent free model formats.

\StartSection{Musician}

The musicians create either new or provide existing tracks for the game. The music falls into two categories. The first is in game ambient music that the user passively listens to. The second is music that is more actively listened to during navigation menus, cinematics, and possibly the separately to be released game soundtrack.

\StartSection{Package Maintainer}

Package maintainers are responsible for building and preparing Avaneya packages for the community. They are comfortable with standard Debian packages, the requisite structure, key signing, and any helper tools like CDBS, debhelper, and lintian that can aid with the tasks at hand. They integrate any hooks that need to execute at various times during user installation to perform required actions, such as registering a file extension or a desktop launchers in a freedesktop.org compliant manner. They know how to maintain an apt repository which is used to roll out our packages. When they do not know on how to do something, they know where to seek help.

Package maintainers must be careful to ensure our packages describe the appropriate dependencies that the game actually requires to function. Without that, a user's package manager cannot make automated intelligent decisions to reconcile dependencies. Sometimes this might mean being very careful to avoid known bugs in specific versions of our dependent runtime libraries and working around this problem possibly by including a dependency that, although already available within the user's distribution, as a custom patched or newer version within our own repository than can be expected within a distribution. This allows us to take advantage, for instance, of a bleeding edge feature like interfacing with an experimental API so that we can support Dolby Digital\high{\registered} (AC--3) or Digital Theatre System (DTS)\high{\registered} realtime surround sound encoding for users with hardware decoders.

There are many packages that need to be maintained, such as packages containing the game's main binary; media files, such as textures, models, audio, and cinematics; user extras, such as a screensaver or Plymouth theme; documentation; debugging symbols; developer related files, and so on. Each of these different types of objects should be contained within a distinct package representing a logical type. This must be in harmony with naming schemes users of major GNU distributions expect within their package management software. By breaking up Avaneya into multiple logical packages, we prevent users from having to update a single large package where it could have instead been split up into multiple packages such that only whatever they need to update, is updated.

Package maintainers need to work with the engineers to adapt the build environment to be flexible for producing packages for all supported architectures, such as {\tt amd64}, {\tt armel}, {\tt armhf}, {\tt i686}, {\tt lpia}, {\tt mips64el}, or what have you. They might also need to be familiar with the various quirks of different architectures and advise the engineers where possible when the crew have determined to commit to a given architecture.

Package maintainers must also be careful to ensure that package contents meet the packaging guidelines of a distribution so that they can be hosted in a given software channel (e.g. Ubuntu's {\it universe} or {\it multiverse}). This is sometimes a very difficult and tedious task and we must be careful to avoid stepping on the wrong toes. There is no point on spending years of labour only to not have it available anywhere where most users would have thought to have looked.

\StartSection{Researcher}

These people provide the background information and attention to detail that makes the game rich. They have an interest in {\it areology\index{areology}} (the study of Mars), terraforming, simulation and complex modelling, social and political issues (e.g. the \about[Genuine Progress Indicator]), and whatever else that might be useful.

\StartSection{Scripter}

Scripters write code in Lua that drives and breathes life into the game engine. They will work closely with the engineers to ensure the engine APIs they require are available as they need them.

\StartSection{SysOp}

System operators administer critical software that form the bridging point between what we produce and our interaction with the general public. A very large number of people are dependent on them to work properly. They must be protected from the nefarious. These areas include moderating forums, chatrooms, mailing lists, managing the issue tracker, and eventually monitoring the Solnet cluster described in \in{section}[Multiplayer: Solnet].

\StartSection{Tester}

Testers try to stress test applications, looking for their weakness and trying to break the application. They spot bugs as well as identify user interface and gameplay issues. They are very important for ensuring that nasty bugs are caught and issues rectified before the general public suffers. They are comfortable using the Launchpad interface, with prejudice for the issue tracker, and know how to generate stack traces and provide the usual critical information engineers require to replicate a problem.

\StartSection{Translator}

Translators are what makes Avaneya available to people of different languages. They ensure cinematic subtitles, graphical user interface, the website, and other interactive elements are properly internationalized. They work with any tools that support standard GNU gettext and language catalogues. They must be sensitive to the needs of a given locality.

\StartSection{Voice Actor}

Cinematics and in game audio often requires real people to play a role.

\StartSection{Web Developer}

Web developers are familiar with standards and work with things like CSS, XHTML, php, MySQL, and so on. They probably will end up coordinating with the system administrators.

\StartSection{Writer}

Writers are so important. They are so often overlooked, but are truly key to ensuring users get the best experience they can. They are needed wherever one must write primarily for humans and not machines.

Writers can be either of the creative or technical kind, whatever they are most comfortable with, or both. Those more interested in creative writing are concerned more with elements of story, dialogue, characters, in game fictional literature, and other creative fictional aspects of the game. Naturally, they work closely with researchers.

Technical writers, on the other hand, are responsible for writing and maintaining technical documentation which is just as important. They work with powerful document engineering software like \BIBTEX\ and \CONTEXT\ for typesetting our literature into books like this one, editing wikis, producing man and info pages, and so on.

All writers are responsible for maintaining, editing, and producing most of the documentation related to Avaneya. They are very important for editing material for grammar, spelling, and clarity. They often work closely with researchers, script writers, and other artists as they make suggestions to each other. 

To get started as a writer with this project, see \in{chapter}[Writer Contributors].

\StopChapter
