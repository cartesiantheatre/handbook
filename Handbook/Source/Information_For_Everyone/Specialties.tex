% Specialties section...
\startsection{Specialities}
Avaneya is a fairly large, creative, and collaborative project, calling upon a variety of different disciplines. These are the different capacities Avaneya contributors fall into. There is no reason why someone who wishes to work in more than one capacity cannot. Many do so already.
@sp 1

@table @emph
\item2D Artist
2D artists work on the graphical user interface managed by cegui, texturing for models, and other things of that sort. Their areas of expertise range from conceptual art, storyboarding, font design, bump maps, and procedural textures (materials). They can work with a range of software from FontForge to GIMP.
@sp 1

\itemAudio Engineer
These folks creatively can take unassuming sounds and transform them into things usable for science fiction game. An example would be sampling the squeak of a chair or a car driving by with a high end portable recorder and remixing it into the sound of a nuclear electric ion-drive propulsion system. You might find them working with software like Ardour, Rosegarden, and portable high resolution audio recorders.
@sp 1

\itemCinematic Artist
Cinematics play an important role in games. They prepare the user and set the stage in ways that would be difficult to do during normal game play. They work with tools like Blender, Cinelerra, and Lombard.
@sp 1

\itemEngineer
The engineers design, discuss, and implement the engine specification, along with writing the Lua script that drive the engine. They work mostly in the languages of C++ and GLSL. The GNU Autotools is useful in the way construction scaffolding is, and they depend on it to ensure the software stays as versatile as possible. There may be some opportunities for assembly level optimizations, but they largely work at a higher level of abstraction that the OGRE 3D rendering engine expects. They coordinate their work with the rest of the team via Bazaar and take care of distribution of pre-compiled binaries via packaging (e.g. debs).
@sp 1

\itemModeller
Modellers produce the 3D game models the user sees during game play. They also work with the 2D artists to ensure models are properly textured. They work with Blender, Wings 3D, or any other modelling program that supports standard patent free model formats.
@sp 1

\itemMusician
The musicians create either new or provide existing tracks for the game. The music falls into two categories. The first is in game ambient music that the user passively listens to. The second is music that is more actively listened to during navigation menus, cinematics, and possibly the separately to be released game soundtrack.
@sp 1

\itemResearcher
These people provide the background information and attention to detail that makes the game rich. They have an interest in @emph{areology} (the study of Mars), terraforming, simulation and complex modelling, social and political issues (e.g. the @emph{Genuine Progress Indicator}), and whatever else that might be useful.
@sp 1

\itemScripter
Scripters write code in Lua that drives and breathes life into the game engine. They will probably work with the engineers to ensure the functionality they require of the AresEngine is exposed safely.
@sp 1

\itemSystem Administrator
System administrators run and administer the user forum, moderate the IRC chatroom, monitor the bug tracker on Launchpad, and so on.
@sp 1

\itemTranslator
Translators are what makes Avaneya available to people of different languages. They ensure cinematic subtitles and the game's GUI, and website, are properly internationalized. They work with any tools that support standard GNU gettext and language catalogues.
@sp 1

\itemVoice Actor
Cinematics and in game audio often requires real people to play a role.
@sp 1

\itemWeb Developer
Web developers are familiar with standards and work with things like CSS, XHTML, php, MySQL, and so on. They probably will end up coordinating with the system administrators.
@sp 1

\itemWriter
Writers work closely with the researchers and other artists to provide dialogue and scripts for storyboarding and other game media.
@sp 1

@end table

