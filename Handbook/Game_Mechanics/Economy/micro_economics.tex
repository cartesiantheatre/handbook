% This is part of the Avaneya Project Crew Handbook.
% Copyright (C) 2010-2015 Cartesian Theatre <info@cartesiantheatre.com>.
% See the file Copying for details on copying conditions.

% Game Mechanics chapter...

\StartSection{Micro Economics}

Micro econommics deals with the actions of workforce units that impact the various markets in a city, from financials to food stuffs. This section deals with the how and why each individual unit will be influencing each of the cities markets and the broader economy.

For further informantion about workforce units refer to section [workforce.tex]. Also of relevance is section [companies???].


\StartSubsection{Consumption}
Through processes described in section [workforce.tex] each workforce unit will have set traits that determine its consumption patters. Each unit has preferences of consumption that will compete with its own resources - such as money - to be purchased. Through the course of a day an individual unit will spend resources to obtain food and water, keep its shelter (house/unit/etc), and purchase luxuries like scented candles.


\StartSubsection{Budget Constraints and Preferences}
%Consumption options will need to be in classes or organised and given heirarchies. %Perfect Substitutes.
%Income categories for consumerables or similar?

For a given amount of resources (money) for a given span of time, both of which are predicted by the unit to be consistent and regular, a unit will cover its needs. A unit with more money will purchase either more of each thing or more expensive versions of these things, being limited only by its interests and aversions (section[workforce.tex]).

For a quick example, lets compare the consumption choices of three units Bob, Roger, and Ian. In this example the time span and monetary amount are arbitrary. 

Bob makes 100 a week. With this he purchases his basics, enough to eat, drink, have a house, and get to work. He eats rice and beans and cheap takeaway; lives in a trailer park and catches public transport to work. 

Roger makes 200 a week. He purchases everything that Bob does but more of it and at a better quality. He eats steak, potatoes and vegetables as well as cheap takeaway; lives in a townhouse and catches public transport to work. In this example Roger does not make quite enough to purcahse a vehicle. Because Roger makes more than Bob, in this example we will say the Roger eats significantly more takeaway than Bob causing health problems.

Ian makes  1000 a week. This amount allows him to eat well and a lot more. He lives in a house and drives his own car to work. Ian's high consumption ability might cause him to eat a lot more food, however he happens to have an interest of "health", making him avoid junk food and over eating. His remaining income is spent of other things.

This example demonstrates that a units income will immediately cause it to purchase more and better things, with this being hindered only by an interest or other variable as defined in the section on workforce units [workforce.tex].


When a unit first gets its allocated resources (money) at the start of each time period, it will automatically map or plan the spending for that time period. While this is what units expect to happen things may change, affecting the units financial status unexpected and causing him to recalculate.

Lets look at another example. Take Ian again, part way through his weeks pay of 1000, his car is stolen, and he is mugged and bashed. Ian is left with only 250 for the remainder of the week, with no car and now medical fees to pay. This unit will then relcalculate its ability to live by meetings its basic needs while attempting to stay within its interests. From this we can see that Ian may live somewhat closes to how Roger lives for the rest of the week.


%formula for unit expense allocation:
Units have two types of financial accounting, ongoing income and current finances.

Ongoing incomes consist of payments the unit will have or is likely to have. A unit's periodic salary payments, business deals or other pending sales, and dividend payments are examples of ongoing incomes. These are payments the unit can expect to receive on a particular date.

Current finances on the other hand are money, or other assets with purchasing power, the unit has which have not been allocated to cover expenses. They are bulk and/or infrequent payments whose date of receipt is not certain.

Ongoing income is used for purchasing things that will require ongoing payments such as rent, mortage, utility bills, and payment plans. It is also used to determine what quality and quantity of food the unit consistenly purchases.

Current finances are used for ad-hoc spending that will either bolster the ongoing spending or will be allocated to random consumerables assoicated with the unit's interests and needs.


%Spreadsheet of all possible expenses.

At the base level a unit will break down its ongoing income amount into amounts, or percentages, that allow it to meet all its needs as best as possible to a base level; in order of hierarchy being oxygen, water, food, shelter, and belonging.

Game data on the costs of all these will be avaliable, that is the cost of housing, water, food, etc will be available for the unit to compare. The unit  allocates a percentage of its funds to meet all of these things in order of hierarchy.

From this point as a base the unit will allocate additional funds that remain, if any, to fulfill its needs to a higher level and to entertain its interests. 

For a detailed explaination on how units fulfill their needs refer to the section on Workforce Unit Needs under Workforce Attributes.


%Example
Lets have another, more detailed example. The games data shows that the cheapest dwelling in "City A" is 50 dollars a week, and the cheapest food is potatoes  meeting the unit's food needs for a week at 10 dollars. Assuming all other factors are free, and the unit is on 1000 dollars a week it has leftover 940 dollars, and in terms of percentages the unit has only used 6 percent of its income to meet its base needs. Table ... below demonstrates this allocation.

Income	1000	
                    Percentage 
Food	  10	  	 1	Subsection{Optimal Choices: Mapping Consumption Patterns}
House	  50	  	 5

Total	  60	  	 6


The unit can then increase its expenditure to meet its needs at a higher level, that is to live more expensively.

	 Percentage Increases	
Food	 1, 1.1, ...,  2,  ...,  83 
Housing	 5, 5.5, ..., 10,  ...,  16.6
Total	 6, 6.6, ..., 12,  ...,  99.6

Units wills naturally seek to maximise their expenditure to live as comfortable as possible with the exception of savings and investing (see subsection ... below).

The general rule is units will distribute their income evenly in attempts to satisfy needs, however if the unit can meet needs with money to spare it will seek to get greater utility out of its money for each purchased good used to meet its needs, in seeking greater utility the unit also factors in interests which determine how much the unit wants a particular purchase.


%Increase in Income and Expenditure 
A units access to high utility products will cause greater distribution of income into that need. Units gain access to higher quality products as they are invented, and as they become more readily available or become more affordable to the unit. A changing variety of products available for the unit to purchase in order to meet its needs means the unit must evaluate the costs against what it seeks to gain. 

%Cost and Utility Ratio
(New Product Cost / (New Product Utility + Interest)) / (Current Product Cost / (Current Product Utility + Interest))

In calculating the cost and ultility ratio we compare 5 things:

(1) New Product Cost (NC) - the cost of the product the unit may purchase to meet its need.
(2) New Product Utility (NU) - the utility the unit will gain from the new product (how much it will meet its need).
(3) Current Product Cost (CC) - the cost of the current product the unit purchases to meet its need.
(4) Current Product Utility (CU) - the utility the unit current gains from the product (how much its meeting its need).
(5) Interest (I) - whether the unit has an interest this product is associated with.

This will determine the level to which the unit will benefit in both utility and cost if it seeks to purchase another good. It shows how much the unit seeks to gain on a new purchase compared with how much money it will spend on this gain. Where the cost and utility ratio is less than 1 (1/1, .66/.66, etc) the unit will conduct the purchase.

%Utility Ratio and Pay Period Factor
The utility of a product or purchase is compared against the degree of need it seeks to feed, replenish, or provide.

In the section on Workforce Atrributes - Calculating Unit's Needs, the various demands of a unit are described. A unit will be able to calculate its basic age needs, on average, for a pay period by method described in that section - that being - comparing its age with its workforce type and expected activity in interests.  
 
From this a base level for each need that is required to be met for a pay period can be established:

Daily Need * Pay Period Duration = Period Need

Total Utility / Pay Period >= Need


From this calculation the unit will purchase goods for each need at the base cost level required to meet them, then use the cost and utility ratio to increase expenditure to the best product at the best price for all needs up the hierarchy.

This transforms the cost and utility ratio formula to: 
(((New Product Cost / (New Product Utility + Interest)) / (Current Product Cost / (Current Product Utility + Interest)) * pay period) / Period Need)

With a number at or less than 1 being required for the unit to approve its overall purchase schedule.


%Utility Ratio Comparing Needs Hierarchy
If all needs are met the unit will seek to benefit the need that will recieve the greatest utility for its level on the hierarchy. A unit will not purchase "Extra Clean Air" at 1000 dollars a tank for no benefit when it can buy fresh vegetables for 10 dollars at a large benefit. Thefore the cost utility ratio forumla needs to be multi axis, comparing the best utility of each product for one need against best utilities offered for products of the other needs.

In this way, as the unit conducts purchases to meet its needs, the final purchases are made after comparing the final cost utility ratio of each product for each need. Then all those with the best outcome (being those lowest from 1), will be purchased (provided they are within the units purchasing means).

%example
Unit uses cost utility ratio to meet food and belonging needs.
Unit goes into world and compares food items while also comparing social activities.
Cost utility ratio is used at shops to find food with a ratio value to .9, and used to calculate bowling with friends with a ratio of .5. After comparing these ratios (this ratio implies the unit is fed, given that it finds food less appealing than social activity) the unit will go bowling with friends.



%Purchasing goods for the period
-Where a units needs are met and will be met for a duration the unit may seek to purchase goods that will be consumed at a later date or used in recipes.

This occurs in much the same way that a unit will breakdown its expenditure to last its pay period. The unit will compare goods that will achive the greatest result for money spent to meet the unit's needs for the entire pay period and  purchase these goods.

%Example
As mentioned, a unit can calculate its average daily need requirements based on its age, worktype, interests and other varialbes see Workforce Attributes - Calculating Units Needs. When a unit goes to purchase goods to meet its needs, it can buy goods that will be consumed at a later date provided these goods are capable of being stored for the duration of the pay period. So if a unit has a pay period of 7 days, it will calculate the base money requirements to meet its needs, then conduct further calculations described previously in order to maximise its spending. Once its maximum spending requirements are obtained it can the use this as the basis for its daily purchases. Then, in some cases units will choose to buy things, say food, on day 1, that they will return home with and sit there for a number of days, say 3, and be consumed on day 4. When consumed the unit will be meeting its relevant needs on day 4.

Instances of purchasing goods and services to last for a later date will not always be applicable. Purchasing a social situation to meet a future need is not possible, for instance it does not make sense for a unit to play 2 rounds of bowling intending the second round to meet a social need that will arise in 4 days. Likewise units will not attend resturants and buy a weeks worth of food. Therefore places and items that are able to provide needs at a later date will be marked as being so.


%Example with advertisements (and other influences, ie third parties influencing consumption patterns and interests to add fake value to a product)

%Example where unit is already meeting some needs over others so disadvantaged factors will offer greater utility

%Example with multiple products in market place
When a unit is seeking to purchase a good and finds a better alternative it will hold the alternative for a duration where a chance for other options to be bought takes place.

This new option is then compared against other alternatives and the time frame is reset. Eventualy the unit will find the purchase that is best value for money and meets its interests.

In a city there will be many different alternatives so to limit the number of choices, units will only make comparisons when they become available to it. This will mostly be when the unit goes to purchase the good, and find that the marketplace contains many alternatives. This could also take place in the forms of advertisements or other influences.




%-------------------->This section may be redundant<-----------------------
%Where units purchase multiple goods for the one fulfillment (ie cooking a meal or making something).
%Workforce unit crafting in workforce_change.tex subsection{workforce crafting}
%Possible crafting will be required for other entities like companies and manufacturing etc. Will expand this if required for other entities anyway, otherway likely to be scrapped as unneeded. 
Units will sometimes purchase multiple, goods being several purchases of single goods, in order to turn these into a final product. This aspect, called crafting, is not always the method in which units will purchase things but the ability to combine multiple items into a single product (such as is done when cooking) is something units will sometimes do. On a side note, crafting is something undertaken by companies, which are akin to workforce units on a large scale.

For a detailed explaination on workforce units crafting refer to subsection "workforce crafting" in the workforce change section (game_mechanics/workforce/workforce change.tex). In this section we will discuss the method of how units make selections of final crafted products and the single items that make them up. 

-recipes compared for utility against eachother
-recipes compared for utility against finished products
-recipe utility factors in units ability to make quality

-individual items of recipe compared against other items for utility.

-compare the outcome of recipes for their utility, find the ingredients, and make; given the tools available.
%-------------------->This section may be redundant<-----------------------



%Spending for families/other units
When a unit needs to purchase a vartiety of goods for itself and one or more additional units it

Family group or its subgroup, the family group that the unit is living with, ie ID:F[gametime][location][.01], refer to Workforce Interaction - Group Dynamics. Similar to above example where units will sometimes buy goods to meet an entire period of needs, unit will buy goods to meet needs of family.
Family needs levels calculated the same. If member of family has need, unit >= certain age will go and purchase goods to fulfill needs and return this to needy unit. Else if no unit of required age unit of greatest age (more applicable to poverty situations, see below). When unit conducts purchase for entire pay period, unit will make purchase to meet needs of entire family and bring home. Family then consume goods that are stored at the home over duration of pay period.

Some needs will not be able to be purchased for the family, similar to how units cannot purchase a bowling social event in the present to meet a future belonging need. 
Young units needing to meet belonging needs will generally find these met by being with family, then as they age being able to go into the world themselves and meet these needs. See Workforce Attributes - Workforce Units Needs for detailed description.
%Process
Units in immediate family group determined (already determined via family group dynamics)
Needs of entire family calculated (already determined via unit needs calculation)
Unit required to meet need for family (ie family unit doesn't have money or can't walk, etc)
Unit able to meet need for family (ie it is possible like buying food and not impossible like buying friends)
Process of Utility Cost Ratio conducted



%Example


%Purchasing when a unit cannot possibly meet all needs
There will be situations where a unit is struck with poverty and will not be able to meet all of its needs. In this circumstance the unit will make the standard calculations as described above, but cut out needs up the hierarchy and recalculate until remaining needs can be met. This is of course provided lower needs have already been met.


%Starvation mode - see Workforce attribute also


%Savings
Units will save percentage of income if consumerables are above base level.

Percentage of income saved will increase as amount spent on consumerables increases.


Base = 0 saved

for each increase in expenditure above the base save an amount equal to the ... where the final result is a unit saving 50 percent and spending 50 percent.



%Loans
Units default avoid loans.

Units will take loans to to pay for goods when:
-Can't meet needs.
-Price of goods exceeds income.
	-as the price of certain goods increases and unit requires, unit taking a loan becomes more certain. 




%Investing

 

%Formula notes
Expenditure = Ongoing income 

Expenditure Variation = ongoing income + ( current finances / ? )

Multi-attribute utility model of riskless choice:
Ratio of - Utility of chosen option / sum of utility of all other options (source 1)

The observation that decision makers can be quite selective in information acquisition, particularly with larger sets of options and attributes, supports the idea that heuristic shortcuts are involved in riskless choice as well (source 1)

consumers cut off at price and availability (source 1)

in source 1 check out Payne et al., 1991)


%Formula for food costs
-checking out workforce_attributes.tex to associate with needs and interests.


%Formula for dwelling costs


%Formula for bills costs



%Example of unit with commitments.
Unit named "Greg" has income of 500 a week. He has pre-established and ongoing expenses of 150 for rent, 100 for bills, and 120 for food, totaling 370. Greg allocated 370 from his 500 each week to go to his ongoing expenses. 

%Example of unit with no commitments.
Unit named "Jim" has income of 500 a week. He has no commitments. He needs to meet his needs and interests. 

As Jim is setting up his inital selection of lifestyle he needs to factor in the costs for water, food, shelter and bills, along with other miscellaneous costs such as transportation, and all these need to meet the 500 dollar limit of his income (more complex debt based example to come).

The model will make available to "Jim" the data required to analyse and average out the various costs for living allowing "Jim" to make the appropraite selection based on the costs required to meet his needs.

Dwelling cost and average bills for this dwelling
Food costs (associated with interests)
Water costs
Transportation costs 




Let's say he has just purchased and consumed some food and water so this is met, and now needs to find shelter. The city he lives in will determine the manner in which he finds accomodation. After factoring in dwelling selection ratios he find a plac

%Ratios for dwelling selection
% -distance vs price
% -crime?
% -amenities
% -access to meet needs
% -access to meet interests


%Example of unit with commitments gradually being established. 

%Exmaple where debt is involved.

%Complete and complex example?

%Final formula encompassing everything.

% source 1 - consumer 1 - 

\StartSubsection{Optimal Choices: Mapping Consumption Patterns}
%Units needs to spend more on consumerables that offer them more, or in some cases at least appear to offer them more.
The complex of consumerables is a categorized mix of a priori value with value added by the producer cross referenced to the units consumption interests.

All goods products available in Avaneya for workforce units to consume are categorized then sub-categorized, for instance food is a category with a sub-category of vegetables. Within the sub-category of vegetables, each vegetable is organized in a hierarchy of goodness or value. Further, producers of vegetables (maybe farmers) have an impact on quality making zuchinis produced by some farmers better (an therfore worth more) than others. Finally some units may desire zuchinis more than others making them will to pay more, and the more units that like and buy zuchinis the greater the demand and price.

Also of note here is that some consumerables may not actually be better but are marketted as better. Marketting both increases the value of a consumerable and - if timed right - creates a consumption interest within the workforce unit. This is as true for cars as it is for zuchinis.   

From this and previous sections it can be seen that what a unit ultimately purcahses will be the best goods and services that it can most afford within a pay period after making a calculation at the start of that pay period so as not to leave it starving and homeless. And also as mentioned before, situations will arise causing units to forgo luxuries or prior determined purchases for less desirable options, or perhaps cause units to take drastic actions to survive (crime). Likewise, goods and services may become cheaper unexpectedly or from month to month, effectively making the unit wealthier within the domain of said purchases.

\StartSubsection{Investing, Savings, and Finances}
Each workforce unit will save an amount of its resources by default. This amount will be a set standard for all units except for those with either an interest that results in saving or with education variables that lead them to value saving or investing. Savings amounts will also be one of the las value placed on the use of a units resources for a given time frame, therefore for some periods of time certain workforce units will not save or may use some of their savings.

Saved resources will be stroed depending on the resource, of primary note is money. Money can be stored in a financial institution such as a bank or in a units place of residence (or perhaps other place such as behind a rock). Primarily units will store monetary resources in banks, however some units may have variables causing them to distrust such institutions, in this case they will store money physically. 


In addition to savings, units with the correct variables and in advanced cities will invest their resources into business' via stocks (and perhaps bonds). Financial investments are described in greater detail in section [...] but we will go briefly into investments as far as they interact with workforce units in the micro economy.

Similar to savings, workforce units with the appropriate variables and enough resources, will set aside an amount of money each week (on top of their savings) that will be invested in corporate stocks  (and or corporate and government bonds). Units will be able to invest money by reaching an appropriate financial instution or communicating with another workforce unit whose current job/occupation variable is label as "broker", this communication can take place directly in person or via distant communitcation such as phone or email. The act of investing has a fee associated with it for units that are not brokers. After this act is undertaken this unit will then need to be tagged with a variable that specifies what and how much of each investment it holds that remain open.     

What financials a unit invests in also varies with its degree of education and interests. Section [...] describes corporate entities. In this section it is noted that corporate entites will have numerical and hierarchical determinations of success and the better companies score of these the better they are to invest in. The better a workforce units level of education the higher up on a this hierarchy they will pick to invest in thier resource in. Further to this, units with only the most basic interest in investment will invest in companies assocaited with things they have other interests in and companies that have a media presence, such as advertisements.

%
%global free markets and units leaving cities
%
