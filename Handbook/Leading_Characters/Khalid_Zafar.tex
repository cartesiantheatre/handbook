% This is part of the Avaneya Project Crew Handbook.
% Copyright (C) 2010, 2011, 2012 Cartesian Theatre <kip@thevertigo.com>.
% See the file Copying for details on copying conditions.

% Khalid Zafar section...
\StartSection{Khalid Zafar}

\placetable[force,nonumber,right]{}
{
    \SetupCharacterTable

    \bTABLEbody

        \bTR 
            \bTD[nc=2] \midaligned{\rotate[rotation=42]{\color[red]{\bft Todo: Character's image goes here.}}} \eTD 
        \eTR

        \bTR 
            \bTD[nc=2] Khalid Zafar \eTD 
        \eTR

        \bTR
            \bTC Born \eTC
            \bTC 9 Pisces, 28 B.R. \eTC
        \eTR
        
        \bTR
            \bTC Birthplace \eTC
            \bTC Haraat al-Daraj, Gaza \eTC
        \eTR
        
        \bTR
            \bTC Gender \eTC
            \bTC Male \eTC
        \eTR
            
        \bTR
            \bTC Nationality \eTC
            \bTC Israeli \eTC
        \eTR
        
        \bTR
            \bTC Ethnicity \eTC
            \bTC Palestinian \eTC
        \eTR
        
        \bTR
          \bTC Hair \eTC
          \bTC Brown \eTC
        \eTR
        
        \bTR
            \bTC Eyes \eTC
            \bTC Green \eTC
        \eTR

        \bTR
            \bTC Age (Selection) \eTC % 19 Scorpius, 11 B.R.
            \bTC 17 MYrs / 32 Yrs \eTC % (11 + (18+318)/669)
        \eTR

        \bTR
            \bTC Age (Year Zero) \eTC
            \bTC 28 MYrs / 54 Yrs \eTC
        \eTR

        \bTR
            \bTC Education \eTC
            \bTC 
                \startitemize[4]
                \startpacked
                \item Electronics &\\Communications\\Engineering
                \item Artificial Intelligence &\\Robotics
                \item Philosophy 
                \stoppacked
                \stopitemize
            \eTC
        \eTR
        
        \bTR
            \bTC Occupation \eTC
            \bTC 
                \startitemize[4]
                \startpacked
                \item Professor of Computer\\Science
                \item Scientist
                \item Software Engineer
                \stoppacked
                \stopitemize
            \eTC
        \eTR
        
        \bTR
            \bTC Mission Titles \eTC
            \bTC 
                \startitemize[4]
                \startpacked
                \item Systems Team Lead
                \stoppacked
                \stopitemize
            \eTC
        \eTR
    \eTABLEbody

\eTABLE
}

Khalid was born 9 Pisces, 28 B.R. in Gaza's\index{Gaza} Daraj Quarter\index{Daraj Quarter}. His father was a fisherman and his mother a baker.

Khalid attended The University of Gaza\index{The University of Gaza} as an undergraduate to pursue an education in computer science. His studies were cut short when the department was destroyed in an air strike. With difficulty, he acquired the necessary paperwork with his father's assistance to leave Gaza to find work in Cairo where he intended to complete his studies. He managed to enrol in Cairo University in Giza after securing work to support himself in a local textile plant that produced North American Union military uniforms. After several years, he ended up graduating with a Bachelor of Science in Electronics and Communications Engineering. 

With the aid of scholarships, he was accepted for graduate studies at Sapienza University of Rome\index{Sapienza University of Rome} where he completed a MSc in Artificial Intelligence and Robotics. His thesis focused on addressing an old problem that had persisted in the field of artificial intelligence since its inception, the problem of {\it general intelligence}\index{Artificial intelligence+General intelligence}. That is, intelligence embodied in a machine that does not simply rely on {\it ad hoc} algorithms to solve specific problems, but sufficiently versatile enough that it can carry out, or even exceed, the cognitive capabilities of an actual human being.

Khalid then left Rome to take up a research position at Cambridge. He completed a PhD in Philosophy four years later. His novel philosophical approach to problem solving found its way into his doctoral thesis where he revisited another old and unsolved topic in the field. His paper, {\it Qualia: The Stubborn Elephant in the Room}\index{Qualia}, quickly rose to become a classic as one of the most cited in the field.

After completing his graduate studies, Khalid remained at Cambridge as both an instructor as well as to conduct postdoctoral research. He travelled to Brescia annually at the invitation of the European Union Association for Artificial Intelligence\index{European Union Association for Artificial Intelligence} as a regular speaker to attend their Symposiums on Artificial Intelligence\index{Symposiums on Artificial Intelligence}.

Khalid's laboratory at Cambridge was contracted to develop a number of technologies necessary for the Avaneya Initiative. Among other technologies, he was directly responsible for writing the integrated onboard artificial intelligence used by the {\it Mars Science Laboratory Curiosity XI}. This unmanned autonomous aerial vehicle was used to survey potential landing sites for the mission.

While still at Cambridge, Khalid was invited to participate in the Avaneya crew selection at UNSA's request. On 19 Scorpius, 11 B.R., after successfully enduring a battery of prescreening, interviews and training in Antarctica and Huelva, European Union, the selection committee presented him with an offer to accompany the Avaneya crew on their mission to Mars. He was informed that he would be travelling in the capacity of Systems Team Lead, a role responsible for leading a team of artificial intelligence specialists, knowledge engineers, cerebral mechanics, cyberneticists, and communication specialists. He accepted the offer without hesitation.

Khalid has been described by his peers as highly intelligent, amiable, and with a quirky sense of humour. Having little choice but to endure hardships in his early life, a positive outlook helped him through the arduous times.

