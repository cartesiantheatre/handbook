\StartSubsection{Taxation}
Through taxation income is gained by taking a portion of what is owned by a cities people, businesses, or industries and using this to fund city amenities, growth, projects, and militia among other things. (Taxation is also discussed in 8.8.9.)

In Avaneya, taxation levels range from taking nothing to taking everything and includes progessive tax systems such as negative income tax whereby money is given to population rather than taken. Managing the taxation levels effectiveness is part of the challenge of running the city and different levels work for different cities and styles of play. In the nature of true sandbox gaming, it may be possible to have 0 percent tax or even 100 percent tax and get it to work if players can divise methods for the population to remain happy or to quell descent. In addition players can create and implement there own styles of taxation through varying degrees of micromanagement on the tax screen. The below figures show the varying methods in which the population (and companies)can be taxed.


%TaxationA.gv
This figure indicates the basic structure of taxation using two base types of income entities, companies and people. Companies and persons generating income are taxed by the city, perhaps at different rates, which in turn generates an income for the city.  



%TaxationB.gv
%TaxationC.gv
In figure "TaxationB" we can see an example of utilizing a basic yet more narrowed method of taxation. In it all companies are taxed at 30 percent of their income except manufacturing companies which are taxed at 10 percent of their income. This example could be extended as in figure "TaxationC" to include farm based companies with the 10 percent taxation bracket and add an extra category of energy production companies and taxing them 20 percent. The point here is that there is no set boundary in which the player is told to base his or her taxation, and the level of detail the player goes into the deciding taxation is his or her own choice. With this in mind, players should be able to go even further into the managing of how entities are taxed. In relation to game design, various business entities will require classifications of various tiers to enable enforcement of player selected taxation styles. 
%TaxationD.gv
In figure ("taxationD") is a similar example to those shown previously with the addition of making an individual company pay a different level of tax to all others. This gives players complete control to determine what industires have more or less money and apply tax to each business appropriately, based on its benefit to the city (or other reason as seen fit by the player). However fair or unfair, this allows players to drive some businesses out of the city and encourage others. The legallity of suddenly changing tax brackets and those of individual companies is dicussed in section (???) on Law. There will be times where breaking various business deals, including the altering of taxation brackets, may incur fines, inspire military action against a players city, or drive down foreign investment.

Various levels of taxation can also be stacked up on all or certain types of companies or specifically named companies.
%TaxationE.gv
Figure ("TaxationE") shows a tax rate of 30 percent of all companies profit within this particular city. In addition to this, all manufacturing companies are taxed an additional 10 percent of profit. Metalworks Incorporated, a specific manufacturing company, gets hit with an additional 3 percent tax on all income. So there are multiple methods for players to sort and implement the taxation of entities in the city.

%TaxationF.gv
Please note that all of these examples could also be implemented in a similar manner for persons and families. In figure ("TaxationF") is a similar taxation system as those above but with the system based around the taxation of people. With a dynamic and creative tax system players can choose just how much to tax to what class of people. In the example families and sinlges are used but this could be 'high' or 'low' income levels, both, or many other things in multiple combinations.

%TaxationG.gv
In addition to these above examples, progressive methods of taxation can be implemented. See figure ("TaxationG"), in this figure negative income tax is partially implemented in both company and personal taxation schemes. All commerce companies are given 10 percent of the profit they generate and all low and middle income workforce are given an amount of money. The purpose of this figure is to represent how government, in the economies of certain cities, can pay the wages of some or maybe all workers in one method (ie paying companies) or another (paying working individuals). This method could be expanded to a point where the government does not tax anyone, but rather gives money to all income entities (business and workers). Several examples are: government income could be generated through initially taxing all companies 100 percent and then redistributing the income; generate income through high fees scheduled in ordinance; or "renting out" city land and resources. The primary message here is that numerous methods of distributing wealth are possible with players being able to be creative, resourceful, and fresh in their approach to managing city finance; with the only hinderance being how a particular strategy matches against the economic principles transcribed into the game.  


%--TAXATION SCREENS--%
The exact method in which players implement taxation in live gameplay can be seen in the image below of the taxation screen.
%Figure showing draft for taxation screen.
%Screen overlays to be put into each section. 
Players can initially choose to implement taxation based on the zone types, being: commercial, commercial &  habitat, habitat, and industrial. These are located on the (LEFT). Below these are other methods of implementing the taxation, in this example the player has set up so regions to micro manage the cities areas of taxation (among other things). Players can click on each of these buttons which will bring them to the next screen offering more options. 

On the (RIGHT) side of the initial taxation screen is a table showing every taxation scheme the player has implemented along with its details such as the taxation type and rate, income the taxation generates and whether tax from this is going up or down from a previous time period.


%Note: dollar signs and R to be changed to Jenya symbol.

From this figure we can see that people or businesses can be required to give a percentage of there income or recieve a percentage of there income. There is also an option to take or give exact amounts, or take or give types of resources. In this image the dollar sign symbolises varying income levels. These levels may be originally set at a default by the game (this may change with ordinance or other gameplay developments) but can be changed by the player. For instance the game might set "R" from R 0 to R15,000, and "RR" from R15,001 to R45,000 but the player can modify this to any value, such as "R" being R0 to R30,000 and "RR" being R30,001 to R100,000. 

On the taxation screen are various types of zones that can be taxed. Players select the zone type they want to manage. This then brings up a screen showing both the current income/profit/etc scheme as well as the currently selected taxation types being one or more of percent, exact monetary value, or resource. 
From this screen players can choose to select and edit either, current income or taxation type.

%current income
The current incomes bring up a graph representing the entire "population" of that zone type on y and its income levels on x. There are options to change what the graph represents such as earnings, take home pay, profit, expenses, loss, consumption. When each of these are selected the player can place vertical lines along x indicating a tax bracket. This can be done multiple times for different variables adding new graphs to the screen as they are created. (The settings of these variables will need to be programmed in such a way as they change with inflation so that there value relative to other varibale of the same city remains the same but the actual numbers change with inflation. This is done simply to be user friendly and can be turned off.) In this way a user has chosen just what to tax, how many variables are to be included in the tax, as what levels indicate a taxation threshold.

%taxation types
Going back to the zonal taxation screen, players can then select the taxation type screen. This brings up a screen showing the previously selected taxation variables each with various options to induce taxation. Represented is each of the tax thresholds, and each of the taxation variables for these thresholds. Players can then implement for each threshold what is taxed and what is subsidied by government. This is done by selecting a taxation type and dragging bars for each variable separately. Each bar represents a taxation type, those being monetary percent, exact monetary value, resource percent, exact resource value. Dragging the bar one way is taxation, the other is subsidies.

%examples
An example of the use of multiple variables in taxation is described:
A business can be taxed 5 percent of its profit, 2 percent of its earnings, R1,000,000 for earning over R5,000,000, and be required to give the government 100 tonnes (or even 1 percent) of some resource for every Z k/watts of energy it uses. The exact opposite could be done, there could only exist taxation of 10 percent of profit and no other taxation. 

Another example may be taxaing a household 5 percent on take home pay, 10 percent on energy consumption over Z k/watts, and subsidizing R5,000 for consuming energy under Z k/watts.


%further
At this stage of taxation management this is just a blanket taxation of all businesses/population in that zone type with that taxable threshold. Players can then go into another screen if they wish to change tax variables for different business types, and can further change variables for individual business names. From here taxation can be even more micromanged by players. Within each of the above mentioned tax types, players can create varying tax levels and types which effect different workforce groups and business types. Taxation factors may then include: income, city location, business type, polution level, family size, education level, workforce type, etc. In turn, higher or lower taxation is a method of shaping workforce and business direction, development, and evolution. 

All types of taxation can be turned on or off giving players flexibilty to create there own methods of managing a city in different situations.

Now this just deals with taxation and incentives. Players can enact ordinance which gives other incentives and limits to zone progress (dicsussed further in ordinance and also in the law chapter) and players can directly effect funding levels by spending government money or other projects. 


%creating boundaries and taxing different areas of the city by there boundary name.
Players can also use the (TOOL_NAME) tool create boroughs and localities for a variety of purposes, one of those being to create areas that receive different tax conditions.
 
%dividends from stock
If stock/shares are implemented, taxation should be able to used on stock just like any other taxable good.


%Taxation systems interaction with economic principles(???).
%Taxation forms a wedge between the buyer and seller price equilibrium

