% This is part of the Avaneya Project Crew Handbook.
% Copyright (C) 2010-2016 Cartesian Theatre <info@cartesiantheatre.com>.
% See the file Copying for details on copying conditions.

% Using Bzr section
\StartSection{Using Bzr}
As discussed already in \in{section}[Rationale for Bzr], we opted for Bzr as our revision control management software. We host the project's Bzr master branch on Launchpad. The branch contains the source code to Avaneya, and all relevant documentation -- including the source to the book you are reading now. Game media we are still trying to figure out.\footnotecite[launchpad2012_bzr_not_ready_digital_media]

This section contains instructions on how to interact with the repository. You can browse the master branch on Launchpad through a web browser at the following link.

\startnarrower[3*left]
\fullahref{https://code.launchpad.net/~avaneya/avaneya/trunk}
\stopnarrower

\StartSubSection{Installing Bzr Under Ubuntu}
To edit source code, you will need a local copy. A browser is fine for casual viewing, but not really a viable option as a contributor. To access the source locally, you will need to download and install Bzr. We will cover the installation on Ubuntu only, but other operating systems should have simple documentation available for them. 

Start by instructing your package manager to fetch and install the latest version of the Bzr software it has available to it.

\startCodeExample
$ sudo aptitude update
$ sudo aptitude install bzr 
\stopCodeExample

Bzr is a purely command line tool written in Python, however, there is a mature and attractive graphical user interface that integrates well into the popular Nautilus file manager for Gnome. If you would like to use it, run the following command to download and install it. 

Note that the second command will restart Nautilus to make it available without having to log out and back in again. Be careful, this will kill Nautilus so make sure it is not doing anything important like copying a file.

\startCodeExample
$ sudo aptitude install nautilus-bzr
$ killall -9 nautilus
\stopCodeExample

To {\tt identify} yourself to Bzr and in any future commits you make, run the following command. Try not to use handles, but use your actual name with proper spelling and punctuation if possible. The reason being is that we would like to maintain clean commit logs for the benefit of everyone.

\startCodeExample
$ bzr whoami "FirstName LastName <your@address.com>"
\stopCodeExample

\StartSubSection{Creation of a Local Branch}

If you do not know whether you have been granted write access to the master branch, you probably only have read access. You will know if you have write access because you are listed as a member of the \href{https://launchpad.net/~avaneya}{Avaneya Project Crew} on Launchpad.

Run the following command to retrieve a local copy of the Avaneya source to a folder called Avaneya in the current working directory. 

\startCodeExample
$ bzr branch lp:avaneya Avaneya
$ cd Avaneya
\stopCodeExample

This retrieves a local copy of the entire repository which is called a {\it branch} in the Bzr terminology. It can function entirely independent of a project's official repository ({\it upstream}). You can take it with you on the road, without internet connectivity, save changes incrementally as {\it revisions}, and even revert back to previous ones in the absence of internet connectivity.

However, if you happen to have write access to the repository, replace the word {\tt branch} in the previous example with {\tt checkout} instead. This would have looked as follows in this case.

\startCodeExample
$ bzr checkout lp:avaneya Avaneya
$ cd Avaneya
\stopCodeExample

The difference here between using {\tt branch} instead of {\tt checkout} is that the former is decentralized and not tied to our master branch. The latter, on the other hand, is centralized and tied to our master branch (write access). 

When saving changes in a branch that was retrieved using the {\tt branch} method, the changes are saved locally only. In the case of a {\tt checkout}, the changes will be saved on the remote project server if you have been given write permissions. We will discuss further how to do commits in \in{section}[Committing Changes].

\StartSubSection{Updating}
If you performed a {\tt checkout} instead of a {\tt branch} while following the instructions in \in{section}[Creation of a Local Branch], you can update your local branch to the latest revision available in Launchpad's master branch by running the following.

\startCodeExample
$ cd Avaneya
$ bzr update
\stopCodeExample

If instead you performed a {\tt branch} in \in{section}[Creation of a Local Branch], run the following substituting the {\tt branch} for {\tt pull}. Note that in either case internet connectivity is required.

\startCodeExample
$ cd Avaneya
$ bzr pull
\stopCodeExample

To view the latest {\it commit log} after either executing a {\tt pull} or an {\tt update}, run the following within the directory containing your local working copy. Hitting the {\tt q} key exits the log.

\startCodeExample
$ bzr log | less
\stopCodeExample

\StartSubSection{Committing Changes}
Go ahead and make any changes you like as you see fit. When you are done making those changes, you can then check to see a list of all the files that you added, modified, or removed that Bzr is tracking. Do this by running the following.

\startCodeExample
$ cd Avaneya/Documentation/Contributors/Handbook
$ bzr status .
added:
  Engineer_Contributors/Images/Fluid_Dynamics.png
modified:
  Makefile
  ReadMe
  Environment.tex
\stopCodeExample

The above example shows the user the status of their local branch within the current working directory. Bzr reports one file named {\tt Fluid_Dynamics.png} was added and three others were modified within the subdirectory containing the source to this handbook.

The next step is to send the changes either back {\it upstream} or locally into your own local branch. If the former, you want to make sure you update your local working copy first to verify there are no conflicts since you began your changes.

Depending on whether you have write access to the repository or not there are two methods to send your changes back upstream. If you have write access to the master branch, you can upload your changes directly into the master branch. Changes will be stored on the remote project server.

If you do not have write access, you generate what is called a {\it bundle} file which can be emailed to the public mailing list described in \in{section}[Public Discussion] for peer review. Your changes will commit locally only until a maintainer merges them into the master branch.

In either case, you need to first perform the commit like in the following example. Note that the {\tt --message} switch is used to specify the log message for the revision. If you need more than one line, omit this switch and you will be prompted to enter a multiline log entry in your default text editor.

\startCodeExample
$ bzr commit --message "Some commit log message..."
\stopCodeExample

At this point you are done if you had write access. If you did not, you need to inform upstream of a merge request so they can synchronize with your local branch's changes. You can do this by telling Bzr to generate the bundle which contains every commit you made locally, including log messages and any other book keeping details.

\startCodeExample
$ bzr bundle > ~/Desktop/MyChanges.bundle
\stopCodeExample

This will generate the bundle file on your desktop called {\tt MyChanges.bundle} which you can then email to the public mailing list described in \in{section}[Public Discussion] for peer review. The file is both machine and human readable making it mailing list friendly.

For the astute reader familiar already with unified diffs and other revision control systems, you may be wondering why the file has a {\tt .bundle} extension rather than a {\tt .patch} or {\tt .diff}. The reason for this is Bzr bundles are not vanilla unified diffs, though certainly similar. Bundles contain metadata specific to Bzr and so we do not recommend using any of the generic patch extensions to avoid confusion.


\StartSubSection{Installing Bzr Under Windows}

Bzr can be used on a Windows operating system using its command prompt or a graphical user interface. The software is available from Canonical at the link below by navigating to the downloads section.

\startnarrower[3*left]
\fullahref{http://bazaar.canonical.com/}
\stopnarrower

As Windows does not have SSH key authentication installed by default (the software to authenticate your login to Avaneya's Launchpad repository) this also requires download and installation. Two things are required, the SSH key generation software (assuming you don't have a key already) and the SSH key authentication software. Both can be found in the download section at the link below.

\startnarrower[3*left]
\fullahref{http://www.chiark.greenend.org.uk/~sgtatham/putty/}
\stopnarrower

At the Putty website, download the appropriate files for your needs and hardware. The first file you download, {\tt putty.exe} allows you to create an SSH key by following the steps. If you have a key already you can use it through {\tt pageant.exe} described below. Otherwise you will need to generate a new one using {\tt putty.exe}. Follow the steps displayed on screen.

Make sure all generated files appropriately and securely. Remember your passphrase for future use because it cannot be retrieved if forgotten. Also store your key in a secure backup to prevent the need for repeated key generation and pairing.

The {\tt pageant.exe} file is the second file you will need. It is the software that will help perform the authentication with Launchpad using your Launchpad ID when Bzr requires access.

To use your key, run {\tt pageant.exe}. This will allow you to select the {\it Add} key button. Using this you can navigate files and folders to find your saved SSH key you either already had or just generated using {\tt putty.exe}.


\StartSubSubSection{Bzr Launchpad Login}
To link Bzr to your Launchpad account open a command prompt. You can do this by searching for {\tt cmd} in Windows search. When the command prompt is open type the following replacing {\it <username>} with your Launchpad login.

\startCodeExample
bzr launchpad-login <username>
\stopCodeExample

\StartSubSubSection{Using Bzr in Command Prompt}
Bzr in the Windows command prompt is similar to using Bzr in a GNU terminal, like that available under Ubuntu. Open the command prompt again and type {\tt bzr}. This will bring up an initial list of commands allowing you to add files, make commits, and more.

To pull files using Bzr in the command prompt, change your current directory to the location of where you would like the Avaneya project by using the {\tt cd} command.

When you have changed directory you are ready to retrieve the Avaneya source by using the pull command. 

\startCodeExample
cd \documents\avaneya\trunk
bzr pull lp:avaneya
\stopCodeExample

You can then access the files, keep them up to date, and make appropriate changes and modifications. Once you are done you will be ready to push changes back to the Launchpad repository. Doing this in command prompt is simple:

\startCodeExample
bzr status
bzr commit -m "Your commit log message."
\stopCodeExample

The command prompt will then display a revision number and notify you that the commit has been completed, such as \quote{Committed revision 412.}

You may notice on the commit logs of our repository that there are specifications for entering certain types of commits and the manner in which it should be worded. Just have a browse through the recent revisions of the Avaneya trunk on Launchpad for examples. It is very simple.

\StartSubSection{Add New Files} 
If you have added new files to your local repository, as in created new files in the trunk or any of its sub--directories, those files are required to be added prior to a commit. First navigate to the appropriate directory using the change directory {\tt cd} command:

\startCodeExample
cd Avaneya/Documentation/Contributors/Handbook/Game_Mechanics
\stopCodeExample

You may then add the files using the {\tt Bzr} commands:

\startCodeExample
bzr add SomeFile.tex
\stopCodeExample

After this you can run the commit command previously mentioned.

\StartSubSubSection{Pushing Files to Launchpad}
At this stage any new files added, changed, and deleted are committed locally but they have not been uploaded to the online repository for everyone to access and pull to their own local systems. To upload them you will need to push your changes.

\startCodeExample
bzr push lp:avaneya
\stopCodeExample

The command prompt will then display attempts to locate your SSH key and appropriate SSH software. After a moment it will display a revision number, such as \quote{Pushed up to revision 412.} Remember that {\tt pageant.exe} must be opened with the correct key added to its list and your Launchpad username set before using Bzr to push to the Avaneya remote repository. These steps were outlined above.


\StartSubSubSection{Using a GUI}

Using Bzr in Windows through Bazaar Explorer is straight forward. Upon opening you will see a list of local projects using Bzr. If you have already pulled the files they will display in the box on the opening screen. Otherwise you will need to get the files from the server.

To pull files simply click the pull button on the top section of the buttons. A new window will open. In the location field enter \quote{lp:avaneya}. Click ok. You can then make changes to your files locally and browse files in the Bazaar Explorer in a similar manner to Windows Explorer.

Once you have made these changes and feel they are at an appropriate stage you can commit them by pressing the commit button. This will bring up another window. The branch should show, by default, your local repository. In the message box type the message you want publicly displayed when your commit is pushed. When you are done you can push files by clicking the push button.

A new window will open. Once again in location type \quote{lp:avaneya} and then click ok. Alternatively, if typing \quote{lp:avaneya} does not work, you can leave the default location in this box and click ok. This will then upload your locally modified files to Avaneya's Launchpad repository.

{\it Remember that Pageant and your private SSH key are required to be available during all remote authentications, whether in command prompt or Bazaar Explorer.}

