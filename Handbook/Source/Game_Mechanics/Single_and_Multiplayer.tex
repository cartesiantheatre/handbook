% This is part of the Avaneya Project Crew Handbook.
% Copyright (C) 2010, 2011, 2012
%   Kshatra Corp.
% See the file License for copying conditions.

% Single & Multiplayer section...
\StartSection{Single & Multiplayer}

Lets start by discussing the different ways users can play. This can be either single or multiplayer.

Single player, as the name suggest, the user plays alone. Well, actually, with the computer and the challenges it presents. This includes a tutorial, campaign, and custom mode. The user can restore from a saved game in all three modes.

In tutorial mode, the user is shown how to interact with elements of the game through the user interface by Khalid Zafar. This helps the user get comfortable finding their way around the game.

In campaign mode, the user assumes the role of Arda Baştürk, or any of the other leading characters described in \in{chapter}[Leading Characters], as they journey through the campaign which takes place near the end of the fictional timeline described in \in{chapter}[Timeline]. 

The custom mode allows users to play specific scenarios without thought of commitment to a full campaign. We will discuss scenarios further in \in{section}[Scenarios]. 

In multiplayer, the user can play over the local area network (LAN) with their friends in any of the available scenarios that allow multiplayer use. As in the case of the single player campaign, the user has the option of loading from a previous saved game.

But multiplayer offers another mode in addition to the former, {\it Solnet}\index{Solnet}. Solnet is the online service that allows users to join much larger groups for internet play. While both methods allow users to play cooperatively with each other, the latter method on a much grander scale. Users, having purchased a reasonable monthly subscription, gain access to the official Solnet server which hosts at least Arcadia Planitia following the fictional timeline's aftermath. 

These subscriptions are critical for ensuring the official server runs on reliable high quality hardware that can handle large concurrent loads. This is also vital for promoting additional development of the game's engine, artwork, music, and other graphical media that continue to make the game better and sustainable.

Playing over Solnet offers another important advantage that makes it distinct from playing over the LAN. Consider the problem of what to do with the world the user plays in after they have signed off from a multiplayer session. This is not a problem for a game hosted on the LAN because it is expected that no one cares after you and your friends pack up and have had enough. That logical cluster of space, objects, states, and so on, is destroyed and no machine need have it persist and updated. But in the case of playing over Solnet, that world must remain. It is persistent because you are sharing the same world with potentially many others who would not appreciate it if it was abrubtly destroyed simply because some of its users went to bed. Thus, the world hosted on Solnet carries on with a life of its own - even in your absence.

The following is a summary of the different modes of play:

\startitemize[4]
    \item Single Player
        \startitemize[4]
        \item Tutorial
        \item Campaign
        \item Custom
        \stopitemize

    \item Multiplayer
        \startitemize[4]
        \item Local Area Network (LAN)
        \item Solnet
        \stopitemize
\stopitemize

That concludes a general overview of who can play with who and by what medium, but we must now discuss what it is that the user does either by themselves or with others. Let us take a look now at the scenarios that users can engage in.

