% This is part of the Avaneya Project Crew Handbook.
% Copyright (C) 2010-2013 Cartesian Theatre <kip@thevertigo.com>.
% See the file Copying for details on copying conditions.

% Game Mechanics chapter...
\StartSection{Workforce & Economy}
What a city can produce, how much it can produce, and what it is capable of producing are largely byproduct of its workforce, the production, import and export ot goods, and internal use of the goods make up the cities economy. This section discusses the variety of workforces, what they represent, how they are created and maintained, and what they are capable of. It also deals with a cities income and expenditures and how these are managed by the player.  

\StartSubSection{Wokrforce}
All objects and buildings are produced by a form of labour, what this is may change with technology or player choice, but this labour force must be able to a) exist, and b) travel to production location.

A workforce is essentially another resource of either population of technology that needs to be either educated or built on a variety of levels and scales and therefore its creation, maintainence, and control needs to be carefully managed by the player. A labour workfoce requires certain factors to bring it into fruition which may include a certain level of education, access to work, and perhaps even restrictions. An intellectual workforce requires different factors to develop then manual labour workforce. While the two are not mutually exclusive, it becomes more difficult to create multiple classes in a single city. Workforce includes all labour factors driving the economy and extends to consumers. How people are raised, educated, and conditioned through experience largely determines their consumer patterns and may be reflected in a population's consumer choices.

Workforce is a dynamic resource that can move globally. If your workforce requires cheap housing and this becomes easier to gain in another city, many may migrate to this city seeking better opportunities. Likewise, they can move within your city to different opportunities. The movement within a city can be controlled to an extent to enable better or worse access to certain locations, increasing and decreasing travel time, or diverting travel for other means (likewise a workforce of machines can be moved with expense or stationary as in a plant). Whether workfoce movement can be depicted visually be small people objects moving around the city is yet to be determined, though would be cool. 

Basicly there are different fundamental types of workforce. There is a robotic/mechanical workforce and a human/biological/social workforce. The mechanical workforce requires research and development into avenues such a robotics and mechanical engineering with a social system in place requiring either the development of these to the eventual production and acceptance or the import of these and the acceptance by the population. The human workforce alternative is made up of human labour performing the tasks required by the society. This may be accepted to varying degrees also. Cities in Avaneya may also have a mix of these two workforce systems, indeed most will. 

There will be underlying conditions that will determing what a cities workforce is comprised of and players will have certain degress of influence over the direction the workforce takes but if certain conditions aren't met the citizenry may not accept or appreciate the current workforce type with the vary consequences of innaction being varried and obvious. How players make these conditions and the direction they take with there workforce is a choice made in the sandbox style of play, it is a creation. They can create anything from human/animal labour pool similar to the Ahmish to complete nanocracies with robotic workforces and human indulging in deccandent pleasure palaces. The survival and logevity rating of their choice is to be contended.



\StartSubSection{Economy}
%All to be expanded upon massively. So much work to do here. Economy may require own section?
A cities economy is a diverse and complex aspect of gameplay in Avaneya. However complex the implementation of the economy becomes, a players use of the economic aspects of the game should be fluent and intuitive allowing for degrees of complexity and involvement depending on a players style of play. This can be done with a good user interface and solid economic construstion. 

To begin we will discuss income streams for a players cities. These are primarily made up of taxation, exports and imports, and city resources and infrastructure. Within each of these categories are several methods which players can use to generate income.

\StartSubsection{Taxation}
Income is gained through taxation by taking a portion of what is owned by a cities people and/or businesses and industries and using this to fund city amenities, growth, projects, and militia among others. 
%[type of level to be set]
Taxation levels can range from taking nothing to attempting to take everything. Managing a taxation levels effectiveness is part of the challenge of running the city and different levels work for different cities and styles of play. Keeping in nature of true sandbox style of play, it would be cool to design the game where there is no ideal taxation level or range, where it may be possible to have 0 tax or even 100 tax and get it to work.

\StartSubSection{Exports and Imports}
Through exports and imports levies, fees, charges or taxation income can be gained for the city from its businesses and industries exporting goods or from other cities importing goods to the players cities. Additionally, exports increase the income of a city by bringing in wealth from other cities. Trade in general may increase the income of a city through established trade routes, larger markets, and peaceful diplomacy backed by financial gains. 

\StartSubSection{City Resources and Infrastructure}
Utilizing resources and land available to the city and the cities infrasture through privitization the city can generate revenue through lease agreements or similar. In addition players can have government run busineess which directly generate profit or loss. If ran correctly these business are a potential income stream. 

%Figure showing flow of city expense and income

The above figure shows how income is also generated from some of the cities expenses. Expenses can be seen of as investments varying in lengths and amounts of returns. Zoning for instance is a relatively cheap investment at city expense which has further investment form private capital to generate a form of growth which can be taxed or processed through fees to generate city income. This simple investment can be made more complex through adding infrastructure which may increase desirability and allow for higher fees or taxes. Other forms of investment, such as funding, divert money into projects the player/city desires allowing for complex growth along specific avenues. This is all explained in the above figure where paths of expense and income can be traced.

Next we will discuss city expenditures.

\StartSubSection{Zoning}
Through zoning a city expends money, representing beurocratic process, to designate lands or areas to be open or available to a development of a certain type, either: agricultural; commercial; habitat; habitat and commercial; industrial; or unzoned.

The zoned lands then opens up the land for building and/or utilization from private investment which may come from other cities or from earth. These zoned lands are also open to the cities player's development. 
%legal action re changing or interfering with zoned lands by player.
What is developed in each of these zone types is also dependant on the location of the city, the available workforce (population), city ordinance and funding, as well as infrastructure. Therefore factors such as designated areas of no polution, or organic produce area, essentially ordinance chosen by the player or cities population effect what growth occures on the zone, as will education of workforce, consumption choices of the workforce, and competition with other cities through taxes, rates, and other conditions. Finally, as mentioned, infrastructure effects zoned developement through accessability to workforce, to consumers, and to imports and exports.



\StartSubSection{Funding}
Funding designates city money to avenues chosen by the player or the cities population. Efficient use of this funding depends on a variety of factors through micro management. Funding can be directed to institutions such as universities or government research labs, or directed to commerical interests such as businesses or industry, or to city infrastructure or projects of similar vain. Funding allows for improvements in specific, directed areas of the city or cities development. Finding keeps students educated (or uneducated), transport network working (or falling apart), law inforcement honest (or corrupt), among others. It also allows for technological developments unique to a city or keeps a cities technological innovativeness in line with other cities, giving it a competitive edge. Funds can be spent wisely or unwisely and depending on a cities style, goal, or projected growth, certain avenues of funding become more or less important.

%Through funding players can generate income by ....

\StartSubSection{City Land} %work in progress, may be scrapped as conflicts with zoning. may expand zoning to include land being bought by other cities.
Players are designated an area of land to be owned by them, for use as they wish, their owned land can expand or contract depending on a variety of circumstances. 

Land ownded by a players city can be leased or sold to various interests with various conditions. In this manner a player can generate income from leasing or selling its cities land. This land may be bought by domestic business interest, terran business or government interests, or by other cities and players. The price of land may be dictated by players but its intrinsic value in dependant on its ordinance, local zoning, resource value, or growth potential. %(whatever growth potential is)
%legal action re changing or interfering with land bought by external interest. 
A potential problem for the city is that this land's use is then largely outside of city control and potential disasters or problems may not be solvable, leaving areas of disruption within the cities limits.



\StartSubSection{Militia}
Construciton of local militia is a state/city expense. This includes training, equipment, upkeep, and research and technology. Militia can be seen as an investment in public safety and there presence may attract investors or other interest wanting/requiring high levels of security.




\StartSubSection{City Business}
The creation of business interests setup, controlled, and owned by the player/city allows for profit from these projects to return to the city player, however it also requires constly outlays and potential for massive loss. Players may wish to setup player/government owned mining companies, housing estates, industries, or agriculture. 



\StartSubSection{Ordinance}
City ordinance is projects, initiatives, or requirements a city sets in place for itself and other bodies. Examples of ordinance may be a organic agriculture innitiative, free higher education, or polution free zones. Ordinances can be broad or specific and can be designated to certain areas of a city or apply to the city as a whole, they can also target certain groups or bodies of a city. Ordinances give players more control over what is produced in the city and where. They can generate income if used correctly by have an initial and/or ongoing cost due to it being implemented by government agents.

Types of ordinance available depend on workforce types, funding levels, curresntly established interets, agents in play, and other ordinance that is active. 




\StartSubSection{Infrastructure}
Infrastrucure is covered in the transport networks section [to be set]. In terms of a cities economy, its infrastructure allows for degrees of access by its population, transfer of workforce, and goods and services, as well as providing access to import and export markets. Types of infrastructure depend on a cities level of development, relationships with neighbouring cities, ordinance and costs. 


From the explaination above it can be seen how a cities economy is a diverse and dynamic aspect of gameplay, interacting with various inputs and outputs. 





\StartSubSection{Resources}
The final component of a cities economy is its resources. Resources themselves don't play a direct role in the economy, its more their volume, demand, value, and how they are utilzed which effects the economy. A city can take advantage to the resources it had available, giving it wealth through cheap materials or exports. Depending on technology, these resources can be obtained and used in different ways. Resources either imported of exported can be used to build or fund infrasture, their extraction can be the backbone of business and zoning, and its effective use may require ordinance being implemented. In addition, resources may attract negative attendtion from external interests and produce conflict. From this is can be seen that resources play a unique role to economic fundamentals, being either the fundamental point on which an economy is based, or hold a tertiary status with the city, being utilized for high levels of processing or political interest.



\StartSubSection{City Buildings}

\StartSubSection{Services}
