% This is part of the Avaneya Project Crew Handbook.
% Copyright (C) 2010, 2011, 2012
%   Kshatra Corp.
% See the file License for copying conditions.

% Preface chapter...
\StartChapter{Preface}

The {\it Avaneya Project Crew Handbook} is a guide for new and current contributors to the Avaneya project, a cerebral science fiction game for GNU. 

This document will likely continue to be revised and improved with time. Therefore, because it is a living document, it is probably better to pass on a \href{\LatestHandbookURL}{link} to the latest revision in our list of project blueprints\footnote[]{See \in{section}[Blueprints] for details on project blueprints.} than it is to send a potentially outdated local copy to another person.

This handbook has several purposes:

\startitemize[4]
\item
To encourage project crew to refine their ideas, think them through, and put them to paper where they are available for consideration;

\item
To define the project as clearly as possible so that everyone involved in contributing understands it;

\item
To provide the project with a definition in a single, consolidated, canonical location;

\item
To politely consider those already engaged in the project from having to explain those things redundant with the handbook.\footnote[rtfm]{It is usually a good idea to ask questions. But as we say in the hacker community, albeit in a slightly more colourful form, consider reading the documentation first.}
\stopitemize

This book is probably not as useful to normal users as it is to contributors. It may even spoil elements of the game for the former, so consider yourself warned. On the other hand, normal users are just as likely to enrich the game through constructive feedback which is best buttressed with a well informed understanding of the project.

It is also important to bear in mind that it is not necessary for users to be familiar with the relevant details contained within this handbook any more than it is for patrons of theatre to be familiar with the details stage technicians working with lighting and acoustics are. However, if you happen to be a stage technician, you need to be.

\StopChapter

