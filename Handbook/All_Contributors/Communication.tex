% This is part of the Avaneya Project Crew Handbook.
% Copyright (C) 2010, 2011, 2012
%   Kshatra Corp.
% See the file License for copying conditions.

% Communication section...
\StartSection{Communication}
The team uses two primary means of intercommunication. The first is through mailing lists and the second is over IRC. The Code of Conduct described in \in{section}[Avaneya Code of Conduct] applies to both.

\StartSubSection{Mailing Lists}
Mailing lists are not new. They have been around for literally decades. They lack the cosmetics and ease of use of a forum, but until our website is ready, it will suffice for the time being. Avaneya has three mailing lists, each serving a different purpose.

\StartSubSubSection{Announcements}
The first is a unidirectional announcement mailing list. It is low volume and only intended to communicate from the project to the general public about major project press releases. Anyone can subscribe to it. It has many hundreds of subscribers and grows on a daily basis.

\blank[2*big]
\startnarrower[3*left]
Subscribe:\crlf
\fullahref{https://www.avaneya.com/lists/?p=subscribe&id=1} 

\blank[2*big]
Browse the archives:\crlf
\fullahref{https://www.avaneya.com/news/announcements/}
\stopnarrower

\StartSubSubSection{Public Discussion}
The second \mailto{avaneya@lists.avaneya.com} is for anyone to discuss anything related to the project. It is open to everyone.

\blank[2*big]
\startnarrower[3*left]
Subscribe:\crlf
\fullahref{http://lists.avaneya.com/listinfo.cgi/avaneya-avaneya.com} 

\blank[2*big]
Browse the archives:\crlf
\fullahref{http://lists.avaneya.com/pipermail/avaneya-avaneya.com/}
\stopnarrower

\StartSubSubSection{Private Discussion}
The third \mailto{avaneya-private@lists.avaneya.com} is open only to project crew. Topics range from game storyline spoilers, unreleased conceptual art, security vulnerabilities, and so on. If you are a contributing member of our community, you should have been sent a subscription invitation. 

Appropriate topics range from game storyline spoilers, unreleased conceptual art, security vulnerabilities, and so on. Like Vegas, what goes on in that list stays in that list.

If you are still confused as to why a free software project should need a private mailing list in the first place, see the first point of the {\it Avaneya Code of Conduct} in \in{section}[Avaneya Code of Conduct].

\blank[2*big]
\startnarrower[3*left]
Subscribe:\crlf
\fullahref{http://lists.avaneya.com/listinfo.cgi/avaneya-private-avaneya.com} 

\blank[2*big]
Browse the archives:\crlf
\fullahref{http://lists.avaneya.com/private.cgi/avaneya-private-avaneya.com/}
\stopnarrower

\StartSubSubSection{Etiquette}
When using either of the latter two mailing lists, you may find the following tips useful.

\startitemize[4]
\item
{\it Do not ever send email formatted in HTML.} Some mail clients cannot render it and you should not assume everyone is using the same software that you are. Even if they are using an HTML capable mail client, there is no guarantee that it will come out the same. People with visually impairments who rely on speech synthesizers can find the latter unreliable when parsing HTML since it is much more difficult for it to interpret. Yes, we do have fans with visual impairments. In other cases, users may even have to pay for additional bandwidth to retrieve your email since HTML encoded email is much larger in size in comparison with plain text. The reasons go on.\footnotecite[html_mail_bad]

\item
When you reply to a post, remember to {\it reply to the list and not just the original sender privately} - unless that was your intention. Mailing lists are setup so that everyone who subscribe to them can benefit. Sometimes this may not happen until years later when a new subscriber searches through old archives to find a solution to a problem they were having that was solved long ago by someone who had the same issue.

\item
When replying to a post, if you have your subscription configured to use batch digest mode,\footnote{When batch digest mode is enabled, the server will {\it batch} together emails into groups and then send it to you as a single compilation to cut down on the amount of email you receive.} {\it you do not need to copy the whole digest}. Just quote the minimum needed for context.

\item
{\it Check the subject heading of your reply} to a message posted on the list to make sure it still reflects the original post. Some mail readers, if you have batch digest mode enabled in your subscription, will change the heading to reflect the batch digest's subject heading, instead of the specific message within it you are replying to.

\item
When you reply, remember to {\it reply at the bottom and not at the top of the message}. Top posting is generally not encouraged because it makes preservation of chronological order difficult to follow for readers.\footnotecite[top_posting]
\stopitemize

\StartSubSection{Internet Relay Chat}

IRC is among the oldest forms of realtime chat over the internet. Avaneya has a channel ({\bf \tt \type{#}avaneya}) on the Freenode server ({\bf \tt irc.freenode.net}). You can use whatever client you like, but it is recommended you use one that supports SSL.

Make sure you register your chosen nick name with the {\it nickserv} bot on Freenode. This ensures you are consistently identifiable to others in the chat room.

Whenever you would like to send someone a message publicly in the channel, you should precede your message with their nick name. This is because many people have their IRC clients configured to alert them audibly when that happens, as opposed to every time anyone says anything in the channel. Usually you only need to type the first few letters of their nick name and hit tab to have your client complete it.


